%&preformat-disser
\RequirePackage[l2tabu,orthodox]{nag} % Раскомментировав, можно в логе получать рекомендации относительно правильного использования пакетов и предупреждения об устаревших и нерекомендуемых пакетах
% Формат А4, 14pt (ГОСТ Р 7.0.11-2011, 5.3.6)
\documentclass[a4paper,14pt,oneside,openany]{memoir}

%%%%%%%%%%%%%%%%%%%%%%%%%%%%%%%%%%%%%%%%%%%%%%%%%%%%%%
%%%% Файл упрощённых настроек шаблона диссертации %%%%
%%%%%%%%%%%%%%%%%%%%%%%%%%%%%%%%%%%%%%%%%%%%%%%%%%%%%%

%%% Инициализирование переменных, не трогать!  %%%
\newcounter{intvl}
\newcounter{otstup}
\newcounter{contnumeq}
\newcounter{contnumfig}
\newcounter{contnumtab}
\newcounter{pgnum}
\newcounter{chapstyle}
\newcounter{headingdelim}
\newcounter{headingalign}
\newcounter{headingsize}
%%%%%%%%%%%%%%%%%%%%%%%%%%%%%%%%%%%%%%%%%%%%%%%%%%%%%%

%%% Область упрощённого управления оформлением %%%

%% Интервал между заголовками и между заголовком и текстом %%
% Заголовки отделяют от текста сверху и снизу
% тремя интервалами (ГОСТ Р 7.0.11-2011, 5.3.5)
\setcounter{intvl}{3}               % Коэффициент кратности к размеру шрифта

%% Отступы у заголовков в тексте %%
\setcounter{otstup}{0}              % 0 --- без отступа; 1 --- абзацный отступ

%% Нумерация формул, таблиц и рисунков %%
% Нумерация формул
\setcounter{contnumeq}{0}   % 0 --- пораздельно (во введении подряд,
                            %       без номера раздела);
                            % 1 --- сквозная нумерация по всей диссертации
% Нумерация рисунков
\setcounter{contnumfig}{0}  % 0 --- пораздельно (во введении подряд,
                            %       без номера раздела);
                            % 1 --- сквозная нумерация по всей диссертации
% Нумерация таблиц
\setcounter{contnumtab}{1}  % 0 --- пораздельно (во введении подряд,
                            %       без номера раздела);
                            % 1 --- сквозная нумерация по всей диссертации

%% Оглавление %%
\setcounter{pgnum}{1}       % 0 --- номера страниц никак не обозначены;
                            % 1 --- Стр. над номерами страниц (дважды
                            %       компилировать после изменения настройки)
\settocdepth{subsection}    % до какого уровня подразделов выносить в оглавление
\setsecnumdepth{subsection} % до какого уровня нумеровать подразделы


%% Текст и форматирование заголовков %%
\setcounter{chapstyle}{1}     % 0 --- разделы только под номером;
                              % 1 --- разделы с названием "Глава" перед номером
\setcounter{headingdelim}{1}  % 0 --- номер отделен пропуском в 1em или \quad;
                              % 1 --- номера разделов и приложений отделены
                              %       точкой с пробелом, подразделы пропуском
                              %       без точки;
                              % 2 --- номера разделов, подразделов и приложений
                              %       отделены точкой с пробелом.

%% Выравнивание заголовков в тексте %%
\setcounter{headingalign}{1}  % 0 --- по центру;
                              % 1 --- по левому краю

%% Размеры заголовков в тексте %%
\setcounter{headingsize}{0}   % 0 --- по ГОСТ, все всегда 14 пт;
                              % 1 --- пропорционально изменяющийся размер
                              %       в зависимости от базового шрифта

%% Подпись таблиц %%

% Смещение строк подписи после первой строки
\newcommand{\tabindent}{0cm}

% Тип форматирования заголовка таблицы:
% plain --- название и текст в одной строке
% split --- название и текст в разных строках
\newcommand{\tabformat}{plain}

%%% Настройки форматирования таблицы `plain`

% Выравнивание по центру подписи, состоящей из одной строки:
% true  --- выравнивать
% false --- не выравнивать
\newcommand{\tabsinglecenter}{false}

% Выравнивание подписи таблиц:
% justified   --- выравнивать как обычный текст («по ширине»)
% centering   --- выравнивать по центру
% centerlast  --- выравнивать по центру только последнюю строку
% centerfirst --- выравнивать по центру только первую строку (не рекомендуется)
% raggedleft  --- выравнивать по правому краю
% raggedright --- выравнивать по левому краю
\newcommand{\tabjust}{justified}

% Разделитель записи «Таблица #» и названия таблицы
\newcommand{\tablabelsep}{~\cyrdash\ }

%%% Настройки форматирования таблицы `split`

% Положение названия таблицы:
% \centering   --- выравнивать по центру
% \raggedleft  --- выравнивать по правому краю
% \raggedright --- выравнивать по левому краю
\newcommand{\splitformatlabel}{\raggedleft}

% Положение текста подписи:
% \centering   --- выравнивать по центру
% \raggedleft  --- выравнивать по правому краю
% \raggedright --- выравнивать по левому краю
\newcommand{\splitformattext}{\raggedright}

%% Подпись рисунков %%
%Разделитель записи «Рисунок #» и названия рисунка
\newcommand{\figlabelsep}{~\cyrdash\ }  % (ГОСТ 2.105, 4.3.1)
                                        % "--- здесь не работает

%%% Цвета гиперссылок %%%
% Latex color definitions: http://latexcolor.com/
\definecolor{linkcolor}{rgb}{0.9,0,0}
\definecolor{citecolor}{rgb}{0,0.6,0}
\definecolor{urlcolor}{rgb}{0,0,1}
%\definecolor{linkcolor}{rgb}{0,0,0} %black
%\definecolor{citecolor}{rgb}{0,0,0} %black
%\definecolor{urlcolor}{rgb}{0,0,0} %black
            % общие настройки шаблона
%%% Проверка используемого TeX-движка %%%
\newif\ifxetexorluatex   % определяем новый условный оператор (http://tex.stackexchange.com/a/47579)
\ifxetex
    \xetexorluatextrue
\else
    \ifluatex
        \xetexorluatextrue
    \else
        \xetexorluatexfalse
    \fi
\fi

\newif\ifsynopsis           % Условие, проверяющее, что документ --- автореферат

\usepackage{etoolbox}[2015/08/02]   % Для продвинутой проверки разных условий
\providebool{presentation}

\usepackage{comment}    % Позволяет убирать блоки текста (добавляет
                        % окружение comment и команду \excludecomment)

%%% Поля и разметка страницы %%%
\usepackage{pdflscape}  % Для включения альбомных страниц
\usepackage{geometry}   % Для последующего задания полей

%%% Математические пакеты %%%
\usepackage{amsthm,amsmath,amscd}   % Математические дополнения от AMS
\usepackage{amsfonts,amssymb}       % Математические дополнения от AMS
\usepackage{mathtools}              % Добавляет окружение multlined
\usepackage{xfrac}                  % Красивые дроби
\usepackage[
    locale = DE,
    list-separator       = {;\,},
    list-final-separator = {;\,},
    list-pair-separator  = {;\,},
    list-units           = single,
    range-units          = single,
    range-phrase={\text{\ensuremath{-}}},
    % quotient-mode        = fraction, % красивые дроби могут не соответствовать ГОСТ
    fraction-function    = \sfrac,
    separate-uncertainty,
    ]{siunitx}[=v2]                 % Размерности SI
\sisetup{inter-unit-product = \ensuremath{{}\cdot{}}}

% Кириллица в нумерации subequations
% Для правильной работы требуется выполнение сразу после загрузки пакетов
\patchcmd{\subequations}{\def\theequation{\theparentequation\alph{equation}}}
{\def\theequation{\theparentequation\asbuk{equation}}}
{\typeout{subequations patched}}{\typeout{subequations not patched}}

%%%% Установки для размера шрифта 14 pt %%%%
%% Формирование переменных и констант для сравнения (один раз для всех подключаемых файлов)%%
%% должно располагаться до вызова пакета fontspec или polyglossia, потому что они сбивают его работу
\newlength{\curtextsize}
\newlength{\bigtextsize}
\setlength{\bigtextsize}{13.9pt}

\makeatletter
%\show\f@size    % неплохо для отслеживания, но вызывает стопорение процесса,
                 % если документ компилируется без команды  -interaction=nonstopmode
\setlength{\curtextsize}{\f@size pt}
\makeatother

%%% Кодировки и шрифты %%%
\ifxetexorluatex
    \ifpresentation
        \providecommand*\autodot{} % quick fix for polyglossia 1.50
    \fi
    \PassOptionsToPackage{no-math}{fontspec}    % https://tex.stackexchange.com/a/26295/104425
    \usepackage{polyglossia}[2014/05/21]        % Поддержка многоязычности
                                        % (fontspec подгружается автоматически)
\else
   %%% Решение проблемы копирования текста в буфер кракозябрами
    \ifnumequal{\value{usealtfont}}{0}{}{
        \input glyphtounicode.tex
        \input glyphtounicode-cmr.tex %from pdfx package
        \pdfgentounicode=1
    }
    \usepackage{cmap}   % Улучшенный поиск русских слов в полученном pdf-файле
    \ifnumequal{\value{usealtfont}}{2}{}{
        \defaulthyphenchar=127  % Если стоит до fontenc, то переносы
                                % не впишутся в выделяемый текст при
                                % копировании его в буфер обмена
    }
    \usepackage{textcomp}
    \usepackage[T1,T2A]{fontenc}                    % Поддержка русских букв
    \ifnumequal{\value{usealtfont}}{1}{% Используется pscyr, при наличии
        \IfFileExists{pscyr.sty}{\usepackage{pscyr}}{}  % Подключение pscyr
    }{}
    \usepackage[utf8]{inputenc}[2014/04/30]         % Кодировка utf8
    \usepackage[english, russian]{babel}[2014/03/24]% Языки: русский, английский
    \makeatletter\AtBeginDocument{\let\@elt\relax}\makeatother % babel 3.40 fix
    \ifnumequal{\value{usealtfont}}{2}{
        % http://dxdy.ru/post1238763.html#p1238763
        \usepackage[scaled=0.914]{XCharter}[2017/12/19] % Подключение русифицированных шрифтов XCharter
        \usepackage[charter, vvarbb, scaled=1.048]{newtxmath}[2017/12/14]
        \ifpresentation
        \else
            \setDisplayskipStretch{-0.078}
        \fi
    }{}
\fi

%%% Оформление абзацев %%%
\ifpresentation
\else
    \indentafterchapter     % Красная строка после заголовков типа chapter
    \usepackage{indentfirst}
\fi

%%% Цвета %%%
\ifpresentation
\else
    \usepackage[dvipsnames, table, hyperref]{xcolor} % Совместимо с tikz
\fi

%%% Таблицы %%%
\usepackage{longtable,ltcaption} % Длинные таблицы
\usepackage{multirow,makecell}   % Улучшенное форматирование таблиц
\usepackage{tabu, tabulary}      % таблицы с автоматически подбирающейся
                                 % шириной столбцов (tabu обязательно
                                 % до hyperref вызывать)
\makeatletter
%https://github.com/tabu-issues-for-future-maintainer/tabu/issues/26
\@ifpackagelater{longtable}{2020/02/07}{
\def\tabuendlongtrial{%
    \LT@echunk  \global\setbox\LT@gbox \hbox{\unhbox\LT@gbox}\kern\wd\LT@gbox
                \LT@get@widths
}%
}{}
\makeatother

\usepackage{threeparttable}      % автоматический подгон ширины подписи таблицы

%%% Общее форматирование
\usepackage{soulutf8}% Поддержка переносоустойчивых подчёркиваний и зачёркиваний
\usepackage{icomma}  % Запятая в десятичных дробях

%%% Оптимизация расстановки переносов и длины последней строки абзаца
\IfFileExists{impnattypo.sty}{% проверка установленности пакета impnattypo
    \ifluatex
        \ifnumequal{\value{draft}}{1}{% Черновик
            \usepackage[hyphenation, lastparline, nosingleletter, homeoarchy,
            rivers, draft]{impnattypo}
        }{% Чистовик
            \usepackage[hyphenation, lastparline, nosingleletter]{impnattypo}
        }
    \else
        \usepackage[hyphenation, lastparline]{impnattypo}
    \fi
}{}

%% Векторная графика

\usepackage{tikz}                   % Продвинутый пакет векторной графики
\usetikzlibrary{chains}             % Для примера tikz рисунка
\usetikzlibrary{shapes.geometric}   % Для примера tikz рисунка
\usetikzlibrary{shapes.symbols}     % Для примера tikz рисунка
\usetikzlibrary{arrows}             % Для примера tikz рисунка

%%% Гиперссылки %%%
\ifxetexorluatex
    \let\CYRDZE\relax
\fi
\usepackage{hyperref}[2012/11/06]

%%% Изображения %%%
\usepackage{graphicx}[2014/04/25]   % Подключаем пакет работы с графикой
\usepackage{caption}                % Подписи рисунков и таблиц
\usepackage{subcaption}             % Подписи подрисунков и подтаблиц
\usepackage{pdfpages}               % Добавление внешних pdf файлов

%%% Счётчики %%%
\usepackage{aliascnt}
\usepackage[figure,table]{totalcount}   % Счётчик рисунков и таблиц
\usepackage{totcount}   % Пакет создания счётчиков на основе последнего номера
                        % подсчитываемого элемента (может требовать дважды
                        % компилировать документ)
\usepackage{totpages}   % Счётчик страниц, совместимый с hyperref (ссылается
                        % на номер последней страницы). Желательно ставить
                        % последним пакетом в преамбуле

%%% Продвинутое управление групповыми ссылками (пока только формулами) %%%
\ifpresentation
\else
    \usepackage[russian]{cleveref} % cleveref имеет сложности со считыванием
    % языка из babel. Такое решение русификации вывода выбрано вместо
    % определения в documentclass из опасности что-то лишнее передать во все
    % остальные пакеты, включая библиографию.

    % Добавление возможности использования пробелов в \labelcref
    % https://tex.stackexchange.com/a/340502/104425
    \usepackage{kvsetkeys}
    \makeatletter
    \let\org@@cref\@cref
    \renewcommand*{\@cref}[2]{%
        \edef\process@me{%
            \noexpand\org@@cref{#1}{\zap@space#2 \@empty}%
        }\process@me
    }
    \makeatother
\fi

\usepackage{placeins} % для \FloatBarrier

\ifnumequal{\value{draft}}{1}{% Черновик
    \usepackage[firstpage]{draftwatermark}
    \SetWatermarkText{DRAFT}
    \SetWatermarkFontSize{14pt}
    \SetWatermarkScale{15}
    \SetWatermarkAngle{45}
}{}

%%% Цитата, не приводимая в автореферате:
% возможно, актуальна только для biblatex
%\newcommand{\citeinsynopsis}[1]{\ifsynopsis\else ~\cite{#1} \fi}

% если текущий процесс запущен библиотекой tikz-external, то прекомпиляция должна быть включена
\ifdefined\tikzexternalrealjob
    \setcounter{imgprecompile}{1}
\fi

\ifnumequal{\value{imgprecompile}}{1}{% Только если у нас включена предкомпиляция
    \usetikzlibrary{external}   % подключение возможности предкомпиляции
    \tikzexternalize[prefix=images/cache/,optimize command away=\includepdf] % activate! % здесь можно указать отдельную папку для скомпилированных файлов
    \ifxetex
        \tikzset{external/up to date check={diff}}
    \fi
}{}

%% кастомные пакеты

\usepackage{subcaption}
\usepackage{stmaryrd}
\usepackage{mathpartir}
\usepackage{tikz}
\usepackage{thm-restate}
\usepackage{thmtools,thm-restate}
\usepackage{xspace}
\usepackage{multicol}
% \usepackage{prooftree}
\usepackage{xifthen}

\usepackage{bussproofs}
\EnableBpAbbreviations
         % Пакеты общие для диссертации и автореферата
\synopsisfalse                      % Этот документ --- не автореферат
\input{Dissertation/dispackages}    % Пакеты для диссертации
\input{Dissertation/userpackages}   % Пакеты для специфических пользовательских задач

%%%%%%%%%%%%%%%%%%%%%%%%%%%%%%%%%%%%%%%%%%%%%%%%%%%%%%
%%%% Файл упрощённых настроек шаблона диссертации %%%%
%%%%%%%%%%%%%%%%%%%%%%%%%%%%%%%%%%%%%%%%%%%%%%%%%%%%%%

%%% Инициализирование переменных, не трогать!  %%%
\newcounter{intvl}
\newcounter{otstup}
\newcounter{contnumeq}
\newcounter{contnumfig}
\newcounter{contnumtab}
\newcounter{pgnum}
\newcounter{chapstyle}
\newcounter{headingdelim}
\newcounter{headingalign}
\newcounter{headingsize}
%%%%%%%%%%%%%%%%%%%%%%%%%%%%%%%%%%%%%%%%%%%%%%%%%%%%%%

%%% Область упрощённого управления оформлением %%%

%% Интервал между заголовками и между заголовком и текстом %%
% Заголовки отделяют от текста сверху и снизу
% тремя интервалами (ГОСТ Р 7.0.11-2011, 5.3.5)
\setcounter{intvl}{3}               % Коэффициент кратности к размеру шрифта

%% Отступы у заголовков в тексте %%
\setcounter{otstup}{0}              % 0 --- без отступа; 1 --- абзацный отступ

%% Нумерация формул, таблиц и рисунков %%
% Нумерация формул
\setcounter{contnumeq}{0}   % 0 --- пораздельно (во введении подряд,
                            %       без номера раздела);
                            % 1 --- сквозная нумерация по всей диссертации
% Нумерация рисунков
\setcounter{contnumfig}{0}  % 0 --- пораздельно (во введении подряд,
                            %       без номера раздела);
                            % 1 --- сквозная нумерация по всей диссертации
% Нумерация таблиц
\setcounter{contnumtab}{1}  % 0 --- пораздельно (во введении подряд,
                            %       без номера раздела);
                            % 1 --- сквозная нумерация по всей диссертации

%% Оглавление %%
\setcounter{pgnum}{1}       % 0 --- номера страниц никак не обозначены;
                            % 1 --- Стр. над номерами страниц (дважды
                            %       компилировать после изменения настройки)
\settocdepth{subsection}    % до какого уровня подразделов выносить в оглавление
\setsecnumdepth{subsection} % до какого уровня нумеровать подразделы


%% Текст и форматирование заголовков %%
\setcounter{chapstyle}{1}     % 0 --- разделы только под номером;
                              % 1 --- разделы с названием "Глава" перед номером
\setcounter{headingdelim}{1}  % 0 --- номер отделен пропуском в 1em или \quad;
                              % 1 --- номера разделов и приложений отделены
                              %       точкой с пробелом, подразделы пропуском
                              %       без точки;
                              % 2 --- номера разделов, подразделов и приложений
                              %       отделены точкой с пробелом.

%% Выравнивание заголовков в тексте %%
\setcounter{headingalign}{1}  % 0 --- по центру;
                              % 1 --- по левому краю

%% Размеры заголовков в тексте %%
\setcounter{headingsize}{0}   % 0 --- по ГОСТ, все всегда 14 пт;
                              % 1 --- пропорционально изменяющийся размер
                              %       в зависимости от базового шрифта

%% Подпись таблиц %%

% Смещение строк подписи после первой строки
\newcommand{\tabindent}{0cm}

% Тип форматирования заголовка таблицы:
% plain --- название и текст в одной строке
% split --- название и текст в разных строках
\newcommand{\tabformat}{plain}

%%% Настройки форматирования таблицы `plain`

% Выравнивание по центру подписи, состоящей из одной строки:
% true  --- выравнивать
% false --- не выравнивать
\newcommand{\tabsinglecenter}{false}

% Выравнивание подписи таблиц:
% justified   --- выравнивать как обычный текст («по ширине»)
% centering   --- выравнивать по центру
% centerlast  --- выравнивать по центру только последнюю строку
% centerfirst --- выравнивать по центру только первую строку (не рекомендуется)
% raggedleft  --- выравнивать по правому краю
% raggedright --- выравнивать по левому краю
\newcommand{\tabjust}{justified}

% Разделитель записи «Таблица #» и названия таблицы
\newcommand{\tablabelsep}{~\cyrdash\ }

%%% Настройки форматирования таблицы `split`

% Положение названия таблицы:
% \centering   --- выравнивать по центру
% \raggedleft  --- выравнивать по правому краю
% \raggedright --- выравнивать по левому краю
\newcommand{\splitformatlabel}{\raggedleft}

% Положение текста подписи:
% \centering   --- выравнивать по центру
% \raggedleft  --- выравнивать по правому краю
% \raggedright --- выравнивать по левому краю
\newcommand{\splitformattext}{\raggedright}

%% Подпись рисунков %%
%Разделитель записи «Рисунок #» и названия рисунка
\newcommand{\figlabelsep}{~\cyrdash\ }  % (ГОСТ 2.105, 4.3.1)
                                        % "--- здесь не работает

%%% Цвета гиперссылок %%%
% Latex color definitions: http://latexcolor.com/
\definecolor{linkcolor}{rgb}{0.9,0,0}
\definecolor{citecolor}{rgb}{0,0.6,0}
\definecolor{urlcolor}{rgb}{0,0,1}
%\definecolor{linkcolor}{rgb}{0,0,0} %black
%\definecolor{citecolor}{rgb}{0,0,0} %black
%\definecolor{urlcolor}{rgb}{0,0,0} %black
      % Упрощённые настройки шаблона

%% programming languages abbreviations

\newcommand{\Java}{Java\xspace}
\newcommand{\JVM}{JVM\xspace}
\newcommand{\CLANG}{C\xspace}
\newcommand{\CPP}{C/C++\xspace}
\newcommand{\JS}{JavaScript\xspace}
\newcommand{\LLVM}{LLVM\xspace}

%% memory models abbreviations

\newcommand{\MM}[1]{\ensuremath{\mathsf{#1}}\xspace}

\newcommand{\SC}{\MM{SC}}
\newcommand{\DRFx}{\MM{DRFx}}

\newcommand{\Intel}{\MM{x86}}
\newcommand{\TSO}{\MM{TSO}}
\newcommand{\SPARC}{\MM{SPARC}}
\newcommand{\ARM}{\MM{ARM}}
\newcommand{\ARMv}[1]{\MM{ARMv{#1}}}
\newcommand{\POWER}{\MM{POWER}}
\newcommand{\RISC}{\MM{RISC\text{-}V}}

\newcommand{\CMM}{\MM{C11}}
\newcommand{\RCMM}{\MM{RC11}}

\newcommand{\Prm}{\MM{Promising}}
\newcommand{\Wkm}{\MM{Weakestmo}}
\newcommand{\WkmS}{\MM{Weakestmo2}}
\newcommand{\MRD}{\MM{MRD}}
\newcommand{\PwP}{\MM{PwP}}

%% proof assistants 

\newcommand{\coq}{\textsc{Coq}\xspace}
\newcommand{\gallina}{\textsc{Gallina}\xspace}
\newcommand{\mathcomp}{\textsc{MathComp}\xspace}
\newcommand{\analysis}{\textsc{MathComp-Analysis}\xspace}
\newcommand{\finmap}{\textsc{finmap}\xspace}
\newcommand{\relationalgebra}{\textsc{relation-algebra}\xspace}
\newcommand{\ssreflect}{\textsc{SSReflect}\xspace}
\newcommand{\equations}{\textsc{Equations}\xspace}

\newcommand{\agda}{\textsc{Agda}\xspace}
\newcommand{\arend}{\textsc{Arend}\xspace}
\newcommand{\idris}{\textsc{Idris}\xspace}
\newcommand{\isabelle}{\textsc{Isabelle/HOL}\xspace}

%% tools 

\newcommand{\hmc}{\textsc{HMC}\xspace}
\newcommand{\hmclbf}{$\hmc_{\lbf}$\xspace}
\newcommand{\RCMC}{\textsc{RCMC}\xspace}
\newcommand{\rcmc}{\textsc{rcmc}\xspace}
\newcommand{\genmc}{\textsc{GenMC}\xspace}
\newcommand{\lockmc}{\textsc{LAPOR}\xspace}
\newcommand{\genmcmath}{\textnormal{\genmc}\xspace}
\newcommand{\Tracer}{\textsc{Tracer}\xspace}
\newcommand{\Herd}{\textsc{Herd}\xspace}
\newcommand{\PPCMEM}{\textsc{PPCMEM}\xspace}
\newcommand{\ARMMEM}{\textsc{ARMMEM}\xspace}
\newcommand{\CPPMEM}{\textsc{CPPMEM}\xspace}
\newcommand{\TriCheck}{\textsc{TriCheck}\xspace}
\newcommand{\rmem}{\textsc{rmem}\xspace}
\newcommand{\Nidhugg}{\textsc{Nidhugg}\xspace}
\newcommand{\CDSChecker}{\textsc{CDS\-Checker}\xspace}
\newcommand{\CBMC}{\textsc{CBMC}\xspace}
\newcommand{\Dartagnan}{\textsc{Dartagnan}\xspace}
\newcommand{\Verisoft}{\textsc{Verisoft}\xspace}
\newcommand{\CHESS}{\textsc{CHESS}\xspace}
\newcommand{\wmc}{\textsc{WMC}\xspace}
           % кастомные макросы

% Новые переменные, которые могут использоваться во всём проекте
% ГОСТ 7.0.11-2011
% 9.2 Оформление текста автореферата диссертации
% 9.2.1 Общая характеристика работы включает в себя следующие основные структурные
% элементы:
% актуальность темы исследования;
\newcommand{\actualityTXT}{Actuality.}
% степень ее разработанности;
\newcommand{\progressTXT}{Background.}
% цели и задачи;
\newcommand{\aimTXT}{Aim}
\newcommand{\tasksTXT}{tasks}
% научную новизну;
\newcommand{\noveltyTXT}{Scientific novelty.}
% теоретическую и практическую значимость работы;
%\newcommand{\influenceTXT}{Теоретическая и практическая значимость}
% или чаще используют просто
\newcommand{\influenceTXT}{Theoretical and practical influence.}
% методологию и методы исследования;
\newcommand{\methodsTXT}{Methodology and research methods.}
% положения, выносимые на защиту;
\newcommand{\defpositionsTXT}{The main results submitted for defense.}
% степень достоверности и апробацию результатов.
\newcommand{\reliabilityTXT}{Reliability}
\newcommand{\probationTXT}{Approbation.}

\newcommand{\contributionTXT}{personal contribution}
\newcommand{\publicationsTXT}{Publications.}


%%% Заголовки библиографии:

% для автореферата:
\newcommand{\bibtitleauthor}{Публикации автора по теме диссертации}

% для стиля библиографии `\insertbiblioauthorgrouped`
\newcommand{\bibtitleauthorvak}{В изданиях из списка ВАК РФ}
\newcommand{\bibtitleauthorscopus}{В изданиях, входящих в международную базу цитирования Scopus}
\newcommand{\bibtitleauthorwos}{В изданиях, входящих в международную базу цитирования Web of Science}
\newcommand{\bibtitleauthorother}{В прочих изданиях}
\newcommand{\bibtitleauthorconf}{В сборниках трудов конференций}
\newcommand{\bibtitleauthorpatent}{Зарегистрированные патенты}
\newcommand{\bibtitleauthorprogram}{Зарегистрированные программы для ЭВМ}

% для стиля библиографии `\insertbiblioauthorimportant`:
\newcommand{\bibtitleauthorimportant}{Наиболее значимые \protect\MakeLowercase\bibtitleauthor}

% для списка литературы в диссертации и списка чужих работ в автореферате:
\newcommand{\bibtitlefull}{Список литературы} % (ГОСТ Р 7.0.11-2011, 4)
         % Новые переменные, для всего проекта

%%% Основные сведения %%%
\newcommand{\thesisAuthorLastName}{Моисеенко}
\newcommand{\thesisAuthorOtherNames}{Евгений Александрович}
\newcommand{\thesisAuthorInitials}{E.\,A.}
\newcommand{\thesisAuthor}             % Диссертация, ФИО автора
{%
    \texorpdfstring{% \texorpdfstring takes two arguments and uses the first for (La)TeX and the second for pdf
        \thesisAuthorLastName~\thesisAuthorOtherNames% так будет отображаться на титульном листе или в тексте, где будет использоваться переменная
    }{%
        \thesisAuthorLastName, \thesisAuthorOtherNames% эта запись для свойств pdf-файла. В таком виде, если pdf будет обработан программами для сбора библиографических сведений, будет правильно представлена фамилия.
    }
}
\newcommand{\thesisAuthorShort}        % Диссертация, ФИО автора инициалами
{\thesisAuthorInitials~\thesisAuthorLastName}
%\newcommand{\thesisUdk}                % Диссертация, УДК
%{\fixme{xxx.xxx}}
\newcommand{\thesisTitle}              % Диссертация, название
{Семантика многопоточных систем с слабыми моделями памяти на основе структур событий}
\newcommand{\thesisSpecialtyNumber}    % Диссертация, специальность, номер
{\fixme{XX.XX.XX}}
\newcommand{\thesisSpecialtyTitle}     % Диссертация, специальность, название (название взято с сайта ВАК для примера)
{\fixme{XXX}}
%% \newcommand{\thesisSpecialtyTwoNumber} % Диссертация, вторая специальность, номер
%% {\fixme{XX.XX.XX}}
%% \newcommand{\thesisSpecialtyTwoTitle}  % Диссертация, вторая специальность, название
%% {\fixme{Теория и~методика физического воспитания, спортивной тренировки,
%% оздоровительной и~адаптивной физической культуры}}
\newcommand{\thesisDegree}             % Диссертация, ученая степень
{кандидата физико-математических наук}
\newcommand{\thesisDegreeShort}        % Диссертация, ученая степень, краткая запись
{канд. физ.-мат. наук}
\newcommand{\thesisCity}               % Диссертация, город написания диссертации
{Санкт-Петербург}
\newcommand{\thesisYear}               % Диссертация, год написания диссертации
{\the\year}
\newcommand{\thesisOrganization}       % Диссертация, организация
{\fixme{Федеральное государственное автономное образовательное учреждение высшего
образования <<Длинное название образовательного учреждения <<АББРЕВИАТУРА>>}}
\newcommand{\thesisOrganizationShort}  % Диссертация, краткое название организации для доклада
{\fixme{НазУчДисРаб}}

\newcommand{\thesisInOrganization}     % Диссертация, организация в предложном падеже: Работа выполнена в ...
{кафедре системного программирования Санкт-Петербургского государственного университета}

%% \newcommand{\supervisorDead}{}           % Рисовать рамку вокруг фамилии
\newcommand{\supervisorFio}              % Научный руководитель, ФИО
{Кознов Дмитрий Владимирович}
\newcommand{\supervisorRegalia}          % Научный руководитель, регалии
{доктор технических наук, доцент, профессор кафедры системного программирования}
\newcommand{\supervisorFioShort}         % Научный руководитель, ФИО
{Д.\,В.~Кознов}
\newcommand{\supervisorRegaliaShort}     % Научный руководитель, регалии
{д-р техн. наук, проф.}

%% \newcommand{\supervisorTwoDead}{}        % Рисовать рамку вокруг фамилии
%% \newcommand{\supervisorTwoFio}           % Второй научный руководитель, ФИО
%% {\fixme{Фамилия Имя Отчество}}
%% \newcommand{\supervisorTwoRegalia}       % Второй научный руководитель, регалии
%% {\fixme{уч. степень, уч. звание}}
%% \newcommand{\supervisorTwoFioShort}      % Второй научный руководитель, ФИО
%% {\fixme{И.\,О.~Фамилия}}
%% \newcommand{\supervisorTwoRegaliaShort}  % Второй научный руководитель, регалии
%% {\fixme{уч.~ст.,~уч.~зв.}}

\newcommand{\opponentOneFio}           % Оппонент 1, ФИО
{\fixme{Фамилия Имя Отчество}}
\newcommand{\opponentOneRegalia}       % Оппонент 1, регалии
{\fixme{доктор физико-математических наук, профессор}}
\newcommand{\opponentOneJobPlace}      % Оппонент 1, место работы
{\fixme{Не очень длинное название для места работы}}
\newcommand{\opponentOneJobPost}       % Оппонент 1, должность
{\fixme{старший научный сотрудник}}

\newcommand{\opponentTwoFio}           % Оппонент 2, ФИО
{\fixme{Фамилия Имя Отчество}}
\newcommand{\opponentTwoRegalia}       % Оппонент 2, регалии
{\fixme{кандидат физико-математических наук}}
\newcommand{\opponentTwoJobPlace}      % Оппонент 2, место работы
{\fixme{Основное место работы c длинным длинным длинным длинным названием}}
\newcommand{\opponentTwoJobPost}       % Оппонент 2, должность
{\fixme{старший научный сотрудник}}

%% \newcommand{\opponentThreeFio}         % Оппонент 3, ФИО
%% {\fixme{Фамилия Имя Отчество}}
%% \newcommand{\opponentThreeRegalia}     % Оппонент 3, регалии
%% {\fixme{кандидат физико-математических наук}}
%% \newcommand{\opponentThreeJobPlace}    % Оппонент 3, место работы
%% {\fixme{Основное место работы c длинным длинным длинным длинным названием}}
%% \newcommand{\opponentThreeJobPost}     % Оппонент 3, должность
%% {\fixme{старший научный сотрудник}}

\newcommand{\leadingOrganizationTitle} % Ведущая организация, дополнительные строки. Удалить, чтобы не отображать в автореферате
{\fixme{XXX}}

\newcommand{\defenseDate}              % Защита, дата
{\fixme{DD mmmmmmmm YYYY~г.~в~XX часов}}
\newcommand{\defenseCouncilNumber}     % Защита, номер диссертационного совета
{\fixme{Д\,123.456.78}}
\newcommand{\defenseCouncilTitle}      % Защита, учреждение диссертационного совета
{\fixme{Название учреждения}}
\newcommand{\defenseCouncilAddress}    % Защита, адрес учреждение диссертационного совета
{\fixme{Адрес}}
\newcommand{\defenseCouncilPhone}      % Телефон для справок
{\fixme{+7~(0000)~00-00-00}}

\newcommand{\defenseSecretaryFio}      % Секретарь диссертационного совета, ФИО
{\fixme{Фамилия Имя Отчество}}
\newcommand{\defenseSecretaryRegalia}  % Секретарь диссертационного совета, регалии
{\fixme{д-р~физ.-мат. наук}}            % Для сокращений есть ГОСТы, например: ГОСТ Р 7.0.12-2011 + http://base.garant.ru/179724/#block_30000

\newcommand{\synopsisLibrary}          % Автореферат, название библиотеки
{\fixme{Название библиотеки}}
\newcommand{\synopsisDate}             % Автореферат, дата рассылки
{\fixme{DD mmmmmmmm}\the\year~года}

% To avoid conflict with beamer class use \providecommand
\providecommand{\keywords}%            % Ключевые слова для метаданных PDF диссертации и автореферата
{}
             % Основные сведения
\input{common/fonts}            % Определение шрифтов (частичное)
\input{common/styles}           % Стили общие для диссертации и автореферата
%%% Переопределение именований, если иначе не сработает %%%
%\gappto\captionsrussian{
%    \renewcommand{\chaptername}{Глава}
%    \renewcommand{\appendixname}{Приложение} % (ГОСТ Р 7.0.11-2011, 5.7)
%}

%%% Изображения %%%
\graphicspath{{images/}{Dissertation/images/}}         % Пути к изображениям

%%% Интервалы %%%
%% По ГОСТ Р 7.0.11-2011, пункту 5.3.6 требуется полуторный интервал
%% Реализация средствами класса (на основе setspace) ближе к типографской классике.
%% И правит сразу и в таблицах (если со звёздочкой)
%\DoubleSpacing*     % Двойной интервал
\OnehalfSpacing*    % Полуторный интервал
%\setSpacing{1.42}   % Полуторный интервал, подобный Ворду (возможно, стоит включать вместе с предыдущей строкой)

%%% Макет страницы %%%
% Выставляем значения полей (ГОСТ 7.0.11-2011, 5.3.7)
\geometry{a4paper, top=2cm, bottom=2cm, left=2.5cm, right=1cm, nofoot, nomarginpar} %, heightrounded, showframe
\setlength{\topskip}{0pt}   %размер дополнительного верхнего поля
\setlength{\footskip}{12.3pt} % снимет warning, согласно https://tex.stackexchange.com/a/334346

%%% Выравнивание и переносы %%%
%% http://tex.stackexchange.com/questions/241343/what-is-the-meaning-of-fussy-sloppy-emergencystretch-tolerance-hbadness
%% http://www.latex-community.org/forum/viewtopic.php?p=70342#p70342
\tolerance 1414
\hbadness 1414
\emergencystretch 1.5em % В случае проблем регулировать в первую очередь
\hfuzz 0.3pt
\vfuzz \hfuzz
%\raggedbottom
%\sloppy                 % Избавляемся от переполнений
\clubpenalty=10000      % Запрещаем разрыв страницы после первой строки абзаца
\widowpenalty=10000     % Запрещаем разрыв страницы после последней строки абзаца
\brokenpenalty=4991     % Ограничение на разрыв страницы, если строка заканчивается переносом

%%% Блок управления параметрами для выравнивания заголовков в тексте %%%
\newlength{\otstuplen}
\setlength{\otstuplen}{\theotstup\parindent}
\ifnumequal{\value{headingalign}}{0}{% выравнивание заголовков в тексте
    \newcommand{\hdngalign}{\centering}                % по центру
    \newcommand{\hdngaligni}{}% по центру
    \setlength{\otstuplen}{0pt}
}{%
    \newcommand{\hdngalign}{}                 % по левому краю
    \newcommand{\hdngaligni}{\hspace{\otstuplen}}      % по левому краю
} % В обоих случаях вроде бы без переноса, как и надо (ГОСТ Р 7.0.11-2011, 5.3.5)

%%% Оглавление %%%
\renewcommand{\cftchapterdotsep}{\cftdotsep}                % отбивка точками до номера страницы начала главы/раздела

%% Переносить слова в заголовке не допускается (ГОСТ Р 7.0.11-2011, 5.3.5). Заголовки в оглавлении должны точно повторять заголовки в тексте (ГОСТ Р 7.0.11-2011, 5.2.3). Прямого указания на запрет переносов в оглавлении нет, но по той же логике невнесения искажений в смысл, лучше в оглавлении не переносить:
\setrmarg{2.55em plus1fil}                             %To have the (sectional) titles in the ToC, etc., typeset ragged right with no hyphenation
\renewcommand{\cftchapterpagefont}{\normalfont}        % нежирные номера страниц у глав в оглавлении
\renewcommand{\cftchapterleader}{\cftdotfill{\cftchapterdotsep}}% нежирные точки до номеров страниц у глав в оглавлении
%\renewcommand{\cftchapterfont}{}                       % нежирные названия глав в оглавлении

\ifnumgreater{\value{headingdelim}}{0}{%
    \renewcommand\cftchapteraftersnum{.\space}       % добавляет точку с пробелом после номера раздела в оглавлении
}{}
\ifnumgreater{\value{headingdelim}}{1}{%
    \renewcommand\cftsectionaftersnum{.\space}       % добавляет точку с пробелом после номера подраздела в оглавлении
    \renewcommand\cftsubsectionaftersnum{.\space}    % добавляет точку с пробелом после номера подподраздела в оглавлении
    \renewcommand\cftsubsubsectionaftersnum{.\space} % добавляет точку с пробелом после номера подподподраздела в оглавлении
    \AfterEndPreamble{% без этого polyglossia сама всё переопределяет
        \setsecnumformat{\csname the#1\endcsname.\space}
    }
}{%
    \AfterEndPreamble{% без этого polyglossia сама всё переопределяет
        \setsecnumformat{\csname the#1\endcsname\quad}
    }
}

\renewcommand*{\cftappendixname}{\appendixname\space} % Слово Приложение в оглавлении

%%% Колонтитулы %%%
% Порядковый номер страницы печатают на середине верхнего поля страницы (ГОСТ Р 7.0.11-2011, 5.3.8)
\makeevenhead{plain}{}{\rmfamily\thepage}{}
\makeoddhead{plain}{}{\rmfamily\thepage}{}
\makeevenfoot{plain}{}{}{}
\makeoddfoot{plain}{}{}{}
\pagestyle{plain}

%%% добавить Стр. над номерами страниц в оглавлении
%%% http://tex.stackexchange.com/a/306950
\newif\ifendTOC

\newcommand*{\tocheader}{
\ifnumequal{\value{pgnum}}{1}{%
    \ifendTOC\else\hbox to \linewidth%
      {\noindent{}~\hfill{Стр.}}\par%
      \ifnumless{\value{page}}{3}{}{%
        \vspace{0.5\onelineskip}
      }
      \afterpage{\tocheader}
    \fi%
}{}%
}%

%%% Оформление заголовков глав, разделов, подразделов %%%
%% Работа должна быть выполнена ... размером шрифта 12-14 пунктов (ГОСТ Р 7.0.11-2011, 5.3.8). То есть не должно быть надписей шрифтом более 14. Так и поставим.
%% Эти установки будут давать одинаковый результат независимо от выбора базовым шрифтом 12 пт или 14 пт
\newcommand{\basegostsectionfont}{\fontsize{16pt}{18pt}\selectfont\bfseries}

\makechapterstyle{thesisgost}{%
    \chapterstyle{default}
    \setlength{\beforechapskip}{0pt}
    \setlength{\midchapskip}{0pt}
    \setlength{\afterchapskip}{\theintvl\curtextsize}
    \renewcommand*{\chapnamefont}{\basegostsectionfont}
    \renewcommand*{\chapnumfont}{\basegostsectionfont}
    \renewcommand*{\chaptitlefont}{\basegostsectionfont}
    \renewcommand*{\chapterheadstart}{}
    \ifnumgreater{\value{headingdelim}}{0}{%
        \renewcommand*{\afterchapternum}{.\space}   % добавляет точку с пробелом после номера раздела
    }{%
        \renewcommand*{\afterchapternum}{\quad}     % добавляет \quad после номера раздела
    }
    \renewcommand*{\printchapternum}{\hdngaligni\hdngalign\chapnumfont \thechapter}
    \renewcommand*{\printchaptername}{}
    \renewcommand*{\printchapternonum}{\hdngaligni\hdngalign}
}

\makeatletter
\makechapterstyle{thesisgostchapname}{%
    \chapterstyle{thesisgost}
    \renewcommand*{\printchapternum}{\chapnumfont \thechapter}
    \renewcommand*{\printchaptername}{\hdngaligni\hdngalign\chapnamefont \@chapapp} %
}
\makeatother

\chapterstyle{thesisgost}

\setsecheadstyle{\basegostsectionfont\hdngalign}
\setsecindent{\otstuplen}

\setsubsecheadstyle{\basegostsectionfont\hdngalign}
\setsubsecindent{\otstuplen}

\setsubsubsecheadstyle{\basegostsectionfont\hdngalign}
\setsubsubsecindent{\otstuplen}

\sethangfrom{\noindent #1} %все заголовки подразделов центрируются с учетом номера, как block

\ifnumequal{\value{chapstyle}}{1}{%
    \chapterstyle{thesisgostchapname}
    \renewcommand*{\cftchaptername}{\chaptername\space} % будет вписано слово Глава перед каждым номером раздела в оглавлении
}{}%

%%% Интервалы между заголовками

% Заголовки отделяют от текста сверху и снизу тремя интервалами (ГОСТ Р 7.0.11-2011, 5.3.5).
%% \setbeforesecskip{\theintvl\curtextsize} 
%% \setaftersecskip{\theintvl\curtextsize}
%% \setbeforesubsecskip{\theintvl\curtextsize}
%% \setaftersubsecskip{\theintvl\curtextsize}
%% \setbeforesubsubsecskip{\theintvl\curtextsize}
%% \setaftersubsubsecskip{\theintvl\curtextsize}

% Вместо рекомендации ГОСТ используем одиночный интервал
\setbeforesecskip{\curtextsize} 
\setaftersecskip{\curtextsize}
\setbeforesubsecskip{\curtextsize}
\setaftersubsecskip{\curtextsize}
\setbeforesubsubsecskip{\curtextsize}
\setaftersubsubsecskip{\curtextsize}


%%% Вертикальные интервалы глав (\chapter) в оглавлении как и у заголовков
% раскомментировать следующие 2
% \setlength{\cftbeforechapterskip}{0pt plus 0pt}   % ИЛИ эти 2 строки из учебника
% \renewcommand*{\insertchapterspace}{}
% или эту
% \renewcommand*{\cftbeforechapterskip}{0em}


%%% Блок дополнительного управления размерами заголовков
\ifnumequal{\value{headingsize}}{1}{% Пропорциональные заголовки и базовый шрифт 14 пт
    \renewcommand{\basegostsectionfont}{\large\bfseries}
    \renewcommand*{\chapnamefont}{\Large\bfseries}
    \renewcommand*{\chapnumfont}{\Large\bfseries}
    \renewcommand*{\chaptitlefont}{\Large\bfseries}
}{}

%%% Счётчики %%%

%% Упрощённые настройки шаблона диссертации: нумерация формул, таблиц, рисунков
\ifnumequal{\value{contnumeq}}{1}{%
    \counterwithout{equation}{chapter} % Убираем связанность номера формулы с номером главы/раздела
}{}
\ifnumequal{\value{contnumfig}}{1}{%
    \counterwithout{figure}{chapter}   % Убираем связанность номера рисунка с номером главы/раздела
}{}
\ifnumequal{\value{contnumtab}}{1}{%
    \counterwithout{table}{chapter}    % Убираем связанность номера таблицы с номером главы/раздела
}{}

\AfterEndPreamble{
%% регистрируем счётчики в системе totcounter
    \regtotcounter{totalcount@figure}
    \regtotcounter{totalcount@table}       % Если иным способом поставить в преамбуле то ошибка в числе таблиц
    \regtotcounter{TotPages}               % Если иным способом поставить в преамбуле то ошибка в числе страниц
    \newtotcounter{totalappendix}
    \newtotcounter{totalchapter}
}
  % Стили для диссертации
% для вертикального центрирования ячеек в tabulary
\def\zz{\ifx\[$\else\aftergroup\zzz\fi}
%$ \] % <-- чиним подсветку синтаксиса в некоторых редакторах
\def\zzz{\setbox0\lastbox
\dimen0\dimexpr\extrarowheight + \ht0-\dp0\relax
\setbox0\hbox{\raise-.5\dimen0\box0}%
\ht0=\dimexpr\ht0+\extrarowheight\relax
\dp0=\dimexpr\dp0+\extrarowheight\relax
\box0
}

\lstdefinelanguage{Renhanced}%
{keywords={abbreviate,abline,abs,acos,acosh,action,add1,add,%
        aggregate,alias,Alias,alist,all,anova,any,aov,aperm,append,apply,%
        approx,approxfun,apropos,Arg,args,array,arrows,as,asin,asinh,%
        atan,atan2,atanh,attach,attr,attributes,autoload,autoloader,ave,%
        axis,backsolve,barplot,basename,besselI,besselJ,besselK,besselY,%
        beta,binomial,body,box,boxplot,break,browser,bug,builtins,bxp,by,%
        c,C,call,Call,case,cat,category,cbind,ceiling,character,char,%
        charmatch,check,chol,chol2inv,choose,chull,class,close,cm,codes,%
        coef,coefficients,co,col,colnames,colors,colours,commandArgs,%
        comment,complete,complex,conflicts,Conj,contents,contour,%
        contrasts,contr,control,helmert,contrib,convolve,cooks,coords,%
        distance,coplot,cor,cos,cosh,count,fields,cov,covratio,wt,CRAN,%
        create,crossprod,cummax,cummin,cumprod,cumsum,curve,cut,cycle,D,%
        data,dataentry,date,dbeta,dbinom,dcauchy,dchisq,de,debug,%
        debugger,Defunct,default,delay,delete,deltat,demo,de,density,%
        deparse,dependencies,Deprecated,deriv,description,detach,%
        dev2bitmap,dev,cur,deviance,off,prev,,dexp,df,dfbetas,dffits,%
        dgamma,dgeom,dget,dhyper,diag,diff,digamma,dim,dimnames,dir,%
        dirname,dlnorm,dlogis,dnbinom,dnchisq,dnorm,do,dotplot,double,%
        download,dpois,dput,drop,drop1,dsignrank,dt,dummy,dump,dunif,%
        duplicated,dweibull,dwilcox,dyn,edit,eff,effects,eigen,else,%
        emacs,end,environment,env,erase,eval,equal,evalq,example,exists,%
        exit,exp,expand,expression,External,extract,extractAIC,factor,%
        fail,family,fft,file,filled,find,fitted,fivenum,fix,floor,for,%
        For,formals,format,formatC,formula,Fortran,forwardsolve,frame,%
        frequency,ftable,ftable2table,function,gamma,Gamma,gammaCody,%
        gaussian,gc,gcinfo,gctorture,get,getenv,geterrmessage,getOption,%
        getwd,gl,glm,globalenv,gnome,GNOME,graphics,gray,grep,grey,grid,%
        gsub,hasTsp,hat,heat,help,hist,home,hsv,httpclient,I,identify,if,%
        ifelse,Im,image,\%in\%,index,influence,measures,inherits,install,%
        installed,integer,interaction,interactive,Internal,intersect,%
        inverse,invisible,IQR,is,jitter,kappa,kronecker,labels,lapply,%
        layout,lbeta,lchoose,lcm,legend,length,levels,lgamma,library,%
        licence,license,lines,list,lm,load,local,locator,log,log10,log1p,%
        log2,logical,loglin,lower,lowess,ls,lsfit,lsf,ls,machine,Machine,%
        mad,mahalanobis,make,link,margin,match,Math,matlines,mat,matplot,%
        matpoints,matrix,max,mean,median,memory,menu,merge,methods,min,%
        missing,Mod,mode,model,response,mosaicplot,mtext,mvfft,na,nan,%
        names,omit,nargs,nchar,ncol,NCOL,new,next,NextMethod,nextn,%
        nlevels,nlm,noquote,NotYetImplemented,NotYetUsed,nrow,NROW,null,%
        numeric,\%o\%,objects,offset,old,on,Ops,optim,optimise,optimize,%
        options,or,order,ordered,outer,package,packages,page,pairlist,%
        pairs,palette,panel,par,parent,parse,paste,path,pbeta,pbinom,%
        pcauchy,pchisq,pentagamma,persp,pexp,pf,pgamma,pgeom,phyper,pico,%
        pictex,piechart,Platform,plnorm,plogis,plot,pmatch,pmax,pmin,%
        pnbinom,pnchisq,pnorm,points,poisson,poly,polygon,polyroot,pos,%
        postscript,power,ppoints,ppois,predict,preplot,pretty,Primitive,%
        print,prmatrix,proc,prod,profile,proj,prompt,prop,provide,%
        psignrank,ps,pt,ptukey,punif,pweibull,pwilcox,q,qbeta,qbinom,%
        qcauchy,qchisq,qexp,qf,qgamma,qgeom,qhyper,qlnorm,qlogis,qnbinom,%
        qnchisq,qnorm,qpois,qqline,qqnorm,qqplot,qr,Q,qty,qy,qsignrank,%
        qt,qtukey,quantile,quasi,quit,qunif,quote,qweibull,qwilcox,%
        rainbow,range,rank,rbeta,rbind,rbinom,rcauchy,rchisq,Re,read,csv,%
        csv2,fwf,readline,socket,real,Recall,rect,reformulate,regexpr,%
        relevel,remove,rep,repeat,replace,replications,report,require,%
        resid,residuals,restart,return,rev,rexp,rf,rgamma,rgb,rgeom,R,%
        rhyper,rle,rlnorm,rlogis,rm,rnbinom,RNGkind,rnorm,round,row,%
        rownames,rowsum,rpois,rsignrank,rstandard,rstudent,rt,rug,runif,%
        rweibull,rwilcox,sample,sapply,save,scale,scan,scan,screen,sd,se,%
        search,searchpaths,segments,seq,sequence,setdiff,setequal,set,%
        setwd,show,sign,signif,sin,single,sinh,sink,solve,sort,source,%
        spline,splinefun,split,sqrt,stars,start,stat,stem,step,stop,%
        storage,strstrheight,stripplot,strsplit,structure,strwidth,sub,%
        subset,substitute,substr,substring,sum,summary,sunflowerplot,svd,%
        sweep,switch,symbol,symbols,symnum,sys,status,system,t,table,%
        tabulate,tan,tanh,tapply,tempfile,terms,terrain,tetragamma,text,%
        time,title,topo,trace,traceback,transform,tri,trigamma,trunc,try,%
        ts,tsp,typeof,unclass,undebug,undoc,union,unique,uniroot,unix,%
        unlink,unlist,unname,untrace,update,upper,url,UseMethod,var,%
        variable,vector,Version,vi,warning,warnings,weighted,weights,%
        which,while,window,write,\%x\%,x11,X11,xedit,xemacs,xinch,xor,%
        xpdrows,xy,xyinch,yinch,zapsmall,zip},%
    otherkeywords={!,!=,~,$,*,\%,\&,\%/\%,\%*\%,\%\%,<-,<<-},%$
    alsoother={._$},%$
    sensitive,%
    morecomment=[l]\#,%
    morestring=[d]",%
    morestring=[d]'% 2001 Robert Denham
}%

%решаем проблему с кириллицей в комментариях (в pdflatex) https://tex.stackexchange.com/a/103712
\lstset{extendedchars=true,keepspaces=true,literate={Ö}{{\"O}}1
    {Ä}{{\"A}}1
    {Ü}{{\"U}}1
    {ß}{{\ss}}1
    {ü}{{\"u}}1
    {ä}{{\"a}}1
    {ö}{{\"o}}1
    {~}{{\textasciitilde}}1
    {а}{{\selectfont\char224}}1
    {б}{{\selectfont\char225}}1
    {в}{{\selectfont\char226}}1
    {г}{{\selectfont\char227}}1
    {д}{{\selectfont\char228}}1
    {е}{{\selectfont\char229}}1
    {ё}{{\"e}}1
    {ж}{{\selectfont\char230}}1
    {з}{{\selectfont\char231}}1
    {и}{{\selectfont\char232}}1
    {й}{{\selectfont\char233}}1
    {к}{{\selectfont\char234}}1
    {л}{{\selectfont\char235}}1
    {м}{{\selectfont\char236}}1
    {н}{{\selectfont\char237}}1
    {о}{{\selectfont\char238}}1
    {п}{{\selectfont\char239}}1
    {р}{{\selectfont\char240}}1
    {с}{{\selectfont\char241}}1
    {т}{{\selectfont\char242}}1
    {у}{{\selectfont\char243}}1
    {ф}{{\selectfont\char244}}1
    {х}{{\selectfont\char245}}1
    {ц}{{\selectfont\char246}}1
    {ч}{{\selectfont\char247}}1
    {ш}{{\selectfont\char248}}1
    {щ}{{\selectfont\char249}}1
    {ъ}{{\selectfont\char250}}1
    {ы}{{\selectfont\char251}}1
    {ь}{{\selectfont\char252}}1
    {э}{{\selectfont\char253}}1
    {ю}{{\selectfont\char254}}1
    {я}{{\selectfont\char255}}1
    {А}{{\selectfont\char192}}1
    {Б}{{\selectfont\char193}}1
    {В}{{\selectfont\char194}}1
    {Г}{{\selectfont\char195}}1
    {Д}{{\selectfont\char196}}1
    {Е}{{\selectfont\char197}}1
    {Ё}{{\"E}}1
    {Ж}{{\selectfont\char198}}1
    {З}{{\selectfont\char199}}1
    {И}{{\selectfont\char200}}1
    {Й}{{\selectfont\char201}}1
    {К}{{\selectfont\char202}}1
    {Л}{{\selectfont\char203}}1
    {М}{{\selectfont\char204}}1
    {Н}{{\selectfont\char205}}1
    {О}{{\selectfont\char206}}1
    {П}{{\selectfont\char207}}1
    {Р}{{\selectfont\char208}}1
    {С}{{\selectfont\char209}}1
    {Т}{{\selectfont\char210}}1
    {У}{{\selectfont\char211}}1
    {Ф}{{\selectfont\char212}}1
    {Х}{{\selectfont\char213}}1
    {Ц}{{\selectfont\char214}}1
    {Ч}{{\selectfont\char215}}1
    {Ш}{{\selectfont\char216}}1
    {Щ}{{\selectfont\char217}}1
    {Ъ}{{\selectfont\char218}}1
    {Ы}{{\selectfont\char219}}1
    {Ь}{{\selectfont\char220}}1
    {Э}{{\selectfont\char221}}1
    {Ю}{{\selectfont\char222}}1
    {Я}{{\selectfont\char223}}1
    {і}{{\selectfont\char105}}1
    {ї}{{\selectfont\char168}}1
    {є}{{\selectfont\char185}}1
    {ґ}{{\selectfont\char160}}1
    {І}{{\selectfont\char73}}1
    {Ї}{{\selectfont\char136}}1
    {Є}{{\selectfont\char153}}1
    {Ґ}{{\selectfont\char128}}1
}

% Ширина текста минус ширина надписи 999
\newlength{\twless}
\newlength{\lmarg}
\setlength{\lmarg}{\widthof{999}}   % ширина надписи 999
\setlength{\twless}{\textwidth-\lmarg}

\lstset{ %
%    language=R,                     %  Язык указать здесь, если во всех листингах преимущественно один язык, в результате часть настроек может пойти только для этого языка
    numbers=left,                   % where to put the line-numbers
    numberstyle=\fontsize{12pt}{14pt}\selectfont\color{Gray},  % the style that is used for the line-numbers
    firstnumber=1,                  % в этой и следующей строках задаётся поведение нумерации 5, 10, 15...
    stepnumber=5,                   % the step between two line-numbers. If it's 1, each line will be numbered
    numbersep=5pt,                  % how far the line-numbers are from the code
    backgroundcolor=\color{white},  % choose the background color. You must add \usepackage{color}
    showspaces=false,               % show spaces adding particular underscores
    showstringspaces=false,         % underline spaces within strings
    showtabs=false,                 % show tabs within strings adding particular underscores
    frame=leftline,                 % adds a frame of different types around the code
    rulecolor=\color{black},        % if not set, the frame-color may be changed on line-breaks within not-black text (e.g. commens (green here))
    tabsize=2,                      % sets default tabsize to 2 spaces
    captionpos=t,                   % sets the caption-position to top
    breaklines=true,                % sets automatic line breaking
    breakatwhitespace=false,        % sets if automatic breaks should only happen at whitespace
%    title=\lstname,                 % show the filename of files included with \lstinputlisting;
    % also try caption instead of title
    basicstyle=\fontsize{12pt}{14pt}\selectfont\ttfamily,% the size of the fonts that are used for the code
%    keywordstyle=\color{blue},      % keyword style
    commentstyle=\color{ForestGreen}\emph,% comment style
    stringstyle=\color{Mahogany},   % string literal style
    escapeinside={\%*}{*)},         % if you want to add a comment within your code
    morekeywords={*,...},           % if you want to add more keywords to the set
    inputencoding=utf8,             % кодировка кода
    xleftmargin={\lmarg},           % Чтобы весь код и полоска с номерами строк была смещена влево, так чтобы цифры не вылезали за пределы текста слева
}

%http://tex.stackexchange.com/questions/26872/smaller-frame-with-listings
% Окружение, чтобы листинг был компактнее обведен рамкой, если она задается, а не на всю ширину текста
\makeatletter
\newenvironment{SmallListing}[1][]
{\lstset{#1}\VerbatimEnvironment\begin{VerbatimOut}{VerbEnv.tmp}}
{\end{VerbatimOut}\settowidth\@tempdima{%
        \lstinputlisting{VerbEnv.tmp}}
    \minipage{\@tempdima}\lstinputlisting{VerbEnv.tmp}\endminipage}
\makeatother

\DefineVerbatimEnvironment% с шрифтом 12 пт
{Verb}{Verbatim}
{fontsize=\fontsize{12pt}{14pt}\selectfont}

%% \newfloat[chapter]{ListingEnv}{lol}{Листинг}

\renewcommand{\lstlistingname}{Листинг}

%Общие счётчики окружений листингов
%http://tex.stackexchange.com/questions/145546/how-to-make-figure-and-listing-share-their-counter
% Если смешивать плавающие и не плавающие окружения, то могут быть проблемы с нумерацией
\makeatletter
\AfterEndPreamble{% https://tex.stackexchange.com/a/252682
    \let\c@ListingEnv\relax % drop existing counter "ListingEnv"
    \newaliascnt{ListingEnv}{lstlisting} % команда требует пакет aliascnt
    \let\ftype@lstlisting\ftype@ListingEnv % give the floats the same precedence
}
\makeatother

% значок С++ — используйте команду \cpp
\newcommand{\cpp}{%
    C\nolinebreak\hspace{-.05em}%
    \raisebox{.2ex}{+}\nolinebreak\hspace{-.10em}%
    \raisebox{.2ex}{+}%
}

%%%  Чересстрочное форматирование таблиц
%% http://tex.stackexchange.com/questions/278362/apply-italic-formatting-to-every-other-row
\newcounter{rowcnt}
\newcommand\altshape{\ifnumodd{\value{rowcnt}}{\color{red}}{\vspace*{-1ex}\itshape}}
% \AtBeginEnvironment{tabular}{\setcounter{rowcnt}{1}}
% \AtEndEnvironment{tabular}{\setcounter{rowcnt}{0}}

%%% Ради примера во второй главе
\let\originalepsilon\epsilon
\let\originalphi\phi
\let\originalkappa\kappa
\let\originalle\le
\let\originalleq\leq
\let\originalge\ge
\let\originalgeq\geq
\let\originalemptyset\emptyset
\let\originaltan\tan
\let\originalcot\cot
\let\originalcsc\csc

%%% Русская традиция начертания математических знаков
\renewcommand{\le}{\ensuremath{\leqslant}}
\renewcommand{\leq}{\ensuremath{\leqslant}}
\renewcommand{\ge}{\ensuremath{\geqslant}}
\renewcommand{\geq}{\ensuremath{\geqslant}}
\renewcommand{\emptyset}{\varnothing}

%%% Русская традиция начертания математических функций (на случай копирования из зарубежных источников)
\renewcommand{\tan}{\operatorname{tg}}
\renewcommand{\cot}{\operatorname{ctg}}
\renewcommand{\csc}{\operatorname{cosec}}

%%% Русская традиция начертания греческих букв (греческие буквы вертикальные, через пакет upgreek)
\renewcommand{\epsilon}{\ensuremath{\upvarepsilon}}   %  русская традиция записи
\renewcommand{\phi}{\ensuremath{\upvarphi}}
%\renewcommand{\kappa}{\ensuremath{\varkappa}}
\renewcommand{\alpha}{\upalpha}
\renewcommand{\beta}{\upbeta}
\renewcommand{\gamma}{\upgamma}
\renewcommand{\delta}{\updelta}
\renewcommand{\varepsilon}{\upvarepsilon}
\renewcommand{\zeta}{\upzeta}
\renewcommand{\eta}{\upeta}
\renewcommand{\theta}{\uptheta}
\renewcommand{\vartheta}{\upvartheta}
\renewcommand{\iota}{\upiota}
\renewcommand{\kappa}{\upkappa}
\renewcommand{\lambda}{\uplambda}
\renewcommand{\mu}{\upmu}
\renewcommand{\nu}{\upnu}
\renewcommand{\xi}{\upxi}
\renewcommand{\pi}{\uppi}
\renewcommand{\varpi}{\upvarpi}
\renewcommand{\rho}{\uprho}
%\renewcommand{\varrho}{\upvarrho}
\renewcommand{\sigma}{\upsigma}
%\renewcommand{\varsigma}{\upvarsigma}
\renewcommand{\tau}{\uptau}
\renewcommand{\upsilon}{\upupsilon}
\renewcommand{\varphi}{\upvarphi}
\renewcommand{\chi}{\upchi}
\renewcommand{\psi}{\uppsi}
\renewcommand{\omega}{\upomega}
 % Стили для специфических пользовательских задач

%%% Библиография. Выбор движка для реализации %%%
% Здесь только проверка установленного ключа. Сама настройка выбора движка
% размещена в common/setup.tex
\ifnumequal{\value{bibliosel}}{0}{%
    \input{biblio/predefined}   % Встроенная реализация с загрузкой файла через движок bibtex8
}{
    %%% Реализация библиографии пакетами biblatex и biblatex-gost с использованием движка biber %%%

\usepackage{csquotes} % biblatex рекомендует его подключать. Пакет для оформления сложных блоков цитирования.
%%% Загрузка пакета с основными настройками %%%
\makeatletter
\ifnumequal{\value{draft}}{0}{% Чистовик
\usepackage[%
backend=biber,% движок
bibencoding=utf8,% кодировка bib файла
sorting=none,% настройка сортировки списка литературы
style=gost-numeric,% стиль цитирования и библиографии (по ГОСТ)
language=autobib,% получение языка из babel/polyglossia, default: autobib % если ставить autocite или auto, то цитаты в тексте с указанием страницы, получат указание страницы на языке оригинала
autolang=other,% многоязычная библиография
clearlang=true,% внутренний сброс поля language, если он совпадает с языком из babel/polyglossia
defernumbers=true,% нумерация проставляется после двух компиляций, зато позволяет выцеплять библиографию по ключевым словам и нумеровать не из большего списка
sortcites=true,% сортировать номера затекстовых ссылок при цитировании (если в квадратных скобках несколько ссылок, то отображаться будут отсортированно, а не абы как)
doi=false,% Показывать или нет ссылки на DOI
isbn=false,% Показывать или нет ISBN, ISSN, ISRN
]{biblatex}[2016/09/17]
\ltx@iffilelater{biblatex-gost.def}{2017/05/03}%
{\toggletrue{bbx:gostbibliography}%
\renewcommand*{\revsdnamepunct}{\addcomma}}{}
}{%Черновик
\usepackage[%
backend=biber,% движок
bibencoding=utf8,% кодировка bib файла
sorting=none,% настройка сортировки списка литературы
% defernumbers=true, % откомментируйте, если требуется правильная нумерация ссылок на литературу в режиме черновика. Замедляет сборку
]{biblatex}[2016/09/17]%
}
\makeatother

\providebool{blxmc} % biblatex version needs and has MakeCapital workaround
\boolfalse{blxmc} % setting our new boolean flag to default false
\ifxetexorluatex
\else
% Исправление случая неподдержки знака номера в pdflatex
    \DefineBibliographyStrings{russian}{number={\textnumero}}

% Исправление случая отсутствия прописных букв в некоторых случаях
% https://github.com/plk/biblatex/issues/960#issuecomment-596658282
    \ifdefmacro{\ExplSyntaxOn}{}{\usepackage{expl3}}
    \makeatletter
    \ltx@ifpackagelater{biblatex}{2020/02/23}{
    % Assuming this version of biblatex defines MakeCapital correctly
    }{
        \ltx@ifpackagelater{biblatex}{2019/12/01}{
            % Assuming this version of biblatex defines MakeCapital incorrectly
            \usepackage{expl3}[2020/02/25]
            \@ifpackagelater{expl3}{2020/02/25}{
                \booltrue{blxmc} % setting our new boolean flag to true
            }{}
        }{}
    }
    \makeatother
    \ifblxmc
        \typeout{Assuming this version of biblatex defines MakeCapital
        incorrectly}
        \usepackage{xparse}
        \makeatletter
        \ExplSyntaxOn
        \NewDocumentCommand \blx@maketext@lowercase {m}
          {
            \text_lowercase:n {#1}
          }

        \NewDocumentCommand \blx@maketext@uppercase {m}
          {
            \text_uppercase:n {#1}
          }

        \RenewDocumentCommand \MakeCapital {m}
          {
            \text_titlecase_first:n {#1}
          }
        \ExplSyntaxOff

        \protected\def\blx@biblcstring#1#2#3{%
          \blx@begunit
          \blx@hyphenreset
          \blx@bibstringsimple
          \lowercase{\edef\blx@tempa{#3}}%
          \ifcsundef{#2@\blx@tempa}
            {\blx@warn@nostring\blx@tempa
             \blx@endnounit}
            {#1{\blx@maketext@lowercase{\csuse{#2@\blx@tempa}}}%
             \blx@endunit}}

        \protected\def\blx@bibucstring#1#2#3{%
          \blx@begunit
          \blx@hyphenreset
          \blx@bibstringsimple
          \lowercase{\edef\blx@tempa{#3}}%
          \ifcsundef{#2@\blx@tempa}
            {\blx@warn@nostring\blx@tempa
             \blx@endnounit}
            {#1{\blx@maketext@uppercase{\csuse{#2@\blx@tempa}}}%
             \blx@endunit}}
        \makeatother
    \fi
\fi

\ifsynopsis
\ifnumgreater{\value{usefootcite}}{0}{
    \ExecuteBibliographyOptions{autocite=footnote}
    \newbibmacro*{cite:full}{%
        \printtext[bibhypertarget]{%
            \usedriver{%
                \DeclareNameAlias{sortname}{default}%
            }{%
                \thefield{entrytype}%
            }%
        }%
        \usebibmacro{shorthandintro}%
    }
    \DeclareCiteCommand{\smartcite}[\mkbibfootnote]{%
        \usebibmacro{prenote}%
    }{%
        \usebibmacro{citeindex}%
        \usebibmacro{cite:full}%
    }{%
        \multicitedelim%
    }{%
        \usebibmacro{postnote}%
    }
}{}
\fi

%%% Подключение файлов bib %%%
\addbibresource[label=bl-external]{biblio/external.bib}
\addbibresource[label=bl-author]{biblio/author.bib}
\addbibresource[label=bl-registered]{biblio/registered.bib}

%http://tex.stackexchange.com/a/141831/79756
%There is a way to automatically map the language field to the langid field. The following lines in the preamble should be enough to do that.
%This command will copy the language field into the langid field and will then delete the contents of the language field. The language field will only be deleted if it was successfully copied into the langid field.
\DeclareSourcemap{ %модификация bib файла перед тем, как им займётся biblatex
    \maps{
        \map{% перекидываем значения полей language в поля langid, которыми пользуется biblatex
            \step[fieldsource=language, fieldset=langid, origfieldval, final]
            \step[fieldset=language, null]
        }
        \map{% перекидываем значения полей numpages в поля pagetotal, которыми пользуется biblatex
            \step[fieldsource=numpages, fieldset=pagetotal, origfieldval, final]
            \step[fieldset=numpages, null]
        }
        \map{% перекидываем значения полей pagestotal в поля pagetotal, которыми пользуется biblatex
            \step[fieldsource=pagestotal, fieldset=pagetotal, origfieldval, final]
            \step[fieldset=pagestotal, null]
        }
        \map[overwrite]{% перекидываем значения полей shortjournal, если они есть, в поля journal, которыми пользуется biblatex
            \step[fieldsource=shortjournal, final]
            \step[fieldset=journal, origfieldval]
            \step[fieldset=shortjournal, null]
        }
        \map[overwrite]{% перекидываем значения полей shortbooktitle, если они есть, в поля booktitle, которыми пользуется biblatex
            \step[fieldsource=shortbooktitle, final]
            \step[fieldset=booktitle, origfieldval]
            \step[fieldset=shortbooktitle, null]
        }
        \map{% если в поле medium написано "Электронный ресурс", то устанавливаем поле media, которым пользуется biblatex, в значение eresource.
            \step[fieldsource=medium,
            match=\regexp{Электронный\s+ресурс},
            final]
            \step[fieldset=media, fieldvalue=eresource]
            \step[fieldset=medium, null]
        }
        \map[overwrite]{% стираем значения всех полей issn
            \step[fieldset=issn, null]
        }
        \map[overwrite]{% стираем значения всех полей abstract, поскольку ими не пользуемся, а там бывают "неприятные" латеху символы
            \step[fieldsource=abstract]
            \step[fieldset=abstract,null]
        }
        \map[overwrite]{ % переделка формата записи даты
            \step[fieldsource=urldate,
            match=\regexp{([0-9]{2})\.([0-9]{2})\.([0-9]{4})},
            replace={$3-$2-$1$4}, % $4 вставлен исключительно ради нормальной работы программ подсветки синтаксиса, которые некорректно обрабатывают $ в таких конструкциях
            final]
        }
        \map[overwrite]{ % стираем ключевые слова
            \step[fieldsource=keywords]
            \step[fieldset=keywords,null]
        }
        % реализация foreach различается для biblatex v3.12 и v3.13.
        % Для версии v3.13 эта конструкция заменяет последующие 7 структур map
        % \map[overwrite,foreach={authorvak,authorscopus,authorwos,authorconf,authorother,authorparent,authorprogram}]{ % записываем информацию о типе публикации в ключевые слова
        %     \step[fieldsource=$MAPLOOP,final=true]
        %     \step[fieldset=keywords,fieldvalue={,biblio$MAPLOOP},append=true]
        % }
        \map[overwrite]{ % записываем информацию о типе публикации в ключевые слова
            \step[fieldsource=authorvak,final=true]
            \step[fieldset=keywords,fieldvalue={,biblioauthorvak},append=true]
        }
        \map[overwrite]{ % записываем информацию о типе публикации в ключевые слова
            \step[fieldsource=authorscopus,final=true]
            \step[fieldset=keywords,fieldvalue={,biblioauthorscopus},append=true]
        }
        \map[overwrite]{ % записываем информацию о типе публикации в ключевые слова
            \step[fieldsource=authorwos,final=true]
            \step[fieldset=keywords,fieldvalue={,biblioauthorwos},append=true]
        }
        \map[overwrite]{ % записываем информацию о типе публикации в ключевые слова
            \step[fieldsource=authorconf,final=true]
            \step[fieldset=keywords,fieldvalue={,biblioauthorconf},append=true]
        }
        \map[overwrite]{ % записываем информацию о типе публикации в ключевые слова
            \step[fieldsource=authorother,final=true]
            \step[fieldset=keywords,fieldvalue={,biblioauthorother},append=true]
        }
        \map[overwrite]{ % записываем информацию о типе публикации в ключевые слова
            \step[fieldsource=authorpatent,final=true]
            \step[fieldset=keywords,fieldvalue={,biblioauthorpatent},append=true]
        }
        \map[overwrite]{ % записываем информацию о типе публикации в ключевые слова
            \step[fieldsource=authorprogram,final=true]
            \step[fieldset=keywords,fieldvalue={,biblioauthorprogram},append=true]
        }
        \map[overwrite]{ % добавляем ключевые слова, чтобы различать источники
            \perdatasource{biblio/external.bib}
            \step[fieldset=keywords, fieldvalue={,biblioexternal},append=true]
        }
        \map[overwrite]{ % добавляем ключевые слова, чтобы различать источники
            \perdatasource{biblio/author.bib}
            \step[fieldset=keywords, fieldvalue={,biblioauthor},append=true]
        }
        \map[overwrite]{ % добавляем ключевые слова, чтобы различать источники
            \perdatasource{biblio/registered.bib}
            \step[fieldset=keywords, fieldvalue={,biblioregistered},append=true]
        }
        \map[overwrite]{ % добавляем ключевые слова, чтобы различать источники
            \step[fieldset=keywords, fieldvalue={,bibliofull},append=true]
        }
%        \map[overwrite]{% стираем значения всех полей series
%            \step[fieldset=series, null]
%        }
        \map[overwrite]{% перекидываем значения полей howpublished в поля organization для типа online
            \step[typesource=online, typetarget=online, final]
            \step[fieldsource=howpublished, fieldset=organization, origfieldval]
            \step[fieldset=howpublished, null]
        }
    }
}

\ifnumequal{\value{mediadisplay}}{1}{
    \DeclareSourcemap{
        \maps{%
            % \map{% использование media=text по умолчанию
            %     \step[fieldset=media, fieldvalue=text]
            % }
        }
    }
}{}
\ifnumequal{\value{mediadisplay}}{2}{
    \DeclareSourcemap{
        \maps{%
            \map[overwrite]{% удаление всех записей media
                \step[fieldset=media, null]
            }
        }
    }
}{}
\ifnumequal{\value{mediadisplay}}{3}{
    \DeclareSourcemap{
        \maps{
            \map[overwrite]{% стираем значения всех полей media=text
                \step[fieldsource=media,match={text},final]
                \step[fieldset=media, null]
            }
        }
    }
}{}
\ifnumequal{\value{mediadisplay}}{4}{
    \DeclareSourcemap{
        \maps{
            \map[overwrite]{% стираем значения всех полей media=eresource
                \step[fieldsource=media,match={eresource},final]
                \step[fieldset=media, null]
            }
        }
    }
}{}

\ifsynopsis
\else
\DeclareSourcemap{ %модификация bib файла перед тем, как им займётся biblatex
    \maps{
        \map[overwrite]{% стираем значения всех полей addendum
            \perdatasource{biblio/author.bib}
            \step[fieldset=addendum, null] %чтобы избавиться от информации об объёме авторских статей, в отличие от автореферата
        }
    }
}
\fi

\ifpresentation
% удаляем лишние поля в списке литературы презентации
% их названия можно узнать в файле presentation.bbl
\DeclareSourcemap{
    \maps{
    \map[overwrite,foreach={%
        % {{{ Список лишних полей в презентации
        address,%
        chapter,%
        edition,%
        editor,%
        eid,%
        howpublished,%
        institution,%
        key,%
        month,%
        note,%
        number,%
        organization,%
        pages,%
        publisher,%
        school,%
        series,%
        type,%
        media,%
        url,%
        doi,%
        location,%
        volume,%
        % Список лишних полей в презентации }}}
    }]{
        \perdatasource{biblio/author.bib}
        \step[fieldset=$MAPLOOP,null]
    }
    }
}
\fi

\defbibfilter{vakscopuswos}{%
    keyword=biblioauthorvak or keyword=biblioauthorscopus or keyword=biblioauthorwos
}

\defbibfilter{scopuswos}{%
    keyword=biblioauthorscopus or keyword=biblioauthorwos
}

\defbibfilter{papersregistered}{%
    keyword=biblioauthor or keyword=biblioregistered
}

%%% Убираем неразрывные пробелы перед двоеточием и точкой с запятой %%%
%\makeatletter
%\ifnumequal{\value{draft}}{0}{% Чистовик
%    \renewcommand*{\addcolondelim}{%
%      \begingroup%
%      \def\abx@colon{%
%        \ifdim\lastkern>\z@\unkern\fi%
%        \abx@puncthook{:}\space}%
%      \addcolon%
%      \endgroup}
%
%    \renewcommand*{\addsemicolondelim}{%
%      \begingroup%
%      \def\abx@semicolon{%
%        \ifdim\lastkern>\z@\unkern\fi%
%        \abx@puncthook{;}\space}%
%      \addsemicolon%
%      \endgroup}
%}{}
%\makeatother

%%% Правка записей типа thesis, чтобы дважды не писался автор
%\ifnumequal{\value{draft}}{0}{% Чистовик
%\DeclareBibliographyDriver{thesis}{%
%  \usebibmacro{bibindex}%
%  \usebibmacro{begentry}%
%  \usebibmacro{heading}%
%  \newunit
%  \usebibmacro{author}%
%  \setunit*{\labelnamepunct}%
%  \usebibmacro{thesistitle}%
%  \setunit{\respdelim}%
%  %\printnames[last-first:full]{author}%Вот эту строчку нужно убрать, чтобы автор диссертации не дублировался
%  \newunit\newblock
%  \printlist[semicolondelim]{specdata}%
%  \newunit
%  \usebibmacro{institution+location+date}%
%  \newunit\newblock
%  \usebibmacro{chapter+pages}%
%  \newunit
%  \printfield{pagetotal}%
%  \newunit\newblock
%  \usebibmacro{doi+eprint+url+note}%
%  \newunit\newblock
%  \usebibmacro{addendum+pubstate}%
%  \setunit{\bibpagerefpunct}\newblock
%  \usebibmacro{pageref}%
%  \newunit\newblock
%  \usebibmacro{related:init}%
%  \usebibmacro{related}%
%  \usebibmacro{finentry}}
%}{}

%\newbibmacro{string+doi}[1]{% новая макрокоманда на простановку ссылки на doi
%    \iffieldundef{doi}{#1}{\href{http://dx.doi.org/\thefield{doi}}{#1}}}

%\ifnumequal{\value{draft}}{0}{% Чистовик
%\renewcommand*{\mkgostheading}[1]{\usebibmacro{string+doi}{#1}} % ссылка на doi с авторов. стоящих впереди записи
%\renewcommand*{\mkgostheading}[1]{#1} % только лишь убираем курсив с авторов
%}{}
%\DeclareFieldFormat{title}{\usebibmacro{string+doi}{#1}} % ссылка на doi с названия работы
%\DeclareFieldFormat{journaltitle}{\usebibmacro{string+doi}{#1}} % ссылка на doi с названия журнала
%%% Тире как разделитель в библиографии традиционной руской длины:
\renewcommand*{\newblockpunct}{\addperiod\addnbspace\cyrdash\space\bibsentence}
%%% Убрать тире из разделителей элементов в библиографии:
%\renewcommand*{\newblockpunct}{%
%    \addperiod\space\bibsentence}%block punct.,\bibsentence is for vol,etc.
%%% Изменение точки с запятой на запятую в перечислении библиографических
%%% ссылок:
%\renewcommand*{\multicitedelim}{\addcomma\space}

%%% Возвращаем запись «Режим доступа» %%%
%\DefineBibliographyStrings{english}{%
%    urlfrom = {Mode of access}
%}
%\DeclareFieldFormat{url}{\bibstring{urlfrom}\addcolon\space\url{#1}}

%%% В списке литературы обозначение одной буквой диапазона страниц англоязычного источника %%%
\DefineBibliographyStrings{english}{%
    pages = {p\adddot} %заглавность буквы затем по месту определяется работой самого biblatex
}

%%% В ссылке на источник в основном тексте с указанием конкретной страницы обозначение одной большой буквой %%%
%\DefineBibliographyStrings{russian}{%
%    page = {C\adddot}
%}

%%% Исправление длины тире в диапазонах %%%
% \cyrdash --- тире «русской» длины, \textendash --- en-dash
\DefineBibliographyExtras{russian}{%
  \protected\def\bibrangedash{%
    \cyrdash\penalty\value{abbrvpenalty}}% almost unbreakable dash
  \protected\def\bibdaterangesep{\bibrangedash}%тире для дат
}
\DefineBibliographyExtras{english}{%
  \protected\def\bibrangedash{%
    \cyrdash\penalty\value{abbrvpenalty}}% almost unbreakable dash
  \protected\def\bibdaterangesep{\bibrangedash}%тире для дат
}

%Set higher penalty for breaking in number, dates and pages ranges
\setcounter{abbrvpenalty}{10000} % default is \hyphenpenalty which is 12

%Set higher penalty for breaking in names
\setcounter{highnamepenalty}{10000} % If you prefer the traditional BibTeX behavior (no linebreaks at highnamepenalty breakpoints), set it to ‘infinite’ (10 000 or higher).
\setcounter{lownamepenalty}{10000}

%%% Set low penalties for breaks at uppercase letters and lowercase letters
%\setcounter{biburllcpenalty}{500} %управляет разрывами ссылок после маленьких букв RTFM biburllcpenalty
%\setcounter{biburlucpenalty}{3000} %управляет разрывами ссылок после больших букв, RTFM biburlucpenalty

%%% Список литературы с красной строки (без висячего отступа) %%%
%\defbibenvironment{bibliography} % переопределяем окружение библиографии из gost-numeric.bbx пакета biblatex-gost
%  {\list
%     {\printtext[labelnumberwidth]{%
%       \printfield{prefixnumber}%
%       \printfield{labelnumber}}}
%     {%
%      \setlength{\labelwidth}{\labelnumberwidth}%
%      \setlength{\leftmargin}{0pt}% default is \labelwidth
%      \setlength{\labelsep}{\widthof{\ }}% Управляет длиной отступа после точки % default is \biblabelsep
%      \setlength{\itemsep}{\bibitemsep}% Управление дополнительным вертикальным разрывом между записями. \bibitemsep по умолчанию соответствует \itemsep списков в документе.
%      \setlength{\itemindent}{\bibhang}% Пользуемся тем, что \bibhang по умолчанию принимает значение \parindent (абзацного отступа), который переназначен в styles.tex
%      \addtolength{\itemindent}{\labelwidth}% Сдвигаем правее на величину номера с точкой
%      \addtolength{\itemindent}{\labelsep}% Сдвигаем ещё правее на отступ после точки
%      \setlength{\parsep}{\bibparsep}%
%     }%
%      \renewcommand*{\makelabel}[1]{\hss##1}%
%  }
%  {\endlist}
%  {\item}

%%% Макросы автоматического подсчёта количества авторских публикаций.
% Печатают невидимую (пустую) библиографию, считая количество источников.
% http://tex.stackexchange.com/a/66851/79756
%
\makeatletter
    \newtotcounter{citenum}
    \defbibenvironment{counter}
        {\setcounter{citenum}{0}\renewcommand{\blx@driver}[1]{}} % begin code: убирает весь выводимый текст
        {} % end code
        {\stepcounter{citenum}} % item code: cчитает "печатаемые в библиографию" источники

    \newtotcounter{citeauthorvak}
    \defbibenvironment{countauthorvak}
        {\setcounter{citeauthorvak}{0}\renewcommand{\blx@driver}[1]{}}
        {}
        {\stepcounter{citeauthorvak}}

    \newtotcounter{citeauthorscopus}
    \defbibenvironment{countauthorscopus}
        {\setcounter{citeauthorscopus}{0}\renewcommand{\blx@driver}[1]{}}
        {}
        {\stepcounter{citeauthorscopus}}

    \newtotcounter{citeauthorwos}
    \defbibenvironment{countauthorwos}
        {\setcounter{citeauthorwos}{0}\renewcommand{\blx@driver}[1]{}}
        {}
        {\stepcounter{citeauthorwos}}

    \newtotcounter{citeauthorother}
    \defbibenvironment{countauthorother}
        {\setcounter{citeauthorother}{0}\renewcommand{\blx@driver}[1]{}}
        {}
        {\stepcounter{citeauthorother}}

    \newtotcounter{citeauthorconf}
    \defbibenvironment{countauthorconf}
        {\setcounter{citeauthorconf}{0}\renewcommand{\blx@driver}[1]{}}
        {}
        {\stepcounter{citeauthorconf}}

    \newtotcounter{citeauthor}
    \defbibenvironment{countauthor}
        {\setcounter{citeauthor}{0}\renewcommand{\blx@driver}[1]{}}
        {}
        {\stepcounter{citeauthor}}

    \newtotcounter{citeauthorvakscopuswos}
    \defbibenvironment{countauthorvakscopuswos}
        {\setcounter{citeauthorvakscopuswos}{0}\renewcommand{\blx@driver}[1]{}}
        {}
        {\stepcounter{citeauthorvakscopuswos}}

    \newtotcounter{citeauthorscopuswos}
    \defbibenvironment{countauthorscopuswos}
        {\setcounter{citeauthorscopuswos}{0}\renewcommand{\blx@driver}[1]{}}
        {}
        {\stepcounter{citeauthorscopuswos}}

    \newtotcounter{citeregistered}
    \defbibenvironment{countregistered}
        {\setcounter{citeregistered}{0}\renewcommand{\blx@driver}[1]{}}
        {}
        {\stepcounter{citeregistered}}

    \newtotcounter{citeauthorpatent}
    \defbibenvironment{countauthorpatent}
        {\setcounter{citeauthorpatent}{0}\renewcommand{\blx@driver}[1]{}}
        {}
        {\stepcounter{citeauthorpatent}}

    \newtotcounter{citeauthorprogram}
    \defbibenvironment{countauthorprogram}
        {\setcounter{citeauthorprogram}{0}\renewcommand{\blx@driver}[1]{}}
        {}
        {\stepcounter{citeauthorprogram}}

    \newtotcounter{citeexternal}
    \defbibenvironment{countexternal}
        {\setcounter{citeexternal}{0}\renewcommand{\blx@driver}[1]{}}
        {}
        {\stepcounter{citeexternal}}
\makeatother

\defbibheading{nobibheading}{} % пустой заголовок, для подсчёта публикаций с помощью невидимой библиографии
\defbibheading{pubgroup}{\section*{#1}} % обычный стиль, заголовок-секция
\defbibheading{pubsubgroup}{\noindent\textbf{#1}} % для подразделов "по типу источника"

%%%Сортировка списка литературы Русский-Английский (предварительно удалить dissertation.bbl) (начало)
%%%Источник: https://github.com/odomanov/biblatex-gost/wiki/%D0%9A%D0%B0%D0%BA-%D1%81%D0%B4%D0%B5%D0%BB%D0%B0%D1%82%D1%8C,-%D1%87%D1%82%D0%BE%D0%B1%D1%8B-%D1%80%D1%83%D1%81%D1%81%D0%BA%D0%BE%D1%8F%D0%B7%D1%8B%D1%87%D0%BD%D1%8B%D0%B5-%D0%B8%D1%81%D1%82%D0%BE%D1%87%D0%BD%D0%B8%D0%BA%D0%B8-%D0%BF%D1%80%D0%B5%D0%B4%D1%88%D0%B5%D1%81%D1%82%D0%B2%D0%BE%D0%B2%D0%B0%D0%BB%D0%B8-%D0%BE%D1%81%D1%82%D0%B0%D0%BB%D1%8C%D0%BD%D1%8B%D0%BC
%\DeclareSourcemap{
%    \maps[datatype=bibtex]{
%        \map{
%            \step[fieldset=langid, fieldvalue={tempruorder}]
%        }
%        \map[overwrite]{
%            \step[fieldsource=langid, match=russian, final]
%            \step[fieldsource=presort,
%            match=\regexp{(.+)},
%            replace=\regexp{aa$1}]
%        }
%        \map{
%            \step[fieldsource=langid, match=russian, final]
%            \step[fieldset=presort, fieldvalue={az}]
%        }
%        \map[overwrite]{
%            \step[fieldsource=langid, notmatch=russian, final]
%            \step[fieldsource=presort,
%            match=\regexp{(.+)},
%            replace=\regexp{za$1}]
%        }
%        \map{
%            \step[fieldsource=langid, notmatch=russian, final]
%            \step[fieldset=presort, fieldvalue={zz}]
%        }
%        \map{
%            \step[fieldsource=langid, match={tempruorder}, final]
%            \step[fieldset=langid, null]
%        }
%    }
%}
%Сортировка списка литературы (конец)

%%% Создание команд для вывода списка литературы %%%
\newcommand*{\insertbibliofull}{
    \printbibliography[keyword=bibliofull,section=0,title=\bibtitlefull]
    \ifnumequal{\value{draft}}{0}{
      \printbibliography[heading=nobibheading,env=counter,keyword=bibliofull,section=0]
    }{}
}
\newcommand*{\insertbiblioauthor}{
    \printbibliography[heading=pubgroup, section=0, filter=papersregistered, title=\bibtitleauthor]
}
\newcommand*{\insertbiblioauthorimportant}{
    \printbibliography[heading=pubgroup, section=2, filter=papersregistered, title=\bibtitleauthorimportant]
}

% Вариант вывода печатных работ автора, с группировкой по типу источника.
% Порядок команд `\printbibliography` должен соответствовать порядку в файле common/characteristic.tex
\newcommand*{\insertbiblioauthorgrouped}{
    \section*{\bibtitleauthor}
    \ifsynopsis
    \printbibliography[heading=pubsubgroup, section=0, keyword=biblioauthorvak,    title=\bibtitleauthorvak,resetnumbers=true] % Работы автора из списка ВАК (сброс нумерации)
    \else
    \printbibliography[heading=pubsubgroup, section=0, keyword=biblioauthorvak,    title=\bibtitleauthorvak,resetnumbers=false] % Работы автора из списка ВАК (сквозная нумерация)
    \fi
    \printbibliography[heading=pubsubgroup, section=0, keyword=biblioauthorwos,    title=\bibtitleauthorwos,resetnumbers=false]% Работы автора, индексируемые Web of Science
    \printbibliography[heading=pubsubgroup, section=0, keyword=biblioauthorscopus, title=\bibtitleauthorscopus,resetnumbers=false]% Работы автора, индексируемые Scopus
    \printbibliography[heading=pubsubgroup, section=0, keyword=biblioauthorpatent, title=\bibtitleauthorpatent,resetnumbers=false]% Патенты
    \printbibliography[heading=pubsubgroup, section=0, keyword=biblioauthorprogram,title=\bibtitleauthorprogram,resetnumbers=false]% Программы для ЭВМ
    \printbibliography[heading=pubsubgroup, section=0, keyword=biblioauthorconf,   title=\bibtitleauthorconf,resetnumbers=false]% Тезисы конференций
    \printbibliography[heading=pubsubgroup, section=0, keyword=biblioauthorother,  title=\bibtitleauthorother,resetnumbers=false]% Прочие работы автора
}

\newcommand*{\insertbiblioexternal}{
    \printbibliography[heading=pubgroup,    section=0, keyword=biblioexternal,     title=\bibtitlefull]
}
     % Реализация пакетом biblatex через движок biber
}

% Вывести информацию о выбранных опциях в лог сборки
\typeout{Selected options:}
\typeout{Draft mode: \arabic{draft}}
\typeout{Font: \arabic{fontfamily}}
\typeout{AltFont: \arabic{usealtfont}}
\typeout{Bibliography backend: \arabic{bibliosel}}
\typeout{Precompile images: \arabic{imgprecompile}}
% Вывести информацию о версиях используемых библиотек в лог сборки
\listfiles

%%% Управление компиляцией отдельных частей диссертации %%%
% Необходимо сначала иметь полностью скомпилированный документ, чтобы все
% промежуточные файлы были в наличии
% Затем, для вывода отдельных частей можно воспользоваться командой \includeonly
% Ниже примеры использования команды:
%
%\includeonly{Dissertation/part2}
%\includeonly{Dissertation/contents,Dissertation/appendix,Dissertation/conclusion}
%
% Если все команды закомментированы, то документ будет выведен в PDF файл полностью

\begin{document}
%%% Переопределение именований типовых разделов
% https://tex.stackexchange.com/a/156050
\gappto\captionsrussian{%%% Переопределение именований %%%
\renewcommand{\contentsname}{Оглавление}% (ГОСТ Р 7.0.11-2011, 4)
\renewcommand{\figurename}{Рисунок}% (ГОСТ Р 7.0.11-2011, 5.3.9)
\renewcommand{\tablename}{Таблица}% (ГОСТ Р 7.0.11-2011, 5.3.10)
\renewcommand{\listfigurename}{Список рисунков}%
\renewcommand{\listtablename}{Список таблиц}%
\renewcommand{\bibname}{\bibtitlefull}%
% Переопределения названий для nomencl. Так как опция russian не для utf8
\renewcommand{\nomname}{Список сокращений и условных обозначений}%
\renewcommand{\eqdeclaration}[1]{, см.~(#1)}%
\renewcommand{\pagedeclaration}[1]{, стр.~#1}%
\renewcommand{\nomAname}{Латинские буквы}%
\renewcommand{\nomGname}{Греческие буквы}%
\renewcommand{\nomXname}{Верхние индексы}%
\renewcommand{\nomZname}{Индексы}%\unskip} % for polyglossia and babel
%%% Переопределение именований %%%
\renewcommand{\contentsname}{Оглавление}% (ГОСТ Р 7.0.11-2011, 4)
\renewcommand{\figurename}{Рисунок}% (ГОСТ Р 7.0.11-2011, 5.3.9)
\renewcommand{\tablename}{Таблица}% (ГОСТ Р 7.0.11-2011, 5.3.10)
\renewcommand{\listfigurename}{Список рисунков}%
\renewcommand{\listtablename}{Список таблиц}%
\renewcommand{\bibname}{\bibtitlefull}%
% Переопределения названий для nomencl. Так как опция russian не для utf8
\renewcommand{\nomname}{Список сокращений и условных обозначений}%
\renewcommand{\eqdeclaration}[1]{, см.~(#1)}%
\renewcommand{\pagedeclaration}[1]{, стр.~#1}%
\renewcommand{\nomAname}{Латинские буквы}%
\renewcommand{\nomGname}{Греческие буквы}%
\renewcommand{\nomXname}{Верхние индексы}%
\renewcommand{\nomZname}{Индексы}%

%%% Структура диссертации (ГОСТ Р 7.0.11-2011, 4)
% Титульный лист (ГОСТ Р 7.0.11-2001, 5.1)
\thispagestyle{empty}
\begin{center}
\thesisOrganization
\end{center}
%
\vspace{0pt plus4fill} %число перед fill = кратность относительно некоторого расстояния fill, кусками которого заполнены пустые места
\IfFileExists{images/logo.pdf}{
  \begin{minipage}[b]{0.5\linewidth}
    \begin{flushleft}
      %% \includegraphics[height=3.5cm]{logo}
    \end{flushleft}
  \end{minipage}%
  \begin{minipage}[b]{0.5\linewidth}
    \begin{flushright}
      На правах рукописи\\
%      \textsl {УДК \thesisUdk}
    \end{flushright}
  \end{minipage}
}{
\begin{flushright}
На правах рукописи

%\textsl {УДК \thesisUdk}
\end{flushright}
}
%
\vspace{0pt plus6fill} %число перед fill = кратность относительно некоторого расстояния fill, кусками которого заполнены пустые места
\begin{center}
{\large \thesisAuthor}
\end{center}
%
\vspace{0pt plus1fill} %число перед fill = кратность относительно некоторого расстояния fill, кусками которого заполнены пустые места
\begin{center}
\textbf {\large %\MakeUppercase
\thesisTitle}

\vspace{0pt plus2fill} %число перед fill = кратность относительно некоторого расстояния fill, кусками которого заполнены пустые места
{%\small
Специальность \thesisSpecialtyNumber\ "---

<<\thesisSpecialtyTitle>>
}

\ifdefined\thesisSpecialtyTwoNumber
{%\small
Специальность \thesisSpecialtyTwoNumber\ "---

<<\thesisSpecialtyTwoTitle>>
}
\fi

\vspace{0pt plus2fill} %число перед fill = кратность относительно некоторого расстояния fill, кусками которого заполнены пустые места
Диссертация на соискание учёной степени

\thesisDegree
\end{center}
%
\vspace{0pt plus4fill} %число перед fill = кратность относительно некоторого расстояния fill, кусками которого заполнены пустые места
\begin{flushright}
\ifdefined\supervisorTwoFio
Научные руководители:

\supervisorRegalia

\ifdefined\supervisorDead
\framebox{\supervisorFio}
\else
\supervisorFio
\fi

\supervisorTwoRegalia

\ifdefined\supervisorTwoDead
\framebox{\supervisorTwoFio}
\else
\supervisorTwoFio
\fi
\else
Научный руководитель:

\supervisorRegalia

\ifdefined\supervisorDead
\framebox{\supervisorFio}
\else
\supervisorFio
\fi
\fi

\end{flushright}
%
\vspace{0pt plus4fill} %число перед fill = кратность относительно некоторого расстояния fill, кусками которого заполнены пустые места
{\centering\thesisCity\ "--- \thesisYear\par}
           % Титульный лист
\include{Dissertation/contents}        % Оглавление
\ifnumequal{\value{contnumfig}}{1}{}{\counterwithout{figure}{chapter}}
\ifnumequal{\value{contnumtab}}{1}{}{\counterwithout{table}{chapter}}
\chapter*{Введение}                         % Заголовок
\addcontentsline{toc}{chapter}{Введение}    % Добавляем его в оглавление

\newcommand{\actuality}{\textbf{\actualityTXT}}
\newcommand{\progress}{\textbf{\progressTXT}}
\newcommand{\aim}{{\textbf\aimTXT}}
\newcommand{\tasks}{\textbf{\tasksTXT}}
\newcommand{\novelty}{\textbf{\noveltyTXT}}
\newcommand{\influence}{\textbf{\influenceTXT}}
\newcommand{\methods}{\textbf{\methodsTXT}}
\newcommand{\defpositions}{\textbf{\defpositionsTXT}}
\newcommand{\reliability}{\textbf{\reliabilityTXT}}
\newcommand{\probation}{\textbf{\probationTXT}}
\newcommand{\contribution}{\textbf{\contributionTXT}}
\newcommand{\publications}{\textbf{\publicationsTXT}}


% {\actuality} Обзор, введение в тему, обозначение места данной работы в
% мировых исследованиях и~т.\:п., можно использовать ссылки на~другие
% работы~\autocite{Gosele1999161,Lermontov}
% (если их~нет, то~в~автореферате
% автоматически пропадёт раздел <<Список литературы>>). Внимание! Ссылки
% на~другие работы в~разделе общей характеристики работы можно
% использовать только при использовании \verb!biblatex! (из-за технических
% ограничений \verb!bibtex8!. Это связано с тем, что одна
% и~та~же~характеристика используются и~в~тексте диссертации, и в
% автореферате. В~последнем, согласно ГОСТ, должен присутствовать список
% работ автора по~теме диссертации, а~\verb!bibtex8! не~умеет выводить в~одном
% файле два списка литературы).
% При использовании \verb!biblatex! возможно использование исключительно
% в~автореферате подстрочных ссылок
% для других работ командой \verb!\autocite!, а~также цитирование
% собственных работ командой \verb!\cite!. Для этого в~файле
% \verb!common/setup.tex! необходимо присвоить положительное значение
% счётчику \verb!\setcounter{usefootcite}{1}!.

% Для генерации содержимого титульного листа автореферата, диссертации
% и~презентации используются данные из файла \verb!common/data.tex!. Если,
% например, вы меняете название диссертации, то оно автоматически
% появится в~итоговых файлах после очередного запуска \LaTeX. Согласно
% ГОСТ 7.0.11-2011 <<5.1.1 Титульный лист является первой страницей
% диссертации, служит источником информации, необходимой для обработки и
% поиска документа>>. Наличие логотипа организации на~титульном листе
% упрощает обработку и~поиск, для этого разметите логотип вашей
% организации в папке images в~формате PDF (лучше найти его в векторном
% варианте, чтобы он хорошо смотрелся при печати) под именем
% \verb!logo.pdf!. Настроить размер изображения с логотипом можно
% в~соответствующих местах файлов \verb!title.tex!  отдельно для
% диссертации и автореферата. Если вам логотип не~нужен, то просто
% удалите файл с~логотипом.

% \ifsynopsis
% Этот абзац появляется только в~автореферате.
% Для формирования блоков, которые будут обрабатываться только в~автореферате,
% заведена проверка условия \verb!\!\verb!ifsynopsis!.
% Значение условия задаётся в~основном файле документа (\verb!synopsis.tex! для
% автореферата).
% \else
% Этот абзац появляется только в~диссертации.
% Через проверку условия \verb!\!\verb!ifsynopsis!, задаваемого в~основном файле
% документа (\verb!dissertation.tex! для диссертации), можно сделать новую
% команду, обеспечивающую появление цитаты в~диссертации, но~не~в~автореферате.
% \fi

% {\progress}
% Этот раздел должен быть отдельным структурным элементом по
% ГОСТ, но он, как правило, включается в описание актуальности
% темы. Нужен он отдельным структурынм элемементом или нет ---
% смотрите другие диссертации вашего совета, скорее всего не нужен.

{\actuality} 
В современном мире многопоточные программные системы распространены повсеместно. 
Разработка и тестирование программного обеспечения для таких систем на порядок 
сложнее и существенно более трудозатратно, чем для последовательных систем. 
По этой причине крайне актуальной является задача 
верификации многопоточных программ.
Решение этой задачи в свою очередь требует наличия 
строгой математической спецификации семантики многопоточных программ.

Формальная семантика многопоточных программ, потоки которых работают 
с разделяемой памятью, называется \emph{моделью памяти}. 
Главной целью модели памяти является задание множества 
допустимых \emph{сценариев исполнения} программы.
Современные многопоточные системы и языки программирования 
вследствие применения компиляторами и процессорами
различных оптимизаций при сборке и выполнении программ
допускают так называемые \emph{слабые сценарии исполнения},
то есть такие сценарии, которые не могут быть получены в результате 
простого поочередного исполнения инструкций различными потоками. 
\emph{Слабые модели памяти} призваны описать множество 
допустимых слабых сценариев исполнения программы. 
Оказывается, что вопрос какие именно слабые сценарии поведения 
следует допускать, а какие нет, не является однозначным 
и зависит от требований к самой многопоточной системе 
или языку программирования. 
По этой причине в последние годы появилось (и продолжает появляться) 
множество различных моделей памяти для современных мультипроцессоров, например, \Intel~\autocite{Sewell-al:CACM10}, 
\ARM~\autocite{Pulte-al:POPL18}, 
\POWER~\autocite{Sarkar-al:PLDI11}
языков программирования, например, 
\CPP~\autocite{Batty-al:POPL11},
\Java~\autocite{Manson-al:POPL05}, 
\JS~\autocite{Watt-al:PLDI2020}, 
\OCaml~\autocite{Dolan-al:PLDI18},
а также распределенных систем%
~\autocite{Jagadeesan-al:ESOP2018,Lahav-Boker:PLDI2020}.
В связи с этим встает задача формализации 
существующих моделей памяти и создание теории
для разработки будущих моделей памяти.


Одним из способов задания моделей памяти
является использование семантических доменов
\emph{истинной конкурентности} (\emph{true concurrency semantics}).
Этот класс моделей позволяет выразить независимость (параллельность) атомарных событий, а также 
причинно-следственные связи между ними,
что ведет к более компактному представлению пространства состояний программы.
Всё это упрощает рассуждения 
о поведении многопоточных программ как для человека, 
так и для программных средств при автоматической и интерактивной верификации. 

\emph{Структуры событий} (\emph{event structures}) являются одним из семантических доменов, 
относящихся к классу моделей истинной конкурентности.
В наиболее простом варианте структура событий состоит из множества атомарных событий,
функции, присваивающей каждому событию семантическую метку,
отношения причинно-следственной связи и отношения конфликта между событиями.
Классическая теория структур событий была разработана M.Nielsen, G.Plotkin и G.Winskel
для описания семантики исчисления взаимодействующих систем (Calculus of Communicating Systems, CCS).
Следует отметить, что данное исчисление является достаточно простой моделью параллельных вычислений
и не позволяет описывать слабые сценарии поведения многопоточных программ.

Для описания слабых моделей памяти исследователями было предложено несколько формализмов,
основанных на структурах событий, в частности,
модель A.Jeffrey и J.Riely~\autocite{Jeffrey-Riely:LICS16},
модель J.Pichon-Pharabod и P. Sewell~\autocite{PichonPharabod-Sewell:POPL16},
модель \Wkm~\autocite{Chakraborty-Vafeiadis:POPL19},
модель \MRD~\autocite{Paviotti-al:ESOP20}.
Отметим, что данные модели вводят новые классы
структур событий, несовместимые с классической теорией,
что не позволяет применять известные результаты
о структурах событий к данным моделям.
Кроме того, в рамках данных моделей даже для небольших программ
вычислительно затратно перечисление возможных слабых сценариев поведения,
что препятствует разработке эффективных средств верификации многопоточных программ.

Таким образом, возникает потребность в создании формальной семантики 
многопоточных программ на основе структур событий, 
которая, с одной стороны, позволяла бы описывать слабые сценарии поведения, 
с другой стороны, допускала бы разработку  
инструментов для автоматической и интерактивной верификации. 

{\progress}

Теория структур событий была разработана M.Nielsen, G.Plotkin и G.Winskel в 1980-1990 годы%
~\autocite{Nielsen:REX93,Sassone:MFCS1993,Vaandrager:TCS1991,Winskel-TCS:09} 
для задания денотационной семантики 
исчисления параллельных взаимодействующих систем (Calculus of Communicating Systems, CCS)%
~\autocite{Winskel:ICALP1982}.
Относительно недавно эта теория также была использована 
для задания семантики пи-исчисления процессов ($\pi$-calculus)%
~\autocite{Varacca-Nobuko:TCS10,Crafa-al:FSCCS12,Hildebrandt-al:LATA2017}.
Но CCS и пи-исчисление не позволяют описывать 
слабые сценарии поведения многопоточных программ.

Теория слабых моделей памяти также активно развивалась, начиная с 1990-ых годов. 
На сегодняшний день существует множество моделей памяти, 
описывающих поведение мультипроцессоров, 
многопоточных языков программирования и распределенных систем. 
Эти модели, в свою очередь, можно разделить на несколько классов.

Модели, \emph{сохраняющие программный порядок}, образуют широкий класс,
включающий, в том числе, модель \TSO процессоров семейства \Intel~\autocite{Sewell-al:CACM10},
модели последовательной согласованности (sequential consistency)~\autocite{Lamport:TC79}
причинной согласованности (causal consistency)~\autocite{Lahav-Boker:PLDI2020},
и согласованности в конечном счёте (eventual consistency)~\autocite{Jagadeesan-al:ESOP2018},
а также модели памяти некоторых языков программирования, например,
модель памяти языка \OCaml~\autocite{Dolan-al:PLDI18}.
Общим недостатком моделей данного класса является то,
что в рамках этих моделей не поддерживаются \emph{оптимальные схемы компиляции} 
в целевой код для современных мультипроцессоров \ARM и \POWER.
Это означает, что реализация данных моделей на этих мультипроцессорах
влечет дополнительные накладные расходы и может приводить
к увелечению времени исполнения программ~\autocite{Ou-Demsky:OOPSLA18}. 

Модели памяти мультипроцессоров,
например \ARM~\autocite{Pulte-al:POPL18} и \POWER~\autocite{Sarkar-al:PLDI11}, 
как правило, принадлежат к классу моделей, \emph{сохраняющих синтаксические зависимости}. 
Основное ограничение моделей, принадлежащих к данному классу, заключается в том, 
что они не поддерживают некоторые трансформации программ, 
применяемые оптимизирующими компиляторами, например, свертку констант.
По этой причине модели данного класса, как правило,
не применяются в качестве моделей памяти для языков программирования.  

Таким образом, модели памяти двух вышеупомянутых классов 
не отвечают требованиям, предъявляемым к моделям памяти для таких языков как \CPP и \Java. 
С целью преодоления этих ограничений исследователями были предложены модели  
\Prm~\autocite{Kang-al:POPL17}, \Wkm~\autocite{Chakraborty-Vafeiadis:POPL19}, 
\MRD~\autocite{Paviotti-al:ESOP20}, \PwP~\autocite{Jagadeesan-al:OOPSLA2020}
и другие~\autocite{Jeffrey-Riely:LICS16,PichonPharabod-Sewell:POPL16},
которые обычно относят к классу моделей, \emph{сохраняющих семантические зависимости}.
Данные модели, как правило, поддерживают оптимальные схемы компиляции
для современных мультипроцессоров и поддерживают широкий спектр оптимизаций программ. 
Тем не менее, классы моделей, сохраняющих программный порядок и синтаксические зависимости, 
хорошо изучены, в то время как свойства класса моделей,
сохраняющих семантические зависимости, по-прежнему активно исследуются.
В частности, для моделей данного класса практический не исследованы
вопросы построения эффективных инструментов автоматической и интерактивной верификации. 

%% Некоторые из вышеупомянутых моделей, сохраняющих семантические зависимости,
%% основаны на теории структур событий%
%% ~\autocite{Jeffrey-Riely:LICS16,PichonPharabod-Sewell:POPL16,
%% Chakraborty-Vafeiadis:POPL19,Paviotti-al:ESOP20}.
%% Общий недостаток данных моделей заключается в том,
%% что они вводят новые классы структур событий, 
%% несовместимые с классическими определениями.
%% Это затрудняет применение уже существующей классической теории структур событий
%% для решения проблем, возникающих в теории слабых моделей памяти. 

Рассмотрим для примера язык \CPP.
Для описания модели памяти данного языка исследовательским
сообществом было выработано несколько подходов.
Модель \RCMM~\autocite{Lahav-al:PLDI17}
относится к классу моделей, сохраняющих программный порядок.
Данная модель является относительно простой и
предоставляет ряд важных и полезных свойств.
Для данной модели также были разработаны эффективные
средства верификации многопоточных программ,
например, инструмент проверки моделей \genmc~\autocite{Kokologiannakis:PLDI2019}.
Однако данная модель не поддерживает
оптимальную схему компиляции в модели мультипроцессоров \ARM и \POWER.
С другой стороны, модели \Prm и \Wkm,
относящиеся к классу моделей, сохраняющих семантические зависимости,
поддерживают оптимальную схему компиляции и широкий набор оптимизаций программ.
Но данные модели существенно более сложные, их свойства слабо изучены, и для них
не разработаны эффективные методы верификации программ. 

В контексте данной работы наибольший интерес
представляет именно модель \Wkm~\autocite{Chakraborty-Vafeiadis:POPL19},
так как она основана на теории структур событий
и для нее было формально доказано наличие ряда важных для практики свойств.
Отметим, что у данной модели тем не менее есть ряд недостаков:
данная модель не укладывается в классическую теорию структур событий,
для нее не была доказана корректность оптимальной схемы
компиляции в модели современных мультипроцессоров,
а также для данной модели не были ранее разработаны
средства верификации программ.

%% Для большей гибкости, данный язык предоставляет несколько
%% режимов доступа к разделяемым переменным (\emph{access modes}):
%% \emph{последовательно согласованный режим} (\emph{sequentiall consistent}),
%% режимы \emph{захвата и освобождения} (\emph{acquire/release}),
%% гарантирующие причинную согласованность~\autocite{Lahav-al:POPL16},
%% \emph{ослабленный режим} (\emph{relaxed}),
%% гарантирующий когерентность~\autocite{Alglave-al:TOPLAS14}
%% и \emph{неатомарный режим} для неконкурентных обращений к памяти. 

%% Для описания подмножества модели памяти \CPP
%% исследователями была разработана модель \RCMM~\autocite{Lahav-al:PLDI17}.
%% Данная модель полностью описывает все возможные сценарии
%% поведения многопоточных программ, которые
%% не используют режим ослабленных обращений.  

%% Среди слабых моделей памяти, сохраняющих семантические зависимости
%% и основанных на структурах событий, в контексте данной работы наибольший интерес
%% представляет модель \Wkm~\autocite{Chakraborty-Vafeiadis:POPL19},
%% поскольку для данной модели было формально доказано наличие ряда важных для практики свойств.
%% В частности, для этой модели была доказана корректность
%% локальных трансформаций программ и теорема о свободе от гонок.
%% Тем не менее отметим, что корректность оптимальной схемы
%% компиляции из модели \Wkm в модели современных мультипроцессоров
%% \emph{не была ранее доказана}, что является существенным недостатком,
%% так как наличие данного свойства является одним из базовых требований,
%% предъявляемых к классу моделей памяти, сохраняющих семантические зависимости.

{\aim} данной работы является адаптация теории структур событий
для описания слабых моделей памяти и разработка на основе этих исследований 
инструментов для автоматической и интерактивной верификации многопоточных программ. 

Для достижения данной цели были сформулированы следующие {\tasks}.
\begin{enumerate}[beginpenalty=10000] % https://tex.stackexchange.com/a/476052/104425
  \item
    Формализовать в системе для интерактивного доказательства теорем \coq
    классическую теорию структур событий. Показать, что
    в данную теорию укладывается класс моделей памяти,
    сохраняющих программный порядок и, в частности, модель \RCMM.
  \item
    Формализовать в системе для интерактивного доказательства теорем \coq
    теорию структур событий модели \Wkm.
    Доказать корректность оптимальной схемы компиляции
    из модели \Wkm в модели памяти современных мультипроцессоров.
  \item
    Разработать строгую версию модели \Wkm, 
    допускающую реализацию эффективных инструментов автоматической верификации
    и доказать, что для неё сохраняются основные свойства \Wkm  
    (в частности, корректность компиляции, корректность локальных трансформаций программ, 
     теорема о свободе от гонок).
  \item
    Разработать алгоритм проверки моделей (model~checking) для предложенной модели.
\end{enumerate}

~\newline

{\methods}

Диссертационное исследование базируется на теории формальных семантик. 
Используются классические и хорошо изученные формализмы, в частности, 
системы помеченных переходов, языки помеченных частично упорядоченных мультимножеств и структуры событий. 

Для формализации некоторых теорем и доказательств, представленных в данной работе, 
использовалась система интерактивного доказательства теорем \coq 
и библиотека формализованных математических теорий \mathcomp.

%% При разработке алгоритма проверки моделей использовались техника \emph{редукции частичных порядков}.
%% Предложенный алгоритм был внедрен в систему \genmc --- 
%% инструмент для автоматической верификации многопоточных программ написанных на языке \CLANG.

{\defpositions}
\begin{enumerate}[beginpenalty=10000] % https://tex.stackexchange.com/a/476052/104425
  \item Предложена формальная семантика на основе классической теории структур событий, 
    покрывающая класс слабых моделей памяти, сохраняющих программный порядок;
    данная семантика формализована в системе \coq.
  \item Доказана корректность оптимальной схемы компиляции из модели \Wkm
    в модели современных мультипроцессоров \TSO, \ARM и \POWER;
    модель \Wkm и доказательство теоремы о корректности компиляции
    формализованы в системе \coq.
  \item Предложена модель \WkmS, расширяющая модель \Wkm 
  новыми свойствами \emph{свободы от буферизации
  операций чтения} и \emph{локальности сертификациии}, 
  которые позволяют проводить эффективную верификацию программ в данной модели;
  также доказано сохранение основных свойств модели \Wkm: корректности компиляции,
  свойства корректности локальных трансформаций программ,
  теоремы о свободе от гонок.
  \item Для модели \WkmS разработан алгоритм автоматической 
    верификации программ методом проверки моделей; 
    в ряде экспериментов показана лучшая эффективность 
    данного алгоритма по сравнению с аналогами.
\end{enumerate}
% В папке Documents можно ознакомиться с решением совета из Томского~ГУ
% (в~файле \verb+Def_positions.pdf+), где обоснованно даются рекомендации
% по~формулировкам защищаемых положений.

{\novelty}
\begin{enumerate}[beginpenalty=10000] % https://tex.stackexchange.com/a/476052/104425

  \item Впервые предложена семантика, основанная на классической теории структур событий,
    которая покрывает класс слабых моделей памяти, сохраняющих программный порядок.
    %% что позволяет применить известные теоретические результаты 
    %% о структурах событий к данному классу моделей.

  \item Впервые доказана корректность оптимальной схемы компиляции
    из модели памяти, основанной на структурах событий (\Wkm), 
    в модели памяти современных мультипроцессоров.

  \item Впервые предложена модель памяти (\WkmS),
    принадлежащая к классу моделей, сохраняющих семантические зависимости, 
    и при этом допускающая реализацию эффективных методов автоматической верификации программ. 

  \item Разработан новый алгоритм проверки моделей для \WkmS,
    который является существенно более эффективным по сравнению с другими алгоритмами
    (\CDSChecker~\autocite{Norris-Demsky:OOPSLA2013}, \rmem~\autocite{Pulte-al:PLDI2019}),
    которые поддерживают класс моделей памяти, сохраняющих семантические зависимости.

\end{enumerate}

{\influence} 

Новая семантика на основе теории структур событий 
для класса слабых моделей памяти, сохраняющих программный порядок,
соединяет классическую теорию структур событий.
%% с теорией слабых моделей памяти и позволяет применить известные результаты 
%% о структурах событий в новой предметной области.  
Формализация этой семантики в системе \coq открывает 
путь к дальнейшей разработке инструментов для  
интерактивной верификации многоточных программ  
с учетом слабых сценариев исполнения. 
 
Новые свойства предложенной модели \WkmS ---
свобода от буферизации операций чтения (load buffering race freedom)
и локальности сертификации (certification locality), --- 
также могут быть добавлены в другие модели памяти 
с целью разработки методов автоматической верификации программ в этих моделях. 
Наличие данных свойств позволяет оптимизировать алгоритм 
проверки моделей и таким образом существенно увеличить его эффективность.

Предложенный  алгоритм проверки моделей может быть использован на практике
для отладки и верификации многопоточных алгоритмов и структур данных 
с учетом слабых сценариев исполнения, допустимых стандартом языка \CLANG. 

{\reliability} полученных результатов обеспечивается 
формальными доказательствами, разработанными в том числе с использованием
систем интерактивного доказательства теорем, 
а также инженерными экспериментами. 
Результаты находятся в соответствии с результатами, полученными другими авторами.

{\probation}
Основные результаты работы докладывались~на
следующих научных конференциях и семинарах:
Surrey Concurrency Workshop (23-24 июля 2019, Университет Суррея, Великобритания),
The European Conference on Object-Oriented Programming
(ECOOP, 15-17 ноября 2020, Берлин, Германия, онлайн конференция),
Spring/Summer Young Researchers' Colloquium on Software Engineering
(27-28 мая 2021, Москва, Россия),
внутренние семинары JetBrains Research
(18 ноября 2018, 13 апреля 2020, Санкт-Петербург, Россия). \\
\fixme{добавить будущие мероприятия по мере проведения}.

% {\contribution} Автор принимал активное участие \ldots

\ifnumequal{\value{bibliosel}}{0}
{%%% Встроенная реализация с загрузкой файла через движок bibtex8. (При желании, внутри можно использовать обычные ссылки, наподобие `\cite{vakbib1,vakbib2}`).
    {\publications} Основные результаты по теме диссертации изложены
    в~XX~печатных изданиях,
    X из которых изданы в журналах, рекомендованных ВАК,
    X "--- в тезисах докладов.
}%
{%%% Реализация пакетом biblatex через движок biber
    \begin{refsection}[bl-author, bl-registered]
        % Это refsection=1.
        % Процитированные здесь работы:
        %  * подсчитываются, для автоматического составления фразы "Основные результаты ..."
        %  * попадают в авторскую библиографию, при usefootcite==0 и стиле `\insertbiblioauthor` или `\insertbiblioauthorgrouped`
        %  * нумеруются там в зависимости от порядка команд `\printbibliography` в этом разделе.
        %  * при использовании `\insertbiblioauthorgrouped`, порядок команд `\printbibliography` в нём должен быть тем же (см. biblio/biblatex.tex)
        %
        % Невидимый библиографический список для подсчёта количества публикаций:
        \printbibliography[heading=nobibheading, section=1, env=countauthorvak,          keyword=biblioauthorvak]%
        \printbibliography[heading=nobibheading, section=1, env=countauthorwos,          keyword=biblioauthorwos]%
        \printbibliography[heading=nobibheading, section=1, env=countauthorscopus,       keyword=biblioauthorscopus]%
        \printbibliography[heading=nobibheading, section=1, env=countauthorconf,         keyword=biblioauthorconf]%
        \printbibliography[heading=nobibheading, section=1, env=countauthorother,        keyword=biblioauthorother]%
        \printbibliography[heading=nobibheading, section=1, env=countregistered,         keyword=biblioregistered]%
        \printbibliography[heading=nobibheading, section=1, env=countauthorpatent,       keyword=biblioauthorpatent]%
        \printbibliography[heading=nobibheading, section=1, env=countauthorprogram,      keyword=biblioauthorprogram]%
        \printbibliography[heading=nobibheading, section=1, env=countauthor,             keyword=biblioauthor]%
        \printbibliography[heading=nobibheading, section=1, env=countauthorvakscopuswos, filter=vakscopuswos]%
        \printbibliography[heading=nobibheading, section=1, env=countauthorscopuswos,    filter=scopuswos]%
        %
        \nocite{*}%
        %
        {\publications} Основные результаты по теме диссертации изложены в~\arabic{citeauthor}~печатных изданиях,
        \arabic{citeauthorvak} из которых изданы в журналах, рекомендованных ВАК\sloppy%
        \ifnum \value{citeauthorscopuswos}>0%
            , \arabic{citeauthorscopuswos} "--- в~периодических научных журналах, индексируемых Web of~Science и Scopus\sloppy%
        \fi%
        \ifnum \value{citeauthorconf}>0%
            , \arabic{citeauthorconf} "--- в~тезисах докладов.
        \else%
            .
        \fi%
        \ifnum \value{citeregistered}=1%
            \ifnum \value{citeauthorpatent}=1%
                Зарегистрирован \arabic{citeauthorpatent} патент.
            \fi%
            \ifnum \value{citeauthorprogram}=1%
                Зарегистрирована \arabic{citeauthorprogram} программа для ЭВМ.
            \fi%
        \fi%
        \ifnum \value{citeregistered}>1%
            Зарегистрированы\ %
            \ifnum \value{citeauthorpatent}>0%
            \formbytotal{citeauthorpatent}{патент}{}{а}{}\sloppy%
            \ifnum \value{citeauthorprogram}=0 . \else \ и~\fi%
            \fi%
            \ifnum \value{citeauthorprogram}>0%
            \formbytotal{citeauthorprogram}{программ}{а}{ы}{} для ЭВМ.
            \fi%
        \fi%
        % К публикациям, в которых излагаются основные научные результаты диссертации на соискание учёной
        % степени, в рецензируемых изданиях приравниваются патенты на изобретения, патенты (свидетельства) на
        % полезную модель, патенты на промышленный образец, патенты на селекционные достижения, свидетельства
        % на программу для электронных вычислительных машин, базу данных, топологию интегральных микросхем,
        % зарегистрированные в установленном порядке.(в ред. Постановления Правительства РФ от 21.04.2016 N 335)
    \end{refsection}%
    \begin{refsection}[bl-author, bl-registered]
        % Это refsection=2.
        % Процитированные здесь работы:
        %  * попадают в авторскую библиографию, при usefootcite==0 и стиле `\insertbiblioauthorimportant`.
        %  * ни на что не влияют в противном случае
        \nocite{Moiseenko-al:OOPSLA22}
        \nocite{Moiseenko-al:ECOOP20}
        \nocite{Moiseenko-al:STJITMO22}
        \nocite{Moiseenko-al:PCS21}
        \nocite{Gladstein-al:ISPRAS21}
        % \nocite{vakbib2}%vak
        % \nocite{patbib1}%patent
        % \nocite{progbib1}%program
        % \nocite{bib1}%other
        % \nocite{confbib1}%conf
    \end{refsection}%
        %
        % Всё, что вне этих двух refsection, это refsection=0,
        %  * для диссертации - это нормальные ссылки, попадающие в обычную библиографию
        %  * для автореферата:
        %     * при usefootcite==0, ссылка корректно сработает только для источника из `external.bib`. Для своих работ --- напечатает "[0]" (и даже Warning не вылезет).
        %     * при usefootcite==1, ссылка сработает нормально. В авторской библиографии будут только процитированные в refsection=0 работы.
}

% При использовании пакета \verb!biblatex! будут подсчитаны все работы, добавленные
% в файл \verb!biblio/author.bib!. Для правильного подсчёта работ в~различных
% системах цитирования требуется использовать поля:
% \begin{itemize}
%         \item \texttt{authorvak} если публикация индексирована ВАК,
%         \item \texttt{authorscopus} если публикация индексирована Scopus,
%         \item \texttt{authorwos} если публикация индексирована Web of Science,
%         \item \texttt{authorconf} для докладов конференций,
%         \item \texttt{authorpatent} для патентов,
%         \item \texttt{authorprogram} для зарегистрированных программ для ЭВМ,
%         \item \texttt{authorother} для других публикаций.
% \end{itemize}
% Для подсчёта используются счётчики:
% \begin{itemize}
%         \item \texttt{citeauthorvak} для работ, индексируемых ВАК,
%         \item \texttt{citeauthorscopus} для работ, индексируемых Scopus,
%         \item \texttt{citeauthorwos} для работ, индексируемых Web of Science,
%         \item \texttt{citeauthorvakscopuswos} для работ, индексируемых одной из трёх баз,
%         \item \texttt{citeauthorscopuswos} для работ, индексируемых Scopus или Web of~Science,
%         \item \texttt{citeauthorconf} для докладов на конференциях,
%         \item \texttt{citeauthorother} для остальных работ,
%         \item \texttt{citeauthorpatent} для патентов,
%         \item \texttt{citeauthorprogram} для зарегистрированных программ для ЭВМ,
%         \item \texttt{citeauthor} для суммарного количества работ.
% \end{itemize}

% Счётчик \texttt{citeexternal} используется для подсчёта процитированных публикаций;
% \texttt{citeregistered} "--- для подсчёта суммарного количества патентов и программ для ЭВМ.

% Для добавления в список публикаций автора работ, которые не были процитированы в
% автореферате, требуется их~перечислить с использованием команды \verb!\nocite! в
% \verb!Synopsis/content.tex!.

Личный вклад автора в публикациях, выполненных с соавторами, распределён следующим образом.
В работе \cite{Gladstein-al:ISPRAS21} автор предложил
метод кодирования семантики параллельной регистровой машины с
моделью памяти, сохраняющей программный порядок, в терминах простых структур событий;
соавторы участвовали в формализации данного метода в системе \coq.
В работе \cite{Moiseenko-al:STJITMO22} автор предложил
метод кодирования семантического домена языков частично упорядоченных мультимножеств
с использованием фактор-типов, 
соавторы участвовали в обсуждении данного метода и его формализации в системе \coq.
В работе \cite{Moiseenko-al:PCS21} автор выполнил сбор и анализ данных
о существующих моделях памяти языков программирования;
соавторы участвовали в формулировке выводов данного анализа.
В работе \cite{Moiseenko-al:ECOOP20} автор выполнил
формализацию доказательства корректности компиляции из
модели \Wkm в модели современных мультипроцессоров;
соавторы участвовали в обсуждении данного доказательства
и его формализации в системе \coq.
В работе \cite{Moiseenko-al:OOPSLA22} автор
формализовал новые свойства модели \WkmS,
а именно, свойства свободы от буферизации операций чтения и локальности сертификации,
доказал сохранение полезных свойств модели \Wkm в \WkmS,
а также разработал прототип алгоритма проверки моделей для \WkmS;
соавторы участвовали в обсуждении формализации модели \WkmS
и доказательстве ее свойств,
оказывали помощь при реализации нового алгоритма,
а также провели эксперименты по сравнению нового алгоритма с аналогами.
 % Характеристика работы по структуре во введении и в автореферате не отличается (ГОСТ Р 7.0.11, пункты 5.3.1 и 9.2.1), потому её загружаем из одного и того же внешнего файла, предварительно задав форму выделения некоторым параметрам

\textbf{Объем и структура работы.} Диссертация состоит из~введения,
\formbytotal{totalchapter}{глав}{ы}{}{},
заключения и
\formbytotal{totalappendix}{приложен}{ия}{ий}{}.
%% на случай ошибок оставляю исходный кусок на месте, закомментированным
%Полный объём диссертации составляет  \ref*{TotPages}~страницу
%с~\totalfigures{}~рисунками и~\totaltables{}~таблицами. Список литературы
%содержит \total{citenum}~наименований.
%
Полный объём диссертации составляет
\formbytotal{TotPages}{страниц}{у}{ы}{}, включая
\formbytotal{totalcount@figure}{рисун}{ок}{ка}{ков} и
\formbytotal{totalcount@table}{таблиц}{у}{ы}{}.
Список литературы содержит
\formbytotal{citenum}{наименован}{ие}{ия}{ий}.
    % Введение
\ifnumequal{\value{contnumfig}}{1}{\counterwithout{figure}{chapter}
}{\counterwithin{figure}{chapter}}
\ifnumequal{\value{contnumtab}}{1}{\counterwithout{table}{chapter}
}{\counterwithin{table}{chapter}}

\chapter{Обзор}
\label{ch:review}
           % Глава 1: Обзор
\chapter{Структуры событий для моделей памяти сохраняющих программный порядок}
\label{ch:porf-evenstruct}

В данной главе описан предложенный в диссертации метод 
кодирования слабых моделей памяти сохраняющих программный порядок 
с помощью простых структур событий с предикатом консистентности. 
Этот результат позволяет утверждать, что теория  
моделей памяти сохраняющих программный порядок%
\footnote{Везде далее в этом разделе 
под термином ``модель памяти'' будем подразумевать 
модель памяти сохраняющую программный порядок, 
если явно не утверждается иное.} 
может быть сведена к теории простых структур событий, 
и, как следствие, позволяет применить множество 
уже известных результатов о структурах событий%
~\cite{Winskel:86,Vaandrager:TCS1991,Sassone:MFCS1993,Nielsen:REX93,Winskel-TCS:09}
к проблемам слабых моделей памяти.

Теория простых структур событий и метод сведения 
моделей памяти к ним были формализованы в системе 
доказательства теорем \coq. 
Полученная в результате библиотека может быть использована 
для формализации доказательства свойств структур событий 
и моделей памяти, а также при разработке других
инструментов для интерактивной верификации многоточных программ.
%% Технические вопросы представления структур событий 
%% и других упоминаемых в данной главе формализмов 
%% также кратко рассматриваются в данной главе.  

Данная глава организована следующим образом. 
В разделе \cref{sec:pomset-graphs} описывается метод 
сведения графов сценариев исполнения к языкам помсетов. 
Далее, в разделе \cref{sec:mm-eventstruct} этот результат 
используется для сведения моделей памяти к простым структурам событий. 
В разделе \cref{sec:eventstruct-opsem} описывается 
операционная семантика для инкрементального построения 
структуры событий по заданной многопоточной программе 
и доказываются основные её свойства. 
%% Наконец, в \cref{sec:mm-eventstruct} кратко рассматриваются 
%% технические вопросы представления структур событий 
%% и других упоминаемых в данной главе формализмов.

\section{Сведение графов сценариев исполнения к языкам помсетов}
\label{sec:pomset-graphs}

%% Эквивалентность задания моделей памяти в терминах
%% графов сценариев исполнения и в терминах языков помсетов.

Напомним, что и формализм графов сценариев исполнения,
широко распространненый в сообществе для задания слабых моделей памяти%
~\cite{Alglave-al:TOPLAS14}, 
и формализм помсетов, известный по классическим работам%
~\cite{Pratt:CONCUR84,Gischer:TCS88} 
в области семантики многопоточных программ, основаны на теории частичных порядков. 
Главное отличие между ними заключается в том, что помсет состоит из единственного
отношения частичного порядка --- отношения причинно-следственной связи, 
в то время как граф сценариев исполнения состоит из
нескольких отношений, наделенных различной семантикой. 

Основная идея представленного метода сведения заключается в том 
что в рамках моделей сохраняющих программный порядок 
объединение отношения программного порядка и отношения ``читает-из''
может рассматриваться как аналог отношения причинно-следственной связи.
Таким образом множество графов сценариев исполнения можно факторизовать 
по отношению эквивалентности индуцированных отношений причинно-следственной связи.
И наоборот, можно построить частичную функцию, 
которая принимая на вход помсет и пытается выполнить 
разбиение отношение причинно-следственной связи на 
программный порядок и отношение ``читает-из''. 

Рассмотрим данное построение более формально. 
Для начала, введем формальное определение 
аксиоматических моделей памяти сохраняющих программный порядок.

\begin{definition}
Будем говорить, что граф сценария исполнения $G$ 
сохраняет программный порядок, если выполняются следующие условия: 
\begin{itemize}
  \item $G.\lR \suq \cod{G.\lRF}$; 
    \labelAxiom{$\lRF$-complete}{ax:rf-complete}

  \item $\lPO \cup \lRF$ является ацикличным отношением.
    \labelAxiom{$\lPORF$-acyclic}{ax:porf-acyc}
\end{itemize}
Обозначим множество всех таких графов как $\PorfExecG$.
Также будем говорить, что модель памяти $M$, 
заданная в аксиоматическом стиле, сохраняет программный порядок, 
если для любой программы $P$ каждый соответствующий ей
граф сценария исполнения $G \in \sem{P}_M$ сохраняет программный порядок, 
то есть ${\sem{P}_M \suq \PorfExecG}$.
\end{definition}

Далее определим функцию для построения помсета по графу сценария исполнения.

\begin{definition}
Определим функцию $\gpom : \PorfExecG \fun \Pom[\TidMemLab]$ следующим образом. 
Пусть $G \in \PorfExecG$. Тогда $\gpom(G) = \tup{E, \lab, \ca}$ где
\begin{itemize}
  \item $E = G.\lE$, 
  \item $\lab(e) = \tup{G.\lTID(e), G.\lLAB(e)}$,
  \item $\ca {}\defeq{} (G.\lPO \cup G.\lRF)^*$.
\end{itemize}
\end{definition}

Также предъявим обратную функцию, выполняющую построение 
графа сценария исполнения по помсету.
Для этого сначала определим подмножество помсетов, 
для которых такое построение возможно. 

\begin{definition}
Пусть $p = \tup{E, \lab, \ca} \in \Pom[\TidMemLab]$
и пусть $\simeq$ это некоторое отношение эквивалентности на метках событий. 
Будем называть помсет $p$ \emph{разделяемым на потоки относительно $\simeq$} 
(\emph{threaded pomset}) если для любого класса эквивалентности $\eset \suq E$
относительно $\simeq$ верно, что сужение помсета на события этого класса $p\rst{\eset}$ 
порождает линейно упорядоченное мультимножество. 
Множество всех таких помсетов будем обозначать как $\ThrdPom[\TidMemLab, \simeq]$.
\end{definition}

\begin{definition}
Пусть $p = \tup{E, \lab, \ca} \in \ThrdPom[\TidMemLab, \lEQTID]$. 
В таком случае определим индуцированное отношение 
программный порядка следующим образом:
$$ \lPO(p) \defeq \sca \cap \lEQTID. $$
Также определим множество кандидатов 
на отношение ``читает-из'' $\lRFs(p)$ 
таким образом, что $\lRF \in \lRFs(p)$ 
при выполнении следующих условий:
\begin{itemize}
  \item $\lRF {}\suq{} \lEQLOC \cap \lEQVAL$,
  \item $E \cap \lR \suq \cod{\lRF}$, 
  \item $\ca {}={} (\lPO(p) \cup \lRF)^*$.
\end{itemize}
%
Соответствующим образом определим 
множество графов-кандидатов $\ExecGs{p}$. 
Положим $G \in \ExecGs{p}$ если $p \in \ThrdPom[\TidMemLab]$
и выполнены следующие условия:
\begin{itemize}
  \item $G.\lE = E$,
  \item $\tup{G.\lTID(e), G.\lLAB(e)} = \lab(e)$, 
  \item $G.\lPO = \lPO(p)$, 
  \item $G.\lRF \in \lRFs(p)$. 
\end{itemize}
Если же $p \not\in \ThrdPom[\TidMemLab]$ то положим $\ExecGs{p} = \emptyset$.
%

Множество всех помсетов для которых $\ExecGs{p} \neq \emptyset$
будем обозначать как $\PorfPom[\TidMemLab]$.
Также зададим частично определенную функцию 
${\pomg : \Pom[\TidMemLab] \pfun \ExecG}$,
которая для данного помсета выбирает 
один из соответствующих ему графов.
\begin{equation*}
  \pomg(p) = \begin{cases*}
    G      & такой что $G \in \ExecGs{p}$   \\
    \bot   & если $\ExecGs{p} = \emptyset$. \\
  \end{cases*}
\end{equation*}
%
\end{definition}

\begin{proposition}
Для любого помсета $p \in \Pom[\TidMemLab]$
верно, что $\ExecGs{p} \suq \PorfExecG$
и, следовательно, $\pomg(p) \in \PorfExecG$.
\end{proposition}

Наконец, покажем, что произвольная модель памяти
сохраняющая программный порядок может быть 
представлена как язык помсетов. 

\begin{definition}
Пусть $M$ --- это заданная в аксиоматическом стиле 
модель памяти сохраняющая программный порядок,
то есть $M \suq \PorfExecG$.
Построим соответствующий ей язык помсетов 
$\wmmlang{M} \in \Pomlang[\TidMemLab]$ следующим образом.
Будем считать, что помсет принадлежит этому языку, 
если найдется хотя бы один граф, допустимый $M$,
который соответствует этому помсету.  
$$ p \in \wmmlang{M} \defeq \exists G \in \ExecGs{p} \ldotp G \in M $$
\end{definition}
 
\section{Кодирование модели памяти с помощью структуры событий}
\label{sec:mm-eventstruct}

В предыдущем разделе было показано, что 
модель памяти сохраняющая программный порядок 
может быть представлена как язык помсетов. 
Это наблюдение позволяет заключить, 
что модель памяти также может быть представлена 
и как простая структура событий с предикатом консистентности,
ведь, как было упомянуто в разделе~\cref{sec:pomsets-eventstruct}, 
данный класс структур событий позволяет выразить произвольный язык помсетов. 

В этом разделе будет также показано,
что на самом деле модель памяти сохраняющая программный порядок 
может быть представлена как простая структура событий специального вида.
В данной структуре событий все события одного потока образуют дерево. 
Ветви этого дерева соответствуют различным путям исполнения данного потока,
а события, принадлежащие разным веткам дерева, находятся в отношении конфликта. 
Cтруктуры событий данного вида будем называть 
\emph{разделяемыми на потоки} (\emph{threaded prime event structures}).

Основной особенностью структуры событий принадлежашей 
данному подклассу является то,
что она может быть закодирована с помощью 
только одного отношения частичного порядка
и некоторого отношения эквивалентности на метках.
Отношение конфликта при этом не требуется хранить явно, 
оно индуцируется двумя предыдущими отношениями. 
Такое представление, зачастую, позволяет существенно 
упростить рассуждения о структурах событий. 
Например, можно заметить, что две структуры событий 
данного класса изоморфны тогда и только тогда, 
когда они изоморфны как помеченные частично упорядоченные множества. 

Введем формальные определения описанных выше объектов. 

\begin{definition}
Частично упорядоченное множество $\tup{E, \ca}$ 
называется \emph{префиксно-линейно упорядоченным} 
(\emph{downward-total}) если выполняется следующие условие:
$$ x \ca z \wedge y \ca z \implies x \ca y \vee y \ca x. $$
Если в дополнение к этому частично упорядоченное множество являетя 
префикс-конечным, то будем называть такое множество \emph{лесом}.
Если кроме того существует наименьший элемент $e_0 \in E$, 
тогда будем называть такое множество \emph{деревом}, 
а $e_0$ --- корнем этого дерева. 
\end{definition}

\begin{proposition}
Для леса $\tup{E, \ca}$ можно задать 
частично определенную функцию $\pred : E \pfun E$, 
возращающую родителя данного элемента: 
$$ \pred(e) = e' \;{}\iff{}\; e' \ica e $$
\end{proposition}

\begin{proposition}
Размеченный лес $\tup{E, \lab, \ca}$ порождает простую структуру событий 
без волнений (confusion-free) $\tup{E, \lab, \ca, \cf}$, 
где отношение непосредственного конфликта задается следующим образом: 
$$ e_1 \icf e_2 \defeq e_1 \neq e_2 \wedge \pred(e_1) = \pred(e_2). $$
\end{proposition}

\begin{definition}
Пусть $S = \tup{E, \lab, \ca, \cf} \in \PrimeES[L]$ --- 
это простая структура событий
с бинарным конфликтом, и пусть $\simeq$ это некоторое 
отношение эквивалентности на метках $L$.
Рассмотрим сужение структуры событий на классы эквивалентности 
${ S\rst{\simeq} \defeq \tup{E, \lab, \ca {}\cap{} \simeq, \cf {}\cap{} \simeq} }$.
Будем говорить, что структура \emph{разделима на потоки относительно отношения $\simeq$}
если $S\rst{\simeq}$ образует размеченный лес и, кроме того, 
отношение конфликта во всей структуре событий является 
продолжением отношения конфликта в суженной структуре, то есть 
$${ \icf_S = \icf_{S\rst{\simeq}} }.$$
Будем обозначать множество всех таких структур 
как $\ThrdPrimeES[L,\simeq]$
\end{definition}

\begin{proposition}
Пусть $p = \tup{E, \lab, \ca}$ это 
размеченное частично упорядоченное множество, 
а $\simeq$ отношение эквивалентности на метках событий.
Предположим, что сужение отношения порядка на классы эквивалентности 
${ p\rst{\simeq} \defeq \tup{E, \lab, \ca {}\cap{} \simeq} }$ образует лес.
Рассмотрим отношение непосредственного конфликта $\icf$, порождаемое этим лесом. 
Если отношение $\cf$, определенное как продолжение $\icf$ вдоль отношения $\ca$,
иррефлексивно, тогда $\tup{E, \lab, \ca, \cf}$ является 
простой структурой событий разделимой на потоки относительно отношения $\simeq$.  
\end{proposition}

Рассмотрим структуру событий $S \in \ThrdPrimeES[\TidMemLab, \lEQTID]$
разделяемую на потоки относительно отношения $\lEQTID$.
Для данной модели памяти $M$ можно дополнить $S$ предикатом консистентности, 
который будет отфильтровывать все конфигурации, 
не принадлежащие языку $\wmmlang{M}$.

\begin{definition}
Пусть ${S = \tup{E, \lab, \ca, \cf} \in \ThrdPrimeES[\TidMemLab, \lEQTID]}$
и ${M \suq \PorfExecG}$.
Определим простую структуру событий с предикатом консистентности
${\wmmpes{S}{M} = \tup{E, \lab, \ca, \cons}}$ таким образом, что:
\begin{itemize}
  \item $E {}\defeq{} E_S$,
  \item $\lab {}\defeq{} \lab_S$,
  \item $\ca {}\defeq{} \ca_S$,
  \item $C \in \cons {}\defeq{} C \not\in \gcf_S \wedge S\rst{C} \in \wmmlang{M}$.
\end{itemize}
\end{definition}

Корректность структуры событий $\wmmpes{S}{M}$ относительно языка 
порождаемого $M$ вытекает напрямую из определения. 

\begin{proposition}
\label{prop:thrd-es-sound}
Для любой структуры событий $S = \ThrdPrimeES[\TidMemLab, \lEQTID]$
и любой модели памяти $M \suq \PorfExecG$
язык помсетов, порождаемый $\wmmpes{S}{M}$, 
корректен относительно языка $\wmmlang{M}$, то есть:
$$ \pomlang{\wmmpes{S}{M}} \suq \wmmlang{M}. $$
\end{proposition}

Тем не менее, нетрудно заметить, что структура событий $\wmmpes{S}{M}$
не обязана быть полной относительно языка порождаемого $M$.
В следующем разделе будет рассмотрена операционная семантика 
для инкрементального построения структуры $S$ 
порождающей структуру $\wmmpes{S}{M}$ 
полную относительно языка $\sem{P}_M$ для 
любой заверщающейся программы $P$.

\section{Операционная семантика построения структуры событий}
\label{sec:eventstruct-opsem}

В данном разделе представлена операционная семантика 
для инкрементального построения структуры событий.
Предложенная семантика предоставляет конкретную
процедуру построения структуры событий, 
кодирующую все возможные сценарии поведения 
заданной программы~$P$ в заданной модели памяти~$M$.  
Также доказываются основные свойства данной семантики, 
а именно \emph{конфлюэнтность}, \emph{полнота} и \emph{терминируемость}.

Отметим, что помимо модели памяти $M$, данная семантика также параметризована
последовательной семантикой потоков программы $P$, 
заданной в терминах системы помеченных переходов $\LTS$. 
Это, в частности, позволяет абстрагироваться от деталей 
реализации последовательной семантики 
и комбинировать предложенную семантику построения структуры событий 
с различными моделями последовательных вычислений. 

Прежде чем перейти к рассмотрению операционной семантики, 
введем несколько вспомогательных определений, 
помогающих связать структуру событий с 
последовательной семантикой. 

\begin{definition}
Пусть $\LTS = \tup{\State, L, \ltr{}{}{}}$ это система помеченных переходов, 
а $p = \tup{E, \lab, \ca}$ это лес размеченный метками типа $\Step{L}{\State}$. 
Будем говорить, что $p$ \emph{корректен} (\emph{sound}) 
относительно $\LTS$ если выполняются следующие условия:
\begin{itemize}
  \item для любого события $e \in E$ его метка 
    $\lab(e) = \step{l}{\state}{\state'}$ 
    образует валидный переход, то есть $\ltr{l}{\state}{\state'}$;
  %
  \item метки соседних событий $e_1 \ica e_2$ \emph{сопряжены} в том смысле, что если 
    $\lab(e_1) = \step{l_1}{\state_1}{\state'_1}$ и 
    $\lab(e_2) = \step{l_2}{\state_2}{\state'_2}$ тогда 
    $\state'_1 = \state_2$.
\end{itemize}
Будем говорить, что структура событий разделимая на потоки
корректна относительно системы переходов, 
если соответствующей этой структуре лес корректен. 
\end{definition}

\begin{proposition}
Пусть $p$ это лес, корректный относительно системы переходов~$\LTS$.
Тогда для любого событий $e \in E$ верно, что
метки событий его префикс $\dwset{e}$, упорядоченного согласно отношению $\ca$, 
образуют валидную трассу системы $\LTS$.

Аналогичное утверждение верно для структуры событий разделимой на потоки,
с точностью до того, что для такой структуры необходимо сузить
префикс на события того же потока, что и рассматриваемое событие $e$.  
\end{proposition}

Для того чтобы хранить локальные состояния потоков в структуре событий
дополним тип меток $\TidMemLab$ до типа 
$\ThrdMemLab \defeq \Tid \times \Lab \times \State \times \State$
расширив его парой состояний, формирующих переход.
Также определим функцию $\lSTEP$, извлекающую из метки $\ell \in \ThrdMemLab$ 
тройку образующую помеченный переход:
$$ \lSTEP(\tup{t, \el, \state, \state'}) = \step{\el}{\state}{\state'}. $$

Для удобства при необходимости мы также будем трактовать 
структуру событий $S \in \PrimeES[\ThrdMemLab]$ как 
структуру с метками типа $\TidMemLab$, то есть 
$S \in \PrimeES[\ThrdMemLab]$.

\begin{definition}
Предположим, что многопоточная программа $P$ задана как 
система помеченных переходов $\LTS = \tup{\State, L, \ltr{}{}{}}$ и 
функция инициализации потоков $\initst : \Tid \pfun \State$.
Будем говорить, что структура событий 
$S \in \in \ThrdPrimeES[\ThrdMemLab, \lEQTID]$ 
\emph{корректна} (\emph{sound}) относительно программы $P$ 
если она корректна относительно $\LTS$ и, кроме того, 
для любого потока $t \in \Tid$ сужение $S$ на 
события этого потока $S\rst{t}$ порождает дерево с корнем $e_t$ таким, 
что $\stlab(e_t) = \step{\lTS}{\initst(t)}{\initst(t)}$.
\end{definition}

Сформулируем утверждение, суть которого заключается в том, 
что структура событий корректная относительно программы $P$
кодирует графы сценариев исполнения, которые 
допускаются моделью памяти $M$ для программы $P$.

\begin{lemma}
Рассмотрим струтуру событий $S \in \ThrdPrimeES[\ThrdMemLab, \lEQTID]$,
модель памяти $M \suq \PorfExecG$ и программу $P$.
Тогда для любого помсета $p$, принадлежащего языку $\wmmpes{S}{M}$, 
и для любого соответствующего ему графа $G \in \ExecGs{p}$ 
верно, что этот граф корректен относительно $P$ в модели $M$, 
то есть $G \in \sem{P}_M$.
\end{lemma}

Перейдем к рассмотрению процедуры построения структуры событий. 
Начнем с вспомогательного правила перехода, 
которое позволяет расширить произвольный помсет 
путем добавления в его конец нового события.

\newcommand{\PomAddEventRule}{{(Add~Event)}\xspace}
\newcommand{\PorfAddEventRule}{{(ES~Step)}\xspace}

\begin{center}
  \AXC{$e \not\in E$}
  \AXC{$\eset \subseteq E$}
  %
  \RightLabel{\PomAddEventRule}
  \BIC{$\tup{E, \lab, \ca}
        \pomStep{\tup{e, \ell, \eset}}
        \tup{E \uplus \set{e}, \updmap{\lab}{e}{\ell}, \ca \uplus \dwset{\eset} \times \set{e}}$}
  \DisplayProof
\end{center}

Это правило добавляет в помсет $p = \tup{E, \lab, \ca}$
новое событие $e$ c меткой $l$ и множеством событий-предшественников $\eset$. 

Далее, рассмотрим правило перехода, которое 
выполняет построение интересующей нас структуры событий, 
также путем добавления одного нового события. 

\begin{center}
  
  \AXC{$\ltr{\el}{\state}{\state'}$}
  \noLine
  \UIC{$\lSTEP(\ell) = \step{\el}{\state}{\state'}$}
  \noLine
  \UIC{$p \pomStep{\tup{e, \ell, \set{e_{\lPO}, e_{\lRF}}}} p'$}
  %
  \AXC{$\stlab(e_{\lPO}) \posync \lSTEP(\ell)$}
  \noLine
  \UIC{$\dlab(e_{\lRF}) \rfsync \lLAB(\ell)$}
  %
  \AXC{$p' \in \DetPom$}
  \noLine
  \UIC{$\upset{e_{\lPO}} \cap \dwset{e_{\lRF}} \subseteq \emptyset$}
  \noLine
  \UIC{$p'\rst{\dwset{e'}} \in \wmmlang{M}$}
  %
  \RightLabel{\PorfAddEventRule}
  \TIC{$p \esStep{\tup{e, \ell, e_{\lPO}, e_{\lRF}}} p'$}
  \DisplayProof
\end{center}

Данное правило добавляет в помсет $p = \tup{E, \lab, \ca}$ 
событие $e$ c меткой $\ell$, такой что $\lSTEP(\ell)$ 
образует валидный переход в рамках последовательной семантики.
Данное правило также недетерминированным образом выбирает 
для нового события двух предков $e_{\lPO}$ и $e_{\lRF}$.
Кроме того требуется, чтобы метка события $e_{\lPO}$ 
была сопряжена c меткой нового события $\ell$. 
Аналогично, требуется чтобы метка $e_{\lRF}$
была согласована с $\ell$ в смысле отношения $\lRF$.

\begin{center}
  \AXC{$$}
  \RightLabel{}
  \UIC{$\step{l_1}{\state}{\state'} \posync \step{l_2}{\state'}{\state''}$}
  \DisplayProof
  \rulehskip
  %
  \AXC{$$}
  \RightLabel{}
  \UIC{$\tup{t, \tslab} \rfsync \tup{t,\wlab{o}{x}{v}}$}
  \DisplayProof
  \rulevspace
  
  \AXC{$$}
  \RightLabel{}
  \UIC{$\tup{t, \tslab} \rfsync \tup{t,\rlab{o}{x}{\initval}}$}
  \DisplayProof
  \rulehskip
  % 
  \AXC{$$}
  \RightLabel{}
  \UIC{$\tup{t, \wlab{o}{x}{v}} \rfsync \tup{t, \rlab{o}{x}{v}}$}
  \DisplayProof
\end{center}

Предусловие $p' \in \DetPom$ проверяет, что новый помсет является детерминированным.

\begin{definition}
Помсет $p \in \Pom[L]$ назывется \emph{детерминированным}, 
если все его события с одинаковой меткой и префиксом равны:
$$ \lab(e_1) = \lab(e_2) \wedge \dwsset{e_1} = \dwsset{e_2} \implies e_1 = e_2. $$
Другими словами, детерминированный помсет не может 
содержать дублирующиеся события. 
Множество всех таких помсетов будем обозначать как $\DetPom[L]$.
\end{definition}

Требование на детерминированность помсета необходимо для 
обеспечения завершимости построения структуры событий.
В противном случае, правило \PorfAddEventRule могло бы 
добавлять одно и то же событие в помсет неограниченное 
количество раз. 

\begin{proposition}
Помсет $p$, построенный по правилу \PorfAddEventRule, 
то есть $p_0 \esStep{}^* p$, является детерминированным: $p \in \DetPom$.
\end{proposition}

Предусловие $\upset{e_{\lPO}} \cap \dwset{e_{\lRF}}$ гарантирует, 
что отношение конфликта $\cf$, порождаемое $p'$, будет иррефлексивно. 
Действительно, в обновленной структуре $p'$, новое событие $e$
будет находиться в конфликте со всеми потомками события $e_{\lPO}$.
Если бы среди них нашелся хотя бы один предок события $e_{\lRF}$, 
то отношение конфликта можно было бы продлить до $e_{\lRF}$
и, следовательно, до $e$, получив тем что $e$ находится 
в конфликте с самим собой. 

\begin{proposition}
Помсет $p$, построенный по правилу \PorfAddEventRule, 
то есть $p_0 \esStep{}^* p$, 
порождает структуру событий $S = \tup{E, \lab, \ca, \cf}$
разделимую на потоки относительно $\lEQTID$: 
$S \in \ThrdPrimeES[\TidMemLab, \lEQTID]$.
\end{proposition}

Наконец, рассмотрим назначение последнего предусловия 
$p'\rst{\dwset{e'}} \in \wmmlang{M}$.
Это предусловие проверят, что префикс нового события $e'$
формирует консистентный согласно модели $M$ помсет. 
Строго говоря, данное предусловие не является необходимым.
Действительно, согласно \cref{prop:thrd-es-sound},
структура событий $\wmmpes{S}{M}$, 
порождаемая помсетом $p$ и дополненная 
предикатом консистентности, уже является корректной
относительно модели памяти $M$, 
то есть она кодирует только графы, допустимые $M$.
Поэтому, вообще говоря, для обеспечения корректности 
нет необходимости каким либо образом дополнительно отфильтровывать 
неконсистентные графы во время построения структуры событий.
Тем не менее, данное предусловие позволяет на раннем 
этапе предотвратить добавление событий, 
которые заведомо не могут быть включены ни в один консистетный помсет.    
Другими словами, добавление рассматриваемого предусловия
в правило \PorfAddEventRule заранее отсекает те ветви структуры событий, 
исследование которых не приведет к появлению новых консистентных помсетов.  

Отметим, что описанная выше оптимизация полагается на то, 
что язык, порождаемый $M$, является префикс-замкнутым. 
Иными словами, для любого консистентного помсета $p \in \wmmlang{M}$
должно быть справедливо, что его сужение на произвольный префикс
остается консистентным, то есть для любого $\eset \suq E$
верно, что $p\rst{\eset} \in \wmmlang{M}$.
В противном случае, не гарантируется полнота операционной семантики. 
Действительно, несложно построить пример искуственной модели памяти $M$
такой, что некоторый помсет $p$ является консистентным, 
но при этом его сужение на префикс некоторого события $e$ неконсистентно.
То есть $p \in \wmmlang{M}$, но $p\rst{\dwset{e}} \not\in \wmmlang{M}$.
В таком случае, построить $p$ инкрементально с помощью правила \PorfAddEventRule
не получится, так как на шаге добавления события $e$ 
процесс построения оборвется. 

%% \begin{proposition}
%% Пусть $S$

%% Пусть $p$ это лес, корректный относительно системы переходов~$\LTS$.
%% Тогда для любого событий $e \in E$ верно, что
%% метки событий его префикс $\dwset{e}$, упорядоченного согласно отношению $\ca$, 
%% образуют валидную трассу системы $\LTS$.

%% Аналогичное утверждение верно для структуры событий разделимой на потоки,
%% с точностью до того, что для такой структуры необходимо сузить
%% префикс на события того же потока, что и рассматриваемое событие $e$.  
%% \end{proposition}


%% для любого событий $e \in E$
%% его префикс $\dwset{e}$, упорядоченный согласно $\ca$, 
%% образует валидную трассу $$

%% с выделенным начальным состоянием $$

%% и ${M \suq \PorfExecG}$.
%% Определим простую структуру событий с предикатом консистентности
%% ${\wmmpes{S}{M} = \tup{E, \lab, \ca, \cons}}$ таким образом, что:
%% \begin{itemize}
%%   \item $E {}\defeq{} E_S$,
%%   \item $\lab {}\defeq{} \lab_S$,
%%   \item $\ca {}\defeq{} \ca_S$,
%%   \item $C \in \cons {}\defeq{} C \not\in \gcf_S \wedge S\rst{C} \in \wmmlang{M}$.
%% \end{itemize}


\todo{}

%% \newcommand{\PomAddEventRule}{{(Add Event)}\xspace}
\newcommand{\PorfAddEventRule}{{(ES Step)}\xspace}
% \newcommand{\ESStoreRule}{{(Store)}\xspace}
% \newcommand{\ESLoadRule}{{(Load)}\xspace}
% \newcommand{\ESLoadBotRule}{{(Load-Bottom)}\xspace}

\begin{figure*}[t]
\begin{center}

  \AXC{$e \not\in E$}
  \AXC{$\eset \subseteq E$}
  %
  \RightLabel{\PomAddEventRule}
  \BIC{$\tup{E, \lab, \ca}
        \pomStep{\tup{e, \ell, \eset}}
        \tup{E \uplus \set{e}, \updmap{\lab}{e}{\ell}, \ca \uplus \dwset{\eset} \times \set{e}}$}
  \DisplayProof

  \rulevspace

  \AXC{$\ltr{\el}{\state}{\state'}$}
  \noLine
  \UIC{$\lSTEP(\ell) = \step{\el}{\state}{\state'}$}
  %
  \AXC{$p \pomStep{\tup{e, \ell, \set{e_{\lPO}, e_{\lRF}}}} p'$}
  %
  \AXC{$\stlab(e_{\lPO}) \posync \lSTEP(\ell)$}
  \noLine
  \UIC{$\dlab(e_{\lRF}) \rfsync \lLAB(\ell)$}
  %
  \RightLabel{\PorfAddEventRule}
  \TIC{$p \esStep{\tup{e, \ell, e_{\lPO}, e_{\lRF}}} p'$}
  \DisplayProof

  \rulevspace

  %% \AXC{$p \pomStep{\tup{e, \ell, \set{e} \setminus \set{e'}}} p'$}
  %% \AXC{$\ell' \ssync \ell$}
  %% \AXC{$(\ell \neq \ell' = \lab(e') \wedge e' \in E) \vee 
  %%       (\ell = \ell' \wedge e = e')$}
  %% \RightLabel{\RfAddEventRule}
  %% \TIC{$p \rfStep{\tup{e, \ell, e'}} p'$}
  %% \DisplayProof

  %% \rulevspace

  %% \AXC{$p_{\lPO} \poStep{\tup{e, \ell, e_\lPO}} p'_{\lPO}$}
  %% \AXC{$p_{\lRF} \rfStep{\tup{e, \ell, e_\lRF}} p'_{\lRF}$}
  %% \AXC{$ \upset{e_{\lPO}} \cap \dwset{e_{\lRF}} \subseteq \emptyset $}
  %% \RightLabel{\PorfAddEventRule}
  %% \TIC{$\tup{p_{\lPO}, p_{\lRF}} \porfStep{\tup{e, \ell, e'}} \tup{p'_{\lPO}, p'_{\lRF}}$}
  %% \DisplayProof
  
  % \AXC{$e = \fresh(\first(\eventSeq))$}
  % \AXC{$e_{\lPO} \in \eventSeq$}
  % \AXC{$e_{\lRF} \in \eventSeq$}
  % \RightLabel{\ESAddEventRule}
  % \TIC{$\tup{\eventSeq, \lLAB, \lfPO, \lfRF}
  %       \esstep{\tup{e, \ell, e_{\lPO}, e_{\lRF}}}
  %       \tup{e :: \eventSeq, \updmap{\lLAB}{e}{\ell}, \updmap{\lfPO}{e}{e_{\lPO}}, \updmap{\lfRF}{e}{e_{\lRF}}}$}
  % \DisplayProof

  % \rulevspace

  % \AXC{$e \in S.\eventSeq$}
  % \AXC{$\thrdst = \appmap{\thrdmap}{e}$}
  % \AXC{$P \vdash \thrdst \thrdstep{\eps} \thrdst'$}
  % \RightLabel{\ESIdleRule}
  % \TIC{$P \vdash \tup{S, \thrdmap} \fullstep{\eps} \tup{S, \updmap{\thrdmap}{e}{\thrdst'}}$}
  % \DisplayProof

  % \rulevspace

  % \AXC{$\thrdst = \appmap{\thrdmap}{e_{\lPO}}$}
  % \AXC{$P \vdash \thrdst \thrdstep{\ell} \thrdst'$}
  % \AXC{$S \esstep{\tup{e, \ell, e_{\lPO}, \bot}} S'$}
  % \AXC{$l = \wlab{x}{v}$}
  % \RightLabel{\ESStoreRule}
  % \QIC{$P \vdash \tup{S, \thrdmap} \fullstep{\tup{e, \ell, e_{\lPO}, \bot}} \tup{S', \updmap{\thrdmap}{e}{\thrdst'}}$}
  % \DisplayProof

  % \rulevspace

  % \AXC{$\thrdst = \appmap{\thrdmap}{e_{\lPO}}$}
  % \AXC{$P \vdash \thrdst \thrdstep{\ell} \thrdst'$}
  % \AXC{$S \esstep{\tup{e, \ell, e_{\lPO}, e_{\lRF}}} S'$}
  % \AXC{$\neg (e \cfRel e_{\lRF})$}
  % \AXC{$l = \rlab{x}{v}$}
  % \noLine
  % \UIC{$\appmap{\lLAB}{e_{\lRF}} = \wlab{x}{v}$}
  % \RightLabel{\ESLoadRule}
  % \QuinaryInfC{$P \vdash \tup{S, \thrdmap} \fullstep{\tup{e, \ell, e_{\lPO}, e_{\lRF}}} \tup{S', \updmap{\thrdmap}{e}{\thrdst'}}$}
  % \DisplayProof

  % \rulevspace

  % \AXC{$\thrdst = \appmap{\thrdmap}{e_{\lPO}}$}
  % \AXC{$P \vdash \thrdst \thrdstep{\ell} \thrdst'$}
  % \AXC{$S \esstep{\tup{e, \ell, e_{\lPO}, \bot}} S'$}
  % \AXC{$l = \rlab{x}{\bot}$}
  % \RightLabel{\ESLoadBotRule}
  % \QIC{$P \vdash \tup{S, \thrdmap} \fullstep{\tup{e, \ell, e_{\lPO}, \bot}} \tup{S', \updmap{\thrdmap}{e}{\thrdst'}}$}
  % \DisplayProof
  
  \caption{Operational Semantics}
  \label{fig:eventstruct-opsem}

\end{center}
\end{figure*}


%% Операционная семантика для построения структуры событий
%% по многопоточной программе с произвольной моделью памяти
%% сохраняющей программный порядок.
%% Свойства полученной семантики:

%% \begin{itemize}
%%   \item конфлюэнтность (confluence);
%%   \item корректность (soundness);
%%   \item полнота (completness),
%%     для случая конечных программ;
%%   \item терминируемость (termination),
%%     для случая завершающихся программ (well-founded).
%% \end{itemize}
 % Глава 2: Структуры событий для моделей,
                                        %          сохраняющих программный порядок
\chapter{Модель \Wkm и корректность компиляции}
\label{ch:weakestmo-imm}
    % Глава 2: Модель Weakestmo и корректность компиляции 
\chapter{Модель \WkmS и свойства свободы от буфферезации операций чтения и локальности сертификациии}
\label{ch:weakestmo2}

\section{Недостатки модели \Wkm}

\section{Cвобода от буфферезации операций чтения}

\section{Локальность сертификации}

\section{Формализация модели \WkmS}

\subsection{Свобода от гонок и буфферезации операций чтения}

\subsection{Корректность компиляции}

\subsection{Корректность локальных трансформаций}
       % Глава 3: Модель Weakestmo2 и свойства свободы от 
                                        %          буфферезации операций чтения и локальности сертификации
\chapter{Верификация методом проверки модели для \WkmS}
\label{ch:mc-weakestmo2}

Постановка задачи проверки моделей (model checking)
для многопоточных программ в рамках модели \WkmS.
Введение алгоритма \genmc.
Краткое описание модифицированного алгоритма \genmc
для модели \WkmS.
Краткий анонс полученных результатов сравнения нового алгоритма
с аналогами. 

\section{Алгоритм \genmc для моделей сохраняющих программный порядок}

Описание алгоритма \genmc. Пример работы алгоритма. 

\section{Алгоритм \wmc для модели \WkmS}

Описание модифицированной версии алгоритма \genmc --- \wmc ---
для модели \WkmS. Пример работы алгоритма.
Обсуждение открытых проблем и заделов
для будущей работы в алгоритме. 
    % Глава 4: Верификация методом проверки модели для Weakestmo2
\chapter{Сравнения и соотнесения}
\label{ch:related-work}

\section{Формализация моделей памяти в системах интерактивного доказательства теорем}

Обзор работ по формализации моделей памяти
в системах интерактивного доказательства теорем.
Сравнение данной работы (формализация теории структур событий
и модели \Wkm) с другими работами.

\section{Модели памяти сохраняющие семантические зависимости}

Сравнение модели \WkmS с другими моделями
сохраняющими семантическими зависимости
(в частности, \Prm~\cite{Kang-al:POPL17}, \PwP~\cite{Jagadeesan-al:OOPSLA2020}).

\section{Сравнение алгоритма \wmc с аналогами}

Сравнение алгоритма проверки моделей с аналогами (evaluation). Выводы. 
     % Глава 5: Сравнения и соотнесения

%% \chapter{Обзор}
\label{ch:review}

В данной главе приводится введение в предметную область
данного диссертационного исследования --- слабые модели памяти.
Рассматриваются требования, предъявляемые к моделям памяти,
обсуждаются существующие модели памяти и их классификация.
Рассматриваются недостатки существующих моделей памяти
и открытые исследовательские вопросы.
Также вводятся необходимые математические формализмы,
используемые для описания семантики многопоточных программ
и слабых моделей памяти. 

\section{Слабые модели памяти}
\label{sec:models-intro}

Напомним, что в контексте данного исследования
моделью памяти называется формальная семантика
многопоточных программ, оперирующих с разделяемой памятью.
В данной работе будем преимущественно говорить о моделях памяти 
высокоуровневых языков программирования~\cite{Moiseenko-al:PCS21}, 
таких как, например, \CPP, \Java и другие. 

Одной из наиболее простых для понимания моделей памяти
является модель \emph{последовательной согласованности}
(\emph{sequential consistency})~\cite{Lamport:TC79}.
В рамках данной модели каждый допустимый
сценарий поведения многопоточной программы
является результатом поочередного исполнения
атомарных обращений к разделяемой памяти из параллельных потоков.

Рассмотрим, например, параллельную программу, показанную ниже.

\begin{equation*}
\inarrII{
\writeInst{}{x}{1} \\
\readInst{}{r_1}{y} \\
\kw{if} {r_1 = 0} ~\{ \\
~~\rfcomment{критическая секция} \\
\}
}{
\writeInst{}{y}{1} \\
\readInst{}{r_2}{x} \\
\kw{if} {r_2 = 0} ~\{ \\
~~\rfcomment{критическая секция} \\
\}
}
\tag{Dekker's Lock}\label{ex:Dekker}
\end{equation*}


Данная программа является упрощенной версией
алгоритма блокировки Деккера~\cite{Dijkstra:68}.
В этой программе два потока соревнуются за доступ к критической секции.
Каждый поток, для того чтобы обозначить свое намерение войти в критическую секцию,
устанавливает значение переменной $x$ или $y$ соответственно
\footnote{В данной работе разделяемые переменные
будем обозначать как $x$, $y$, $z$..., 
а локальные переменные как $r_1$, $r_2$, $r_3$...}.
Право войти в критическую секцию получает тот поток, 
который успеет прочитать значение переменной до его установки другим потоком.

В рамках модели последовательной согласованности
в результате выполнения данной программы 
либо один из потоков прочитает значение~\tcode{1}, а другой~\tcode{0}, 
либо оба прочитают значение~\tcode{1} (в этом случае ни один из потоков
не войдет в критическую секцию).
То есть в результате получим один из следующих исходов:
${[r_1=0, r_2=1]}$, ${[r_1=1,r_2=0]}$ или ${[r_1=1,r_2=1]}$. 
Данные сценарии поведения будем называть 
\emph{последовательно согласованными}.

Алгоритм Деккера полагается на тот факт, что оба 
потока не могут одновременно прочитать значение~\tcode{0}.
В противном случае не гарантируется свойство \emph{взаимного исключения}, 
то есть два потока могут одновременно войти в критическую секцию. 
Тем не менее на практике можно также наблюдать
сценарий поведения данной программы, нарушающий это предположение, 
то есть в результате которого имеем ${[r_1=0,r_2=0]}$.
Например, данный сценарий поведения можно наблюдать,
если перевести приведенный выше псевдокод алгоритма Деккера
на язык \CLANG, скомпилировать его с помощью \GCC
и запустить получившийся код на процессорах семейства \IntelX.

Подобные сценарии поведения, 
не укладывающиеся в модель последовательной согласованности, 
принято называть \emph{слабыми сценариями}.
Слабые сценарии поведения могут появляться 
в результате выполнения различных оптимизаций 
компилятором при сборке программы или процессором при ее исполнении. 
Например, в случае программы \ref{ex:Dekker}, оптимизатор может выполнить 
\emph{переупорядочивание независимых инструкций} 
записи в переменную $x$ и чтения из переменной $y$ в левом потоке.
Для оптимизированной версии программы сценарий поведения, 
ведущий к результату ${[r_1=0, r_2=0]}$, уже является последовательно согласованным.

Современные многопоточные языки программирования 
как правило предоставляют \emph{слабые модели памяти},
то есть модели, допускающие слабые сценарии поведения,
поскольку более строгая модель последовательной согласованности
не допускает применение широкого спектра оптимизаций
и, следовательно, реализовация данной модели на практике
приводит к значительным накладным расходам%
~\cite{Marino-al:PLDI11,Singh-al:ISCA12,Liu-al:OOPSLA17,Liu-al:PLDI19}. 

Как было продемонстрировано выше на примере алгоритма Деккера, 
чтобы гарантировать корректность многопоточных программ
разрабочикам необходимо учитывать слабые сценарии поведения.
Таким образом, важно понимать, насколько слабую модель памяти
предоставляет язык программирования, какие оптимизации 
допускаются этой моделью памяти и какие гарантии 
эта модель предлагает программисту. 

Далее в разделе \ref{sec:models-primitives} 
вводятся программные примитивы для работы с разделяемой памятью. 
В разделе \ref{sec:models-requirements} более подробно обсуждаются 
различные требования, предъявляемые к моделям памяти, 
а в разделе \ref{sec:models-classes} обсуждаются существующие модели и их классификация. 
Наконец в разделе \ref{sec:models-summary} приводится сравнение различных 
классов моделей памяти и обозначаются открытые исследовательские проблемы, 
некоторые из которых были решены в рамках данного 
диссертационного исследования. 

\subsection{Программные примитивы предоставляемые моделью памяти}
\label{sec:models-primitives}

В рамках данного исследования будем считать, что
разделяемая память представляет собой отображение
из адресов переменных\footnote{В данной работе адрес переменной 
также иногда будем называть \emph{локацией}.} в их значения. 
Таким образом, будем подразумевать, 
что разделяемая память состоит из взаимно непересекающихся, 
типизированных адресуемых ячеек памяти%
\footnote{В теории моделей памяти также иногда определяют 
разделяемую память как нетипизированную последовательность байт, 
допускающую обращения \emph{смешанного размера} (\emph{mixed-size accesses}). 
В контексте данной работы смешанные обращения не рассматриваются.}

Основные операции, которые предоставляет абстракция разделяемой памяти --- 
это операция записи в разделяемую переменную и операция чтения из разделяемой переменной. 
Также, будем считать, что разделяемая память предоставляет атомарные операции 
\emph{чтения-модификации-записи} (\emph{read-modify-write}), 
в частности, операцию сравнения и замены (\emph{compare-and-swap}, \CAS), 
операцию атомарного обмена (\emph{exchange}, \EXCHG) 
и операцию атомарного инкремента (\emph{fetch-and-add}, \FADD).
Операция сравнения и замены атомарно выполняет сравнение 
текущего и ожидаемого значений переменной и в случае 
их совпадения заменяет значение переменной на желаемое.
Операция обмена атомарно заменяет значение переменной 
и возвращает ее прежнее значение. 
Наконец, операция атомарного инкремента прибавляет 
к значению переменной заданную величину и
возвращает ее значение до модификации.
Все инструкции обращения к разделяемой памяти 
приведены на Рис.\ref{fig:primitives}. 

\begin{center}
\begin{figure}[hb]
\begin{tabular}{l@{\hskip 40pt}|l} 

    \hline

      $\writeInst{o}{x}{v}$ 
    & \makecell[l]{
        Инструкция записи значения $v$ \\
        в разделяемую переменную $x$   \\
        с режимом доступа $o$.
      } 
    \\ 
    \hline

      $\readInst{o}{r}{x}$ 
    & \makecell[l]{
        Инструкция чтения значения    \\
        из разделяемой переменной $x$ \\
        в локальную переменную $r$    \\
        с режимом доступа $o$.
      } 
    \\ 
    \hline

      $\casInst{o_s}{o_f}{r}{x}{v_e}{v_d}$ 
    & \makecell[l]{
        Инструкция атомарного сравнения                 \\ 
        разделяемой переменной $x$                      \\
        c ожидаемым значением $v_e$ и                   \\ 
        заменой на желаемое значение $v_d$;             \\ 
        прочитанное значение присваивается              \\
        в локальную переменную $r$;                     \\
        в случае успеха операции сравнения применяется  \\ 
        режим доступа $o_s$, иначе $o_f$.
      } 
    \\ 
    \hline

      $\exchgInst{o}{r}{x}{v}$ 
    & \makecell[l]{
        Инструкция атомарного обмена значения           \\
        разделяемой переменной $x$ на значение $v_e$,   \\
        аннотированная режимом доступа $o$;             \\ 
        прочитанное значение присваивается              \\
        в локальную переменную $r$.                     \\
      } 
    \\ 
    \hline

      $\faiInst{o}{r}{x}{v}$ 
    & \makecell[l]{
        Инструкция атомарного инкремента значения       \\
        разделяемой переменной $x$ на значение $v$,     \\
        аннотированная режимом доступа $o$;             \\ 
        прочитанное значение присваивается              \\
        в локальную переменную $r$.                     \\
      } 
    \\ 
    \hline

\end{tabular}
\caption{Список используемых программных примитивов}
\label{fig:primitives}
\end{figure}
\end{center}


Помимо этого, модели памяти, как правило, различают 
несколько видов обращений к разделяемой памяти и позволяют 
программисту аннотировать эти обращения 
\emph{режимом доступа} (\emph{access mode}).
Режимы доступа отличаются гарантиями, 
которые они предоставляют пользователю. 
Выделяют следующие режимы доступа: 
\emph{неатомарный режим} (\emph{non-atomic}), 
\emph{ослабленный режим} (\emph{relaxed} или \emph{opaque} в терминологии \Java),
режим \emph{захвата} (\emph{acquire}), 
режим \emph{освобождения} (\emph{release}), 
их комбинированный режим \emph{захвата-освобождения} (\emph{acquire-release}), 
а также \emph{последовательно согласованный режим} 
(\emph{sequentially consistent} или \emph{volatile} в \Java).
Эти режимы обозначаются как $\na$, $\rlx$, $\acq$, $\rel$, $\acqrel$ и $\sco$ соответственно.
При этом режим $\acq$ применим только к операциям чтения,
а режим $\rel$ --- только к операциям записи.
Режимы обращения упорядочены согласно строгости предоставляемых гарантий, 
как показано на следующей диаграмме. 

 \[\inarr{
 \begin{tikzpicture}[yscale=0.6,xscale=1.3]

   \node (na)     at (-2.6,  0)  {$\na$};
   \node (rlx)    at (-1.3,  0)  {$\rlx$};
   \node (rel)    at (0   ,  1)  {$\rel$};
   \node (acq)    at (0   , -1)  {$\acq$};
   \node (acqrel) at (1.5 ,  0)  {$\acqrel$};
   \node (sc)     at (3   ,  0)  {$\sco$};

   \path[->] (na)   edge[line width=0.742mm] 
                    node[fill=white, anchor=center, pos=0.5] {$\squ$} 
             (rlx);

   \path[->] (rlx) edge[line width=0.742mm] 
                   node[fill=white,anchor=center,pos=0.5] 
                   {\rotatebox[origin=c]{45} {$\squ$}} 
             (rel); 

   \path[->] (rlx) edge[line width=0.742mm] 
                   node[fill=white,anchor=center,pos=0.5] 
                   {\rotatebox[origin=c]{-45} {$\squ$}}
             (acq); 
   
   \path[->] (rel) edge[line width=0.742mm] 
                   node[fill=white,anchor=center,pos=0.5] 
                   {\rotatebox[origin=c]{-45} {$\squ$}}
             (acqrel); 

   \path[->] (acq) edge[line width=0.742mm] 
                   node[fill=white,anchor=center,pos=0.5] 
                   {\rotatebox[origin=c]{45} {$\squ$}}
             (acqrel); 

   \path[->] (acqrel) edge[line width=0.742mm] 
                      node[fill=white, anchor=center, pos=0.5] {$\squ$} 
             (sc);

 \end{tikzpicture}
 }\]


Неатомарные обращения, аннотированные режимом $\na$, 
не предполагается использовать для конкурентного доступа 
к разделяемой переменной из параллельных потоков программы. 
В зависимости от конкретного языка программирования
конкурентные неатомарные обращения либо полностью запрещены
(например, в \Haskell~\cite{Vollmer-al:PPoPP17} и \Rust~\cite{RustBook:19}), 
либо могут приводить к неопределенному поведению
(например, в \CPP~\cite{Batty-al:POPL11}),
либо не предоставляют практический никаких гарантий о порядке,
в котором потоки могут наблюдать эти обращения
(например, в \Java~\cite{Manson-al:POPL05}). 

Для ослабленных обращений, аннотированных режимом $\rlx$, 
как правило, гарантируется только выполнение свойства 
\emph{когерентности}~\cite{Alglave-al:TOPLAS14}.
Это свойство обеспечивает \emph{последовательную согласованность 
по каждой отдельной локации в памяти}.
В частности, из этого следует, что программа, 
состоящая из ослабленных обращений только к одной переменной, 
допускает только последовательно согласованные сценарии исполнения.

Обращения, аннотированные режимами захвата $\acq$ и освобождения $\rel$,
используются для поддержки идиомы передачи сообщений~\cite{Lahav-al:POPL16}.
Поток, выполняющий отправку сообщения, должен аннотировать соответствующую 
операцию записи в разделяемую память режимом доступа $\rel$, 
а поток, ожидающий это сообщение, должен аннотировать 
операцию чтения режимом доступа $\acq$.

Наконец, последовательно согласованные обращения, 
то есть аннотированные режимом $\sco$, 
при правильном использовании гарантируют 
семантику последовательной согласованности%
~\cite{Boehm-Adve:PLDI08, Lahav-al:PLDI17}.

\subsection{Требования к моделям памяти}
\label{sec:models-requirements}

Как уже упоминалось ранее на примере алгоритма Деккера, 
при разработке многопоточных программ необходимо 
учитывать модель памяти, предоставляемую языком программирования.
К моделям памяти предъявляются противоречивые требования. 
С одной стороны, более строгая модель допускает меньше сценариев поведения 
и предоставляет больше гарантий разработчику.
С другой стороны, более слабая модель позволяет 
выполнять большее количество различных оптимизаций, 
что приводит к повышению производительности программы. 
Таким образом, в дизайне модели памяти необходимо 
найти разумный компромисс между этими конфликтующими запросами.

В этом разделе будут более подробно описан 
набор типичных требований и критериев, предъявляемых 
к моделям памяти языков программирования. 

\subsubsection*{Оптимальность и корректность схемы компиляции}

\emph{Схемой компиляции} называется отображение
примитивов языка программирования в инструкции 
языка ассемблера конкретного семейства процессоров.
Будем подразумевать, что и выкоуровневый язык программирования и 
язык ассеблера в данном случае предоставляют одинаковый 
набор программных примитивов, описанных в разделе \ref{sec:models-primitives}.

\emph{Оптимальная} схема компиляции позволяет 
компилировать инструкций обращения к разделяемой памяти 
из языка программирования в инструкции целевого процессора
без необходимости вставки барьеров памяти и 
без усиления режима доступа обращений. 
Другими словами, наличие у языка программирование оптимальной схемы компиляции
позволяет компилировать программы на этом языке 
в эффективный код для целевого процессора.  
Напротив, использование неоптимальных схем компиляции
может приводить к снижению производительности кода
из-за наличия многочисленных барьеров памяти.
Но в то же время вставка дополнительных барьеров памяти
может предотвратить появление слабых сценариев поведения, 
допустимых спецификацией архитектуры процессора. 

\emph{Корректность} схемы компиляции гарантирует,
что множество сценариев поведения, допустимых моделью памяти процессора 
для скомпилированной версии программы,  
является подмножеством сценариев поведения исходной программы, 
допустимой моделью памяти языка программирования. 

Чтобы пояснить введеные выше понятия оптимальности 
и корректности схемы компиляции, рассмотрим пример. 
Программа \ref{ex:sb}, показанная ниже, является 
еще более упрощенным фрагментом алгоритма Деккера. 

\begin{equation*}
\inarrII{
   \writeInst{}{x}{1}   \\
   \readInst{}{r_1}{y}  \\
}{
  \writeInst{}{y}{1}   \\
  \readInst{}{r_2}{x}  \\
}
\tag{SB}\label{ex:sb}
\end{equation*}

Допустим, что язык программирования должен предоставлять
последовательно согласованную модель памяти и должен
поддерживать компиляцию в ассемблерный код процессоров семейства \IntelX.

Рассмотрим схему компиляции, которая 
компилирует инструкции чтения и записи 
разделяемых переменных в инструкцию \texttt{MOV}%
\footnote{В архитектуре \IntelX инструкция \texttt{MOV} 
используется для обычного чтения и записи в память.}. 
Такая схема компиляции является оптимальной, 
так как она не вставляет никакие дополнительные барьры памяти
и не усиливает режимы доступа обращений. 
Однако данная схема не является корректной, так как,
спецификации модели памяти~\IntelX, в частности, 
допускает для программы \ref{ex:sb} сценарий исполнения 
с результатом ${[r_1=0, r_2=0]}$, который 
не является последовательно согласованным.
Данный результат может появиться вследствие 
\emph{буферизации операций записи} --- 
операция записи ${\writeInst{}{x}{1}}$ может быть буферизована
и исполнена процессором после выполнения всех остальных инструкций программы.

С другой стороны, рассмотрим схему компиляции, 
которая вставляет инструкцию \texttt{mfence}
%% \footnote{В архитектуре \IntelX инструкция \texttt{mfence} 
%% является барьером памяти.}. 
после каждой операции записи%
~\cite{Sewell-al:CACM10, Batty-al:POPL11}.
Инструкция \texttt{mfence} является специальным барьером памяти 
в системе команд процессоров \IntelX. 
Выполнение данной инструкции приводит к сбросу буфера записей в основную память. 
Для программы \ref{ex:sb}, скомпилированной описанным выше способом,
результат ${[r_1=0, r_2=0]}$ уже является запрещенным 
моделью памяти процессоров \IntelX. 
Таким образом, альтернативная схема компиляции 
является корректной, но не оптимальной%
\footnote{На практике использование данной схемы 
компиляции может приводить к замедлению 
на 10-30\%~\cite{Marino-al:PLDI11, Liu-al:OOPSLA17}.}. 

К сожалению, модель последовательной согласованности 
не обладает оптимальной и корректной схемой компиляции 
для современных мультипроцессоров семейств 
\IntelX, \ARM и \POWER.
Это является одной из причин ослабления моделей памяти 
высокопроизводительных языков программирования. 

\subsubsection*{Корректность трансформаций кода}

Другим немаловажным требованием, предъявляемым к моделям памяти, 
является корректность трансформаций исходного кода, 
то есть правил переписывания исходного кода, 
применяемых при оптимизации программы компилятором.

\emph{Корректность} трансформации гарантирует,
что множество сценариев поведения программы, 
полученной после применения трансформации, 
является подмножеством допустимых сценариев 
поведения оригинальной программы.

Возвращаясь к программе \ref{ex:sb},
вновь рассмотрим модель последовательной согласованности 
и трансформацию \emph{перестановки независимых инструкций}.
Допустим что данная трансформация применяется к левому потоку 
и переставляет местами операции записи и чтения,
как показано ниже. 

\bigskip

\begin{minipage}{0.42\linewidth}
\begin{equation*}
\inarrII{
   \writeInst{}{x}{1}   \\
   \readInst{}{r_1}{y}  \\
}{
  \writeInst{}{y}{1}   \\
  \readInst{}{r_2}{x}  \\
}
% \tag{SB}\label{ex:sb-src}
\end{equation*}
\end{minipage}\hfill%
\begin{minipage}{0.05\linewidth}
\Large~\\ $\leadsto$
\end{minipage}\hfill%
\begin{minipage}{0.42\linewidth}
\begin{equation*}
\inarrII{
   \readInst{}{r_1}{y}  \\
   \writeInst{}{x}{1}   \\
}{
  \writeInst{}{y}{1}   \\
  \readInst{}{r_2}{x}  \\
}
% \tag{SBtr}\label{ex:sb-tgt}
\end{equation*}
\end{minipage}

\bigskip

Для исходной версии программы (слева), 
результат $[r_1=0, r_2=0]$ \textbf{не является} 
последовательно согласованным, но для трансформированной 
версии программы этот результат уже \textbf{является} 
последовательно согласованным. 
Из чего можно сделать вывод, что перестановка независимых инструкций 
не является корректной трансформацией с точки зрения 
модели последовательной согласованности. 

В теории моделей памяти рассматривается вопрос корректности 
широкого набора базовых трансформаций.
Подробный список этих трансформаций с пояснениями 
может быть найден в работе~\cite{Moiseenko-al:PCS21}.
В данной работе будут обсуждаться только некоторые 
конкретные трансформации, которые будут вводиться по мере необходимости. 

%% Далее от требований о поддержке эффективной компиляции
%% и возможности проводить различные оптимизации, 
%% которые влекут к ослаблению модели памяти,
%% перейдем к требованиям о предоставляемых гарантиях, 
%% которые наоборот влекут к необходимости усиления модели памяти.

\subsubsection*{Гарантии для программ свободных от гонок}

Наиболее базовая гарантия, ожидаемая от модели памяти, 
требует, чтобы для программ, не содержащих \emph{гонок по данным}%
\footnote{Напомним, что гонкой по данным называется пара конкурентных 
обращений к одной и той же разделяемой переменной,
такая что как минимум одно из этих обращений является операцией записи.} 
допускались только последовательно согласованные сценарии исполнения. 
Это свойство также называется \emph{свободой от гонок}
(\emph{data-race freedom}, \DRF)~\cite{Manson-al:POPL05}.

В чуть более формальной формулировке, утвержается, 
что слабая модель памяти $M$ удовлетворяет свойству \DRF
если для любой программы $P$, которая не содержит 
гонок ни в одном последовательно согласованном сценарии исполнения,
модель $M$ допускает только последовательно согласованном сценарии%
\footnote{Свойство свободы от гонок в приведенной выше формулировке
также называется \DRFM{SC} по названию модели памяти 
sequential consistency. 
Можно также рассматривать свойство \DRF от другой 
произвольной модели $\mathsf{M}$, в этом случае это 
свойство называется \DRFM{M}.}.

Таким образом, свойство \DRF позволяет свести рассуждения о поведении 
многопоточной программы в слабой модели к рассуждениям о поведении 
этой же программы в более простой модели последовательной согласованности.
Для этого достаточно показать, что программа не имеет гонок в модели \SC. 

\subsubsection*{Спекулятивное исполнение}

Модели памяти также можно разделить по тому, 
требуют ли они спекулятивного исполнения инструкций или нет.
Рассмотрим еще один пример. 

\bigskip

\begin{equation*}
\inarrII{
  \readInst{}{r_1}{x}     \\
  \writeInst{}{y}{1}      \\
}{
  \readInst{}{r_2}{y}     \\
  \writeInst{}{x}{r_2}    \\
}
\tag{LB}\label{ex:lb-spec}
\end{equation*}

\bigskip

Некоторые модели памяти, в частности, 
модели семейств мультипроцессоров \ARM и \POWER,
допускают для данной программы сценарий поведения, 
ведущий к результату ${[r_1=1, r_2=1]}$. 
Однако данный результат не может быть получен 
путем исполнения инструкции согласно их 
порядку внутри потоков (\emph{in-order execution}).
Для того чтобы получить этот результат, необходимо 
использовать \emph{спекулятивное исполнение}
(\emph{speculative execution})~\cite{Boudol-Petri:ESOP10,Boehm-Demsky:MSPC14}.
Например, данный результат можно получить, если 
буферизировать операцию чтения $\readInst{}{r_1}{x}$ в левом потоке
и исполнить инструкцию записи $\writeInst{}{y}{1}$ вне очереди%
\footnote{Из этого происходит название приведенной программы --- 
буферизация операции чтения (\emph{load buffering}, \ref{ex:lb-spec})}.

Важно отметить, что неограниченное использование 
спекулятивного исполнения может привести к нежелательным последствиям. 
Чтобы продемонстрировать проблему, рассмотрим следующий вариант 
программы с буферизацией операции чтения.

\bigskip

\begin{equation*}
\inarrII{
  \readInst{}{r_1}{x}   \\
  \writeInst{}{y}{r_1}  \\
}{
  \readInst{}{r_2}{y}   \\
  \writeInst{}{x}{r_2}  \\
}
\tag{LB+dep}\label{ex:lb+dep-spec}
\end{equation*}

\bigskip

Сценарий исполнения, в котором сначала происходит 
спекулятивное исполнение операции записи 
в переменную \tcode{y} значения \tcode{1} в левом потоке, 
затем чтение этого значения и его запись в переменную \tcode{x}
в правом потоке, а затем вновь чтение его обратно в левом потоке, 
ведет к циклу причинно-следственных связей 
и неожиданному результату ${[r_1=1, r_2=1]}$.
Прочитанные значения \tcode{1} в примере выше 
также называются \emph{значениями из воздуха} 
(\emph{out of thin-air})~\cite{Batty-al:ESOP15}.
Для того чтобы запретить появление подобных значений 
из воздуха модель памяти должна ограничивать использование
спекулятивного исполнения. 
Более подробно возможные решения этой проблемы 
обсуждаются в разделах \ref{sec:models-classes}, 
\ref{sec:exec-graphs} и \ref{sec:wkmo-eventstruct}.

\subsubsection*{Поддержка методов автоматической верификации программ}

Многопоточные программы являются источниками нетривиальных
трудновоспроизводимых ошибок. 
Тестирование многопоточных программ, как правило, 
оказывается недостаточно эффективным методом поиска 
и предотвращения ошибок из-за недетерминированного 
поведения данного типа программ.
В контексте слабых моделей памяти эта проблема встает еще более остро 
из-за того, что слабые модели памяти допускают еще больше 
возможных сценариев поведения программы. 

По этой причине крайне актуальной становится проблема
разработки средств автоматической верификации программ, 
например, методом проверки моделей~\cite{Baier:2008},
которые учитывали бы слабые сценарии поведения.
Однако для некоторых моделей памяти проблема верификации 
программ оказываются слишком вычислительно сложной, 
из-за огромного пространства возможный состояний программы. 
В частности, процесс верификации может быть существенно затруднен, 
если модель памяти допускает спекулятивное исполнение.
Более подробно эта проблема обсуждается в разделе \ref{sec:models-classes}.  

\subsection{Классы моделей памяти}
\label{sec:models-classes}

Существующие слабые модели памяти языков программирования 
можно разбить на несколько классов в зависимости от того, 
поддерживают ли они оптимальную схему компиляции, 
насколько широкий класс трансформаций они поддерживают 
и насколько сильные гарантии для рассуждения о поведении многопоточных программ
они предоставляют. 

В данном исследовании будем рассматривать четыре класса моделей памяти: 
\emph{модели сохраняющие программный порядок}
(\emph{program order preserving models}); 
\emph{модели сохраняющие синтаксические зависимости} 
(\emph{syntactic dependency preserving models});
\emph{модели сохраняющие семантические зависимости} 
(\emph{semantic dependency preserving models}) и 
\emph{модели допускающие значения из воздуха} 
(\emph{out of thin-air models}).
Чтобы продемонстрировать разницу между этими классами 
будем использовать несколько вариаций программы 
буферизации операции чтений 
\ref{ex:lb-nodep}, \ref{ex:lb-fakedep} и \ref{ex:lb-dep}, 
которые приведены ниже. 

\begin{center}
\begin{minipage}{.32\linewidth}
{\small
\begin{equation}
\inarrII{
  \readInst{}{a}{x} \rfcomment{1} \\
  \writeInst{}{y}{1} \\
}{\readInst{}{b}{y} \rfcomment{1} \\
  \writeInst{}{x}{b}  \\
}%
\tag{LB-nodep}\label{ex:lb-nodep}
\end{equation}
}
\end{minipage}
%
\hfill\vline\hfill
\begin{minipage}{.32\linewidth}
{\small
\begin{equation}
\inarrII{
  \readInst{}{a}{x} \rfcomment{1} \\
  \writeInst{}{y}{1 + a * 0} \\
}{\readInst{}{b}{y} \rfcomment{1} \\
  \writeInst{}{x}{b}  \\
}
\tag{LB-fakedep}\label{ex:lb-fakedep}
\end{equation}
}
\end{minipage}
%
\hfill\vline\hfill
%
\begin{minipage}{.32\linewidth}
{\small
\begin{equation}
\inarrII{
  \readInst{}{a}{x} \nocomment{1} \\
  \writeInst{}{y}{a} \\
}{\readInst{}{b}{y} \nocomment{1} \\
  \writeInst{}{x}{b}  \\
}
\tag{LB-dep}\label{ex:lb-dep}
\end{equation}
}
\end{minipage}
\end{center}


Краткое резюме сравнения различных классов моделей памяти, 
относительно критериев, введеных 
в разделе \ref{sec:models-requirements}, 
приведено в Таблице \cref{table:models-classes}.

\begin{table}[t]
\small

\newcommand{\rotateAngle}{270}

\newcolumntype{Y}{>{\centering\arraybackslash}X}

\def\arraystretch{2}
\setlength\tabcolsep{3pt} %\setlength\tabcolsep{2pt}
\setlength\extrarowheight{6pt}

\begin{center}
\begin{tabularx}{\linewidth}{|*{7}{Y|}} 

%% \hline

%% \rotatebox[origin=c]{\rotateAngle}{
%%   \makecell{Класс моделей памяти}  
%% } &

%% \rotatebox[origin=c]{\rotateAngle}{
%%   \makecell{Оптимальность\\схемы компиляции}
%% } &

%% \rotatebox[origin=c]{\rotateAngle}{
%%   \makecell{Корректность\\трансформаций}
%% } &

%% \rotatebox[origin=c]{\rotateAngle}{
%%   \makecell{Гарантии для программ\\свободных от гонок\\(\DRF)} 
%% } &

%% \rotatebox[origin=c]{\rotateAngle}{
%%   \makecell{Спекулятивное\\исполнение}
%% } &

%% \rotatebox[origin=c]{\rotateAngle}{
%%   \makecell{Поддержка методов\\автоматической верификации}
%% }

%% \\

\hline 

Сlass  & 
Compil. &
Transf.~ &
\DRF &
In-Order &
no-OOTA &
Auto. Ver.

\\
\hline

\makecell{Program\\Order\\Preserving} & 
 \badcell & \badcell & \okcell & \okcell & \okcell & \okcell

\\
\hline

\makecell{Syntax.\\Dep.\\Preserving} &
 \okcell & \badcell & \okcell & \badcell & \okcell & \okcell

\\
\hline

\makecell{Semantic\\Dep.\\Preserving} & 
 \okcell & \okcell & \okcell & \badcell & \okcell & \textbf{?}

\\
\hline

\makecell{Out of\\Thin-Air} & 
 \okcell & \okcell & \badcell & \badcell & \badcell & \badcell

\\
\hline


\end{tabularx}
\end{center}

\captionsetup{justification=centering}
\caption{Классы моделей памяти и их свойства}
\label{table:models-classes}
\end{table}



\subsubsection*{Модели памяти сохраняющие программный порядок}

В рамках моделей, сохраняющих программный порядок, 
эффекты от выполнения обращений к разделяемой памяти
наблюдаются потоками согласно их \emph{программному порядку}
то есть в линейном порядке в котором соответствующие 
инструкции обращения к памяти расположены в потоке%
\footnote{Заметим, что при этом данный класс моделей 
все же позволяет операциям чтения наблюдать ``устаревшие'' значение, 
в отличие от модели последовательной согласованности.}. 
Другими словами, в рамках данных моделей запрещено 
спекулятивное исполнение инструкций. 
Модели памяти, принадлежащие к этому классу 
запрещают сценарий поведения с результатом ${[r_1=1,r_2=1]}$
для всех трех програм \ref{ex:lb-nodep}, \ref{ex:lb-fakedep} и \ref{ex:lb-dep}.

Модели памяти, сохраняющие программный порядок, 
предоставляют довольно много гарантий о поведении программ. 
В частности, они обладают свойством свободы от гонок \DRF, 
запрещают спекулятивное исполнение и, следовательно, 
появление значений из воздуха~\cite{Lahav-al:PLDI17}. 
Также для данного класса моделей разработаны эффективные 
алгоритмы автоматической верификации 
методом проверки моделей~\cite{Kokologiannakis-al:POPL2017, Kokologiannakis:PLDI2019}.
С другой стороны, данный класс моделей не поддерживает
оптимальную схему компиляции в модели мультипроцессоров
\ARM и \POWER, а также не поддерживает трансформацию 
перестановки операции чтения после независимой
от нее операции записи (\emph{load/store reordering}). 

К данному классу относится модель~\RCMM~\cite{Lahav-al:PLDI17}, 
покрывающая подмножество сценариев поведения, допустимых моделью памяти языка \CLANG,
модель \TSO процессоров семейства \Intel~\cite{Sewell-al:CACM10},
модель причинной согласованности (causal consistency)~\cite{Lahav-Boker:PLDI2020},
а также, например, модель памяти языка \OCaml~\cite{Dolan-al:PLDI18}.

\subsubsection*{Модели памяти сохраняющие синтаксические зависимости}

Модели памяти, сохраняющих синтаксические зависимости, 
ослабляют требование линейности программного порядка 
и вводят понятие частичного \emph{сохраняемого программного порядка}. 
Операции обращения к раздеяемой памяти из одного и того же потока
находятся в отношении сохраняемого программного порядка 
если между соответствующими им инструкциями есть 
\emph{синтаксические зависимости}, например, 
\emph{зависимость по данным} или \emph{по управлению}. 

Например, в программе \ref{ex:lb} между инструкциями в 
левом потоке нет синтаксической зависимости, 
поэтому эти инструкции могут быть выполнены в произвольном порядке. 
Следовательно, модель памяти, сохраняющая синтаксические зависимости, 
допускает сценарий поведения с результатом ${[r_1=1,r_2=1]}$ для этой программы. 
В то же время этот результат запрещен для программ 
\ref{prog:lb-dep} and \ref{prog:lb-fakedep}, 
поскольку в этих программах существует зависимость 
между инструкциями в обоих потоках. 

Модели данного класса поддерживают оптимальные схемы компиляции
в модели современных мультипроцессоров. 
Также, в отличие от моделей сохраняющих программный порядок, 
данный класс моделей допускает переупорядочивание 
инструкции чтения после независящей от нее инструкции записи. 
Но при этом, данный класс запрещает множество других трансформаций
которые могут удалять синтаксические зависимости между инструкциями. 
Примером такой трансформации является \emph{свертка констант}~\cite{Muchnick:ACDI97}.
Можно видеть, например, что в случае программы 
свертка констант может преобразовать инструкцию 
$\writeInst{}{y}{1 + a * 0}$ в инструкцию $\writeInst{}{y}{1}$, 
удалив зависимость от предшествующей инструкции чтения $\readInst{}{a}{x}$.
После применения этой трансформации сценарий поведения с результатом ${[r_1=1,r_2=1]}$
становится допустим моделью памяти.  

Модели памяти, сохраняющие синтаксические зависимости, 
предоставляют более слабые гарантии, по сравнению с 
моделями, сохраняющими программный порядок. 
Эти модели все еще обладают свойством свободы от гонок (\DRF), 
но в отличие от моделей предыдущего класса 
допускают спекулятивное исполнение 
синтаксический независимых инструкций. 
Несмотря на это, для данного класса моделей все же 
существуют достаточно эффективные алгоритмы автоматической верификации 
методом проверки моделей%
~\cite{Abdulla-al:CAV2016,Pulte-al:PLDI2019,Kokologiannakis-Vafeiadis:ASPLOS2020}.

Модели сохраняющие синтаксические зависимости
практический не используются как модели памяти для
языков программирования, в частности, потому что 
модели этого класса запрещают применение широкого 
класса крайне важных трансформаций (например, свертку констант).  
Вместе с тем, большинство моделей современных мултьтипроцессоров, 
например, \ARM~\cite{Pulte-al:POPL18} и \POWER~\cite{Sarkar-al:PLDI11}, 
попадают именно в этот класс. 
Также к данному классу относится 
модель памяти ядра \Linux~\cite{Alglave-al:ASPLOS18}

\subsubsection*{Модели памяти сохраняющие семантические зависимости}

Модели памяти сохраняющие семантические зависимости
отслеживают вместо синтаксических зависимостей между операциями
\emph{семантические зависимости}.
Например, в случае программы \ref{ex:lb-fakedep} можно сказать,
что хотя между инструкциями $\readInst{}{a}{x}$ и $\writeInst{}{y}{1 + a * 0}$
есть синтаксическая зависимость, семантический они независимы,
так как на самом деле записываемое значение $1 + a * 0$
не зависит от значения $a$, прочитанного первой инструкцией.
Таким образом, модели данного класса допускают
сценарий исполнения с результатом ${[r_1=1,r_2=1]}$ для программ
\ref{ex:lb-nodep} и \ref{ex:lb-fakedep}, но не для \ref{ex:lb-dep}.

Модели памяти сохраняющие семантические зависимости
поддерживают оптимальные схемы компиляции и
применение широкого класса различных трансформаций программ.
В то же время данные модели предоставляют гарантию \DRF,
но также допускают спекулятивное исполнение.
Модели данного класса использую различные
концепутально сложные формализмы, чтобы дать строгое определение
понятию семантических зависимостей между операциями.
Эта сложность, в частности, приводит к тому,
что автоматическая верификация программ в этих моделях
существенно затруднена, а вопрос построения
инструментов для верификации программ не изучен.

К данному классу относятся различные модели,
предложенные в качестве моделей памяти для
языков \CPP и \Java, в частности, модели
\Prm~\cite{Kang-al:POPL17},
\Wkm~\cite{Chakraborty-Vafeiadis:POPL19}, 
\PwP~\cite{Jagadeesan-al:OOPSLA2020}
и другие~\cite{Jeffrey-Riely:LICS16,PichonPharabod-Sewell:POPL16,Paviotti-al:ESOP20}.

\subsubsection*{Модели памяти допускающие значения из воздуха}

Также упомянем класс моделей памяти допускающих значения из воздуха.
Такие модели допускают сценарий исполнения с результатом ${[r_1=1,r_2=1]}$
для всех трех программ \ref{ex:lb-nodep}, \ref{ex:lb-fakedep} и \ref{ex:lb-dep}.

Модели данного класса предоставляют оптимальные схемы компиляции и
допускают применение широкого класса различных трансформаций программ.
Однако ценой этого является возможность появления значений из воздуха.
Наличие значений из воздуха препятствует как неформальному
так строго формальному рассуждению о поведении программ,
приводит к нарушению гарантии \DRF и невозможности
построения каких-либо инструментов верификации программ%
~\cite{Boehm-Demsky:MSPC14, Batty-al:ESOP15}. 

Все эти недостатки данного класса моделей
привели к консенсусу в исследовательском сообществе,
что модели, допускающие значения из воздуха,
не подходят на роль моделей памяти 
для языков программирования.
Тем не менее, с точки зрения истории развития
теории моделей памяти, стоит отметить, например,
что первоначальная версия модели памяти для языков \CPP
допускала значения из воздуха~\cite{Batty-al:POPL11}.

\subsection{Сравнение моделей памяти и открытые проблемы}
\label{sec:models-summary}

Подведение итога обзора моделей памяти.
Фрагмент сравнительной таблицы из обзорной статьи~\cite{Moiseenko-al:PCS21}.
Обозначение research gaps, которые закрывает данная диссертация, а именно:

\begin{itemize}
  \item классическая теория структур событий для моделей памяти
    сохраняющих программный порядок;
  \item корректность компиляции для модели \Wkm;
  \item автоматическая верификация многопоточных программ
    (model checking) в модели \WkmS.
\end{itemize}

\section{Формальная семантика параллельных программ и моделей памяти}

В этом разделе приводятся определения различных формализмов,
используемых для задания семантики многопоточных программ и моделей памяти.
В разделе \cref{sec:lts} дано определение \emph{систем помеченных переходов}
и \emph{операционных семантик с чередованием} 
(\emph{interleaving operational semanitcs}).
В разделе \cref{sec:pomsets-eventstruct} приводится альтернативный способ
задания семантики многопоточных программ без чередования 
(\emph{non-interleaving}) 
с помощью семантических доменов \emph{истинной конкурентности} 
(\emph{true concurrency}), а именно 
\emph{языков частично упорядоченных мультимножеств} и \emph{структур событий}.
В разделе \cref{sec:exec-graphs} вводится понятие графов сценариев исполнения
и аксиоматических моделей памяти, а также приводится краткое сравнение
графов сценариев исполнения и частично упорядоченных мультимножеств.
Наконец, в разделе \cref{sec:wkmo-eventstruct} вводится
специальный тип структур событий, используемых в модели \Wkm.

\subsection{Системы помеченных переходов}
\label{sec:lts}

Системы помеченных переходов являются традиционным 
способом задания операционной семантики. 
Системы помеченных переходов это (потенциально бесконечный) граф, 
вершины в котором представляют внутрение состояния системы, а
ребра соответствуют выполнению шага вычислений. 
Метка ребра задает видимый эффект выполения данного шага вычислений.
%% Множество трасс автомата задает его ``последовательную'' спецификацию, 
%% а список меток, индуцируемый трассой, определяет
%% наблюдаемое поведение автомата. 

\begin{definition}
  \label{def:lts}
  \emph{Система помеченных переходов} --- это тройка
    $\LTS \defeq \tup{\State, \Label, \TrRel}$, где 
  \begin{itemize}
    \item $\State$ --- множество состояний;
    \item $\Label$ --- множество меток, также называемое \emph{алфавитом};
    \item $R \subseteq L \times S \times S$ --- помеченное отношение перехода.
  \end{itemize}
  Для обозначения наличия перехода между состояниями используется следующая нотация:
    \[
    \begin{array}{lcr@{\hspace{3em}}lcr}
    \ltr[R]{\ell}{s}{s'} & \defeq & \step{\ell}{s}{s'} \in \TrRel                     &
    \tr[R]{s}{s'}        & \defeq & \exists \ell \ldotp \step{\ell}{s}{s'} \in \TrRel \\
    \end{array}
    \]
  \emph{Трассой} помеченной системой переходов называется чередующаяся последовательность  
  состояний $s_0, s_1, \ldots, s_n \in \Label$ 
  и меток $\ell_1, \ldots, \ell_n \in L$, 
  такая что выполняется условие
  $$s_0 \xrightarrow{\ell_1} s_1 \xrightarrow{\ell_2} s_2 \xrightarrow{\ell_3} \ldots \xrightarrow{\ell_n} s_n$$
  Язык, принимаемый системой переходов в начальном состоянии $s_0$ 
  это множество последовательностей слов над алфавитом $\Label$, 
  таких что для каждого слова существует соответствующая 
  трасса, начинающася в $s_0$:
  $$ \langof{\LTS, s_0} \defeq \set{ 
       \ell_1 \ldots \ell_n ~|~ \exists s_1, \ldots, s_n \ldotp 
       s_0 \xrightarrow{\ell_1} s_1 \xrightarrow{\ell_2} \ldots \xrightarrow{\ell_n} s_n
     } 
  $$
  
\end{definition}

\begin{figure}[t]
  \centering

    \begin{subfigure}[t]{0.48\textwidth}
    \centering
    \begin{tikzpicture}[xscale=1.5,yscale=1.5]

      \tikzset{
        smallstate/.style={state,
          inner sep=3pt,
          outer sep=3pt,
          minimum size=0pt,
        },
      }

      \node[smallstate] (s0) at (1,3) {$s_0$};
      \node[smallstate] (s1) at (1,2) {$s_1$};
      \node[smallstate] (s2) at (0,1) {$s_2$};
      \node[smallstate] (s3) at (2,1) {$s_3$};
      \node[smallstate] (s4) at (1,0) {$s_4$};

      \draw[->] (s0) edge[right,pos=0.5] node {$a$} (s1);

      \draw[->] (s1) edge[bend right,above,pos=0.5] node {$c$} (s2);
      \draw[->] (s1) edge[bend left ,below,pos=0.5] node {$d$} (s2);

      \draw[->] (s1) edge[right,pos=0.5] node {$b$} (s3);

      \draw[->] (s2) edge[left,pos=0.5]  node {$b$} (s4);

      \draw[->] (s3) edge[bend right,above,pos=0.5] node {$c$} (s4);
      \draw[->] (s3) edge[bend left ,below,pos=0.5] node {$d$} (s4);

    \end{tikzpicture}

    \caption{Система помеченных переходов}
    \label{fig:lts-ex}
    \end{subfigure}
    \hfill
    %
    \begin{subfigure}[t]{0.48\textwidth}
    \centering
    \begin{tikzpicture}[xscale=1.5,yscale=1.5]
      \node (a1) at (0, 2) {$a$};
      \node (b1) at (0, 1) {$b$};
      \node (c1) at (0, 0) {$c$};

      \draw[po] (a1) edge (b1);
      \draw[po] (b1) edge (c1);

      \node (a2) at (1, 2) {$a$};
      \node (c2) at (1, 1) {$c$};
      \node (b2) at (1, 0) {$b$};

      \draw[po] (a2) edge (c2);
      \draw[po] (c2) edge (b2);

      \node (a3) at (2, 2) {$a$};
      \node (b3) at (2, 1) {$b$};
      \node (d3) at (2, 0) {$d$};

      \draw[po] (a3) edge (b3);
      \draw[po] (b3) edge (d3);

      \node (a4) at (3, 2) {$a$};
      \node (d4) at (3, 1) {$d$};
      \node (b4) at (3, 0) {$b$};

      \draw[po] (a4) edge (d4);
      \draw[po] (d4) edge (b4);
    \end{tikzpicture}

    \caption{Язык принимаемый системой переходов в начальном состоянии $s_0$.}
    \label{fig:lang-ex}
    \end{subfigure}

  \label{fig:lts-lang-ex}
  \caption{
    Пример системы помеченных переходов и принимаемого ей языка
  }
%% как обычного языка,
%%     языка помеченных частично упорядоченных мультимножеств
    %% и структуры событий. 
\end{figure}


На \cref{fig:lts-ex} показан пример системы помеченных переходов, 
а на \cref{fig:lang-ex} пример языка, 
принимаемаемого этой системой в состоянии $s_0$.

В рамках так называемой операционной семантики с чередованием
(interleaving semantics) поведение многопоточной программы
определяется как поочередное исполнение атомарных действий параллельных потоков.

\begin{definition}
  \label{def:lts-par}
  Параллельной композицией двух систем переходов $\LTS_1$ и $\LTS_2$
  будем называть систему переходов
  $\parlts{\LTS_1}{\LTS_2} \defeq \tup{\State_1 \times \State_2, \Label, \TrRel_{\parSymb}}$
  где $\TrRel_{\parSymb}$ определяется следующим образом:
  \begin{itemize}
    \item $\ltr[\TrRel_1]{\ell}{s_1}{s'_1}$ влечет
          $\ltr[\TrRel_{\parSymb}]{\ell}{\tup{s_1, s_2}}{\tup{s'_1, s_2}}$, и
    \item $\ltr[\TrRel_2]{\ell}{s_2}{s'_2}$ влечет
          $\ltr[\TrRel_{\parSymb}]{\ell}{\tup{s_1, s_2}}{\tup{s_1, s'_2}}$.
  \end{itemize}
\end{definition}

Если при этом потоки имеют доступ к общему ресурсу 
(например, разделяемой памяти), то в таком случае можно
семантику ресурса также задать с помощью системы переходов
а затем рассмотреть произведение параллельной композиции потоков и ресурса.  

\begin{definition}
  \label{def:lts-par}
  Произведением двух систем переходов $\LTS_1$ и $\LTS_2$
  будем называть систему переходов
  $\prodlts{\LTS_1}{\LTS_2} \defeq \tup{\State_1 \times \State_2, \Label, \TrRel_{\prodSymb}}$
  где $\TrRel_{\prodSymb}$ определяется следующим образом:
  \begin{itemize}
    \item $\ltr[\TrRel_{\prodSymb}]{\ell}{\tup{s_1, s_2}}{\tup{s'_1, s'_2}}$ 
      тогда и только тогда, когда 
      $\ltr[\TrRel_1]{\ell}{s_1}{s'_1}$ и $\ltr[\TrRel_2]{\ell}{s_2}{s'_2}$.
  \end{itemize}
\end{definition}

Таким образом, если система помеченных переходов $\LTS_{\thrdSymb}$ 
задает семантику потоков, а система $\LTS_{\resSymb}$ --- 
семантику разделяемого ресурса, тогда 
$(\LTS_{\thrdSymb} \parSymb \dots \parSymb \LTS_{\thrdSymb}) \prodSymb \LTS_{\resSymb}$
задает семантику всей системы, состоящей из $n$ потоков и разделяемого ресурса.

\subsection{Языки помсетов и простые структуры событий}
\label{sec:pomsets-eventstruct}

Операционные семантики с чередованием представляют 
простой и интуитивно понятный подход для моделирования
многопоточных программ. Однако его недостаток заключается в том, 
что с ростом программы экспоненциально растет количество трасс, 
допустимых операционной семантикой. 

В попытке преодолеть эту проблему, исследователями 
были предложены различные альтернативные 
подходы к заданию семантики многопоточных программ, 
которые позволяют более компактно представить пространство 
возможных сценариев поведения таких программ. 
Данный класс семантик принято называть 
\emph{семантиками без чередования} (\emph{non-interleaving semantics})
или также \emph{истинно конкурентными семантиками}
(\emph{true concurrent semantics})~\cite{Nielsen:REX93}.
Из данного класса семантик в рамках 
данной диссертации интерес представляют 
\emph{языки частично упорядоченных мультимножеств}
(кратко --- языки помсетов)~\cite{Pratt:CONCUR84,Gischer:TCS88}, 
и \emph{структуры событий}~\cite{Winskel:86}.

Языки помеченных частично упорядоченных множеств является
обощением понятия обычных ``последовательных'' языков, 
то есть множества слов данного алфавита. 
Обобщение заключается в переходе от линейного порядка 
на символах алфавита в рамках слова к частичному порядку.
Частично упорядоченное множество соответствует одному
сценарию исполнения многопоточной программы.
Элементы этого множества представляют
атомарные шаги вычисления и называются \emph{событиями}.
Каждому событию ставится в соответствие семантическая \emph{метка} ---
символ заданного алфавита.
Если событие $e_1$ упорядочено перед ~$e_2$, $e_1 \ca e_2$, 
тогда считается что событие~$e_2$ в
сценарии исполнения программы зависит от события~$e_1$.
Если не выполняется ни $e_1 \ca e_2$ ни $e_2 \ca e_1$ 
тогда события $e_1$ и $e_2$ считаются параллельными, 
$e_1 \co e_2$. 

\begin{definition}
  \label{def:lposet}
  \emph{Помеченное частично упорядоченное множество} над множеством меток $\Label$, 
  это тройка $\tup{\Event, \lab, \ca}$, где 
  \begin{itemize}
    \item $\Event$ это множество \emph{событий};
    \item $\lab : \Event \fun \Label$ \emph{функция разметки событий};
    \item $\ca \subseteq \Event \times \Event$ это частичный порядок 
      \emph{причинно-следственной связи} между событиями. 
  \end{itemize}
  Множество всех помеченных частично упорядоченных множеств над алфавитом $\Label$
  будем обозначать как $\lPoset[\Label]$. 
\end{definition}

Отметим, что при работе c помеченными частично упорядоченными множествами
конкретные идентификаторы событий как правило неважны,
важна лишь их разметка и отношение причинно-следственной связи между ними.
Таким образом, с помеченными частично упорядоченными множествами 
работают по модулю переименования событий, то есть с точностью до изоморфизма.

\begin{definition}
  \label{def:lposet-morph}
  Рассмотрим $p, q \in \lPoset[\Label]$. Функция $f : E_p \fun E_q$ называется:
  \begin{itemize}
    \item \emph{сохраняющей метки} если ${\lab_q(f(e)) = \lab_p(e)}$;
    \item \emph{сохраняющей порядок} если ${e_1 \ca_p e_2}$ влечет ${f(e_1) \ca_q f(e_2)}$;
    \item \emph{вкладывающей порядок} если ${e_1 \ca_p e_2}$ тогда и только тогда, когда ${f(e_1) \ca_q f(e_2)}$.
  \end{itemize}
  \emph{Гомоморфизмом} частично упорядоченных помеченных множеств называется
  функция, сохраняющая метки и порядок. Если более того эта фунцкия
  биективна и вкладываюет порядок, то она называется изоморфизмом.
  Запись $p \iso q$ означает что $p$ и $q$ изоморфны,
  то есть существует изоморфная функция между носителями $p$ и $q$.
\end{definition}

\begin{definition}
  \label{def:pomset}
  Помеченные частично упорядоченные мультимножества, 
  или, кратко, \emph{помсеты} (от англ. \emph{partially ordered multiset, pomset}), 
  это классы помеченных частично упорядоченных множеств по модулю изоморфизма: 
  $${\Pom[\Label] \defeq \lPoset[\Label] / {\iso}}.$$ 
  Язык помсетов --- это множество помсетов: 
  $${\Pomlang[\Label] \defeq \pwset{\Pom[\Label]}}.$$ 
\end{definition}

\begin{figure}[t]
  \centering
  \begin{subfigure}[t]{0.48\textwidth}
    \centering
    \begin{tikzpicture}[xscale=1.5,yscale=1.5]
      \node (a1) at (1, 1) {$a$};
      \node (b1) at (0, 0) {$b$};
      \node (c1) at (2, 0) {$c$};

      \draw[po] (a1) edge (b1);
      \draw[po] (a1) edge (c1);

      \node (a2) at (4, 1) {$a$};
      \node (b2) at (3, 0) {$b$};
      \node (d2) at (5, 0) {$d$};

      \draw[po] (a2) edge (b2);
      \draw[po] (a2) edge (d2);
    \end{tikzpicture}
    \caption{Язык помсетов}
    \label{fig:pom-ex}
  \end{subfigure}
  \hfill
  \begin{subfigure}[t]{0.48\textwidth}
    \centering
    \begin{tikzpicture}[xscale=1.5,yscale=1.5]
      \node (a) at (1, 1) {$a$};
      \node (b) at (0, 0) {$b$};
      \node (c) at (1, 0) {$c$};
      \node (d) at (2, 0) {$d$};

      \draw[po] (a) edge (b);
      \draw[po] (a) edge (c);
      \draw[po] (a) edge (d);
      \draw[cf] (c) -> (d);
    \end{tikzpicture}
    \caption{Простая структура событий}
    \label{fig:es-ex}
  \end{subfigure}

  \label{fig:pom-es-ex}
  \caption{
    Пример кодирования языка системы переходов 
    как языка помеченных частично упорядоченных мультимножеств
    и как простой структуры событий. 
  }
\end{figure}


На \cref{fig:pom-ex} можно видеть пример языка помсетов. 
Этот язык помсетов кодирует обычный язык, показанный \cref{fig:lang-ex}, 
так как каждое слово из обычного языка является дополнением некоторого 
частичного упорядоченного множества из языка помcетов
до линейно упорядоченного множества. 
Формально, связь языка помсетов и обычного языка можно установить 
с помощью понятия \emph{линеаризации помсета}.

\begin{definition}
  \label{def:pomset-subs}
  Помсет $p$ \emph{поглощается} $q$, $p \subs q$, 
  если существует биективный гомоморфизм из $q$ в $p$.
  В таком случае также говорят, что $p$ более упорядочено чем $q$.
\end{definition}

\begin{definition}
  \label{def:pomset-lin}
  Помсет $p$ является линеаризацией $q$, 
  что обозначается как $p \in \Lin{q}$,
  если $p$ является линейно упорядоченным и 
  поглощается $q$, $p \subs q$.
  Линеаризация языка помсетов $P$ определяется 
  как объединение линеаризации всех входящих в язык помсетов:
  $$ \Lin{P} \defeq \bigcup_{p \in P}\Lin{p} $$
  Наконец, можно сказать что язык помсетов $P \in \Pomlang[\Label]$
  соответствует обычному языку $L \in \Lang{\Label}$, если $\Lin{P} = L$.
\end{definition}

Множество помсетов можно объединить в одну \emph{структуру событий},
и таким образом представить язык помсетов как одно частично упорядоченное множество.
Существует множество видов структур событий~\cite{}, 
в контексте данной работы будем рассматривать класс \emph{простых структур событий}.
Везде далее под термином \emph{структура событий} будем подразумевать 
именно простую структуру событий, если иное не сказано явно. 

По сравнению с помеченным частично упорядоченным множеством, 
простые структуры событий позволяют дополнительно выразить тот факт, 
что два события $e_1$ и $e_2$ находятся в конфликте.
Это означает, что эти два события не могут одновременно 
принадлежать одному сценарию исполнения программы. 

\begin{definition}
  \label{def:lposet-dwfin}
  Будем говорить, что помеченное частично упорядоченное множество 
  $p = \tup{\Event, \lab, \ca}$ является \emph{префикс конечным} 
  если каждое событие имеет конечное число предшественников, 
  то есть для любого $e \in \Event$ множество 
  $\dwset{e} \defeq \set{e' ~|~ e' \ca e}$ конечно.
\end{definition}

\begin{definition}
  \label{def:prime-es}
  \emph{Простая структура событий с бинарным конфликтом} над множеством меток $\Label$ 
  это кортеж $\tup{\Event, \lab, \ca, \cf}$, где 
  $\tup{\Event, \lab, \ca}$ это префикс-конечное помеченное 
  частично упорядоченное множество, 
  а $\cf \suq \Event \times \Event$ --- это \emph{бинарное отношение конфликта}, 
  которое является иррефлексивным, симметричным и 
  удовлетворяет свойству \emph{наследственности}:
  $$ e_1 \cf e_2 ~\text{и}~ e_2 \ca e_3 ~\text{влечет}~ e_1 \cf e_3.$$
\end{definition}

Отметим, что зачастую язык помсетов может иметь более сложную структуру 
конфликтности между событиями, которая не может быть сведена 
к бинарному конфликту между парой событий. 
В таком случае рассматривают простые структуры событий более общего вида. 

\begin{definition}
  \label{def:prime-cons-es}
  \emph{Простая структура событий с предикатом консистентости} над множеством меток $\Label$ 
  это кортеж $\tup{\Event, \lab, \ca, \Cons}$, где 
  $\tup{\Event, \lab, \ca}$ это префикс-конечное помеченное 
  частично упорядоченное множество, 
  а $\Cons \suq \pwfset{\Event}$ --- это \emph{предикат консистентности}, 
  который должен удовлетворять следующим условиям:
  \begin{enumerate}
    \item \label{ax:prime-cons-emp}
      $\emptyset \in \Cons$,
    \item \label{ax:prime-cons-subs}
      $X \subseteq Y$ и $Y \in \Cons$ влечет $X \in \Cons$,
    \item \label{ax:prime-cons-ca}
      $e_1 \ca e_2$ и $\set{e_2} \cup X \in \Cons$ 
      влечет $\set{e_1} \cup X \in \Cons$.
  \end{enumerate}
\end{definition}

Можно видеть, что простые структуры событий с бинарным конфликтом
являются частным случаем простых структур событий 
с предикатом консистентности. 
Действительно, для простой структуры событий с бинарным конфликтом
можно определить предикат консистентности следующим образом:
$$X \in \Cons \iff \forall e_1~e_2 \in X \ldotp \neg e_1 \cf e_2.$$

Наконец, формально определим язык помсетов, порождаемый структурой событий. 

\begin{definition}
  \label{def:es-cfg}
  Пусть $S = \tup{\Event, \lab, \ca, \Cons}$ простая структура событий 
  с предикатом консистентности. Тогда подмножество событий 
  $X \suq \Event$ называется \emph{конфигурацией} структуры $S$ 
  если оно является префикс-замкнутым, а все его конечные подмножества 
  являются консистентными, то есть 
  \begin{itemize}
    \item $\dwset{X} \defeq {e' ~|~ \exists e \in X \ldotp~ e' \ca e } \suq X$, 
    \item $Y \finsubseteq X$ implies $Y \in \Cons$.
  \end{itemize}
  Будем обозначать как $\Cfg{S}$ множество всех конфигураций $S$.
\end{definition}

\begin{definition}
  \label{def:es-pomlang}
  Язык помсетов, порождаемый структурой событий $S = \tup{\Event, \lab, \ca, \Cons}$, 
  определяется следующим образом:
  $$ \pomlang{S} \defeq \set{p ~|~ \exists X \in \Cfg{S} \ldotp p = S\rst{X} }$$
  где $S\rst{X} \defeq {X, \lab\rst{X}, \ca\rst{X}}$ это сужение 
  структуры событий $S$ на консистентное подмножество событий $X$.
\end{definition}

\subsection{Графы сценариев исполнения}
\label{sec:exec-graphs}

Далее перейдем к описанию формализмов, используемых для
формального определения моделей памяти.

Напомним, что под моделью памяти понимается
семантика многопоточной системы, оперирующей с разделяемой памятью.
Поэтому, как уже упоминалось в \cref{sec:lts},
модель памяти можно определить в терминах операционной семантики.
В таком случае система переходов
$(\LTS_{\thrdSymb} \parSymb \dots \parSymb \LTS_{\thrdSymb}) \prodSymb \LTS_{\memSymb}$
будет описывать многопоточную систему, состояющую из $n$ потоков
и разделямой памяти, где $\LTS_{\thrdSymb}$ --- это система переходов,
описывающая поведение потоков, а $\LTS_{\memSymb}$ --- система переходов,
описывающая поведение разделяемой памяти. 

Как уже отмечалось, в контексте моделирования
многопоточных систем одним из недостатков подхода,
основанного на операционной семантике, 
является экспоненциально рост количество трасс системы.
В контексте слабых моделей памяти у данного подхода
также есть и другой недостаток.
Проблема заключается в том, что для кодирования каждой
отдельной модели памяти необходимо разработать
собственное представление этой модели памяти 
в терминах системы переходов $\LTS_{\memSymb}$.
При этом такая система может быть устроено
достаточно сложным образом и требовать моделирования
множества различных структур данных, например,
буферов операций, очередей сообщений,
многоуровневых кэшей и так далее.  

Поэтому для спецификации моделей памяти
зачастую используют альтернативный \emph{аксиоматический стиль}.
В аксиоматическом стиле задано большинство моделей
современных мультипроцессоров, например,
\Intel~\cite{Sewell-al:CACM10}, 
\POWER~\cite{Sarkar-al:PLDI11,Alglave-al:TOPLAS14}),
\ARM~\cite{Pulte-al:POPL18,Alglave-al:TOPLAS14}),
и некоторых языков программирования,
например, \OCaml~\cite{Dolan-al:PLDI18}, \JS~\cite{Watt-al:PLDI2020}.

Модель памяти в аксиоматическом стиле
определяется как множество консистентных 
\emph{графов сценариев исполнения} (\emph{execution graphs}).
В этом графе вершинами являются атомарные события,
а ребра формируют различные отношения между этими событиями.
Графы сценариев исполнения похожи на помсеты,
главное отличие между ними заключается в том, что 
помсет состоит из единственного
отношения причинно-следственной связи, 
а граф сценариев исполнения состоит из
нескольких отношений, наделенных различной семантикой.
Например, отношение \emph{программного порядка} (\emph{program order}) $\lPO$ 
задает порядок, в котором выполняются события в каждом потоке,
а отношение \emph{читает-из} (\emph{reads-from}) $\lRF$, 
для каждого события записи указывает 
какие события чтения выполняют операцию чтения из него. 
На Рис.\cref{fig:LB-nodep-execs} показаны примеры графов сценариев исполнения 
соответствующих программе \ref{ex:LB-nodep}.

{
\newcommand{\XScale}{1}
\newcommand{\YScale}{0.7}

\begin{figure}[t]
  \begin{subfigure}[b]{.24\textwidth}\centering
  \begin{tikzpicture}[xscale=\XScale,yscale=\YScale]

  \node at (-1,1.8) {$\circledb{A}$};

  \node (init) at (1,  1.5) {$\Init$};

  \node (i11) at ( 0,  0) {$\rlab{}{x}{0}{}$};
  \node (i12) at ( 0, -2) {$\wlab{}{y}{1}{}$};

  \node (i21) at ( 2,  0) {$\rlab{}{y}{0}{}$};
  \node (i22) at ( 2, -2) {$\wlab{}{x}{0}{}$};

  \draw[po] (i11) edge node[right] {\small$\lPO$} (i12);
  \draw[po] (i21) edge node[left ] {\small$\lPO$} (i22);
  %% \draw[ppo,bend left=10] (i21) edge node[right] {\small$\lPPO$} (i22);

  \draw[rf,bend right=60] (init) edge node[above,pos=0.5] {\small$\lRF$} (i11);
  \draw[rf,bend left =60] (init) edge node[above,pos=0.5] {\small$\lRF$} (i21);

  \draw[po] (init) edge node[left]  {\small$\lPO$} (i11);
  \draw[po] (init) edge node[right] {\small$\lPO$} (i21);
  \end{tikzpicture}
  %% \caption{$\Glb$: Execution graph of \ref{ex:LB}.}
  %% \label{fig:lbWeak1}
  \end{subfigure}\hfill
  %
  \begin{subfigure}[b]{.24\textwidth}\centering
  \begin{tikzpicture}[xscale=\XScale,yscale=\YScale]

  \node at (0,1.8) {$\circledb{B}$};

  \node (init) at (1,  1.5) {$\Init$};

  \node (i11) at ( 0,  0) {$\rlab{}{x}{0}{}$};
  \node (i12) at ( 0, -2) {$\wlab{}{y}{1}{}$};

  \node (i21) at ( 2,  0) {$\rlab{}{y}{0}{}$};
  \node (i22) at ( 2, -2) {$\wlab{}{x}{0}{}$};

  \draw[po] (i11) edge node[right] {\small$\lPO$} (i12);
  \draw[po] (i21) edge node[left ] {\small$\lPO$} (i22);
  %% \draw[ppo,bend left=10] (i21) edge node[right] {\small$\lPPO$} (i22);

  \draw[rf] (i22)  edge node[below] {\small$\lRF$} (i11);
  \draw[rf,bend left=60] (init) edge node[above,pos=0.5] {\small$\lRF$} (i21);

  \draw[po] (init) edge node[left]  {\small$\lPO$} (i11);
  \draw[po] (init) edge node[right] {\small$\lPO$} (i21);
  \end{tikzpicture}
  %% \caption{Execution of \ref{ex:LB-TA} and \ref{ex:LB-fake}.}
  %% \label{fig:LB-nodep-execs}
  \end{subfigure}\hfill
  %
  \begin{subfigure}[b]{.24\textwidth}\centering
  \begin{tikzpicture}[xscale=\XScale,yscale=\YScale]

  \node at (0,1.8) {$\circledb{C}$};

  \node (init) at (1,  1.5) {$\Init$};

  \node (i11) at ( 0,  0) {$\rlab{}{x}{0}{}$};
  \node (i12) at ( 0, -2) {$\wlab{}{y}{1}{}$};

  \node (i21) at ( 2,  0) {$\rlab{}{y}{1}{}$};
  \node (i22) at ( 2, -2) {$\wlab{}{x}{1}{}$};

  \draw[po] (i11) edge node[right] {\small$\lPO$} (i12);
  \draw[po] (i21) edge node[left ] {\small$\lPO$} (i22);
  %% \draw[ppo,bend left=10] (i21) edge node[right] {\small$\lPPO$} (i22);

  \draw[rf] (init) edge node[below] {} (i11);
  \draw[rf] (i12)  edge node[below] {\small$\lRF$} (i21);

  \draw[po] (init) edge node[left]  {\small$\lPO$} (i11);
  \draw[po] (init) edge node[right] {\small$\lPO$} (i21);
  \end{tikzpicture}
  %% \caption{$\Glb$: Execution graph of \ref{ex:LB}.}
  %% \label{fig:lbWeak1}
  \end{subfigure}\hfill
  %
  \begin{subfigure}[b]{.24\textwidth}\centering
  \begin{tikzpicture}[xscale=\XScale,yscale=\YScale]

  \node at (0,1.8) {$\circledb{D}$};

  \node (init) at (1,  1.5) {$\Init$};

  \node (i11) at ( 0,  0) {$\rlab{}{x}{1}{}$};
  \node (i12) at ( 0, -2) {$\wlab{}{y}{1}{}$};

  \node (i21) at ( 2,  0) {$\rlab{}{y}{1}{}$};
  \node (i22) at ( 2, -2) {$\wlab{}{x}{1}{}$};

  \draw[po] (i11) edge node[right] {\small$\lPO$} (i12);
  \draw[po] (i21) edge node[left ] {\small$\lPO$} (i22);
  %% \draw[ppo,bend left=10] (i21) edge node[right] {\small$\lPPO$} (i22);

  \draw[rf] (i22) edge node[below] {}             (i11);
  \draw[rf] (i12) edge node[below] {\small$\lRF$} (i21);

  \draw[po] (init) edge node[left]  {\small$\lPO$} (i11);
  \draw[po] (init) edge node[right] {\small$\lPO$} (i21);
  \end{tikzpicture}
  %% \caption{Execution of \ref{ex:LB-TA} and \ref{ex:LB-fake}.}
  %% \label{fig:LB-nodep-execs}
  \end{subfigure}

\caption{Графы сценариев исполнения программы \ref{ex:LB-nodep}.}
\label{fig:LB-nodep-execs}
\end{figure}

}


Далее введем формальное определение графов сценариев исполнения.
Но сначала необходимо также ввести тип семантических меток (алфавита)
для описания абстракции разделяемой памяти.

\begin{definition}
  \label{def:mem-aux}
  Введем следующие множества:
  \begin{itemize}
    \item $\Tid \suq \N$ обозначает множество \emph{идентификаторов потоков}, 
      а поток с идентификатором $t_0 \defeq 0$
      обозначает выделенный \emph{инициализирующий} поток;
    \item $\Loc$ обозначает множество \emph{разделяемых переменных} 
      (или \emph{локаций});
    \item $\Mod \defeq \set{\na, \rlx, \acq, \rel, \acqrel, \sco}$
      обозначает множество \emph{режимов доступа} (\emph{access modes})
      к разделяемым переменным;
    \item $\Val$ обозначает множество возможных \emph{значений}. 
  \end{itemize}  
\end{definition}

\begin{definition}
  \label{def:mem-lab}
  Определим множество меток $\MemLab$, 
  соответствующих абстракции разделяемой памяти. 
  Метка $l \in \MemLab$ принимает одну из следующих форм:
  \begin{itemize}
    \item $\rlab{o}{x}{v}$ --- метка операции чтения значения $v$ из переменной $x$, 
      аннотированная режимом доступа $o$;
    \item $\wlab{o}{x}{v}$ --- метка операции записи значения $v$ в переменную $x$, 
      аннотированная режимом доступа $o$;
    \item $\lF^o$ --- метка операции барьера, аннотированная режимом $o$.
  \end{itemize}
  Если у метки опущен режим доступа, то будем считать что 
  она аннотирована режимом $\rlx$.
\end{definition}

\begin{definition}
  \label{def:exec-graph}
  \emph{Граф сценария исполнения} (\emph{execution graph}) $G$ 
  это кортеж $\tup{\lE, \lLAB, \lPO, \lRMW, \lRF, \lCO}$.
  Компоненты этого кортежа определены следующим образом.
  \begin{itemize}

    \item $\lE \suq \N$ --- это множество событий.

    \item $\lTID : \lE \fun \Tid$ --- это функция, 
      которая присваивает каждому событию идентификатор потока.
      Множество событий, принадлежащих инициализирующему потоку,
      определяется как ${\lEi \defeq \set{e \in \lE \sth \lTID(e) = t_0}}$.

    \item $\lLAB : \lE \fun \MemLab$ --- это функция, 
      которая назначает каждому событию метку. 
      Данная функция также индуцирует частично определенные функции
      $\lTYP$, $\lLOC$, $\lMOD$, $\lVAL$, которые возвращают
      тип, локацию, режим доступа и значение метки соответственно. 
      Также положим, что $\lR$, $\lW$ и $\lF$ обозначают подмножества 
      событий с меткой операции чтения, записи и барьера соответственно.

    \item $\lPO \suq \lE \times \lE$ --- это отношение 
      \emph{программного порядка} (\emph{program order}).
      Это отношение строгого частичного порядка на событиях, 
      которое полностью упорядочивает все события внутри одного потока
      согласно потоку управления программы. 
      Дополнительно полагается, что инициализирующие события $\lEi$ 
      упорядочены программным порядоком раньше всех других событий.
      Также введем отношение \emph{непосредственного программного порядка}
      (\emph{immediate program order}): 
      будем считать событие $e_1$ непосредственным $\lPO$-предшественником 
      события $e_2$ если $e_1$ предшествует $e_2$ 
      и между ними нет других событий.
      \begin{equation*}
        \lPOimm \defeq \lPO \setminus (\lPO \seqc \lPO)
      \end{equation*}

    \item $\lRMW \suq \lRex \seqc \lPOimm \cap \lEQLOC \seqc \lWex$ ---
      отношение соединяющие \emph{атомарные пары событий чтения-записи}. 
      Если $\tup{r, w} \in \lRMW$ тогда считается, что данная пара событий
      возникла в ходе исполнения одной инструкции атомарного чтения-записи, 
      например, инструкции \emph{атомарного инкремента} (\emph{fetch-and-add}, \FADD), 
      или инструкции \emph{атомарного сравнения с обменом} 
      (\emph{compare-and-swap}, \CAS).

    \item $\lRF \suq [\lW] \seqc \lEQLOC \cap \lEQVAL \seqc [\lR]$ --- отношение 
      \emph{читает-из} (\emph{reads-from}). 
      Это отношение связывает событие-запись с событиями-чтениями, 
      которые выполняют операцию чтения из него. 
      Для каждого события чтения должно существовать 
      событие записи, из которого выполняется чтение: 
      $$ r \in \lR \implies \exists w \in \lW \ldotp \tup{w, r} \in \lRF.$$
      Более того, каждое событие чтения может быть связано только с одним событием записи:
      $$ \tup{w_1,r} \in \lRF \wedge \tup{w_2,r} \in \lRF \implies w_1 = w_2.$$

      Дополнительно будем рассматривать внутреннею (\emph{internal}) 
      и внешнюю (\emph{external}) $\lRFE$ версию отношения ``читает-из''
      (обозначается как $\lRFI$ и $\lRFE$ соответcтвенно), 
      в зависимости от того принадлежит ли пара событий записи и чтения
      одному потоку или разным потокам.
      \[\def\arraystretch{1}
       \begin{array}{c@{\qquad}c@{\qquad}c@{\qquad}c}
         \lRFI \defeq \lRF \cap \lPO      &
         \lRFE \defeq \lRF \setminus \lPO
       \end{array}
      \]

    \item $\lCO \suq [\lW] \seqc \lEQLOC \seqc [\lW]$ --- это отношение 
      \emph{когерентности}. Это отношение строгого частичного порядка на событиях, 
      которое полностью упорядочивает все операции записи в одну локацию. 
      Это отношение представляет порядок, в котором операции записи 
      продвигаются в основную память и становятся видимы другим потокам. 
      \begin{equation*}
        \forall w_1, w_2 \in \lW \ldotp~ 
          \lLOC(w_1) = \lLOC(w_2) \implies \tup{w_1, w_2} \in \lCO \cup \lCO^{-1}
      \end{equation*}
      По аналогии с отношением ``читает-из'' также определим
      внутреннею и внешнюю версии отношения когерентности.
      \[\def\arraystretch{1}
       \begin{array}{c@{\qquad}c@{\qquad}c@{\qquad}c}
         \lCOI \defeq \lCO \cap \lPO      &
         \lCOE \defeq \lCO \setminus \lPO
       \end{array}
      \]

  \end{itemize}

  Множество всех графов сценариев исполнения будем обозначать как~$\ExecG$.
\end{definition}

\begin{definition}
  \label{def:ax-memory-model}
  \emph{Аксиоматическая модель памяти} (\emph{axiomatic memory model}) $M$ 
  задается как подмножество графов сценариев исполнения: $M \suq \ExecG$.
  Граф $G$ называется \emph{консистентым} с точки зрения модели $M$, 
  или просто $M$-\emph{консистентым}, если $G \in M$.
\end{definition}

Модели памяти, сохраняющие программный порядок, накладывают 
ограничение консистентности требующее, чтобы объединение 
отношений программного порядка и ``читает-из'' было ацикличным. 

\begin{definition}
Будем говорить, что граф сценария исполнения $G$ 
\emph{сохраняет программный порядок}, если выполняются следующее условие: 
\begin{itemize}
  \item $\lPO \cup \lRF$ является ацикличным отношением.
    \labelAxiom{$\lPORF$-acyclic}{ax:porf-acyc}
\end{itemize}
Обозначим множество всех таких графов как $\PorfExecG$.
Также будем говорить, что модель памяти $M$, 
заданная в аксиоматическом стиле, сохраняет программный порядок, 
если любой $M$-консистентный граф сохраняет программный порядок, 
то есть ${M \suq \PorfExecG}$.
\end{definition}

Например, среди графов, показанных на Рис.\cref{fig:LB-nodep-execs}, 
графы \circledb{A}, \circledb{B} и \circledb{C} сохраняют программный порядок, 
а граф \circledb{D} --- нет, так как он содержит $\lPO \cup \lRF$ цикл.
Таким образом, сценарий исполнения программы \ref{ex:LB-nodep},
соответствующий графу \circledb{D},
и в результате которого в локальные переменные $a$ и $b$ записано значение $1$,
запрещен моделями памяти, сохраняющими программный порядок.

Модели памяти, сохраняющие синтаксические зависимости, 
могут допускать некоторые $\lPO \cup \lRF$ цикличные графы. 
Данные модели гарантируют сохранение порядка между событиями
одного потока только если они связаны отношением 
\emph{сохраняемого программного порядка} (\emph{preserved program order}) $\lPPO$, 
которое является подмножеством отношения программного порядка $\lPO$. 
Отношение сохраняемого программного порядка
строится с помощью отношения \emph{синтаксических зависимостей} между событиями, 
данное отношение включает отношения зависимости по данным, по управлению, и другие. 

\begin{definition}
  \label{def:imm-exec-graph}
  \emph{Расширенным графом сценария исполнения} будем называть
  обычный граф сценария исполнения (\cref{def:exec-graph}), дополненный отношениями 
  \emph{зависимости по данным} (\emph{data dependency}) $\lDATA$, 
  \emph{зависимости по потоку управления} (\emph{control dependency}) $\lCTRL$, 
  \emph{зависимости по целевому адресу} (\emph{address dependency}) $\lADDR$, 
  и \emph{зависимость по операции \CAS} (\emph{\CAS dependency}) $\lRMWDEP$.
\end{definition}

{
\newcommand{\XScale}{1}
\newcommand{\YScale}{0.7}

\begin{figure}[b]
  \begin{subfigure}[b]{.44\textwidth}\centering
  \begin{tikzpicture}[xscale=\XScale,yscale=\YScale]

  %% \node at (0,1.5) {$\circledb{A}$};

  \node (init) at (1,  1.5) {$\Init$};
  \node (i11) at ( 0,  0) {$\rlab{}{x}{1}{}$};
  \node (i12) at ( 0, -2) {$\wlab{}{y}{1}{}$};
  \node (i21) at ( 2,  0) {$\rlab{}{y}{1}{}$};
  \node (i22) at ( 2, -2) {$\wlab{}{x}{1}{}$};

  \draw[po] (i11) edge node[right] {\small$\lPO$} (i12);
  \draw[po] (i21) edge node[left ] {\small$\lPO$} (i22);
  \draw[ppo,bend left=10] (i21) edge node[right] {\small$\lPPO$} (i22);

  \draw[rf] (i22) edge node[below,pos=0.5] {}             (i11);
  \draw[rf] (i12) edge node[below,pos=0.5] {\small$\lRF$} (i21);

  \draw[po] (init) edge node[left]  {\small$\lPO$} (i11);
  \draw[po] (init) edge node[right] {\small$\lPO$} (i21);
  \end{tikzpicture}
  \caption{Граф соответствующий программе \ref{ex:lb-nodep}.}
  \label{fig:LB-nodep-ppo-exec}
  \end{subfigure}\hfill
  %
  \begin{subfigure}[b]{.55\textwidth}\centering
  \begin{tikzpicture}[xscale=\XScale,yscale=\YScale]

  %% \node at (0,1.5) {$\circledb{B}$};

  \node (init) at (1,  1.5) {$\Init$};
  \node (i11) at ( 0,  0) {$\rlab{}{x}{1}{}$};
  \node (i12) at ( 0, -2) {$\wlab{}{y}{1}{}$};
  \node (i21) at ( 2,  0) {$\rlab{}{y}{1}{}$};
  \node (i22) at ( 2, -2) {$\wlab{}{x}{1}{}$};

  \draw[po] (i11) edge node[right] {\small$\lPO$} (i12);
  \draw[po] (i21) edge node[left ] {\small$\lPO$} (i22);
  \draw[ppo,bend right=10] (i11) edge node[left ] {\small$\lPPO$} (i12);
  \draw[ppo,bend left =10] (i21) edge node[right] {\small$\lPPO$} (i22);

  \draw[rf] (i22)  edge node[below]         {}             (i11);
  \draw[rf] (i12)  edge node[below,pos=0.5] {\small$\lRF$} (i21);

  \draw[po] (init) edge node[left]  {\small$\lPO$} (i11);
  \draw[po] (init) edge node[right] {\small$\lPO$} (i21);
  \end{tikzpicture}
  \caption{Граф соответствующий программам 
    \ref{ex:lb-fakedep}~и~\ref{ex:lb-dep}.
  }
  \label{fig:LB-dep-ppo-exec}
  \end{subfigure}

\caption{Графы сценариев исполнения обосновывающие результат ${a=b=1}$.}
\label{fig:LB-ppo-execs}
\end{figure}
}


В контексте моделей памяти, сохраняющих синтаксические зависимости,
под графом сценария исполнения будем подразумевать расширенный граф, 
который дополнен отношениями зависимости. 

Точное определение сохраняемого программного порядка может 
варьироваться в зависимости от конкретной модели памяти, 
но как правило оно включает как минимум объединение 
отношений зависимости, представленных выше. 
Далее модели памяти, сохраняющие синтаксические зависимости, 
накладывают ограничение консистентности требующее, чтобы объединение 
отношений сохраняемого программного порядка и 
внешнего отношения ``читает-из'' было ацикличным. 

\begin{definition}
Будем говорить, что граф сценария исполнения $G$ 
\emph{сохраняет синтаксические зависимости}, если выполняются следующие условия: 
\begin{itemize}
  \item $\lDEPS \suq \lPPO$;
    \labelAxiom{$\lPPO$-deps}{ax:ppo-deps}
  \item $\lPPO \cup \lRFE$ является ацикличным отношением.
    \labelAxiom{$\lPPORF$-acyclic}{ax:pporf-acyc}
\end{itemize}
Обозначим множество всех таких графов как $\PporfExecG$.
Также будем говорить, что модель памяти $M$, 
заданная в аксиоматическом стиле, сохраняет программный порядок, 
если любой $M$-консистентный граф сохраняет программный порядок, 
то есть ${M \suq \PporfExecG}$.
\end{definition}

Рассмотрим, например, пару $\lPO \cup \lRF$ цикличных графов, 
изображенных на Рис.\cref{fig:LB-ppo-execs}.
Данные графы соответствуют сценарию исполнения 
с результатом $a = b = 1$. 
Отметим, что при этом граф, показанный на 
Рис.\cref{fig:LB-nodep-ppo-exec}, является $\lPPO \cup \lRFE$
ацикличным, а граф на Рис.\cref{fig:LB-dep-ppo-exec} содержит такой цикл. 
Это объясняется тем, что в программу \ref{ex:LB-nodep} инстукции 
в левом потоке не связаны зависистью по данным, 
а в программах \ref{ex:LB-fakedep}~и~\ref{ex:LB-dep}
инструкции в обоих потоках связаны зависистью по данным. 
Таким образом, модели памяти, сохраняющие синтаксические зависимости, 
допускают сценарию исполнения с результатом $a = b = 1$ 
для программы \ref{ex:LB-nodep}, 
но не для программ \ref{ex:LB-fakedep}~и~\ref{ex:LB-dep}. 

%% \begin{definition}
%%   \label{def:imm-deps-rel}
%%   Для расширенного графа сценария исполнения определим 
%%   объединенное отношение \emph{зависимости} (\emph{dependency}) 
%%   следующим образом:
%%   $$ \lDEPS \defeq \lDATA \cup \lCTRL \cup \lADDR \seq \lPO^? \cup \lRMWDEP. $$
%% \end{definition}

%% \begin{definition}
%%   \label{def:imm-deps-rel}
%%   Для расширенного графа сценария исполнения определим 
%%   объединенное отношение \emph{зависимости} (\emph{dependency}) 
%%   следующим образом:
%%   $$ \lDEPS \defeq \lDATA \cup \lCTRL \cup \lADDR \seq \lPO^? \cup \lRMWDEP. $$
%% \end{definition}

\subsection{Структуры событий в модели \Wkm}
\label{sec:wkmo-eventstruct}

Определение класса структур событий, использующихся в модели \Wkm.
           
%% \chapter{Структуры событий для моделей памяти сохраняющих программный порядок}
\label{ch:porf-evenstruct}

В данной главе описан предложенный в диссертации метод 
кодирования слабых моделей памяти сохраняющих программный порядок 
с помощью простых структур событий с предикатом консистентности. 
Этот результат позволяет утверждать, что теория  
моделей памяти сохраняющих программный порядок%
\footnote{Везде далее в этом разделе 
под термином ``модель памяти'' будем подразумевать 
модель памяти сохраняющую программный порядок, если явно не указано иное.}.
может быть сведена к теории простых структур событий, 
и, как следствие, позволяет применить уже известныю теорию о структурах событий%
~\cite{Winskel:86,Vaandrager:TCS1991,Sassone:MFCS1993,Nielsen:REX93,Winskel-TCS:09}
к проблемам слабых моделей памяти.

Теория простых структур событий и метод сведения 
моделей памяти к ним были формализованы в системе 
доказательства теорем \coq. 
Полученная в результате библиотека может быть использована 
для формализации доказательства свойств структур событий 
и моделей памяти, а также при разработке других
инструментов для интерактивной верификации многоточных программ.
%% Технические вопросы представления структур событий 
%% и других упоминаемых в данной главе формализмов 
%% также кратко рассматриваются в данной главе.  

Данная глава организована следующим образом. 
В разделе \cref{sec:pomset-graphs} описывается метод 
сведения графов сценариев исполнения к языкам помсетов. 
Далее в разделе \cref{sec:mm-eventstruct} этот результат 
используется для сведения моделей памяти к простым структурам событий. 
В разделе \cref{sec:eventstruct-opsem} описывается 
операционная семантика для инкрементального построения 
структуры событий по заданной многопоточной программе 
и доказываются основные её свойства. 
%% Наконец, в \cref{sec:mm-eventstruct} кратко рассматриваются 
%% технические вопросы представления структур событий 
%% и других упоминаемых в данной главе формализмов.

\section{Сведение графов сценариев исполнения к языкам помсетов}
\label{sec:pomset-graphs}

%% Эквивалентность задания моделей памяти в терминах
%% графов сценариев исполнения и в терминах языков помсетов.

Напомним, что и формализм графов сценариев исполнения,
широко распространненый в сообществе для задания слабых моделей памяти%
~\cite{Alglave-al:TOPLAS14}, 
и формализм помсетов, известный по классическим работам%
~\cite{Pratt:CONCUR84,Gischer:TCS88} 
в области семантики многопоточных программ, основаны на теории частичных порядков. 
Главное отличие между ними заключается в том, что помсет состоит из единственного
отношения частичного порядка --- отношения причинно-следственной связи, 
в то время как граф сценариев исполнения состоит из
нескольких отношений, наделенных различной семантикой. 

Основная идея представленного метода сведения заключается в том 
что в рамках моделей сохраняющих программный порядок 
объединение отношения программного порядка и отношения ``читает-из''
может рассматриваться как аналог отношения причинно-следственной связи.
Таким образом множество графов сценариев исполнения можно факторизовать 
по отношению эквивалентности индуцированных отношений причинно-следственной связи.
И наоборот, можно построить частичную функцию, 
которая принимая на вход помсет и пытается выполнить 
разбиение отношения причинно-следственной связи на 
программный порядок и отношение ``читает-из''. 

Рассмотрим данное построение более формально. 
Для начала, введем формальное определение 
аксиоматических моделей памяти сохраняющих программный порядок.

\begin{definition}
Будем говорить, что граф сценария исполнения $G$ 
сохраняет программный порядок, если выполняются следующие условия: 
\begin{itemize}
  \item $G.\lR \suq \cod{G.\lRF}$; 
    \labelAxiom{$\lRF$-complete}{ax:rf-complete}

  \item $\lPO \cup \lRF$ является ацикличным отношением.
    \labelAxiom{$\lPORF$-acyclic}{ax:porf-acyc}
\end{itemize}
Обозначим множество всех таких графов как $\PorfExecG$.
Также будем говорить, что модель памяти $M$, 
заданная в аксиоматическом стиле, сохраняет программный порядок, 
если для любой программы $P$ каждый соответствующий ей
граф сценария исполнения $G \in \sem{P}_M$ сохраняет программный порядок, 
то есть ${\sem{P}_M \suq \PorfExecG}$.
\end{definition}

Далее определим функцию для построения помсета по графу сценария исполнения.

\begin{definition}
Определим функцию $\gpom : \PorfExecG \fun \Pom[\TidMemLab]$ следующим образом. 
Пусть $G \in \PorfExecG$. Тогда $\gpom(G) = \tup{E, \lab, \ca}$ где
\begin{itemize}
  \item $E = G.\lE$, 
  \item $\lab(e) = \tup{G.\lTID(e), G.\lLAB(e)}$,
  \item $\ca {}\defeq{} (G.\lPO \cup G.\lRF)^*$.
\end{itemize}
\end{definition}

Также предъявим обратную функцию, выполняющую построение 
графа сценария исполнения по помсету.
Для этого сначала определим подмножество помсетов, 
для которых такое построение возможно. 

\begin{definition}
Пусть $p = \tup{E, \lab, \ca} \in \Pom[\Label]$
и пусть $\simeq$ это некоторое отношение эквивалентности на множестве меток $\Label$. 
Будем называть помсет $p$ \emph{разделяемым на потоки относительно $\simeq$} 
(\emph{threaded pomset}) если для любого класса эквивалентности $\eset \suq E$
относительно $\simeq$ верно, что сужение помсета на события этого класса $p\rst{\eset}$ 
порождает линейно упорядоченное мультимножество. 
Множество всех таких помсетов будем обозначать как $\ThrdPom[\Label, \simeq]$.
\end{definition}

\begin{definition}
Пусть $p = \tup{E, \lab, \ca} \in \ThrdPom[\TidMemLab, \lEQTID]$. 
В таком случае определим индуцированное отношение 
программный порядка следующим образом:
$$ \lPO(p) \defeq \sca \cap \lEQTID. $$
Также определим множество кандидатов 
на отношение ``читает-из'' $\lRFs(p)$ 
таким образом, что $\lRF \in \lRFs(p)$ 
при выполнении следующих условий:
\begin{itemize}
  \item $\lRF {}\suq{} \lEQLOC \cap \lEQVAL$,
  \item $E \cap \lR \suq \cod{\lRF}$, 
  \item $\ca {}={} (\lPO(p) \cup \lRF)^*$.
\end{itemize}
%
Соответствующим образом определим 
множество графов-кандидатов $\ExecGs{p}$. 
Положим $G \in \ExecGs{p}$ если $p \in \ThrdPom[\TidMemLab]$
и выполнены следующие условия:
\begin{itemize}
  \item $G.\lE = E$,
  \item $\tup{G.\lTID(e), G.\lLAB(e)} = \lab(e)$, 
  \item $G.\lPO = \lPO(p)$, 
  \item $G.\lRF \in \lRFs(p)$. 
\end{itemize}
Если же $p \not\in \ThrdPom[\TidMemLab]$ то положим $\ExecGs{p} = \emptyset$.
%
Понятие множества графов-кандидатов также можно расширить на уровень языков помсетов:
$$ \ExecGs{P} \defeq \set{ G ~|~ \exists p \in P \ldotp G \in \ExecGs{p} }. $$
%
Множество всех помсетов для которых $\ExecGs{p} \neq \emptyset$
будем обозначать как $\PorfPom[\TidMemLab]$.
Также зададим частично определенную функцию 
${\pomg : \Pom[\TidMemLab] \pfun \ExecG}$,
которая для данного помсета выбирает 
один из соответствующих ему графов.
\begin{equation*}
  \pomg(p) = \begin{cases*}
    G      & такой что $G \in \ExecGs{p}$   \\
    \bot   & если $\ExecGs{p} = \emptyset$. \\
  \end{cases*}
\end{equation*}
%
\end{definition}

\begin{proposition}
Для любого помсета $p \in \Pom[\TidMemLab]$
верно, что $\ExecGs{p} \suq \PorfExecG$
и, следовательно, $\pomg(p) \in \PorfExecG$.
\end{proposition}

Наконец, покажем, что произвольная модель памяти
сохраняющая программный порядок может быть 
представлена как язык помсетов. 

\begin{definition}
Пусть $M$ --- это заданная в аксиоматическом стиле 
модель памяти сохраняющая программный порядок,
то есть $M \suq \PorfExecG$.
Построим соответствующий ей язык помсетов 
$\wmmlang{M} \in \Pomlang[\TidMemLab]$ следующим образом.
Будем считать, что помсет принадлежит этому языку, 
если найдется хотя бы один граф, допустимый $M$,
который соответствует этому помсету: 
$$ p \in \wmmlang{M} \defeq \exists G \in \ExecGs{p} \ldotp G \in M. $$
\end{definition}
 
\section{Кодирование модели памяти с помощью структуры событий}
\label{sec:mm-eventstruct}

В предыдущем разделе было показано, что 
модель памяти сохраняющая программный порядок 
может быть представлена как язык помсетов. 
Это наблюдение позволяет заключить, 
что модель памяти также может быть представлена 
и как простая структура событий с предикатом консистентности,
ведь, как было упомянуто в разделе~\cref{sec:pomsets-eventstruct}, 
данный класс структур событий позволяет выразить произвольный язык помсетов. 

В этом разделе будет также показано,
что на самом деле модель памяти сохраняющая программный порядок 
может быть представлена как простая структура событий специального вида.
В данной структуре событий все события одного потока образуют дерево. 
Ветви этого дерева соответствуют различным путям исполнения данного потока,
а события, принадлежащие разным веткам дерева, находятся в отношении конфликта. 
Cтруктуры событий данного вида будем называть 
\emph{разделяемыми на потоки} (\emph{threaded prime event structures}).

Основной особенностью структуры событий принадлежашей 
данному подклассу является то,
что она может быть закодирована с помощью 
только одного отношения частичного порядка
и некоторого отношения эквивалентности на метках.
Отношение конфликта при этом не требуется хранить явно, 
оно индуцируется двумя предыдущими отношениями. 
Такое представление, зачастую, позволяет существенно 
упростить рассуждения о структурах событий. 
Например, можно заметить, что две структуры событий 
данного класса изоморфны тогда и только тогда, 
когда они изоморфны как помеченные частично упорядоченные множества. 

\begin{figure}[h]
\small

\newcommand{\XScale}{1.5}
\newcommand{\YScale}{1.5}

  \begin{subfigure}{0.25\linewidth}
  \begin{equation*}
    \inarrII{
      \readInst{}{a}{x}
    }{
      \writeInst{}{x}{1}
    }
  \end{equation*}
  \end{subfigure}
  %
  \begin{subfigure}{0.7\linewidth}
  \begin{center}
    \begin{tikzpicture}[xscale=\XScale, yscale=\YScale]
        \node (TS1) at (1.5,1) {$ts_1: \tup{t_1,\lTS}$};
        \node (TS2) at (6  ,1) {$ts_2: \tup{t_2,\lTS}$};

        \node (rx1) at (0,0) {$r_1: \tup{t_1,\rlab{}{x}{\initval}}$};
        \node (rx2) at (3,0) {$r_2: \tup{t_1,\rlab{}{x}{1}}$};
        \node (wx)  at (6,0) {$w: \tup{t_2,\wlab{}{x}{1}}$};

        \draw[ca] (TS1) -- (rx1);
        \draw[ca] (TS1) -- (rx2);
        \draw[ca] (TS2) -- (wx);
        \draw[ca] (wx)  -- (rx2);

        \draw[cf] (rx1) -- (rx2);
    \end{tikzpicture}
  \end{center}
  \end{subfigure}

  \caption{Пример программы и соответствующей ей 
    разделямой на потоки простой структуры событий.}
  \label{fig:thrd-es-example}
\end{figure}

На рис.\cref{fig:thrd-es-example} можно видеть пример многопоточной программы 
и соответствующей ей разделяемой на потоки простой структуры событий. 
Действительно, можно видеть, что внутри каждого потока 
структура событий формирует дерево. 
Каждая ветка этого дерева соответствует одному 
из последовательных сценариев исполнения соответствующего потока.  
События, имеющее общего родителя в таком дереве
(например, события $r_1$ и $r_2$), 
находятся в отношении непосредственного конфликта. 

Далее введем формальные определения описанных выше объектов. 

\begin{definition}
Частично упорядоченное множество $\tup{E, \ca}$ 
называется \emph{префиксно-линейно упорядоченным} 
(\emph{downward-total}) если выполняется следующие условие:
$$ x \ca z \wedge y \ca z \implies x \ca y \vee y \ca x. $$
Если в дополнение к этому частично упорядоченное множество являетя 
префикс-конечным, то будем называть такое множество \emph{лесом}.
Если кроме того существует наименьший элемент $e_0 \in E$, 
тогда будем называть такое множество \emph{деревом}, 
а $e_0$ --- корнем этого дерева. 
\end{definition}

\begin{proposition}
Для леса $\tup{E, \ca}$ можно задать 
частично определенную функцию $\pred : E \pfun E$, 
возращающую родителя данного элемента: 
$$ \pred(e) = e' \;{}\iff{}\; e' \ica e $$
\end{proposition}

\begin{proposition}
Размеченный лес $\tup{E, \lab, \ca}$ порождает простую структуру событий 
без волнений (confusion-free) $S(p) = \tup{E, \lab, \ca, \cf}$, 
где отношение непосредственного конфликта задается следующим образом: 
$$ e_1 \icf e_2 \defeq e_1 \neq e_2 \wedge \pred(e_1) = \pred(e_2). $$
Далее в этом разделе для обозначения структуры событий, порождаемой $p$, 
будем писать просто $S$, если $p$ можно вывести из контекста. 
\end{proposition}

\begin{definition}
Пусть $S = \tup{E, \lab, \ca, \cf} \in \PrimeES[\Label]$ --- 
это простая структура событий
с бинарным конфликтом, и пусть $\simeq$ это некоторое 
отношение эквивалентности на метках $L$.
Рассмотрим сужение структуры событий на классы эквивалентности 
${ S\rst{\simeq} \defeq \tup{E, \lab, \ca {}\cap{} \simeq, \cf {}\cap{} \simeq} }$.
Будем говорить, что структура \emph{разделима на потоки относительно отношения $\simeq$}
(\emph{threaded prime event structure})
если $S\rst{\simeq}$ образует размеченный лес и, кроме того, 
отношение конфликта во всей структуре событий является 
продолжением отношения конфликта в суженной структуре, то есть 
$${ \icf_S = \icf_{S\rst{\simeq}} }.$$
Будем обозначать множество всех таких структур 
как $\ThrdPrimeES[\Label,\simeq]$
\end{definition}

\begin{proposition}
Пусть $p = \tup{E, \lab, \ca}$ это 
размеченное частично упорядоченное множество, 
а $\simeq$ это отношение эквивалентности на метках событий.
Предположим, что сужение отношения порядка на классы эквивалентности 
${ p\rst{\simeq} \defeq \tup{E, \lab, \ca {}\cap{} \simeq} }$ образует лес.
Рассмотрим отношение непосредственного конфликта $\icf$, порождаемое этим лесом. 
Если отношение $\cf$, определенное как продолжение $\icf$ вдоль отношения $\ca$,
иррефлексивно, тогда $\tup{E, \lab, \ca, \cf}$ является 
простой структурой событий разделимой на потоки относительно отношения $\simeq$.  
\end{proposition}

Рассмотрим структуру событий $S \in \ThrdPrimeES[\TidMemLab, \lEQTID]$
разделяемую на потоки относительно отношения $\lEQTID$.
Определим множество графов, которые она кодирует, следующим образом:
$$ \ExecGs{S} \defeq \ExecGs{\pomlang{S}}  $$

Для данной модели памяти $M$ можно дополнить $S$ предикатом консистентности, 
который будет отфильтровывать все конфигурации, 
не принадлежащие языку $\wmmlang{M}$.

\begin{definition}
Пусть ${S = \tup{E, \lab, \ca, \cf} \in \ThrdPrimeES[\TidMemLab, \lEQTID]}$
и ${M \suq \PorfExecG}$.
Определим простую структуру событий с предикатом консистентности
${\wmmpes{S}{M} = \tup{E, \lab, \ca, \Cons}}$ таким образом, что:
\begin{itemize}
  \item $E {}\defeq{} E_S$,
  \item $\lab {}\defeq{} \lab_S$,
  \item $\ca {}\defeq{} \ca_S$,
  \item $C \in \Cons {}\defeq{} C \not\in \Gcf_S \wedge S\rst{C} \in \wmmlang{M}$.
\end{itemize}
\end{definition}

Корректность структуры событий $\wmmpes{S}{M}$ относительно языка 
порождаемого $M$ вытекает напрямую из определения. 

\begin{proposition}
\label{prop:thrd-es-sound}
Для любой структуры событий $S = \ThrdPrimeES[\TidMemLab, \lEQTID]$
и любой модели памяти $M \suq \PorfExecG$
язык помсетов, порождаемый $\wmmpes{S}{M}$, 
корректен относительно языка $\wmmlang{M}$, то есть:
$$ \pomlang{\wmmpes{S}{M}} \suq \wmmlang{M}. $$
Из этого, в частности, следует что:
$$ \ExecGs{S} \suq M. $$
\end{proposition}

Тем не менее, нетрудно заметить, что структура событий $\wmmpes{S}{M}$
не обязана быть полной относительно языка порождаемого $M$.
В следующем разделе будет рассмотрена операционная семантика 
для инкрементального построения структуры $S$, 
порождающей структуру $\wmmpes{S}{M}$ 
полную относительно языка $\sem{P}_M$ для 
любой заверщающейся программы $P$.

\section{Операционная семантика построения структуры событий}
\label{sec:eventstruct-opsem}

В данном разделе представлена операционная семантика 
для инкрементального построения структуры событий.
Предложенная семантика предоставляет конкретную
процедуру построения структуры событий, 
кодирующую все возможные сценарии поведения 
заданной программы~$P$ в заданной модели памяти~$M$.  
Также доказываются основные свойства данной семантики, 
а именно \emph{корректность}, \emph{полнота}, и \emph{конфлюэнтность}. 
%\emph{терминируемость}.

Отметим, что помимо модели памяти $M$, данная семантика также параметризована
последовательной семантикой потоков программы $P$, 
заданной в терминах системы помеченных переходов $\LTS$. 
Это, в частности, позволяет абстрагироваться от деталей 
реализации последовательной семантики 
и комбинировать предложенную семантику построения структуры событий 
с различными моделями последовательных вычислений. 

На рис.\cref{fig:thrd-es-opsem-example} представлен
пример инкрементального построения структуры событий
для программы, показанной на рис.\cref{fig:thrd-es-example}. 
Отметим, что метки данной структуры также содержат 
информацию о переходах $\ltr{l}{\state}{\state'}$ 
в последовательной семантике. 
Эта информация как раз и будет использована
для того, чтобы связать структуру событий 
с последовательной семантикой.  

\begin{figure}[h]
\small

\newcommand{\XScale}{1.5}
\newcommand{\YScale}{1.5}

  \begin{center}
    \begin{tikzpicture}[xscale=\XScale, yscale=\YScale]
        \node (TS1) at (1.5,1) 
          {$ts_1: \tup{t_1,\ltr{\lTS}{\state^1_0}{\state^1_0}}$};

        \node (TS2) at (6  ,1) 
          {$ts_2: \tup{t_2,\ltr{\lTS}{\state^2_0}{\state^2_0}}$};

        \node (rx1) at (0,0) 
          {\phantom{$r_1: \tup{t_1,\ltr{\rlab{}{x}{\initval}}{\state^1_0}{\state^1_1}}$}};
        %% \node (rx2) at (3,0) {$r_2: \tup{t_1,\rlab{}{x}{1}}$};
        %% \node (wx)  at (6,0) {$w: \tup{t_2,\wlab{}{x}{1}}$};

        %% \draw[ca] (TS1) -- (rx1);
        %% \draw[ca] (TS1) -- (rx2);
        %% \draw[ca] (TS2) -- (wx);
        %% \draw[ca] (wx)  -- (rx2);

        %% \draw[cf] (rx1) -- (rx2);

       \draw[esrect] (-2,1.5) rectangle (8,0.5);

       \node[rotate=270] (xx) at (3,0) {\Large$\esStep{}$};
    \end{tikzpicture}
    %
    \begin{tikzpicture}[xscale=\XScale, yscale=\YScale]
        \node (TS1) at (1.5,1) 
          {$ts_1: \tup{t_1,\ltr{\lTS}{\state^1_0}{\state^1_0}}$};

        \node (TS2) at (6  ,1) 
          {$ts_2: \tup{t_2,\ltr{\lTS}{\state^2_0}{\state^2_0}}$};

        \node (rx1) at (0,0) 
          {$r_1: \tup{t_1,\ltr{\rlab{}{x}{\initval}}{\state^1_0}{\state^1_1}}$};
        %% \node (rx2) at (3,0) {$r_2: \tup{t_1,\rlab{}{x}{1}}$};
        %% \node (wx)  at (6,0) {$w: \tup{t_2,\wlab{}{x}{1}}$};

        \draw[ca] (TS1) -- (rx1);
        %% \draw[ca] (TS1) -- (rx2);
        %% \draw[ca] (TS2) -- (wx);
        %% \draw[ca] (wx)  -- (rx2);

        %% \draw[cf] (rx1) -- (rx2);

       \draw[esrect] (-2,1.5) rectangle (8,-0.5);

       \node[rotate=270] (xx) at (3,-1) {\Large$\esStep{}$};
    \end{tikzpicture}
    %
    \begin{tikzpicture}[xscale=\XScale, yscale=\YScale]
        \node (TS1) at (1.5,1) 
          {$ts_1: \tup{t_1,\ltr{\lTS}{\state^1_0}{\state^1_0}}$};

        \node (TS2) at (6  ,1) 
          {$ts_2: \tup{t_2,\ltr{\lTS}{\state^2_0}{\state^2_0}}$};

        \node (rx1) at (0,0) 
          {$r_1: \tup{t_1,\ltr{\rlab{}{x}{\initval}}{\state^1_0}{\state^1_1}}$};

        %% \node (rx2) at (3,0) {$r_2: \tup{t_1,\rlab{}{x}{1}}$};

        \node (wx)  at (6,0) 
          {$w: \tup{t_2,\ltr{\wlab{}{x}{1}}{\state^2_0}{\state^2_1}}$};

        \draw[ca] (TS1) -- (rx1);
        %% \draw[ca] (TS1) -- (rx2);
        \draw[ca] (TS2) -- (wx);
        %% \draw[ca] (wx)  -- (rx2);

        %% \draw[cf] (rx1) -- (rx2);

       \draw[esrect] (-2,1.5) rectangle (8,-0.5);

       \node[rotate=270] (xx) at (3,-1) {\Large$\esStep{}$};
    \end{tikzpicture}
    %
    \begin{tikzpicture}[xscale=\XScale, yscale=\YScale]
        \node (TS1) at (1.5,1) 
          {$ts_1: \tup{t_1,\ltr{\lTS}{\state^1_0}{\state^1_0}}$};

        \node (TS2) at (6  ,1) 
          {$ts_2: \tup{t_2,\ltr{\lTS}{\state^2_0}{\state^2_0}}$};

        \node (rx1) at (0,0) 
          {$r_1: \tup{t_1,\ltr{\rlab{}{x}{\initval}}{\state^1_0}{\state^1_1}}$};

        \node (rx2) at (3,0) 
          {$r_2: \tup{t_1,\ltr{\rlab{}{x}{1}}{\state^1_0}{\state^1_3}}$};


        \node (wx)  at (6,0) 
          {$w: \tup{t_2,\ltr{\wlab{}{x}{1}}{\state^2_0}{\state^2_1}}$};


        \draw[ca] (TS1) -- (rx1);
        \draw[ca] (TS1) -- (rx2);
        \draw[ca] (TS2) -- (wx);
        \draw[ca] (wx)  -- (rx2);

        \draw[cf] (rx1) -- (rx2);

       \draw[esrect] (-2,1.5) rectangle (8,-0.5);

        % \draw[then] (3,4) -- (3,3.5);
    \end{tikzpicture}


  \end{center}

  \caption{Пример построения по программе соответствующей ей 
    разделямой на потоки простой структуры событий.}
  \label{fig:thrd-es-opsem-example}
\end{figure}


Прежде чем перейти к рассмотрению операционной семантики, 
введем несколько вспомогательных определений, 
помогающих связать структуру событий с 
последовательной семантикой. 

\begin{definition}
Пусть $\LTS = \tup{\State, \Label, \ltr{}{}{}}$ это система помеченных переходов, 
а $p = \tup{E, \lab, \ca}$ это лес размеченный метками типа $\Step{\Label}{\State}$. 
Будем говорить, что $p$ \emph{корректен} (\emph{sound}) 
относительно $\LTS$ если выполняются следующие условия:
\begin{itemize}
  \item для любого события $e \in E$ его метка 
    $\lab(e) = \step{l}{\state}{\state'}$ 
    образует валидный переход, то есть $\ltr{l}{\state}{\state'}$;
  %
  \item метки соседних событий $e_1 \ica e_2$ \emph{сопряжены} в том смысле, что если 
    $\lab(e_1) = \step{l_1}{\state_1}{\state'_1}$ и 
    $\lab(e_2) = \step{l_2}{\state_2}{\state'_2}$ тогда 
    $\state'_1 = \state_2$.
\end{itemize}
Будем говорить, что разделимая на потоки структура событий 
корректна относительно системы переходов, 
если соответствующей этой структуре лес корректен. 
\end{definition}

\begin{proposition}
Пусть $p$ это лес, корректный относительно системы переходов~$\LTS$.
Тогда для любого события $e \in E$ верно, что
метки событий его префикса $\dwset{e}$, упорядоченного согласно отношению $\ca$, 
образуют валидную трассу системы $\LTS$.

Аналогичное утверждение верно для структуры событий разделимой на потоки,
с точностью до того, что для такой структуры необходимо сузить
префикс на события того же потока, что и рассматриваемое событие $e$.  
\end{proposition}

Для того чтобы хранить локальные состояния потоков в структуре событий
дополним тип меток $\TidMemLab$ до типа 
$\ThrdMemLab \defeq \Tid \times \State \times \Lab \times \State$
расширив его парой состояний, формирующих переход.
Также определим функцию $\lSTEP$, извлекающую из метки $\ell \in \ThrdMemLab$ 
тройку, образующую помеченный переход:
$$ \lSTEP(\tup{t, \state, \el, \state'}) = \step{\el}{\state}{\state'}. $$

Для удобства при необходимости мы также будем трактовать 
структуру событий $S \in \PrimeES[\ThrdMemLab]$ как 
структуру с метками типа $\TidMemLab$, то есть 
$S \in \PrimeES[\ThrdMemLab]$.

\begin{definition}
Предположим, что многопоточная программа $P$ задана как 
система помеченных переходов $\LTS = \tup{\State, \Label, \ltr{}{}{}}$ и 
функция инициализации потоков $\initst : \Tid \pfun \State$%
\footnote{Везде далее в этом разделе, говоря о программе $P$,
будем подразумевать, что она задана точно таким же образом.}.
Будем говорить, что структура событий 
$S \in \ThrdPrimeES[\ThrdMemLab, \lEQTID]$ 
\emph{корректна} (\emph{sound}) относительно программы $P$ 
если она корректна относительно $\LTS$ и, кроме того, 
для любого потока $t \in \Tid$ сужение $S$ на 
события этого потока $S\rst{t}$ порождает дерево с корнем $e_t$ таким, 
что $\stlab(e_t) = \step{\lTS}{\initst(t)}{\initst(t)}$.
\end{definition}

Сформулируем утверждение, суть которого заключается в том, 
что структура событий, корректная относительно программы $P$,
кодирует графы сценариев исполнения, которые 
допускаются моделью памяти $M$ для программы $P$.

\begin{lemma}
\label{lm:thrd-es-prog-sound}
Рассмотрим струтуру событий $S \in \ThrdPrimeES[\ThrdMemLab, \lEQTID]$,
модель памяти $M \suq \PorfExecG$ и программу $P$.
Пусть что $S$ корректна относительно $P$. Тогда верно, что
$$ \ExecGs{\wmmpes{S}{M}} \suq \sem{P}_M. $$
\end{lemma}

Перейдем к рассмотрению процедуры построения структуры событий. 
Начнем с вспомогательного правила перехода, 
которое позволяет расширить произвольный помсет 
путем добавления в его конец нового события.

\newcommand{\PomAddEventRule}{{(Add~Event)}\xspace}
\newcommand{\PorfAddEventRule}{{(ES~Step)}\xspace}

\begin{center}
  \AXC{$e \not\in E$}
  \AXC{$\eset \subseteq E$}
  %
  \RightLabel{\PomAddEventRule}
  \BIC{$\tup{E, \lab, \ca}
        \pomStep{\tup{e, \ell, \eset}}
        \tup{E \uplus \set{e}, \updmap{\lab}{e}{\ell}, \ca \uplus \dwset{\eset} \times \set{e}}$}
  \DisplayProof
\end{center}

Это правило добавляет в помсет $p = \tup{E, \lab, \ca}$
новое событие $e$ c меткой $l$ и множеством событий-предшественников $\eset$. 

Далее, рассмотрим правило перехода, которое 
выполняет построение интересующей нас структуры событий, 
также путем добавления одного нового события. 

\begin{center}
  \AXC{$\ltr{\el}{\state}{\state'}$}
  \noLine
  \UIC{$\lSTEP(\ell) = \step{\el}{\state}{\state'}$}
  \noLine
  \UIC{$p \pomStep{\tup{e, \ell, \set{e_{\lPO}, e_{\lRF}}}} p'$}
  %
  \AXC{$\stlab(e_{\lPO}) \posync \lSTEP(\ell)$}
  \noLine
  \UIC{$\dlab(e_{\lRF}) \rfsync \lLAB(\ell)$}
  %
  \AXC{$p' \in \DetPom$}
  \noLine
  \UIC{$\upset{e_{\lPO}} \cap \dwset{e_{\lRF}} \subseteq \emptyset$}
  \noLine
  \UIC{$p'\rst{\dwset{e'}} \in \wmmlang{M}$}
  %
  \RightLabel{\PorfAddEventRule}
  \TIC{$p \esStep{\tup{e, \ell, e_{\lPO}, e_{\lRF}}} p'$}
  \DisplayProof
\end{center}

Данное правило добавляет в помсет $p = \tup{E, \lab, \ca}$ 
событие $e$ c меткой $\ell$, такой что $\lSTEP(\ell)$ 
образует валидный переход в рамках последовательной семантики.
Данное правило также недетерминированным образом выбирает 
для нового события двух предков $e_{\lPO}$ и $e_{\lRF}$.
Кроме того, требуется, чтобы метка события $e_{\lPO}$ 
была сопряжена c меткой нового события $\ell$. 
Аналогично, требуется чтобы метка $e_{\lRF}$
была согласована с $\ell$ в смысле отношения $\lRF$.

\begin{center}
  \AXC{$$}
  \RightLabel{}
  \UIC{$\step{l_1}{\state}{\state'} \posync \step{l_2}{\state'}{\state''}$}
  \DisplayProof
  \rulehskip
  %
  \AXC{$$}
  \RightLabel{}
  \UIC{$\tup{t, \tslab} \rfsync \tup{t,\wlab{o}{x}{v}}$}
  \DisplayProof
  \rulevspace
  
  \AXC{$$}
  \RightLabel{}
  \UIC{$\tup{t, \tslab} \rfsync \tup{t,\rlab{o}{x}{\initval}}$}
  \DisplayProof
  \rulehskip
  % 
  \AXC{$$}
  \RightLabel{}
  \UIC{$\tup{t, \wlab{o}{x}{v}} \rfsync \tup{t, \rlab{o}{x}{v}}$}
  \DisplayProof
\end{center}

Предусловие $p' \in \DetPom$ проверяет, что новый помсет является детерминированным.

\begin{definition}
Помсет $p \in \Pom[\Label]$ назывется \emph{детерминированным}, 
если все его события с одинаковой меткой и префиксом равны:
$$ \lab(e_1) = \lab(e_2) \wedge \dwsset{e_1} = \dwsset{e_2} \implies e_1 = e_2. $$
Другими словами, детерминированный помсет не может 
содержать дублирующиеся события. 
Множество всех таких помсетов будем обозначать как $\DetPom[\Label]$.
\end{definition}

Требование на детерминированность помсета необходимо для 
обеспечения завершимости построения структуры событий.
В противном случае, правило \PorfAddEventRule могло бы 
добавлять одно и то же событие в помсет неограниченное 
количество раз. 

\begin{proposition}
Помсет $p$, построенный c помощью операционной семантики,
то есть $p_0 \esStep{}^* p$, является детерминированным: $p \in \DetPom$.
\end{proposition}

Предусловие $\upset{e_{\lPO}} \cap \dwset{e_{\lRF}}$ гарантирует, 
что отношение конфликта $\cf$, порождаемое $p'$, будет иррефлексивно. 
Действительно, в обновленной структуре $S(p')$, новое событие $e$
будет находиться в конфликте со всеми потомками события $e_{\lPO}$.
Если бы среди них нашелся хотя бы один предок события $e_{\lRF}$, 
то отношение конфликта можно было бы продлить до $e_{\lRF}$
и, следовательно, до $e$, получив тем самым что $e$ находится 
в конфликте с самим собой. 

\begin{proposition}
Помсет $p$, построенный c помощью операционной семантики, 
то есть $p_0 \esStep{}^* p$, 
порождает структуру событий $S = \tup{E, \lab, \ca, \cf}$
разделимую на потоки относительно $\lEQTID$: 
$S \in \ThrdPrimeES[\TidMemLab, \lEQTID]$.
\end{proposition}

Наконец, рассмотрим назначение последнего предусловия 
${p'\rst{\dwset{e'}} \in \wmmlang{M}}$.
Это предусловие проверят, что префикс нового события $e'$
формирует консистентный согласно модели $M$ помсет. 
Строго говоря, данное предусловие не является необходимым.
Действительно, согласно \cref{prop:thrd-es-sound},
структура событий $\wmmpes{S}{M}$, 
порождаемая помсетом $p$ и дополненная 
предикатом консистентности, уже является корректной
относительно модели памяти $M$, 
то есть она кодирует только допустимые моделью $M$ графы.
Поэтому, вообще говоря, для обеспечения корректности 
нет необходимости каким либо образом дополнительно отфильтровывать 
неконсистентные графы во время построения структуры событий.
Тем не менее, данное предусловие позволяет на раннем 
этапе предотвратить добавление событий, 
которые заведомо не могут быть включены ни в один консистетный помсет.    
Другими словами, добавление рассматриваемого предусловия
в правило \PorfAddEventRule заранее отсекает те ветви структуры событий, 
исследование которых не приведет к появлению новых консистентных помсетов.  

Отметим, что описанная выше оптимизация полагается на то, 
что язык, порождаемый $M$, является префикс-замкнутым. 
Иными словами, для любого консистентного помсета $p \in \wmmlang{M}$
должно быть справедливо, что его сужение на произвольный префикс
остается консистентным, то есть для любого $\eset \suq E$
верно, что $p\rst{\dwset{\eset}} \in \wmmlang{M}$.
В противном случае не гарантируется полнота операционной семантики. 
Действительно, несложно построить пример искуственной модели памяти $M$
такой, что некоторый помсет $p$ является консистентным, 
но при этом его сужение на префикс некоторого события $e$ неконсистентно.
То есть $p \in \wmmlang{M}$, но $p\rst{\dwset{e}} \not\in \wmmlang{M}$.
В таком случае, построить $p$ инкрементально с помощью операционной семантики
не получится, так как на шаге добавления события $e$ 
процесс построения оборвется. 

Формализуем описанное выше наблюдение в виде леммы, которая гарантирует, 
что любой помсет, принадлежащий префикс-замкнутуму языку, 
может быть построен с помощью операционной семантики. 
Но сначала определим процедуру построения начального помсета для программы $P$. 

\begin{definition}
Пусть $p_0(P) = \tup{E_0, \lab_0, \ca_0}$ это помсет, 
которые содержит начальные состояния для каждого потока программы $P$:
\begin{itemize}
  \item $E_0 {}\defeq{} \set { e_t | t \in \dom{\initst}}$;
  \item $\lab_0(e_t) {}\defeq{} \step{\lTS}{\initst(t)}{\initst(t)}$;
  \item $\ca_0 {}\defeq{} \set{\tup{e,e} ~|~ e \in E_0} $
\end{itemize}
\end{definition}

\begin{lemma}
\label{lm:es-opsem-prefix-clos}
Рассмотрим программу $P$ и модель памяти $M$.
Предположим, что язык $\wmmlang{M}$ является префикс замкнутым, 
то есть $p \in \wmmlang{M}$ и $q \prefle p$ влечет, что $q \in \wmmlang{M}$. 
Тогда для любого $p \in \wmmlang{M}$ верно, что $p_0(P) \esStep{}^* p$.
\end{lemma}

Перейдем к доказательству корректности и полноты полученной операционной семантики.
Для доказательства корректности сперва заметим, что 
построенная структура событий корректна относительно программы $P$.

\begin{lemma}
\label{lm:es-opsem-prog-sound}
Рассмотрим программу $P$ и модель памяти $M$.
Тогда для любого $p$, достижимого из $p_0(P)$,
то есть $p_0(P) \esStep{}^* p$, верно 
что структура событий $S$, порождаемая $p$,
корректна относительно $P$.
\end{lemma}

\begin{theorem}[Корректность]
Рассмотрим программу $P$ и модель памяти $M$.
Пусть $p$ это помсет, построенный по правилу \PorfAddEventRule, 
то есть $p_0(P) \esStep{}^* p$.
Тогда структура событий событий $S(p)$ корректна относительно
аксиоматической семантики программы $P$ в модели $M$:
$$ \ExecGs{\wmmpes{S}{M}} \suq \sem{P}_M. $$
\end{theorem}

\begin{proof}
Доказательтво этой теоремы напрямую следует из
\cref{lm:thrd-es-prog-sound,lm:es-opsem-prog-sound}.
\end{proof}

Для того, чтобы показать полноту, сначала 
докажем наличие у операционной семантики 
другого крайне важного свойства --- \emph{конфлюэнтности}.

\begin{theorem}[Конфлюэнтность]
\label{thm:es-opsem-confluence}
Система переходов для построения структуры событий $\esStep{}$ 
конфлюэнтна в следующем смысле.
Если $p_1 \esStep{} p_2$ и $p_1 \esStep{} p_3$, тогда существует $p_4$, 
такой что $p_2 \esStep{}^? p_4$ и $p_3 \esStep{}^? p_4$.
%
\begin{center}
\begin{tikzpicture}
  \tikzset{
    esstep/.style={->,line width=0.35mm},
  }
  \node (p1) at (1,2) {$p_1$};
  \node (p2) at (0,1) {$p_2$};
  \node (p3) at (2,1) {$p_3$};
  \node (p4) at (1,0) {$p_4$};
  \draw[esstep] (p1) edge (p2);
  \draw[esstep] (p1) edge (p3);
  \draw[esstep] (p2) edge node[pos=0.7,right] {{\tiny?}} (p4);
  \draw[esstep] (p3) edge node[pos=0.7,left ] {{\tiny?}} (p4);
\end{tikzpicture}
\end{center}
%
\end{theorem}

\begin{proof}
Пусть $p_1 \esStep{\tup{e, l, e_{\lPO}, e_{\lRF}}} p_2$ 
и $p_1 \esStep{e', l', e'_{\lPO}, e'_{\lRF}} p_3$.
Если $l = l'$, $e_{\lPO} = e'_{\lPO}$ и $e_{\lRF} = e'_{\lRF}$, 
тогда $p_2$ и $p_3$ изоморфны и, следовательно, равны%
\footnote{Напомним, что здесь рассматриваются помсеты, 
то есть классы эквивалентности помеченных частичных порядоков 
по отношению изоморфизма.}
Иначе, рассмотрим $p_4$, который получается путем добавления 
в $p_2$ события $e'$, то есть $p_2 \esStep{e', l', e'_{\lPO}, e'_{\lRF}} p_4$.
Аналогично, рассмотрим $p'_4$ такой, что $p_3 \esStep{\tup{e, l, e_{\lPO}, e_{\lRF}}} p_4$.
Можно показать, что $p_4$ и $p'_4$ изоморфны, и, следовательно, равны.
\end{proof}

Для доказательства полноты нам также потребуется следующая лемма. 

\begin{lemma}
\label{lm:es-opsem-lang-mon}
Язык, задаваемый структурой событий, 
монотонно растёт при построении структуры 
с помощью системы переходов $\esStep{}$.
$$ p \esStep{} p' \implies \pomlang{S(p)} \suq \pomlang{S(p')} $$
\end{lemma}

\begin{theorem}[Полнота]
\label{thm:es-opsem-completeness}
Рассмотрим программу $P$ и модель памяти $M$,
такую что её язык $\wmmlang{M}$ префикс-замкнут.
Пусть $p$ это терминальный помсет, построенный с помощью операционной семантики, 
то есть $p_0(P) \esStep{}^* p$ и не существует $p'$ такого, что $p \esStep{} p'$.
Тогда структура событий событий $S(p)$ полна относительно
аксиоматической семантики программы $P$ в модели $M$:
$$ \sem{P}_M \suq \ExecGs{\wmmpes{S}{M}}. $$
\end{theorem}

\begin{proof}

Рассмотрим $G \in \sem{P}_M$ и соответствующий ему помсет ${q = \pomg[G]}$. 
Согласно \cref{lm:es-opsem-prefix-clos} можно заключить, что $p_0(P) \esStep{}^* q$.
Также, нетрудно убедиться, что $q \in \pomlang{S(q)}$.
Далee, использую свойство конфлюэнтности (\cref{thm:es-opsem-confluence}),
можно заключить, что существует $p'$, такой что 
$q \esStep{}^* p'$ и $p \esStep{}^* p'$.
По \cref{lm:es-opsem-lang-mon} имеем, что 
${q \in \pomlang{S(q)} \suq \pomlang{S(p')}}$.
Осталось лишь заметить, что так как $p$ терминальный помсет, то $p = p'$.

\begin{center}
\begin{tikzpicture}[scale=1.2]
  \tikzset{
    esstep/.style={->,line width=0.35mm},
  }
  \node (p0) at (1,2) {$p_0$};
  \node (q)  at (0,1) {$q$};
  \node (p)  at (2,0) {$p$};
  \draw[esstep] (p0) edge node[pos=1.15,right] {{\tiny *}} (q);
  \draw[esstep] (p0) edge node[pos= .95,right] {{\tiny *}} (p);
  \draw[esstep] (q)  edge node[pos=1.15,left ] {{\tiny *}} (p);
\end{tikzpicture}
\end{center}

\end{proof}

           
%% \chapter{Модель \Wkm и корректность компиляции}
\label{ch:weakestmo-imm}

Напомним, что модель \Wkm принадлежит классу моделей,
сохраняющих семантические зависимости.
Как уже упоминалось, одним из требований, предъявляемых к таким моделям,
является корректность оптимальной схемы компиляции
в ассемблерный код современных мультипроцессоров,
в частности \Intel~\cite{Sewell-al:CACM10},
\ARM~\cite{Pulte-al:POPL18} и \POWER~\cite{Alglave-al:TOPLAS14}.

В данной главе описывается выполненное в рамках
диссертационного исследования доказательство
о корректности компиляции из модели \Wkm в модели
современных мультипроцессоров.
Данное доказательство вместе с определением модели \Wkm
были формализованы в системе \coq.

В предложенном доказательстве используется \emph{промежуточная модель памяти}
(\emph{intermediate memory model, \IMM})~\cite{Podkopaev-al:POPL19}.
Данная модель является абстракцией над моделями \Intel, \ARM и \POWER,
которая позволяет скрыть низкоуровневые детали этих моделей.
Так как для модели \IMM ранее уже была доказана
корректность компиляции в модели \Intel, \ARM и \POWER~\cite{Podkopaev-al:POPL19}, 
то задача сводится к доказательству корректности компиляции из модели \Wkm в модель \IMM. 

Будем считать, что модель \Wkm и модель \IMM заданы для языка \LLANG,
а оптимальной схемой компиляции положим
тождественное отображение из языка \LLANG в него же.
Тогда требование о корректности компиляции
из модели \Wkm в модель \IMM сводится к следующей теореме. 

\begin{theorem}
  \label{thm:main}
  Пусть $P$ это программа на языке \LLANG
  и пусть $G$ это \IMM-консистентный граф сценария исполнения этой программы.
  Тогда существует \Wkm-консистентная структура событий $S$,
  соответствующая программе $P$,
  которая содержит граф $G$, то есть $S \rhd G$.
\end{theorem}

\section{Схема доказательства теоремы о корректности компиляции}

\eupp{В этом разделе необходимо переформулировать предложения,
  чтобы избежать плагиата с ВКР.}

Для того чтобы доказать теорему о корректности компиляции
в данной работе был предложен способ построения
необходимой структуры событий с помощью метода
\emph{симуляции}~\cite{Milner:1971}.
А именно, построение структуры событий происходит
инкрементально шаг за шагом с помощью операционной семантики
%% (\todo{ссылка на раздел про оп.сем.})
путем симуляции \emph{обхода \IMM графа}
(\emph{\IMM graph traversal})~\cite[\S6,7]{Podkopaev-al:POPL19}.
Обход графа соответствует некоторому сценарию выполнения программы,
в рамках которого события исполняются согласно
сохраняемому программному порядку ($\lPPO$).

Более формально, обход графа $G$
порождает операционную семантику малого шага 
$G \vdash \TC \travstep{e} \TC'$ где $\TC$ и $\TC'$ это
\emph{конфигурации обхода}.
Конфигурация обхода, в свою очередь, это пара $\tup{C, I}$,
где $C \suq G.\lE$ это множество \emph{покрытых событий}
(\emph{covered events}), а $I \suq G.\lW$ это множество
\emph{выпущенных записей} (\emph{issued writes}).
Покрытие события соответствует выполнению
инструкции программы в обычном порядке,
в то время как выпуск события записи соответствует
спекулятивному исполнению инструкции записи вне очереди.
Событие может быть покрыто если (i) все его
$\lPO$ предшественники уже покрыты, и (ii)
это событие уже выпущено (в случае события записи)
либо оно читает из уже выпущенной записи (в случае события чтения).
Событие записи может быть выпущено если все
события записи из других потоков, от которых
зависит данное событие согласно отношению $\lPPO$, также уже выпущены. 
Эти требования могут быть выражены как следующие инварианты конфигурации обхода.

\[\def\arraystretch{1}
\begin{array}{c@{\qquad}c@{\qquad}c@{\qquad}c}
 \dom{\lPO \seqc [C]} \subseteq C  &
 C \cap \lW \subseteq I             &
 \dom{\lRF \seqc [C]} \subseteq I  &
 \dom{\lRFE \seqc \lPPO \seqc [I]} \subseteq I
\end{array}
\]

Имея операционную семантику обхода \IMM графа $G \vdash \TC \travstep{e} \TC'$
и операционную семантику построения структуры событий $S \esstepcons{e} S'$
доказательство строится на основе метода \emph{симуляции}~\cite{Milner:1971}.
А именно, определяется отношение симуляции $\simrel(P, T, G, \TC, S, X)$,
соединяющее программу $P$, подмножество идентификаторов потоков $T \suq \Tid$,
\IMM граф $G$ и текущую конфигурацию его обхода $\TC$,
текущую структуру событий $S$ и его выделенную конфигурацию $X$.
Далее доказательство строится на следующих трех леммах,
которые утверждают что 
(i) начальная конфигурация обхода и инициализирующая структура событий
связаны отношением симуляции, (ii) каждый шаг обхода графа
может быть симулирован соответствующим шагом построения структуры событий,
(iii) из конечной структуры событий, которая соответствует конечной
конфигурации обхода, может быть извлечен требуемый \IMM граф $G$.

\begin{lemma}[Начало симуляции]
  \label{lm:simstart}
  \quad\\
  Пусть $P$ это программа на языке \LLANG,
  а $G$ это соответствуюий ей \IMM консистентный граф.
  Тогда выполняется $\simrel(P, \lTID(P), G, \TCinit{G}, \ESinit(P), \lEi)$ где
  \begin{itemize}
    \item $\lTID(P)$ --- множество всех идентификаторов потоков программы~$P$;
    \item $\TCinit{G} \defeq \tup{\lEi, \lEi}$ --- это начальная конфигурация обхода,
      содержащая только инициализирующие события;
    \item $\ESinit(P)$ --- это начальная структура событий,
      также содержащая только инициализирующие события.
  \end{itemize}
\end{lemma}

\begin{lemma}[Шаг симуляции]
  \label{lm:simstep}
  \quad\\
  Если выполняется $\simrel(P, T, G, \TC, S, X)$ и ${G \vdash \TC \travstep{} \TC'}$,
  тогда существует $S'$ и $X'$, такие что выполняется
  $\simrel(P, T, G, \TC', S', X')$ и $S \esstepcons{}^* S'$.
\end{lemma}

\begin{lemma}[Окончание симуляции]
  \label{lm:simend}
  \quad\\
  Если выполняется $\simrel(P, \lTID(P), G, \TCfinal{G}, S, X)$,
  где $\TCfinal{G} \defeq \tup{G.\lE, G.\lE}$ --- это конечная
  конфигурация обхода графа, тогда граф сценария исполнения,
  порождаемый конфигурацией $X$, изоморфен $G$,
  или, другими словами, $G$ может быть извлечен из $S$:~~$S \rhd G$.
\end{lemma}

Доказательство теоремы \ref{thm:main}
проводится методом индукции по трассе обхода графа
$G \vdash \TCinit{G} \travstep{}^* \TCfinal{G}$.
Лемма \ref{lm:simstart} используется в качестве базы индукции,
лемма \ref{lm:simstep} --- это шаг индукции,
а лемма \ref{lm:simend} завершает доказательство.

В свою очередь, доказательства лемм \ref{lm:simstart} и \ref{lm:simend}
достаточно прямолинейны (найти их можно в \coq репозитории).
Основная сложность заключена в доказательстве леммы \ref{lm:simstep}.
Более детально это доказательство рассматривается~в~разделе~\cref{sec:simstep}.

\section{Модель \IMM и обход графа сценария исполнения}

В этом разделе приводится формальное 
определение модели памяти \IMM~\cite{Podkopaev-al:POPL19},
а также операционной семантики обхода графов 
сценариев исполнения в модели~\IMM.

\subsection*{Модель \IMM}

Модель \IMM относится к классу моделей, сохраняющих синтаксические зависимости. 
В рамках данных моделей, как правило, 
определяется отношение \emph{сохраняемого программного порядка}
(\emph{preserved program order}) $\lPPO$, 
являющееся подмножеством обычного программного порядка. 
События, связанные сохраняемым программным порядком, 
должны выполнятся согласно этому порядку, 
а несвязанные события могут выполняться в произвольном порядке. 
 
Рассмотрим программы \ref{ex:LB-nodep}, 
\ref{ex:LB-fakedep} и \ref{ex:LB-dep} показанные ниже.

\begin{center}
\begin{minipage}{.32\linewidth}
{\small
\begin{equation}
\inarrII{
  \readInst{}{a}{x} \rfcomment{1} \\
  \writeInst{}{y}{1} \\
}{\readInst{}{b}{y} \rfcomment{1} \\
  \writeInst{}{x}{b}  \\
}%
\tag{LB-nodep}\label{ex:LB-nodep}
\end{equation}
}
\end{minipage}
%
\hfill\vline\hfill
\begin{minipage}{.32\linewidth}
{\small
\begin{equation}
\inarrII{
  \readInst{}{a}{x} \rfcomment{1} \\
  \writeInst{}{y}{1 + a * 0} \\
}{\readInst{}{b}{y} \rfcomment{1} \\
  \writeInst{}{x}{b}  \\
}
\tag{LB-fakedep}\label{ex:LB-fakedep}
\end{equation}
}
\end{minipage}
%
\hfill\vline\hfill
%
\begin{minipage}{.32\linewidth}
{\small
\begin{equation}
\inarrII{
  \readInst{}{a}{x} \nocomment{1} \\
  \writeInst{}{y}{a} \\
}{\readInst{}{b}{y} \nocomment{1} \\
  \writeInst{}{x}{b}  \\
}
\tag{LB-dep}\label{ex:LB-dep}
\end{equation}
}
\end{minipage}
\end{center}


\eupp{В этом разделе далее необходимо переформулировать предложения,
  чтобы избежать плагиата с ВКР.}

Модель \IMM допускает сценарий исполнений 
с результатом $a=b=1$ для программы \ref{ex:LB-nodep}, 
но не для \ref{ex:LB-fakedep} и \ref{ex:LB-dep}.
Соответствующий этому сценарию граф для 
программы \ref{ex:LB-nodep} показан на~\cref{fig:LB-nodep-ppo-exec},
а для для \ref{ex:LB-fakedep} и \ref{ex:LB-dep} на~\cref{fig:LB-dep-ppo-exec}.
Заметим, что в графе, показанном на~\cref{fig:LB-nodep-ppo-exec}, 
события в левом потоке не связаны отношением $\lPPO$.
В графе, показанном на~~\cref{fig:LB-nodep-ppo-exec}, напротив, 
события в обоих потоках связанны отношением $\lPPO$,
так как между соответствющими инструкциями 
есть \emph{зависимость по данным}.
Кроме того, объединение отношений $\lPPO$ и $\lRFE$ образует цикл. 
Именно из-за наличия этого цикла данный 
граф считается неконсистентным с точки зрения модели~\IMM.

{
\newcommand{\XScale}{1}
\newcommand{\YScale}{0.7}

\begin{figure}[b]
  \begin{subfigure}[b]{.44\textwidth}\centering
  \begin{tikzpicture}[xscale=\XScale,yscale=\YScale]

  %% \node at (0,1.5) {$\circledb{A}$};

  \node (init) at (1,  1.5) {$\Init$};
  \node (i11) at ( 0,  0) {$\rlab{}{x}{1}{}$};
  \node (i12) at ( 0, -2) {$\wlab{}{y}{1}{}$};
  \node (i21) at ( 2,  0) {$\rlab{}{y}{1}{}$};
  \node (i22) at ( 2, -2) {$\wlab{}{x}{1}{}$};

  \draw[po] (i11) edge node[right] {\small$\lPO$} (i12);
  \draw[po] (i21) edge node[left ] {\small$\lPO$} (i22);
  \draw[ppo,bend left=10] (i21) edge node[right] {\small$\lPPO$} (i22);

  \draw[rf] (i22) edge node[below,pos=0.5] {}             (i11);
  \draw[rf] (i12) edge node[below,pos=0.5] {\small$\lRF$} (i21);

  \draw[po] (init) edge node[left]  {\small$\lPO$} (i11);
  \draw[po] (init) edge node[right] {\small$\lPO$} (i21);
  \end{tikzpicture}
  \caption{Граф соответствующий программе \ref{ex:lb-nodep}.}
  \label{fig:LB-nodep-ppo-exec}
  \end{subfigure}\hfill
  %
  \begin{subfigure}[b]{.55\textwidth}\centering
  \begin{tikzpicture}[xscale=\XScale,yscale=\YScale]

  %% \node at (0,1.5) {$\circledb{B}$};

  \node (init) at (1,  1.5) {$\Init$};
  \node (i11) at ( 0,  0) {$\rlab{}{x}{1}{}$};
  \node (i12) at ( 0, -2) {$\wlab{}{y}{1}{}$};
  \node (i21) at ( 2,  0) {$\rlab{}{y}{1}{}$};
  \node (i22) at ( 2, -2) {$\wlab{}{x}{1}{}$};

  \draw[po] (i11) edge node[right] {\small$\lPO$} (i12);
  \draw[po] (i21) edge node[left ] {\small$\lPO$} (i22);
  \draw[ppo,bend right=10] (i11) edge node[left ] {\small$\lPPO$} (i12);
  \draw[ppo,bend left =10] (i21) edge node[right] {\small$\lPPO$} (i22);

  \draw[rf] (i22)  edge node[below]         {}             (i11);
  \draw[rf] (i12)  edge node[below,pos=0.5] {\small$\lRF$} (i21);

  \draw[po] (init) edge node[left]  {\small$\lPO$} (i11);
  \draw[po] (init) edge node[right] {\small$\lPO$} (i21);
  \end{tikzpicture}
  \caption{Граф соответствующий программам 
    \ref{ex:lb-fakedep}~и~\ref{ex:lb-dep}.
  }
  \label{fig:LB-dep-ppo-exec}
  \end{subfigure}

\caption{Графы сценариев исполнения обосновывающие результат ${a=b=1}$.}
\label{fig:LB-ppo-execs}
\end{figure}
}


Далее приводится формальное определение понятия 
зависимостей и модели \IMM.

\begin{definition}
  \label{def:imm-exec-graph}
  Графом сценария исполнения в модели \IMM называется 
  обычный граф сценария исполнения, дополненный отношениями 
  \emph{зависимости по данным} (\emph{data dependency}) $\lDATA$, 
  \emph{зависимости по потоку управления} (\emph{control dependency}) $\lCTRL$, 
  и \emph{зависимости по целевому адресу} (\emph{address dependency}) $\lADDR$, 
  и \emph{зависимость по операции \CAS} (\emph{\CAS dependency}) $\lRMWDEP$.
  Объединенное отношение \emph{зависимости} (\emph{dependency}) 
  определяется следующим образом: 
  $$ \lDEPS \defeq \lDATA \cup \lCTRL \cup \lADDR \seqc \lPO^? \cup \lRMWDEP. $$
  В контексте модели \IMM под графом сценария исполнения будем 
  подразумевать граф, дополненный отношениями зависимости. 
\end{definition}

\begin{definition}
  \label{def:imm-aux-rel}
  В модели \IMM для графа $G$ вводятся следующием производные отношения%
  \footnote{Подробное описание приведенных здесь отношений может 
   быть найдено в~\cite{Podkopaev-al:POPL19,Moiseenko-al:ECOOP20}}.

  \begin{itemize}

    \item Отношение \emph{порядка барьеров} (\emph{barrier-order-before}):
      $$ \lBOB \defeq \lPO \seqc [\lW^{\rel\squq}] \cup 
                      [\lR^{\acq\squq}] \seqc \lPO \cup 
                      \lPO \seqc [\lF] \cup [\lF] \seqc \lPO \cup 
                      [\lW^{\rel\squq}] \seqc \lPO_{\lLOC} \seqc [\lW]. $$

    \item Отношение \emph{сохраняемого программного порядка} 
      (\emph{preserved program order}):
      $$ \lPPO \defeq [\lR] \seqc (\lDEPS \cup \lRFI)^+ \seqc [W] $$

    \item Отношение \emph{обхода} (\emph{detour}):
      $$ \lDETOUR \defeq (\lCOE \seqc \lRFE) \cap \lPO. $$

    \item Отношение \emph{синхронизируется-с} (\emph{synchronizes-with}):
     $$ \lSW  \defeq [\lE^{\rel\squq}]             \seqc 
                     ([\lF] \seqc \lPO)^?           \seqc 
                     ([\lW] \seqc \lPO_{\lLOC})^?   \seqc
                     (\lRF \seqc \lRMW)^*           \seqc 
                     \lRF \seqc (\lPO \seqc [\lF])^? \seqc 
                     [\lE^{\acq\squq}]. 
     $$

    \item Отношение \emph{произошло-до} (\emph{happens-before}):
      $$ \lHB \defeq (\lPO \cup \lSW)^+. $$

    \item Отношение \emph{читает-до} (\emph{reads-before} или \emph{from-reads}):
      $$ \lFR \defeq \lRF^{-1} \seqc \lCO. $$

    \item \emph{Расширенный порядок когерентности} 
      (\emph{extended coherence order}):
      $$ \lECO \defeq (\lCO \cup \lRF \cup \lFR)^+. $$

    \item Отношение \emph{последовательно-упорядочен-до}
      (\emph{sequentially consistent before}):
      $$ \lSCB \defeq \lPO \cup
                      \lPO\rst{\neq \lLOC} \seqc \lHB \seqc 
                      \lPO\rst{\neq \lLOC} \cup
                      \lHB\rst{\lLOC} \cup
                      \lCO \cup \lFR. $$

    \item Частиный порядок \emph{базовой последовательной упорядоченности}:
      $$ \lPSCB \defeq ([\lE^\sco] \cup [\lF^\sco] \seqc \lHB^?) \seqc 
                         \lSCB \seqc 
                       ([\lE^\sco] \cup \lHB^?\seqc[\lF^\sco]). 
      $$ 

    \item Частиный порядок \emph{последовательной упорядоченности барьеров}:
      $$ \lPSCF \defeq [\lF^\sco] \seqc 
                       (\lHB \cup \lHB \seqc \lECO \seqc \lHB) \seqc 
                       [\lF^\sco]. 
      $$ 

    \item Частиный порядок \emph{последовательной упорядоченности}:
      $$ \lPSC \defeq \lPSCB \cup \lPSCF. $$ 

    \item Вспомогательное \emph{ацикличное} отношение (\emph{acyclic relation}):
      $$ \lAR \defeq \lRFE \cup \lBOB \cup \lPPO \cup \lDETOUR \cup \lPSCF. $$
    
  \end{itemize}

\end{definition}

\begin{definition}
  \label{def:imm-cons}
  Граф $G$ является консистентным с точки зрения \IMM 
  если выполняются следующие условия:
  
  \begin{itemize}

    \item $\lR \suq \cod{\lRF}$;
      \labelAxiom{$\lRF$-completeness}{ax:rf-complete}

    \item $\lAR$ ациклично;
      \labelAxiom{imm-no-thin-air}{ax:imm-noota}

    \item $\lHB_{\IMM} \seqc \lECO^?$ иррефлексивно;
      \labelAxiom{imm-coherent}{ax:imm-coh}

    \item $\lRMW \cap (\lFR \seqc \lCO) = \emptyset$;
      \labelAxiom{rmw-atomic}{ax:imm-atom}

    \item $\lPSC$ ациклично.
      \labelAxiom{imm-sequential-consistency}{ax:imm-sc}

  \end{itemize}
\end{definition}

\TODO{Семантика обхода графа}

\section{Симуляция обхода графа \IMM}

\eupp{В этом разделе необходимо переформулировать предложения,
  чтобы избежать плагиата с ВКР.}

В данном разделе приводится более подробный обзор
процесса симуляции построения структуры событий по обходу \IMM графа.
В разделе \ref{sec:simrel} описывается
отношение симуляции $\simrel$, 
а в разделе \ref{sec:simstep} на примере графа, 
изображенного на \cref{fig:lb-sim-ex},
рассматривается процесс симуляции шага обхода \IMM графа
семантикой построения структуры событий. 

\begin{figure}[h]
\hfill$\inarrII{
  \readInst{}{r_1}{x} \rfcomment{1} \\[1mm]
  \writeInst{}{y}{r_1}              \\[1mm]
  \writeInst{}{z}{1}                \\
}{
  \readInst{}{r_2}{y} \rfcomment{1} \\[1mm]
  \readInst{}{r_3}{z} \rfcomment{1} \\[1mm]
  \writeInst{}{x}{r_3}              \\
}$
\hfill\vrule\hfill
$\inarr{\begin{tikzpicture}[xscale=1,yscale=1.5]
  \node (init) at (2,  1)  {$\Init$};
  \node (i11)  at (0,  0)   {$\mese{1}{1}{} \rlab{}{x}{1}$};
  \node (i12)  at (0, -1)   {$\mese{1}{2}{} \wlab{}{y}{1}$};
  \node (i13)  at (0, -2)   {$\mese{1}{3}{} \wlab{}{z}{1}$};
  \node (i21)  at (4,  0)   {$\mese{2}{1}{} \rlab{}{y}{1}$};
  \node (i22)  at (4, -1)   {$\mese{2}{2}{} \rlab{}{z}{1}$};
  \node (i23)  at (4, -2)   {$\mese{2}{3}{} \wlab{}{x}{1}$};
  \draw[rf] (i13) edge node[below] {\small$\lRF$} (i22);
  \draw[rf] (i23) edge node[pos=.4,above] {\small\phantom{j}$\lRF$} (i11);
  \draw[rf] (i12) edge node[above] {\small$\lRF$} (i21);
  \draw[ppo,out=230,in=130] (i11) edge node[left ,pos=0.8] {\small$\lPPO$} (i12);
  \draw[ppo,out=310,in=50 ] (i22) edge node[right,pos=0.3] {\small$\lPPO$} (i23);
  \draw[po] (init) edge (i11);
  \draw[po] (init) edge (i21);
  \draw[po] (i11)  edge (i12);
  \draw[po] (i12)  edge (i13);
  \draw[po] (i21)  edge (i22);
  \draw[po] (i22)  edge (i23);
\end{tikzpicture}}$
\caption{Пример программы и соответствующий ей \IMM граф}
\label{fig:lb-sim-ex}
\end{figure}


\subsection*{Отношение симуляции}
\label{sec:simrel}

Далее опишем отношение симуляции $\simrel$.
В целях ясности и простоты изложения, 
в данном разделе будет приведена упрощенная версия формального
определения этого отношения, которая опускает некоторые
технические детали. Полная версия отношения симуляции
может быть найдена в \coq репозитории. 

Отношение симуляции $\simrel(P, T, G, TC, S, X)$
устанавливает взаимосвязь между структурой событий $S$
и графом сценария исполнения $G$ с помощью
функции $\ea : S.\lE \fun G.\lE$, которая отображает
события структуры $S$ в события графа $G$.
Эта функция может быть натуральным образом
расширена на множества событий следующим образом%
\footnote{аналогично образом функций $\ea$ может быть расширена
на бинарные отношения на событиях.}:

\begin{align*}
\text{for } A_S \subseteq S.\lE        & :
  \fmap{A_S} \defeq \set{\ea(e) \in G.\lE \mid e \in A_S} \\
\text{for } A_G \subseteq G.\lE        & :
  \fcomap{A_G} \defeq \set{e \in S.\lE \mid \ea(e) \in A_G}.
\end{align*}

Отношение симуляции $\simrel(P, T, G, \TC, S, X)$ состоит из следующих свойств.

\begin{enumerate}

  \item \label{simrel:events}
    События $S$, принадлежащие потокам из $T$, а также события,
    принадлежащие конфигурации $X$, соответствуют покрытым событиям,
    а также выпущенным событиям и их $\lPO$-предшественникам: 
    \begin{itemize}
      \item $\fmap{S.\lE\rst{T}} = \fmap{X} = C \cup \dom{G.\lPO^? \seqc [I]}$
    \end{itemize}

  \item \label{simrel:lab}
    Метки событий из $S$ совпадают с метками событий из $G$
    по модулю прочитанных или записанных значений. 
    \begin{enumerate}
      \setcounter{enumii}{0}
      \item \label{simrel:lab-eqmval}
        $\forall e \in S.\lE \ldotp\;
          S.\set{\lTID, \lTYP, \lLOC, \lMOD}(e) =
          G.\set{\lTID, \lTYP, \lLOC, \lMOD}(\fmap{e}) $
    \end{enumerate}
    Метки покрытых и выпущенных событий, принадлежащих конфигурации $X$,
    сопадают полностью.
    \begin{enumerate}
      \setcounter{enumii}{1}
      \item \label{simrel:lab-det}
        $\forall e \in X \cap \fcomap{C \cup I} \ldotp~
          S.\lVAL(e) = G.\lVAL(\ea(e))$
    \end{enumerate}

  \item \label{simrel:po}
    Программный порядок в структуре событий $S$
    совпадает с программным порядком в графе $G$:
    \begin{itemize}
      \item $\fmap{S.\lPO} \suq G.\lPO$
    \end{itemize}

  \item \label{simrel:cf}
    Если два события имеют одинаковый образ под действием функции $\ea$,
    то эти события равны или находятся в конфликте.
    \begin{itemize}
      \item $\fcomap{\mathtt{id}} \suq S.\lCF^?$
    \end{itemize}

  \item \label{simrel:jf}
    События чтения в $S$ должны быть обоснованы событиями записи,
    которые наблюдаются соответствующим событием чтением в $G$.
    \begin{enumerate}
      \item \label{simrel:jf-obs}
      \setcounter{enumii}{0}
        $\fmap{S.\lJF} \suq G.\lRF^?\seqc G.\lHB^?$
    \end{enumerate}
    Более того, отношение $\lJF$, ограниченное на события чтения,
    принадлежащие конфигурации $X$, соответствуют
    отношению \emph{стабильной обоснованности} (\emph{stable justification})
    (смотри \cref{def:sjf}) в графе $G$.
    \begin{enumerate}
      \setcounter{enumii}{1}
      \item \label{simrel:jf-sjf}
        $\fmap{S.\lJF \seqc [X]} \suq G.\lSRF_{TC}$
    \end{enumerate}
    %% As a consequence it is possible to derive that
    %% justification for covered events in $X$
    %% corresponds to their justification in $G$:
    %% $\fmap{S.\lJF \seqc [X \cap \fcomap{C}]} \subseteq G.\lRF$. \\
    Только выпущенные события могут быть использованы для
    внешнего обоснования событий чтения. 
    \begin{enumerate}
      \setcounter{enumii}{2}
      \item \label{simrel:jfe-iss}
         $\dom{S.\lJFE} \suq \dom{S.\lEW \seqc [X \cap \fcomap{I}]}$
    \end{enumerate}

  \item \label{simrel:ew}
    Все эквивалентные события записи в $S$ отображаются
    в одно и то же событие записи $G$.
    \begin{enumerate}
      \setcounter{enumii}{0}
      \item \label{simrel:ew-id}
        $\fmap{S.\lEW} \suq \mathtt{id}$
    \end{enumerate}
    Также каждый класс эквивалентности по отношению $S.\lEW^*$
    должен иметь представителя среди выпущенных событий,
    принадлежащих конфигурации $X$.
    \begin{enumerate}
      \setcounter{enumii}{1}
      \item \label{simrel:ew-iss}
        $S.\lEW \suq (S.\lEW \seqc [X \cap \fcomap{I}] \seqc S.\lEW)^?$
    \end{enumerate}

  \item \label{simrel:co}
    Если два события структуры $S$ находящихся в отношении когерентности,
    то их образы под действием функции либо также находятся
    в отношении когерентности, либо равны. 
    \begin{enumerate}
      \setcounter{enumii}{0}
      \item \label{simrel:co-co}
         $\fmap{S.\lCO} \suq G.\lCO^?$
    \end{enumerate}
    Если же ребро отношения когерентности оканчивается
    в событии, принадлежащем конфигурации $X$ и одному из потоков из $T$,
    тогда образ этого ребра принадлежит отношению когерентности в графе $G$.
    \begin{enumerate}
      \setcounter{enumii}{1}
      \item \label{simrel:co-cfg}
         $\fmap{S.\lCO \seqc [X\rst{T}]} \suq G.\lCO$
    \end{enumerate}

  \item \label{simrel:sw-hb}
    Отношения ``синхронизируется-с'' и ``происходит-до''
    в структуре событий $S$ согласованы с соответствующими
    отношениями в графе $G$.
    \begin{enumerate}
      \item \label{simrel:sw}
        $\fmap{S.\lSW} \suq G.\lSW$
      \item \label{simrel:hb}
        $\fmap{S.\lHB} \suq G.\lHB$
    \end{enumerate}
\end{enumerate}

\begin{figure}[h]
$\hfill\inarr{\begin{tikzpicture}[xscale=1,yscale=1.5]
  \node (init) at (2,  1)   {$\Init$};
  \node (i11)  at (0,  0)   {$\mese{1}{1}{} \rlab{}{x}{1}$};
  \node (i12)  at (0, -1)   {$\mese{1}{2}{} \wlab{}{y}{1}$};
  \node (i13)  at (0, -2)   {$\mese{1}{3}{} \wlab{}{z}{1}$};
  \node (i21)  at (4,  0)   {$\mese{2}{1}{} \rlab{}{y}{1}$};
  \node (i22)  at (4, -1)   {$\mese{2}{2}{} \rlab{}{z}{1}$};
  \node (i23)  at (4, -2)   {$\mese{2}{3}{} \wlab{}{x}{1}$};
  %% \node (hh)   at (2, -3.5) {$\inarrC{\text{The execution graph } G \text{ and} \\\text{its traversal configuration } \TCa}$};
  \begin{scope}[on background layer]
     \issuedCoveredBox{init};
     \issuedBox{i13};
%     \issuedBox{i23};
  \end{scope}
  \draw[rf] (i13) edge node[above] {} (i22);
  \draw[rf] (i23) edge node[above] {} (i11);
  \draw[rf] (i12) edge node[above] {} (i21);
% \draw[vf] (init) edge[bend right=20]  node[above left, pos=0.9] {$\lVF$} (i11);
% \draw[vf] (init) edge[bend left=20]  node[above right, pos=0.9] {$\lVF$} (i21);
% \draw[vf] (i13)  edge[bend right=20] node[above] {$\lVF$} (i22);
  \draw[ppo,out=230,in=130] (i11) edge node[left ,pos=0.8] {\small$\lPPO$} (i12);
  \draw[ppo,out=310,in=50 ] (i22) edge node[right,pos=0.3] {\small$\lPPO$} (i23);
  \draw[po] (init) edge (i11);
  \draw[po] (init) edge (i21);
  \draw[po] (i11)  edge (i12);
  \draw[po] (i12)  edge (i13);
  \draw[po] (i21)  edge (i22);
  \draw[po] (i22)  edge (i23);
\end{tikzpicture}}
\hfill\vrule\hfill
\inarr{\begin{tikzpicture}[xscale=1,yscale=1.5]

  \node (init) at (0, 1)   {$\Init$};

  \node (i111) at (-1.5,  0)   {$\mese{1}{1}{1} \rlab{}{x}{0}$};
  \node (i121) at (-1.5, -1)   {$\mese{1}{2}{1} \wlab{}{y}{0}$};
  \node (i131) at (-1.5, -2)   {$\mese{1}{3}{1} \wlab{}{z}{1}$};

  \node (i211) at (0.5,  0)   {\phantom{$\mese{2}{1}{1} \rlab{}{y}{0}$}};
  \node (i221) at (0.5, -1)   {\phantom{$\mese{2}{2}{1} \rlab{}{z}{1}$}};
  \node (i231) at (0.5, -2)   {\phantom{$\mese{2}{3}{1} \wlab{}{x}{1}$}};

  \draw[jf] (init) edge[bend right] node[above]        {\small{$\lJF$}} (i111);

  \draw[po] (init)  edge (i111);
  \draw[po] (i111)  edge (i121);
  \draw[po] (i121)  edge (i131);

  \begin{scope}[on background layer]
    \draw[extractStyle] (-3, 1.5) rectangle (1,-2.5);
  \end{scope}

  %% \node (hh) at (0, -3.5) {$\inarrC{\text{The event structure } \ESa \text{ and} \\\text{the selected execution } \SXa}$};
\end{tikzpicture}}\hfill$
\caption{%
Граф сценария исполнения $G$, 
конфигурация обхода $\TC_a$
и соответствующая этой конфигурации
структура событий $S_a$ вместе с конфигурацией $X_a$.
Покрытые события выделены как 
{\protect\tikz \protect\draw[coveredStyle] (0,0) rectangle ++(0.35,0.35);}
, а выпущенные как
{\protect\tikz \protect\draw[issuedStyle] (0,0) rectangle ++(0.35,0.35);}.
События, принадлежащие конфигурации $X_a$, выдены как 
{\protect\tikz \protect\draw[extractStyle] (0,0) rectangle ++(0.35,0.35);}.
}
\label{fig:lb-sim-ex-travA}
\end{figure}


\TODO{пример}

\subsection*{Шаг симуляции}
\label{sec:simstep}

В данном разделе приводится схема доказательства леммы \ref{lm:simstep}.
А именно, будет показано каким образом
операционная семантика построения структуры событий
симулирует шаг обхода графа \IMM.

Предположим, что для некоторых $P$, $G$, $\TC$, $S$ и $X$
выполняется отношение симуляции $\simrel(P, G, \TC, S, X)$.
Также положим, что в рамках обхода выполняется шаг
$G \vdash \TC \travstep{} \TC'$, который покрывает
или выпускает событие из потока с идентификатором $t$.
По условиям леммы \ref{lm:simstep} требуется предъявить
структуру $S'$ и конфигурацию $X'$,
такие что выполняется $\simrel(P, G, \TC', S', X')$.
Если поток $t$ содержит ещё непокрытые, но уже
выпущенные события записи, то необходимо выполнить
несколько шагов для построения из структуры $S$ структуры $S'$
чтобы добавить все события, $\lPO$-предшествующие непокрытым
событиям записи в потоке $t$.
Будем называть множество этих событий
\emph{сертификационной веткой},
а процесс добавление этих событий --- \emph{сертификацией}.

\begin{figure}[h]
$\hfill\inarr{\begin{tikzpicture}[xscale=1,yscale=1.5]
  \node (init) at (2,  1)   {$\Init$};
  \node (i11)  at (0,  0)   {$\mese{1}{1}{} \rlab{}{x}{1}$};
  \node (i12)  at (0, -1)   {$\mese{1}{2}{} \wlab{}{y}{1}$};
  \node (i13)  at (0, -2)   {$\mese{1}{3}{} \wlab{}{z}{1}$};
  \node (i21)  at (4,  0)   {$\mese{2}{1}{} \rlab{}{y}{1}$};
  \node (i22)  at (4, -1)   {$\mese{2}{2}{} \rlab{}{z}{1}$};
  \node (i23)  at (4, -2)   {$\mese{2}{3}{} \wlab{}{x}{1}$};
  %% \node (hh) at (2, -3.5) {$\inarrC{\text{The traversal configuration } \TCb}$};
  \begin{scope}[on background layer]
     \issuedCoveredBox{init};
     \issuedBox{i13};
     \issuedBox{i23};
  \end{scope}
  \draw[rf] (i13) edge node[above] {} (i22);
  \draw[rf] (i23) edge node[above] {} (i11);
  \draw[rf] (i12) edge node[above] {} (i21);
  \draw[vf] (init) edge[bend right=20]  node[above left, pos=0.9] {$\lVF$} (i11);
  \draw[vf] (init) edge[bend left=20]  node[above right, pos=0.9] {$\lVF$} (i21);
  \draw[vf] (i13)  edge[bend right=20] node[above] {$\lVF$} (i22);
  \draw[ppo,out=230,in=130] (i11) edge node[left ,pos=0.8] {\small$\lPPO$} (i12);
  \draw[ppo,out=310,in=50 ] (i22) edge node[right,pos=0.3] {\small$\lPPO$} (i23);
  \draw[po] (init) edge (i11);
  \draw[po] (init) edge (i21);
  \draw[po] (i11)  edge (i12);
  \draw[po] (i12)  edge (i13);
  \draw[po] (i21)  edge (i22);
  \draw[po] (i22)  edge (i23);
\end{tikzpicture}}
\hfill\vrule\hfill
\inarr{\begin{tikzpicture}[xscale=1,yscale=1.5]

  \node (init)  at (0, 1)      {$\Init$};

  \node (i111)  at (-1.5,  0)  {$\mese{1}{1}{1} \rlab{}{x}{0}$};
  \node (i121)  at (-1.5, -1)  {$\mese{1}{2}{1} \wlab{}{y}{0}$};
  \node (i131)  at (-1.5, -2)  {$\mese{1}{3}{1} \wlab{}{z}{1}$};

  \node (i211)  at (1.5,  0)   {$\mese{2}{1}{1} \rlab{}{y}{0}$};
  \node (i221)  at (1.5, -1)   {$\mese{2}{2}{1} \rlab{}{z}{1}$};
  \node (i231)  at (1.5, -2)   {$\mese{2}{3}{1} \wlab{}{x}{1}$};

  \draw[jf] (init) edge[bend right] node[above]        {\small{$\lJF$}} (i111);
  \draw[jf] (init) edge[bend left ] node[above]        {\small{$\lJF$}} (i211);
  \draw[jf] (i131) edge             node[pos=.5,below] {\small{$\lJF$}} (i221);

  \draw[po] (init)  edge (i111);
  \draw[po] (i111)  edge (i121);
  \draw[po] (i121)  edge (i131);

  \draw[po] (init)  edge (i211);
  \draw[po] (i211)  edge (i221);
  \draw[po] (i221)  edge (i231);

  \begin{scope}[on background layer]
    \draw[extractStyle] (-3, 1.5) rectangle (3,-2.5);
  \end{scope}

  %% \node (hh) at (0, -3.5) {$\inarrC{\text{The event structure } \ESb \text{ and} \\\text{the selected execution } \SXb}$};
\end{tikzpicture}}\hfill$
\caption{%
Граф сценария исполнения $G$, 
конфигурация обхода~$\TC_b$
и соответствующая этой конфигурации
структура событий~$S_b$ вместе с конфигурацией~$X_b$.
%% Покрытые события выделены как 
%% {\protect\tikz \protect\draw[coveredStyle] (0,0) rectangle ++(0.35,0.35);}
%% , а выпущенные как
%% {\protect\tikz \protect\draw[issuedStyle] (0,0) rectangle ++(0.35,0.35);}.
%% События, принадлежащие конфигурации $X_b$, выдены как 
%% {\protect\tikz \protect\draw[extractStyle] (0,0) rectangle ++(0.35,0.35);}.
}
\label{fig:lb-sim-ex-travB}
\end{figure}


Рассмотрим процесс построения сертификационной ветки
на примере шага обхода из конфигурации $\TC_a$ (\cref{fig:lb-sim-ex-travA})
в конфигурацию $\TC_b$ (\cref{fig:lb-sim-ex-travB})
путем выпуска события $\ese{2}{3}{}$.
Чтобы симулировать этот шаг, необходимо выполнить инструкции правого потока
и добавить в структуру событий ветку
$\Br_b = \set{\ese{2}{1}{1},\ese{2}{2}{1},\ese{2}{3}{1}}$
(смотри \cref{fig:lb-sim-ex-travB}).
Для того чтобы добавить эти события, в свою очередь,
выполняется построение трассы операционной семантики потока
${\state \thrdstep{\ese{2}{1}{1}}
         \thrdstep{\ese{2}{2}{1}}
         \thrdstep{\ese{2}{3}{1}}
         \state'}$, 
такой что
(i) она содержит все события правого потока
вплоть до последнего выпущенного события записи $\ese{2}{3}{}$ в графе $G$,
(ii) все эти события должны иметь тот же идентификатор потока,
тип обращения и локацию как и соответствующие события в графе
(то есть $\ese{2}{1}{}, \ese{2}{2}{}, \ese{2}{3}{}$),
(iii) все события, соответствующие покрытым и выпущенным событиям
(в данном случае~$\ese{2}{3}{1}$) должны иметь то же значение,
что и в графе $G$.
Для построения этой трассы используется свойство
\emph{восприимчивости} (\emph{receptiveness})
операционной семантики потока.
Это свойство позволяет выбрать произвольные значения
для всех промежуточных событий чтения в конструируемой трассе,
от которых не зависят (согласно отношению $\lDEPS$) выпущенные события записи%
\footnote{Формальное определение восприимчивости опущено
в данной работе для краткости и может быть найдено в \coq репозитории,
сопровождающем работу~\cite{Podkopaev-al:POPL19}.}.

Помимо этого, при добавлении новой ветки $\Br_b$ в структуру событий
необходимо выполнить следующие требования.
\begin{itemize}
  \item Для каждого события чтения (в данном случае $\ese{2}{1}{1}$ и $\ese{2}{2}{1}$)
    необходимо выбрать событие запись, обосновывающее это чтение.  
  \item Для каждого события записи необходимо определить позицию
    этого события в отношении частичного порядка $\lCO$.
\end{itemize}
Наконец, после завершения процесса сертификации,
новая ветка заменяет собой ветку потока $t$ в конфигурации $X$:
$$ X_b \defeq X_a \setminus S.\lE\rst{t} \cup \Br_b $$
где $S.\lE\rst{t} \defeq \set{e \in S.\lE \sth S.\lTID(e) = t}$.

\paragraph{Обоснование событий чтения.}

Далее рассмотрим процесс выбора обосновывающего события записи
для добавляемого события чтения.
Для этой цели определим отношение \emph{стабильной обоснованности}
в несколько этапов. 

Сначала по графу $G$ и текущей конфигурации обхода $\tup{C, I}$
зададим множество \emph{зафиксированных} (\emph{determined}) событий.
Метки зафиксированных событий а также события записи,
обосновывающие зафиксированные чтения, должны
совпадать в графе $G$, текущей структуре событий $S$,
а также в конструируемой сертификационной ветке $\Br$.

\begin{definition}
\label{def:det}
Множество \emph{зафиксированных событий}
определяется следующим соотношением.
\begin{align*}
  G.D_{\tup{C, I}} &\defeq {}
           C \cup I {}\cup{} \\
     %% &\cup G.\lW \setminus \codom{G.\lPPO} {}\cup{} \\
     &\cup \dom{G.\lRFI^? \seqc G.\lPPO \seqc [I]} {}\cup{} \\
     &\cup \cod{[I] \seqc G.\lRFI} {}\cup{} \\
     &\cup \cod{G.\lRFE \seqc [G.\lE^{\squq\acq}]}
\end{align*}
\end{definition}

Помимо покрытых и выпущенных событий
в множество зафиксированных событий также входят
все $\lPPO$-предшественники выпущенных событий,
все события чтения, читающие локально из некоторого выпущенного события,
а также события захватывающего ($\acq$) чтения,
читающие из другого потока. 

Для графа $G$ и конфигурации обхода $\TC_b$,
показанных на \cref{fig:lb-sim-ex-travB},
множество зафиксированных событий 
состоит из событий $\ese{1}{3}{}$, $\ese{2}{2}{}$ и $\ese{2}{3}{}$.
В то же время события $\ese{1}{1}{}$, $\ese{1}{2}{}$ и $\ese{2}{1}{}$
не являются зафиксированными, и следовательно
метки соответствующих им событий в структуре $S_b$
могут отличаться от меток в графе $G$.

Далее, введем понятие \emph{фронта} с помощью отношения $\lVF$.
Множество $\dom{\lVF \seqc [e]}$ будем называть \emph{фронтом}
события $e$. Это множество содержит все события записи
``наблюдаемые'' событием $e$.
Будем говорить что $e$ \emph{наблюдает} событие записи $w$,
то есть $\tup{w, e} \in G.\lVF_{\TC}$, если
$w$ ``происходит-до'' $e$, либо оно было
прочитано некоторым покрытым событием, ``происходящим-до''~$e$,
либо оно было ранее прочитано некоторым зафиксированным событием
принадлежащем тому же потококу, что и~$e$. 

\begin{definition}
\label{def:vf}
Отношение $\lVF$ определено как:
\begin{align*}
  G.\lVF_{\tup{C,I}} \defeq {}
    [G.\lW] \seqc (G.\lRF \seqc [C])^? \seqc G.\lHB^? \cup
    G.\lRF \seqc [G.D_{\tup{C, I}}] \seqc G.\lPO^?.
\end{align*}
\end{definition}

На \cref{fig:lb-sim-ex-travB} изображено три ребра отношения $G.\lVF_{\TC_b}$.
Все остальные ребра этого отношения могут быть выведены
при помощи следующего наблюдения:

$$ {G.\lVF_{\TC} \seqc G.\lPO \subseteq G.\lVF_{\TC}}. $$

Наконец, можно привести определение отношения стабильной обоснованости.
Оно соединяет событие чтения с $\lCO$ максимальным
наблюдаемым событием записи в ту же локацию.

\begin{definition}
\label{def:sjf}
Отношение \emph{стабильной обоснованности} определяется следующим образом.
\begin{equation*}
  G.\lSRF_{TC} \defeq
    ([G.\lW] \seqc (G.\lVF_{TC} \cap \lEQLOC) \seqc [G.\lR])
    \setminus (G.\lCO \seqc G.\lVF_{TC})
\end{equation*}
\end{definition}

Для графа $G$ и конфигурации $\TC_b$
отношение $\lSRF$ совпадает c показанными
на \cref{fig:lb-sim-ex-travB} ребрами отношения $\lVF$:
$$\tup{\Init, \ese{1}{1}{}}, \tup{\Init, \ese{2}{1}{}},
  \tup{\ese{1}{3}{}, \ese{2}{2}{}} \in G.\lSRF_{\TC_b}.$$

\begin{lemma}
\label{lm:sjf-det}
Если граф $G$ консистентен согласно модели \IMM,
тогда отношение $G.\lSRF$ совпадает с отношением $G.\lRF$
на множестве зафиксированных событий чтения.
$$  G.\lSRF_{\TC} ; [G.D_{\TC}] \subseteq G.\lRF $$
\end{lemma}

Лемма \ref{lm:sjf-det}, в частности, гарантирует, что
выбранные метки для событий чтения в сертификационной ветке,
от которых зависят (согласно отношению $\lDEPS$) выпущенные события записи,
будут согласованы с метками соответствующих событий в графе $G$.
Для всех остальных событий чтения, согласно свойству восприимчивости,
можно безопасно заменить прочитанные значения.

\begin{lemma}
\label{lm:sjf-iss-po}
Если граф $G$ консистентен согласно модели \IMM,
тогда для отношения $G.\lSRF$ выполняется следующие соотношение:
$$  G.\lSRF_{\TC} \suq [I] \seqc G.\lSRF_{\TC} \cup G.\lPO. $$
\end{lemma}

Лемма \ref{lm:sjf-iss-po} позволяет выбрать обосновывающее
событие записи в структуре событий.
Пусть $\tup{w, r} \in G.\lSRF_{\TC}$.
Если при этом $\tup{w, r} \in [I] \seqc G.\lSRF_{\TC}$
тогда, согласно свойству \ref{simrel:ew-iss} отношения симуляции
можно выбрать событие записи $w' \in S.\lE$, которое принадлежит
конфигурации $X$ и при этом соответствует выпущенной записи, то есть $\ea(w') = w$.
Например, в случае конфигурации обхода $\TC_b$, показанной на \cref{fig:lb-sim-ex-travB},
для события чтения $\ese{2}{2}{1}$
обосновывающим событием записи будет $\ese{2}{3}{1}$
Иначе $\tup{w, r} \in G.\lPO$.
В таком случае достаточно просто выбрать $S.\lPO$
предшествующее событие записи, принадлежащее сертификационной ветке $\Br$.

\paragraph{Упорядочивание событий записи.}

Позиция добавляемых событий записи в отношении порядка
$S.\lCO$ структуры событий выбирается на основе порядка
$G.\lCO$ \IMM графа. Тем не менее, из-за наличия конфликтующих событий,
можно гарантировать только лишь что отношение $S.\lCO$ вложено
в рефлексивное замыкание отношения $G.\lCO$,
то есть $\fmap{S.\lCO} \subseteq G.\lCO^?$.

\begin{figure}[h]
\hfill$\inarr{\begin{tikzpicture}[xscale=1,yscale=1.5]
  \node (init) at (2,  1)   {$\Init$};
  \node (i11)  at (0,  0)   {$\mese{1}{1}{} \rlab{}{x}{1}$};
  \node (i12)  at (0, -1)   {$\mese{1}{2}{} \wlab{}{y}{1}$};
  \node (i13)  at (0, -2)   {$\mese{1}{3}{} \wlab{}{z}{1}$};
  \node (i21)  at (4,  0)   {$\mese{2}{1}{} \rlab{}{y}{1}$};
  \node (i22)  at (4, -1)   {$\mese{2}{2}{} \rlab{}{z}{1}$};
  \node (i23)  at (4, -2)   {$\mese{2}{3}{} \wlab{}{x}{1}$};
  %% \node (hh) at (2, -3) {$\inarrC{\text{The traversal configuration } \TCc}$};
  \begin{scope}[on background layer]
     \issuedCoveredBox{init};
     \issuedBox{i13};
     \issuedBox{i23};
     \coveredBox{i11};
  \end{scope}
  \draw[rf] (i13) edge node[above] {} (i22);
  \draw[rf] (i23) edge node[above] {} (i11);
  \draw[rf] (i12) edge node[above] {} (i21);
  %\draw[vf] (init) edge[bend left=20]  node[above right, pos=0.9] {$\lVF$} (i21);
  %\draw[vf] (i13)  edge[bend right=20] node[above] {$\lVF$} (i22);
  \draw[ppo,out=230,in=130] (i11) edge node[left ,pos=0.8] {\small$\lPPO$} (i12);
  \draw[ppo,out=310,in=50 ] (i22) edge node[right,pos=0.3] {\small$\lPPO$} (i23);
  \draw[po] (init) edge (i11);
  \draw[po] (init) edge (i21);
  \draw[po] (i11)  edge (i12);
  \draw[po] (i12)  edge (i13);
  \draw[po] (i21)  edge (i22);
  \draw[po] (i22)  edge (i23);
\end{tikzpicture}}
\hfill\vrule\hfill
\inarr{\begin{tikzpicture}[xscale=1,yscale=1.5]
  \node (init) at (3, 1)     {$\Init$};

  \node (i111)  at (0,  0)   {$\mese{1}{1}{1} \rlab{}{x}{0}$};
  \node (i121)  at (0, -1)   {$\mese{1}{2}{1} \wlab{}{y}{0}$};
  \node (i131)  at (0, -2)   {$\mese{1}{3}{1} \wlab{}{z}{1}$};

  \node (i112)  at (3,  0)   {$\mese{1}{1}{2} \rlab{}{x}{1}$};
  \node (i122)  at (3, -1)   {$\mese{1}{2}{2} \wlab{}{y}{1}$};
  \node (i132)  at (3, -2)   {$\mese{1}{3}{2} \wlab{}{z}{1}$};

  \node (i211)  at (6,  0)   {$\mese{2}{1}{1} \rlab{}{y}{0}$};
  \node (i221)  at (6, -1)   {$\mese{2}{2}{1} \rlab{}{z}{1}$};
  \node (i231)  at (6, -2)   {$\mese{2}{3}{1} \wlab{}{x}{1}$};

  \draw[jf] (init) edge[bend right] node[above]        {} (i111);
  \draw[jf] (init) edge[bend left ] node[above]        {} (i211);
  \draw[jf] (i131) edge             node[pos=.5,below] {} (i221);
  \draw[jf] (i231) edge             node[pos=.5,below] {} (i112);

  \draw[cf] (i111) -- (i112);
  \node at ($.5*(i111) + .5*(i112) - (0, 0.2)$) {\small$\lCF$};

  \draw[co] (i122) edge node[pos=.5,below] {\small$\lCO$} (i121);
  \draw[ew] (i131) edge node[pos=.5,below] {\small$\lEW$} (i132);

  \draw[po] (init)  edge (i111);
  \draw[po] (i111)  edge (i121);
  \draw[po] (i121)  edge (i131);

  \draw[po] (init)  edge (i112);
  \draw[po] (i112)  edge (i122);
  \draw[po] (i122)  edge (i132);

  \draw[po] (init)  edge (i211);
  \draw[po] (i211)  edge (i221);
  \draw[po] (i221)  edge (i231);

  \begin{scope}[on background layer]
    \draw[extractStyle] (1.8,1.5) rectangle (7.2,-2.5);
  \end{scope}

  %% \node (hh) at (3, -3.5) {$\inarrC{\text{The event structure } \ESc \text{ and} \\\text{the selected execution } \SXc}$};
\end{tikzpicture}}$\hfill
\caption{
Граф сценария исполнения~$G$, конфигурация обхода~$\TC_c$
и соответствующая этой конфигурации
структура событий~$S_c$ вместе с конфигурацией~$X_c$.
}
\label{fig:lb-sim-ex-travC}
\end{figure}


Данную особенность можно продемонстрировать 
на примере шага обхода из конфигурации $\TC_b$ (\cref{fig:lb-sim-ex-travB})
в конфигурацию $\TC_c$ (\cref{fig:lb-sim-ex-travC})
путем покрытия события $\ese{1}{1}{}$.
Для симуляции этого шага выполняется построение
структуры событий $S_c$, содержашей новую ветку 
$\Br_c \defeq \set{\ese{1}{1}{2}, \ese{1}{2}{2}, \ese{1}{3}{2}}$.

Рассмотрим события записи $\ese{1}{2}{1}$ и $\ese{1}{2}{2}$.
Так как два этих события имеют различные метки,
они не могут быть объявлены $\lEW$-эквивалентными.
С другой стороны, требуется, чтобы отношение
$S_c.\lCO$ полностью упорядочивало все события записи в одну и ту же локацию
(по модулю $\lEW$-эквивалентных событий).
То есть требуется каким-либо образом упорядочить
события $\ese{1}{2}{1}$ и $\ese{1}{2}{2}$ между собой.
Так как оба эти события отображаются в одно и то же событие $\ese{1}{2}{}$
в графе $G$, отношение $G.\lCO$ не может быть использовано
при выборе направления $S_c.\lCO$ ребра. 

На самом деле может быть выбран любой из двух способов
упорядочить эти события. Тем не менее,
в целях упрощения доказательства оказалось
удобнее выбрать порядок, при котором новые события
оказываются упорядочены ранее в отношении когеретности.
То есть, возвращаясь к примеру на \cref{fig:lb-sim-ex-travC},
стоит добавить ребро $\tup{\ese{1}{2}{2}, \ese{1}{2}{1}} \in S_c.\lCO$.
Используя данное соглашение можно показать, что
$S.\lCO$ ребро, оканчивающиеся событием из новой ветки $\Br_c$,
должно отображаться строго в ребро $G.\lCO$ в графе: 
$\fmap{S_c.\lCO \seqc [\Br_c]} \suq G.\lCO$.

Далее рассмотрим события $\ese{1}{3}{1}$ и $\ese{1}{3}{2}$.
Эти события имеют одинаковую метку и отображаются в одно
и тоже событие $\ese{1}{3}{}$ в графе $G$.
Следовательно, они могут быть объявлены $\lEW$-эквивалентными.
На самом деле, для корректности построения их необходимо объявить таковыми.
Иначе события ветки $\Br_c$ окажутся невидимыми из-за того,
что существует путь $S_c.\lCF \cap (S_c.\lJFE \seqc (S_c.\lPO \cup S_c.\lJF)^*)$
из события $\ese{1}{3}{1}$ в событие $\ese{1}{1}{2}$.
Напомним, что только видимые события могут быть использованы
для извлечения графа сценария исполнения из структуры событий
(смотри \cref{def:cfg}).

В общем случае, новое событие записи $e$
прикрепляется к классу эквивалентности по отношению $S.\lEW$,
представленному событием $w$, таким что
(i) $w$ имет тот же образ графе, что и $e$, то есть $\ea(w) = \ea(e)$;
(ii) $w$ принадлежит конфигурации $X$, а его образ в графе
принадлежит множеству выпущенных событий: $w \in X \cap \fcomap{I}$.
Если такого события $w$ не существует, тогда $e$
упорядочивается в отношении $S.\lCO$ до
множества событий, чьи образы в графе $G.\lCO$-предшествуют $\ea(e)$
и после событий, чьи образы в графе равны или $G.\lCO$-следуют за $\ea(e)$.
Благодаря свойству \ref{simrel:jfe-iss} отношения симуляции,
а именно $\dom{S.\lJFE} \suq \dom{S.\lEW \seqc [X \cap \fcomap{I}]}$,
подобный выбор отношения $S'.\lEW$ гарантирует,
что все события из новой сертификационной ветки будут видимы. 



\chapter*{Заключение}                       % Заголовок
\addcontentsline{toc}{chapter}{Заключение}  % Добавляем его в оглавление

%% Согласно ГОСТ Р 7.0.11-2011:
%% 5.3.3 В заключении диссертации излагают итоги выполненного исследования, рекомендации, перспективы дальнейшей разработки темы.
%% 9.2.3 В заключении автореферата диссертации излагают итоги данного исследования, рекомендации и перспективы дальнейшей разработки темы.
%% Поэтому имеет смысл сделать эту часть общей и загрузить из одного файла в автореферат и в диссертацию:

В ходе работы были достигнуты результаты, перечисленные ниже.

\begin{itemize}

  \item Формализована в системе \coq классическая теория
    простых структур событий. Показано, что данная теория
    покрывает класс слабых моделей памяти сохраняющих программный порядок. 

  \item Формализована в системе \coq модель \Wkm из
    класса моделей сохраняющих семантические зависимости.
    Также формализовано доказательство корректности компиляции
    из данной модели в модели памяти современных мультипроцессоров.

  \item Предложена новая версия модели \Wkm --- \WkmS.
    Доказано сохранение основных свойств \Wkm для модели \WkmS:
    корректности компиляции, корректности локальных трансформаций программ, 
    теорема о свободе от гонок.

  \item Разработан и апробирован новый алгоритм проверки моделей \wmc для модели \WkmS.

\end{itemize}

Можно выделить следующие потенциальные
\textbf{направления дальнейшей разработки тематики}
данного исследования.

%% Во-первых, интересной задачей представляется формализация
%% в системе \coq помимо простых структур событий
%% также других классов структур событий, например,
%% класса стабильных структур событий~\cite{}.
%% Данный класс структур событий представляет интерес, в частности,
%% в контексте задачи задания формальной \emph{денотационной} cемантики
%% многопоточных программ в слабых моделей памяти.

Во-первых, необходимо обобщить теорию, представленную в главе \ref{ch:porf-evenstruct}
для случая моделей памяти сохраняющих синтаксические зависимости.
Также было бы крайне полезно разработать вариант модели \Wkm,
укладывающийся в классическую теорию простых структур событий. 
Достичь этого можно ослабив определение причинно-следственной связи
событий в модели \Wkm таким образом, чтобы удовлетворить
ограничению наследственности отношения конфликта.
Выполнение данных задач позволит создать единообразную
теорию простых структур событий, покрывающую
все три класса слабых моделей памяти.

Во-вторых, интересной задачей представляется
дальнейшее изучение свойств модели \WkmS.
В частности, было бы интересно узнать,
оказываются ли полезными новые свойства \WkmS
(свобода от буфферезации операций чтения и локальности сертификациии)
при разработке других методов верификации многопоточных программ,
например, программной логики для дедуктивной верификации программ,
по аналогии с тем, как эти свойства оказываются полезны
при реализации алгоритма проверки моделей. 

%% Основные результаты работы заключаются в следующем.
%% \input{common/concl}
%% И какая-нибудь заключающая фраза.

%% Последний параграф может включать благодарности.  В заключение автор
%% выражает благодарность и большую признательность научному руководителю
%% Иванову~И.\,И. за поддержку, помощь, обсуждение результатов и~научное
%% руководство. Также автор благодарит Сидорова~А.\,А. и~Петрова~Б.\,Б.
%% за помощь в~работе с~образцами, Рабиновича~В.\,В. за предоставленные
%% образцы и~обсуждение результатов, Занудятину~Г.\,Г. и авторов шаблона
%% *Russian-Phd-LaTeX-Dissertation-Template* за~помощь в оформлении
%% диссертации. Автор также благодарит много разных людей
%% и~всех, кто сделал настоящую работу автора возможной.
      % Заключение
\include{Dissertation/acronyms}        % Список сокращений и условных обозначений
%% \include{Dissertation/dictionary}      % Словарь терминов
\include{Dissertation/references}      % Список литературы
\include{Dissertation/lists}           % Списки таблиц и изображений (иллюстративный материал)

\setcounter{totalchapter}{\value{chapter}} % Подсчёт количества глав

%%% Настройки для приложений
\appendix
% Оформление заголовков приложений ближе к ГОСТ:
\setlength{\midchapskip}{20pt}
\renewcommand*{\afterchapternum}{\par\nobreak\vskip \midchapskip}
\renewcommand\thechapter{\Asbuk{chapter}} % Чтобы приложения русскими буквами нумеровались

%% \include{Dissertation/appendix}        % Приложения

\setcounter{totalappendix}{\value{chapter}} % Подсчёт количества приложений

\end{document}
