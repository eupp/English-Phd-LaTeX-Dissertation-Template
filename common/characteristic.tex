
% {\actuality} Обзор, введение в тему, обозначение места данной работы в
% мировых исследованиях и~т.\:п., можно использовать ссылки на~другие
% работы~\autocite{Gosele1999161,Lermontov}
% (если их~нет, то~в~автореферате
% автоматически пропадёт раздел <<Список литературы>>). Внимание! Ссылки
% на~другие работы в~разделе общей характеристики работы можно
% использовать только при использовании \verb!biblatex! (из-за технических
% ограничений \verb!bibtex8!. Это связано с тем, что одна
% и~та~же~характеристика используются и~в~тексте диссертации, и в
% автореферате. В~последнем, согласно ГОСТ, должен присутствовать список
% работ автора по~теме диссертации, а~\verb!bibtex8! не~умеет выводить в~одном
% файле два списка литературы).
% При использовании \verb!biblatex! возможно использование исключительно
% в~автореферате подстрочных ссылок
% для других работ командой \verb!\autocite!, а~также цитирование
% собственных работ командой \verb!\cite!. Для этого в~файле
% \verb!common/setup.tex! необходимо присвоить положительное значение
% счётчику \verb!\setcounter{usefootcite}{1}!.

% Для генерации содержимого титульного листа автореферата, диссертации
% и~презентации используются данные из файла \verb!common/data.tex!. Если,
% например, вы меняете название диссертации, то оно автоматически
% появится в~итоговых файлах после очередного запуска \LaTeX. Согласно
% ГОСТ 7.0.11-2011 <<5.1.1 Титульный лист является первой страницей
% диссертации, служит источником информации, необходимой для обработки и
% поиска документа>>. Наличие логотипа организации на~титульном листе
% упрощает обработку и~поиск, для этого разметите логотип вашей
% организации в папке images в~формате PDF (лучше найти его в векторном
% варианте, чтобы он хорошо смотрелся при печати) под именем
% \verb!logo.pdf!. Настроить размер изображения с логотипом можно
% в~соответствующих местах файлов \verb!title.tex!  отдельно для
% диссертации и автореферата. Если вам логотип не~нужен, то просто
% удалите файл с~логотипом.

% \ifsynopsis
% Этот абзац появляется только в~автореферате.
% Для формирования блоков, которые будут обрабатываться только в~автореферате,
% заведена проверка условия \verb!\!\verb!ifsynopsis!.
% Значение условия задаётся в~основном файле документа (\verb!synopsis.tex! для
% автореферата).
% \else
% Этот абзац появляется только в~диссертации.
% Через проверку условия \verb!\!\verb!ifsynopsis!, задаваемого в~основном файле
% документа (\verb!dissertation.tex! для диссертации), можно сделать новую
% команду, обеспечивающую появление цитаты в~диссертации, но~не~в~автореферате.
% \fi

% {\progress}
% Этот раздел должен быть отдельным структурным элементом по
% ГОСТ, но он, как правило, включается в описание актуальности
% темы. Нужен он отдельным структурынм элемементом или нет ---
% смотрите другие диссертации вашего совета, скорее всего не нужен.

{\actuality} 
В современном мире многопоточные программные системы распространены повсеместно. 
Разработка и тестирование программного обеспечения для таких систем на порядок 
сложнее и существенно более трудозатратно, чем для последовательных систем. 
По этой причине крайне актуальной является задача 
верификации многопоточных программ . 

Формальная семантика многопоточных программ, потоки которых работают 
с разделяемой памятью, называется \emph{моделью памяти}. 
Одним из способов задания моделей памяти
является использование семантических доменов
\emph{истинной конкурентности} (\emph{true concurrency semantics}).
Этот класс моделей позволяет выразить независимость (параллельность) атомарных событий, а также 
причинно-следственные связи между ними,
что ведет к более компактному представлению пространства состояний программы.
Всё это упрощает рассуждения 
о поведении многопоточных программ как для человека, 
так и для программных средств при автоматической и интерактивной верификации. 

\emph{Структуры событий} (\emph{event structures}) являются одним из семантических доменов, 
относящихся к классу моделей истинной конкурентности.
В наиболее простом варианте структура событий состоит из множества атомарных событий,
функции, присваивающей каждому событию семантическую метку,
отношения причинно-следственной связи и отношения конфликта между событиями.
Классическая теория структур событий была разработана M.Nielsen, G.Plotkin и G.Winskel
для описания семантики исчисления взаимодействующих систем (Calculus of Communicating Systems, CCS).
Следует отметить, что данное исчисление является достаточно простой моделью параллельных вычислений и не позволяет описывать 
слабые сценарии поведения многопоточных программ.

Для преодоления этого недостатка исследователями было
предложено несколько формализмов, основанных на структурах событий
и позволяющих описывать слабые модели памяти,
в частности, модель A.Jeffrey и J.Riely~\autocite{Jeffrey-Riely:LICS16},
модель J.Pichon-Pharabod и P. Sewell~\autocite{PichonPharabod-Sewell:POPL16},
модель \Wkm~\autocite{Chakraborty-Vafeiadis:POPL19},
модель \MRD~\autocite{Paviotti-al:ESOP20}. 
Однако данные модели накладывают на структуры событий
ограничения, несовместимые с классическими определениями, что  не позволяет 
применять известные результаты о структурах событий к данным моделям. 
С практической точки зрения это приводит к существенному увеличению 
пространства состояний программ, что препятствует разработке эффективных 
средств верификации многопоточных программ.

Таким образом, возникает потребность в создании формальной семантики 
многопоточных программ на основе структур событий, 
которая, с одной стороны, позволяла бы описывать слабые сценарии поведения, 
с другой стороны, допускала бы разработку  
инструментов для автоматической и интерактивной верификации. 

{\progress}

Теория структур событий была разработана M.Nielsen, G.Plotkin и G.Winskel
в 1980-1990 годы для задания денотационной семантики 
исчисления параллельных взаимодействующих систем (Calculus of Communicating Systems, CCS)%
~\autocite{Winskel:ICALP1982}.
Относительно недавно эта теория также была использована 
для задания семантики пи-исчисления процессов ($\pi$-calculus)%
~\autocite{Varacca-Nobuko:TCS10,Crafa-al:FSCCS12,Hildebrandt-al:LATA2017}.
Но CCS и пи-исчисление не позволяют описывать 
слабые сценарии поведения многопоточных программ.

Теория слабых моделей памяти также активно развивалась, начиная с 1990-ых годов. 
На сегодняшний день существует множество моделей памяти, 
описывающих поведение мультипроцессоров, 
многопоточных языков программирования и распределенных систем. 
Эти модели, в свою очередь, можно разделить на несколько классов. 
Модели, \emph{сохраняющие программный порядок}, образуют широкий класс, 
включающий, в том числе, модель \RCMM~\autocite{Lahav-al:PLDI17},
которая описывает подмножество сценариев поведения, допустимых моделью памяти языков \CPP,
модель \TSO процессоров семейства \Intel~\autocite{Sewell-al:CACM10},
а также модели распределенных систем, 
например, модель последовательной согласованности (causal consistency)~\autocite{Lahav-Boker:PLDI2020}
и модель согласованности в конечном счёте (eventual consistency)~\autocite{Jagadeesan-al:ESOP2018}.
Модели памяти мультипроцессоров, например \ARMv{8}~\autocite{Pulte-al:POPL18} 
и \POWER~\autocite{Sarkar-al:PLDI11}, 
как правило, принадлежат к классу моделей, \emph{сохраняющих синтаксические зависимости}. 
Основное ограничение моделей, принадлежащих к данным классам, заключается в том, 
что они не поддерживают некоторые трансформации программ, 
применяемые оптимизирующими компиляторами. 
Поэтому эти модели не в полной мере отвечают требованиям, 
предъявляемым к моделям памяти для таких языков как \CPP и \Java. 
С целью преодоления этих ограничений исследователями 
были предложено несколько моделей, в том числе 
\Prm~\autocite{Kang-al:POPL17}, \Wkm~\autocite{Chakraborty-Vafeiadis:POPL19}, 
\MRD~\autocite{Paviotti-al:ESOP20}, \PwP~\autocite{Jagadeesan-al:OOPSLA2020},
которые обычно относят к классу моделей, \emph{сохраняющих семантические зависимости}.

Классы моделей, сохраняющих программный порядок и синтаксические зависимости, 
хорошо изучены, в то время как свойства класса моделей,
сохраняющих семантические зависимости, по-прежнему активно исследуются.
В частности, для моделей данного класса практический не исследованы
вопросы построения эффективных инструментов автоматической и интерактивной верификации. 

Некоторые из вышеупомянутых моделей, сохраняющих семантические зависимости,
основаны на теории структур событий%
~\autocite{Jeffrey-Riely:LICS16,PichonPharabod-Sewell:POPL16,
Chakraborty-Vafeiadis:POPL19,Paviotti-al:ESOP20}.
Общий недостаток данных моделей заключается в том,
что они вводят новые классы структур событий, 
несовместимые с классическими определениями.
Это затрудняет применение уже существующей классической теории структур событий
для решения проблем, возникающих в теории слабых моделей памяти. 

Среди слабых моделей памяти, сохраняющих семантические зависимости
и основанных на структурах событий, в контексте данной работы наибольший интерес
представляет модель \Wkm~\autocite{Chakraborty-Vafeiadis:POPL19},
поскольку для данной модели было формально доказано наличие ряда важных для практики свойств.
В частности, для этой модели была доказана корректность
локальных трансформаций программ и теорема о свободе от гонок.
Тем не менее отметим, что корректность оптимальной схемы
компиляции из модели \Wkm в модели современных мультипроцессоров
\emph{не была ранее доказана}, что является существенным недостатком,
так как наличие данного свойства является одним из базовых требований,
предъявляемых к классу моделей памяти, сохраняющих семантические зависимости.

{\aim} данной работы является адаптация теории структур событий
для описания слабых моделей памяти и разработка на основе этих исследований 
инструментов для автоматической и интерактивной верификации многопоточных программ. 

Для достижения данной цели были сформулированы следующие {\tasks}.
\begin{enumerate}[beginpenalty=10000] % https://tex.stackexchange.com/a/476052/104425
  \item 
    Формализовать в системе для интерактивного доказательства теорем \coq
    семантику  слабых моделей памяти, сохраняющих программный порядок, используя
    классическую теорию структур событий. 
  \item 
    Формализовать в системе для интерактивного доказательства теорем \coq
    модель \Wkm, а также  доказать  корректность компиляции
    из данной модели в модели памяти современных мультипроцессоров.
  \item Разработать формально строгую версию модели \Wkm, 
    допускающую реализацию эффективных инструментов автоматической верификации
    и доказать, что для неё сохраняются  основные свойства \Wkm  
    (в частности, корректность компиляции, корректность локальных трансформаций программ, 
     теорему о свободе от гонок).
  \item Разработать алгоритм проверки моделей (model~checking) для предложенной модели.
\end{enumerate}

~\newline

{\methods}

Диссертационное исследование базируется на теории формальных семантик. 
Используются классические и хорошо изученные формализмы, в частности, 
системы помеченных переходов, языки помеченных частично упорядоченных мультимножеств и структуры событий. 

Для формализации некоторых теорем и доказательств, представленных в данной работе, 
использовалась система интерактивного доказательства теорем \coq 
и библиотека формализованных математических теорий \mathcomp.

%% При разработке алгоритма проверки моделей использовались техника \emph{редукции частичных порядков}.
%% Предложенный алгоритм был внедрен в систему \genmc --- 
%% инструмент для автоматической верификации многопоточных программ написанных на языке \CLANG.

{\defpositions}
\begin{enumerate}[beginpenalty=10000] % https://tex.stackexchange.com/a/476052/104425
  \item Предложена формальная семантика, 
    покрывающая класс слабых моделей памяти, сохраняющих программный порядок, на основе классической теории структур событий;
    данная семантика была формализована в системе \coq.
  \item Модель \Wkm и доказательство теоремы о корректности компиляции
    из данной модели в модели памяти современных мультипроцессоров формализованы в системе \coq.
  \item Предложена модель \WkmS, формализующая и  расширяющая модель \Wkm,  доказано сохранение основных свойств модели \Wkm: корректности компиляции,
    свойства корректности локальных трансформаций программ,
    теоремы о свободе от гонок.
  \item Для модели \WkmS разработан алгоритм автоматической 
    верификации программ методом проверки моделей.
\end{enumerate}
% В папке Documents можно ознакомиться с решением совета из Томского~ГУ
% (в~файле \verb+Def_positions.pdf+), где обоснованно даются рекомендации
% по~формулировкам защищаемых положений.

{\novelty}
\begin{enumerate}[beginpenalty=10000] % https://tex.stackexchange.com/a/476052/104425

  \item Впервые предложена семантика, основанная на классической теории структур событий,
    которая покрывает класс слабых моделей памяти, сохраняющих программный порядок.
    %% что позволяет применить известные теоретические результаты 
    %% о структурах событий к данному классу моделей.

  \item Впервые  в системе \coq формализовано доказательство корректности компиляции
    из модели \Wkm, основанной на структурах событий, 
    в модели современных мультипроцессоров.  

  \item Впервые предложена модель памяти (\WkmS),
    которая принадлежит к классу моделей, сохраняющих семантические зависимости, 
    и при этом допускает реализацию эффективных методов автоматической верификации программ. 

  \item Разработан новый алгоритм проверки моделей для \WkmS,
    который является существенно более эффективным по сравнению с другими алгоритмами
    (\CDSChecker~\autocite{Norris-Demsky:OOPSLA2013}, \rmem~\autocite{Pulte-al:PLDI2019}),
    поддерживающими класс моделей памяти, сохраняющих семантические зависимости.

\end{enumerate}

{\influence} 

Новая семантика на основе теории структур событий 
для класса слабых моделей памяти, сохраняющих программный порядок,
соединяет классическую теорию структур событий 
с теорией слабых моделей памяти и позволяет применить известные результаты 
о структурах событий в новой предметной области.  
Формализация этой семантики в системе \coq открывает 
путь к дальнейшей разработке инструментов для  
интерактивной верификации многоточных программ  
с учетом слабых сценариев исполнения. 
 
Новые свойства предложенной модели \WkmS ---
свобода от буферизации операций чтения (load buffering freedom)
и локальности сертификации (certification locality), --- 
также могут быть добавлены  в другие модели памяти 
с целью разработки методов автоматической верификации программ в этих моделях. 

Предложенный  алгоритм проверки моделей может быть использован на практике
для отладки и верификации многопоточных алгоритмов и структур данных 
с учетом слабых сценариев исполнения, допустимых стандартом языка \CLANG. 

{\reliability} полученных результатов обеспечивается 
формальными доказательствами, разработанными в том числе с использованием
систем интерактивного доказательства теорем, 
а также инженерными экспериментами. 
Результаты находятся в соответствии с результатами, полученными другими авторами.

{\probation}
Основные результаты работы докладывались~на
следующих научных конференциях и семинарах:
Surrey Concurrency Workshop (23-24 июля 2019, Университет Суррея, Великобритания),
The European Conference on Object-Oriented Programming
(ECOOP, 15-17 ноября 2020, онлайн конференция),
Spring/Summer Young Researchers' Colloquium on Software Engineering
(27-28 мая 2021, Москва, Россия),
внутренние семинары JetBrains Research
(18 ноября 2018, 13 апреля 2020, Санкт-Петербург, Россия). \\
\fixme{добавить будущие мероприятия по мере проведения}.

% {\contribution} Автор принимал активное участие \ldots

\ifnumequal{\value{bibliosel}}{0}
{%%% Встроенная реализация с загрузкой файла через движок bibtex8. (При желании, внутри можно использовать обычные ссылки, наподобие `\cite{vakbib1,vakbib2}`).
    {\publications} Основные результаты по теме диссертации изложены
    в~XX~печатных изданиях,
    X из которых изданы в журналах, рекомендованных ВАК,
    X "--- в тезисах докладов.
}%
{%%% Реализация пакетом biblatex через движок biber
    \begin{refsection}[bl-author, bl-registered]
        % Это refsection=1.
        % Процитированные здесь работы:
        %  * подсчитываются, для автоматического составления фразы "Основные результаты ..."
        %  * попадают в авторскую библиографию, при usefootcite==0 и стиле `\insertbiblioauthor` или `\insertbiblioauthorgrouped`
        %  * нумеруются там в зависимости от порядка команд `\printbibliography` в этом разделе.
        %  * при использовании `\insertbiblioauthorgrouped`, порядок команд `\printbibliography` в нём должен быть тем же (см. biblio/biblatex.tex)
        %
        % Невидимый библиографический список для подсчёта количества публикаций:
        \printbibliography[heading=nobibheading, section=1, env=countauthorvak,          keyword=biblioauthorvak]%
        \printbibliography[heading=nobibheading, section=1, env=countauthorwos,          keyword=biblioauthorwos]%
        \printbibliography[heading=nobibheading, section=1, env=countauthorscopus,       keyword=biblioauthorscopus]%
        \printbibliography[heading=nobibheading, section=1, env=countauthorconf,         keyword=biblioauthorconf]%
        \printbibliography[heading=nobibheading, section=1, env=countauthorother,        keyword=biblioauthorother]%
        \printbibliography[heading=nobibheading, section=1, env=countregistered,         keyword=biblioregistered]%
        \printbibliography[heading=nobibheading, section=1, env=countauthorpatent,       keyword=biblioauthorpatent]%
        \printbibliography[heading=nobibheading, section=1, env=countauthorprogram,      keyword=biblioauthorprogram]%
        \printbibliography[heading=nobibheading, section=1, env=countauthor,             keyword=biblioauthor]%
        \printbibliography[heading=nobibheading, section=1, env=countauthorvakscopuswos, filter=vakscopuswos]%
        \printbibliography[heading=nobibheading, section=1, env=countauthorscopuswos,    filter=scopuswos]%
        %
        \nocite{*}%
        %
        {\publications} Основные результаты по теме диссертации изложены в~\arabic{citeauthor}~печатных изданиях,
        \arabic{citeauthorvak} из которых изданы в журналах, рекомендованных ВАК\sloppy%
        \ifnum \value{citeauthorscopuswos}>0%
            , \arabic{citeauthorscopuswos} "--- в~периодических научных журналах, индексируемых Web of~Science и Scopus\sloppy%
        \fi%
        \ifnum \value{citeauthorconf}>0%
            , \arabic{citeauthorconf} "--- в~тезисах докладов.
        \else%
            .
        \fi%
        \ifnum \value{citeregistered}=1%
            \ifnum \value{citeauthorpatent}=1%
                Зарегистрирован \arabic{citeauthorpatent} патент.
            \fi%
            \ifnum \value{citeauthorprogram}=1%
                Зарегистрирована \arabic{citeauthorprogram} программа для ЭВМ.
            \fi%
        \fi%
        \ifnum \value{citeregistered}>1%
            Зарегистрированы\ %
            \ifnum \value{citeauthorpatent}>0%
            \formbytotal{citeauthorpatent}{патент}{}{а}{}\sloppy%
            \ifnum \value{citeauthorprogram}=0 . \else \ и~\fi%
            \fi%
            \ifnum \value{citeauthorprogram}>0%
            \formbytotal{citeauthorprogram}{программ}{а}{ы}{} для ЭВМ.
            \fi%
        \fi%
        % К публикациям, в которых излагаются основные научные результаты диссертации на соискание учёной
        % степени, в рецензируемых изданиях приравниваются патенты на изобретения, патенты (свидетельства) на
        % полезную модель, патенты на промышленный образец, патенты на селекционные достижения, свидетельства
        % на программу для электронных вычислительных машин, базу данных, топологию интегральных микросхем,
        % зарегистрированные в установленном порядке.(в ред. Постановления Правительства РФ от 21.04.2016 N 335)
    \end{refsection}%
    \begin{refsection}[bl-author, bl-registered]
        % Это refsection=2.
        % Процитированные здесь работы:
        %  * попадают в авторскую библиографию, при usefootcite==0 и стиле `\insertbiblioauthorimportant`.
        %  * ни на что не влияют в противном случае
        \nocite{Moiseenko-al:OOPSLA22}
        \nocite{Moiseenko-al:ECOOP20}
        \nocite{Moiseenko-al:STJITMO22}
        \nocite{Moiseenko-al:PCS21}
        \nocite{Gladstein-al:ISPRAS21}
        % \nocite{vakbib2}%vak
        % \nocite{patbib1}%patent
        % \nocite{progbib1}%program
        % \nocite{bib1}%other
        % \nocite{confbib1}%conf
    \end{refsection}%
        %
        % Всё, что вне этих двух refsection, это refsection=0,
        %  * для диссертации - это нормальные ссылки, попадающие в обычную библиографию
        %  * для автореферата:
        %     * при usefootcite==0, ссылка корректно сработает только для источника из `external.bib`. Для своих работ --- напечатает "[0]" (и даже Warning не вылезет).
        %     * при usefootcite==1, ссылка сработает нормально. В авторской библиографии будут только процитированные в refsection=0 работы.
}

% При использовании пакета \verb!biblatex! будут подсчитаны все работы, добавленные
% в файл \verb!biblio/author.bib!. Для правильного подсчёта работ в~различных
% системах цитирования требуется использовать поля:
% \begin{itemize}
%         \item \texttt{authorvak} если публикация индексирована ВАК,
%         \item \texttt{authorscopus} если публикация индексирована Scopus,
%         \item \texttt{authorwos} если публикация индексирована Web of Science,
%         \item \texttt{authorconf} для докладов конференций,
%         \item \texttt{authorpatent} для патентов,
%         \item \texttt{authorprogram} для зарегистрированных программ для ЭВМ,
%         \item \texttt{authorother} для других публикаций.
% \end{itemize}
% Для подсчёта используются счётчики:
% \begin{itemize}
%         \item \texttt{citeauthorvak} для работ, индексируемых ВАК,
%         \item \texttt{citeauthorscopus} для работ, индексируемых Scopus,
%         \item \texttt{citeauthorwos} для работ, индексируемых Web of Science,
%         \item \texttt{citeauthorvakscopuswos} для работ, индексируемых одной из трёх баз,
%         \item \texttt{citeauthorscopuswos} для работ, индексируемых Scopus или Web of~Science,
%         \item \texttt{citeauthorconf} для докладов на конференциях,
%         \item \texttt{citeauthorother} для остальных работ,
%         \item \texttt{citeauthorpatent} для патентов,
%         \item \texttt{citeauthorprogram} для зарегистрированных программ для ЭВМ,
%         \item \texttt{citeauthor} для суммарного количества работ.
% \end{itemize}

% Счётчик \texttt{citeexternal} используется для подсчёта процитированных публикаций;
% \texttt{citeregistered} "--- для подсчёта суммарного количества патентов и программ для ЭВМ.

% Для добавления в список публикаций автора работ, которые не были процитированы в
% автореферате, требуется их~перечислить с использованием команды \verb!\nocite! в
% \verb!Synopsis/content.tex!.

Личный вклад автора в публикациях, выполненных с соавторами, распределён следующим образом.
В работе \cite{Gladstein-al:ISPRAS21} автор предложил
метод кодирования семантики параллельной регистровой машины с
моделью памяти, сохраняющей программный порядок, в терминах простых структур событий;
соавторы участвовали в формализации данного метода в системе \coq.
В работе \cite{Moiseenko-al:STJITMO22} автор предложил
метод кодирования семантического домена языков частично упорядоченных мультимножеств
с использованием фактор-типов, 
соавторы участвовали в обсуждении данного метода и его формализации в системе \coq.
В работе \cite{Moiseenko-al:PCS21} автор выполнил сбор и анализ данных
о существующих моделях памяти языков программирования;
соавторы участвовали в формулировке выводов данного анализа.
В работе \cite{Moiseenko-al:ECOOP20} автор выполнил
формализацию доказательства корректности компиляции из
модели \Wkm в модели современных мультипроцессоров;
соавторы участвовали в обсуждении данного доказательства
и его формализации в системе \coq.
В работе \cite{Moiseenko-al:OOPSLA22} автор
формализовал новые свойства модели \WkmS,
а именно, свойство свободы от буферизации операций чтения и локальности сертификации,
доказал сохранение полезных свойств модели \Wkm в \WkmS,
а также разработал прототип алгоритма проверки моделей для \WkmS;
соавторы участвовали в обсуждении формализации модели \WkmS
и доказательстве ее свойств,
оказывали помощь при реализации нового алгоритма,
а также провели эксперименты по сравнению нового алгоритма с аналогами.
