%&preformat-synopsis
\RequirePackage[l2tabu,orthodox]{nag} % Раскомментировав, можно в логе получать рекомендации относительно правильного использования пакетов и предупреждения об устаревших и нерекомендуемых пакетах

% Откомментируйте, чтобы отключить генерацию закладок в pdf
% \PassOptionsToPackage{bookmarks=false}{hyperref}
\documentclass[a5paper,10pt,twoside,openany,article]{memoir} %,draft

%%%%%%%%%%%%%%%%%%%%%%%%%%%%%%%%%%%%%%%%%%%%%%%%%%%%%%
%%%% Файл упрощённых настроек шаблона диссертации %%%%
%%%%%%%%%%%%%%%%%%%%%%%%%%%%%%%%%%%%%%%%%%%%%%%%%%%%%%

%%% Инициализирование переменных, не трогать!  %%%
\newcounter{intvl}
\newcounter{otstup}
\newcounter{contnumeq}
\newcounter{contnumfig}
\newcounter{contnumtab}
\newcounter{pgnum}
\newcounter{chapstyle}
\newcounter{headingdelim}
\newcounter{headingalign}
\newcounter{headingsize}
%%%%%%%%%%%%%%%%%%%%%%%%%%%%%%%%%%%%%%%%%%%%%%%%%%%%%%

%%% Область упрощённого управления оформлением %%%

%% Интервал между заголовками и между заголовком и текстом %%
% Заголовки отделяют от текста сверху и снизу
% тремя интервалами (ГОСТ Р 7.0.11-2011, 5.3.5)
\setcounter{intvl}{3}               % Коэффициент кратности к размеру шрифта

%% Отступы у заголовков в тексте %%
\setcounter{otstup}{0}              % 0 --- без отступа; 1 --- абзацный отступ

%% Нумерация формул, таблиц и рисунков %%
% Нумерация формул
\setcounter{contnumeq}{0}   % 0 --- пораздельно (во введении подряд,
                            %       без номера раздела);
                            % 1 --- сквозная нумерация по всей диссертации
% Нумерация рисунков
\setcounter{contnumfig}{0}  % 0 --- пораздельно (во введении подряд,
                            %       без номера раздела);
                            % 1 --- сквозная нумерация по всей диссертации
% Нумерация таблиц
\setcounter{contnumtab}{1}  % 0 --- пораздельно (во введении подряд,
                            %       без номера раздела);
                            % 1 --- сквозная нумерация по всей диссертации

%% Оглавление %%
\setcounter{pgnum}{1}       % 0 --- номера страниц никак не обозначены;
                            % 1 --- Стр. над номерами страниц (дважды
                            %       компилировать после изменения настройки)
\settocdepth{subsection}    % до какого уровня подразделов выносить в оглавление
\setsecnumdepth{subsection} % до какого уровня нумеровать подразделы


%% Текст и форматирование заголовков %%
\setcounter{chapstyle}{1}     % 0 --- разделы только под номером;
                              % 1 --- разделы с названием "Глава" перед номером
\setcounter{headingdelim}{1}  % 0 --- номер отделен пропуском в 1em или \quad;
                              % 1 --- номера разделов и приложений отделены
                              %       точкой с пробелом, подразделы пропуском
                              %       без точки;
                              % 2 --- номера разделов, подразделов и приложений
                              %       отделены точкой с пробелом.

%% Выравнивание заголовков в тексте %%
\setcounter{headingalign}{1}  % 0 --- по центру;
                              % 1 --- по левому краю

%% Размеры заголовков в тексте %%
\setcounter{headingsize}{0}   % 0 --- по ГОСТ, все всегда 14 пт;
                              % 1 --- пропорционально изменяющийся размер
                              %       в зависимости от базового шрифта

%% Подпись таблиц %%

% Смещение строк подписи после первой строки
\newcommand{\tabindent}{0cm}

% Тип форматирования заголовка таблицы:
% plain --- название и текст в одной строке
% split --- название и текст в разных строках
\newcommand{\tabformat}{plain}

%%% Настройки форматирования таблицы `plain`

% Выравнивание по центру подписи, состоящей из одной строки:
% true  --- выравнивать
% false --- не выравнивать
\newcommand{\tabsinglecenter}{false}

% Выравнивание подписи таблиц:
% justified   --- выравнивать как обычный текст («по ширине»)
% centering   --- выравнивать по центру
% centerlast  --- выравнивать по центру только последнюю строку
% centerfirst --- выравнивать по центру только первую строку (не рекомендуется)
% raggedleft  --- выравнивать по правому краю
% raggedright --- выравнивать по левому краю
\newcommand{\tabjust}{justified}

% Разделитель записи «Таблица #» и названия таблицы
\newcommand{\tablabelsep}{~\cyrdash\ }

%%% Настройки форматирования таблицы `split`

% Положение названия таблицы:
% \centering   --- выравнивать по центру
% \raggedleft  --- выравнивать по правому краю
% \raggedright --- выравнивать по левому краю
\newcommand{\splitformatlabel}{\raggedleft}

% Положение текста подписи:
% \centering   --- выравнивать по центру
% \raggedleft  --- выравнивать по правому краю
% \raggedright --- выравнивать по левому краю
\newcommand{\splitformattext}{\raggedright}

%% Подпись рисунков %%
%Разделитель записи «Рисунок #» и названия рисунка
\newcommand{\figlabelsep}{~\cyrdash\ }  % (ГОСТ 2.105, 4.3.1)
                                        % "--- здесь не работает

%%% Цвета гиперссылок %%%
% Latex color definitions: http://latexcolor.com/
\definecolor{linkcolor}{rgb}{0.9,0,0}
\definecolor{citecolor}{rgb}{0,0.6,0}
\definecolor{urlcolor}{rgb}{0,0,1}
%\definecolor{linkcolor}{rgb}{0,0,0} %black
%\definecolor{citecolor}{rgb}{0,0,0} %black
%\definecolor{urlcolor}{rgb}{0,0,0} %black
          % общие настройки шаблона
%%% Проверка используемого TeX-движка %%%
\newif\ifxetexorluatex   % определяем новый условный оператор (http://tex.stackexchange.com/a/47579)
\ifxetex
    \xetexorluatextrue
\else
    \ifluatex
        \xetexorluatextrue
    \else
        \xetexorluatexfalse
    \fi
\fi

\newif\ifsynopsis           % Условие, проверяющее, что документ --- автореферат

\usepackage{etoolbox}[2015/08/02]   % Для продвинутой проверки разных условий
\providebool{presentation}

\usepackage{comment}    % Позволяет убирать блоки текста (добавляет
                        % окружение comment и команду \excludecomment)

%%% Поля и разметка страницы %%%
\usepackage{pdflscape}  % Для включения альбомных страниц
\usepackage{geometry}   % Для последующего задания полей

%%% Математические пакеты %%%
\usepackage{amsthm,amsmath,amscd}   % Математические дополнения от AMS
\usepackage{amsfonts,amssymb}       % Математические дополнения от AMS
\usepackage{mathtools}              % Добавляет окружение multlined
\usepackage{xfrac}                  % Красивые дроби
\usepackage[
    locale = DE,
    list-separator       = {;\,},
    list-final-separator = {;\,},
    list-pair-separator  = {;\,},
    list-units           = single,
    range-units          = single,
    range-phrase={\text{\ensuremath{-}}},
    % quotient-mode        = fraction, % красивые дроби могут не соответствовать ГОСТ
    fraction-function    = \sfrac,
    separate-uncertainty,
    ]{siunitx}[=v2]                 % Размерности SI
\sisetup{inter-unit-product = \ensuremath{{}\cdot{}}}

% Кириллица в нумерации subequations
% Для правильной работы требуется выполнение сразу после загрузки пакетов
\patchcmd{\subequations}{\def\theequation{\theparentequation\alph{equation}}}
{\def\theequation{\theparentequation\asbuk{equation}}}
{\typeout{subequations patched}}{\typeout{subequations not patched}}

%%%% Установки для размера шрифта 14 pt %%%%
%% Формирование переменных и констант для сравнения (один раз для всех подключаемых файлов)%%
%% должно располагаться до вызова пакета fontspec или polyglossia, потому что они сбивают его работу
\newlength{\curtextsize}
\newlength{\bigtextsize}
\setlength{\bigtextsize}{13.9pt}

\makeatletter
%\show\f@size    % неплохо для отслеживания, но вызывает стопорение процесса,
                 % если документ компилируется без команды  -interaction=nonstopmode
\setlength{\curtextsize}{\f@size pt}
\makeatother

%%% Кодировки и шрифты %%%
\ifxetexorluatex
    \ifpresentation
        \providecommand*\autodot{} % quick fix for polyglossia 1.50
    \fi
    \PassOptionsToPackage{no-math}{fontspec}    % https://tex.stackexchange.com/a/26295/104425
    \usepackage{polyglossia}[2014/05/21]        % Поддержка многоязычности
                                        % (fontspec подгружается автоматически)
\else
   %%% Решение проблемы копирования текста в буфер кракозябрами
    \ifnumequal{\value{usealtfont}}{0}{}{
        \input glyphtounicode.tex
        \input glyphtounicode-cmr.tex %from pdfx package
        \pdfgentounicode=1
    }
    \usepackage{cmap}   % Улучшенный поиск русских слов в полученном pdf-файле
    \ifnumequal{\value{usealtfont}}{2}{}{
        \defaulthyphenchar=127  % Если стоит до fontenc, то переносы
                                % не впишутся в выделяемый текст при
                                % копировании его в буфер обмена
    }
    \usepackage{textcomp}
    \usepackage[T1,T2A]{fontenc}                    % Поддержка русских букв
    \ifnumequal{\value{usealtfont}}{1}{% Используется pscyr, при наличии
        \IfFileExists{pscyr.sty}{\usepackage{pscyr}}{}  % Подключение pscyr
    }{}
    \usepackage[utf8]{inputenc}[2014/04/30]         % Кодировка utf8
    \usepackage[english, russian]{babel}[2014/03/24]% Языки: русский, английский
    \makeatletter\AtBeginDocument{\let\@elt\relax}\makeatother % babel 3.40 fix
    \ifnumequal{\value{usealtfont}}{2}{
        % http://dxdy.ru/post1238763.html#p1238763
        \usepackage[scaled=0.914]{XCharter}[2017/12/19] % Подключение русифицированных шрифтов XCharter
        \usepackage[charter, vvarbb, scaled=1.048]{newtxmath}[2017/12/14]
        \ifpresentation
        \else
            \setDisplayskipStretch{-0.078}
        \fi
    }{}
\fi

%%% Оформление абзацев %%%
\ifpresentation
\else
    \indentafterchapter     % Красная строка после заголовков типа chapter
    \usepackage{indentfirst}
\fi

%%% Цвета %%%
\ifpresentation
\else
    \usepackage[dvipsnames, table, hyperref]{xcolor} % Совместимо с tikz
\fi

%%% Таблицы %%%
\usepackage{longtable,ltcaption} % Длинные таблицы
\usepackage{multirow,makecell}   % Улучшенное форматирование таблиц
\usepackage{tabu, tabulary}      % таблицы с автоматически подбирающейся
                                 % шириной столбцов (tabu обязательно
                                 % до hyperref вызывать)
\makeatletter
%https://github.com/tabu-issues-for-future-maintainer/tabu/issues/26
\@ifpackagelater{longtable}{2020/02/07}{
\def\tabuendlongtrial{%
    \LT@echunk  \global\setbox\LT@gbox \hbox{\unhbox\LT@gbox}\kern\wd\LT@gbox
                \LT@get@widths
}%
}{}
\makeatother

\usepackage{threeparttable}      % автоматический подгон ширины подписи таблицы

%%% Общее форматирование
\usepackage{soulutf8}% Поддержка переносоустойчивых подчёркиваний и зачёркиваний
\usepackage{icomma}  % Запятая в десятичных дробях

%%% Оптимизация расстановки переносов и длины последней строки абзаца
\IfFileExists{impnattypo.sty}{% проверка установленности пакета impnattypo
    \ifluatex
        \ifnumequal{\value{draft}}{1}{% Черновик
            \usepackage[hyphenation, lastparline, nosingleletter, homeoarchy,
            rivers, draft]{impnattypo}
        }{% Чистовик
            \usepackage[hyphenation, lastparline, nosingleletter]{impnattypo}
        }
    \else
        \usepackage[hyphenation, lastparline]{impnattypo}
    \fi
}{}

%% Векторная графика

\usepackage{tikz}                   % Продвинутый пакет векторной графики
\usetikzlibrary{chains}             % Для примера tikz рисунка
\usetikzlibrary{shapes.geometric}   % Для примера tikz рисунка
\usetikzlibrary{shapes.symbols}     % Для примера tikz рисунка
\usetikzlibrary{arrows}             % Для примера tikz рисунка

%%% Гиперссылки %%%
\ifxetexorluatex
    \let\CYRDZE\relax
\fi
\usepackage{hyperref}[2012/11/06]

%%% Изображения %%%
\usepackage{graphicx}[2014/04/25]   % Подключаем пакет работы с графикой
\usepackage{caption}                % Подписи рисунков и таблиц
\usepackage{subcaption}             % Подписи подрисунков и подтаблиц
\usepackage{pdfpages}               % Добавление внешних pdf файлов

%%% Счётчики %%%
\usepackage{aliascnt}
\usepackage[figure,table]{totalcount}   % Счётчик рисунков и таблиц
\usepackage{totcount}   % Пакет создания счётчиков на основе последнего номера
                        % подсчитываемого элемента (может требовать дважды
                        % компилировать документ)
\usepackage{totpages}   % Счётчик страниц, совместимый с hyperref (ссылается
                        % на номер последней страницы). Желательно ставить
                        % последним пакетом в преамбуле

%%% Продвинутое управление групповыми ссылками (пока только формулами) %%%
\ifpresentation
\else
    \usepackage[russian]{cleveref} % cleveref имеет сложности со считыванием
    % языка из babel. Такое решение русификации вывода выбрано вместо
    % определения в documentclass из опасности что-то лишнее передать во все
    % остальные пакеты, включая библиографию.

    % Добавление возможности использования пробелов в \labelcref
    % https://tex.stackexchange.com/a/340502/104425
    \usepackage{kvsetkeys}
    \makeatletter
    \let\org@@cref\@cref
    \renewcommand*{\@cref}[2]{%
        \edef\process@me{%
            \noexpand\org@@cref{#1}{\zap@space#2 \@empty}%
        }\process@me
    }
    \makeatother
\fi

\usepackage{placeins} % для \FloatBarrier

\ifnumequal{\value{draft}}{1}{% Черновик
    \usepackage[firstpage]{draftwatermark}
    \SetWatermarkText{DRAFT}
    \SetWatermarkFontSize{14pt}
    \SetWatermarkScale{15}
    \SetWatermarkAngle{45}
}{}

%%% Цитата, не приводимая в автореферате:
% возможно, актуальна только для biblatex
%\newcommand{\citeinsynopsis}[1]{\ifsynopsis\else ~\cite{#1} \fi}

% если текущий процесс запущен библиотекой tikz-external, то прекомпиляция должна быть включена
\ifdefined\tikzexternalrealjob
    \setcounter{imgprecompile}{1}
\fi

\ifnumequal{\value{imgprecompile}}{1}{% Только если у нас включена предкомпиляция
    \usetikzlibrary{external}   % подключение возможности предкомпиляции
    \tikzexternalize[prefix=images/cache/,optimize command away=\includepdf] % activate! % здесь можно указать отдельную папку для скомпилированных файлов
    \ifxetex
        \tikzset{external/up to date check={diff}}
    \fi
}{}

%% кастомные пакеты

\usepackage{subcaption}
\usepackage{stmaryrd}
\usepackage{mathpartir}
\usepackage{tikz}
\usepackage{thm-restate}
\usepackage{thmtools,thm-restate}
\usepackage{xspace}
\usepackage{multicol}
% \usepackage{prooftree}
\usepackage{xifthen}

\usepackage{bussproofs}
\EnableBpAbbreviations
       % Пакеты общие для диссертации и автореферата
\synopsistrue                 % Этот документ --- автореферат
\input{Synopsis/synpackages}  % Пакеты для автореферата
\input{Synopsis/userpackages} % Пакеты для специфических пользовательских задач

%% programming languages abbreviations

\newcommand{\Java}{Java\xspace}
\newcommand{\JVM}{JVM\xspace}
\newcommand{\CLANG}{C\xspace}
\newcommand{\CPP}{C/C++\xspace}
\newcommand{\JS}{JavaScript\xspace}
\newcommand{\LLVM}{LLVM\xspace}

%% memory models abbreviations

\newcommand{\MM}[1]{\ensuremath{\mathsf{#1}}\xspace}

\newcommand{\SC}{\MM{SC}}
\newcommand{\DRFx}{\MM{DRFx}}

\newcommand{\Intel}{\MM{x86}}
\newcommand{\TSO}{\MM{TSO}}
\newcommand{\SPARC}{\MM{SPARC}}
\newcommand{\ARM}{\MM{ARM}}
\newcommand{\ARMv}[1]{\MM{ARMv{#1}}}
\newcommand{\POWER}{\MM{POWER}}
\newcommand{\RISC}{\MM{RISC\text{-}V}}

\newcommand{\CMM}{\MM{C11}}
\newcommand{\RCMM}{\MM{RC11}}

\newcommand{\Prm}{\MM{Promising}}
\newcommand{\Wkm}{\MM{Weakestmo}}
\newcommand{\WkmS}{\MM{Weakestmo2}}
\newcommand{\MRD}{\MM{MRD}}
\newcommand{\PwP}{\MM{PwP}}

%% proof assistants 

\newcommand{\coq}{\textsc{Coq}\xspace}
\newcommand{\gallina}{\textsc{Gallina}\xspace}
\newcommand{\mathcomp}{\textsc{MathComp}\xspace}
\newcommand{\analysis}{\textsc{MathComp-Analysis}\xspace}
\newcommand{\finmap}{\textsc{finmap}\xspace}
\newcommand{\relationalgebra}{\textsc{relation-algebra}\xspace}
\newcommand{\ssreflect}{\textsc{SSReflect}\xspace}
\newcommand{\equations}{\textsc{Equations}\xspace}

\newcommand{\agda}{\textsc{Agda}\xspace}
\newcommand{\arend}{\textsc{Arend}\xspace}
\newcommand{\idris}{\textsc{Idris}\xspace}
\newcommand{\isabelle}{\textsc{Isabelle/HOL}\xspace}

%% tools 

\newcommand{\hmc}{\textsc{HMC}\xspace}
\newcommand{\hmclbf}{$\hmc_{\lbf}$\xspace}
\newcommand{\RCMC}{\textsc{RCMC}\xspace}
\newcommand{\rcmc}{\textsc{rcmc}\xspace}
\newcommand{\genmc}{\textsc{GenMC}\xspace}
\newcommand{\lockmc}{\textsc{LAPOR}\xspace}
\newcommand{\genmcmath}{\textnormal{\genmc}\xspace}
\newcommand{\Tracer}{\textsc{Tracer}\xspace}
\newcommand{\Herd}{\textsc{Herd}\xspace}
\newcommand{\PPCMEM}{\textsc{PPCMEM}\xspace}
\newcommand{\ARMMEM}{\textsc{ARMMEM}\xspace}
\newcommand{\CPPMEM}{\textsc{CPPMEM}\xspace}
\newcommand{\TriCheck}{\textsc{TriCheck}\xspace}
\newcommand{\rmem}{\textsc{rmem}\xspace}
\newcommand{\Nidhugg}{\textsc{Nidhugg}\xspace}
\newcommand{\CDSChecker}{\textsc{CDS\-Checker}\xspace}
\newcommand{\CBMC}{\textsc{CBMC}\xspace}
\newcommand{\Dartagnan}{\textsc{Dartagnan}\xspace}
\newcommand{\Verisoft}{\textsc{Verisoft}\xspace}
\newcommand{\CHESS}{\textsc{CHESS}\xspace}
\newcommand{\wmc}{\textsc{WMC}\xspace}
           % кастомные макросы

% Новые переменные, которые могут использоваться во всём проекте
% ГОСТ 7.0.11-2011
% 9.2 Оформление текста автореферата диссертации
% 9.2.1 Общая характеристика работы включает в себя следующие основные структурные
% элементы:
% актуальность темы исследования;
\newcommand{\actualityTXT}{Actuality.}
% степень ее разработанности;
\newcommand{\progressTXT}{Background.}
% цели и задачи;
\newcommand{\aimTXT}{Aim}
\newcommand{\tasksTXT}{tasks}
% научную новизну;
\newcommand{\noveltyTXT}{Scientific novelty.}
% теоретическую и практическую значимость работы;
%\newcommand{\influenceTXT}{Теоретическая и практическая значимость}
% или чаще используют просто
\newcommand{\influenceTXT}{Theoretical and practical influence.}
% методологию и методы исследования;
\newcommand{\methodsTXT}{Methodology and research methods.}
% положения, выносимые на защиту;
\newcommand{\defpositionsTXT}{The main results submitted for defense.}
% степень достоверности и апробацию результатов.
\newcommand{\reliabilityTXT}{Reliability}
\newcommand{\probationTXT}{Approbation.}

\newcommand{\contributionTXT}{personal contribution}
\newcommand{\publicationsTXT}{Publications.}


%%% Заголовки библиографии:

% для автореферата:
\newcommand{\bibtitleauthor}{Публикации автора по теме диссертации}

% для стиля библиографии `\insertbiblioauthorgrouped`
\newcommand{\bibtitleauthorvak}{В изданиях из списка ВАК РФ}
\newcommand{\bibtitleauthorscopus}{В изданиях, входящих в международную базу цитирования Scopus}
\newcommand{\bibtitleauthorwos}{В изданиях, входящих в международную базу цитирования Web of Science}
\newcommand{\bibtitleauthorother}{В прочих изданиях}
\newcommand{\bibtitleauthorconf}{В сборниках трудов конференций}
\newcommand{\bibtitleauthorpatent}{Зарегистрированные патенты}
\newcommand{\bibtitleauthorprogram}{Зарегистрированные программы для ЭВМ}

% для стиля библиографии `\insertbiblioauthorimportant`:
\newcommand{\bibtitleauthorimportant}{Наиболее значимые \protect\MakeLowercase\bibtitleauthor}

% для списка литературы в диссертации и списка чужих работ в автореферате:
\newcommand{\bibtitlefull}{Список литературы} % (ГОСТ Р 7.0.11-2011, 4)
       % Новые переменные, которые могут использоваться во всём проекте
%%%%%%%%%%%%%%%%%%%%%%%%%%%%%%%%%%%%%%%%%%%%%%%%%%%%%%%
%%%% Файл упрощённых настроек шаблона автореферата %%%%
%%%%%%%%%%%%%%%%%%%%%%%%%%%%%%%%%%%%%%%%%%%%%%%%%%%%%%%

%%% Инициализирование переменных, не трогать!  %%%
\newcounter{showperssign}
\newcounter{showsecrsign}
\newcounter{showopplead}
%%%%%%%%%%%%%%%%%%%%%%%%%%%%%%%%%%%%%%%%%%%%%%%%%%%%%%%

%%% Список публикаций %%%
\makeatletter
\@ifundefined{c@usefootcite}{
  \newcounter{usefootcite}
  \setcounter{usefootcite}{0} % 0 --- два списка литературы;
                              % 1 --- список публикаций автора + цитирование
                              %       других работ в сносках
}{}
\makeatother

\makeatletter
\@ifundefined{c@bibgrouped}{
  \newcounter{bibgrouped}
  \setcounter{bibgrouped}{1}  % 0 --- единый список работ автора;
                              % 1 --- сгруппированные работы автора
}{}
\makeatother

%%% Область упрощённого управления оформлением %%%

%% Управление зазором между подрисуночной подписью и основным текстом %%
\setlength{\belowcaptionskip}{10pt plus 20pt minus 2pt}


%% Подпись таблиц %%

% смещение строк подписи после первой
\newcommand{\tabindent}{0cm}

% тип форматирования таблицы
% plain --- название и текст в одной строке
% split --- название и текст в разных строках
\newcommand{\tabformat}{plain}

%%% настройки форматирования таблицы `plain'

% выравнивание по центру подписи, состоящей из одной строки
% true  --- выравнивать
% false --- не выравнивать
\newcommand{\tabsinglecenter}{false}

% выравнивание подписи таблиц
% justified   --- выравнивать как обычный текст
% centering   --- выравнивать по центру
% centerlast  --- выравнивать по центру только последнюю строку
% centerfirst --- выравнивать по центру только первую строку
% raggedleft  --- выравнивать по правому краю
% raggedright --- выравнивать по левому краю
\newcommand{\tabjust}{justified}

% Разделитель записи «Таблица #» и названия таблицы
\newcommand{\tablabelsep}{~\cyrdash\ }

%%% настройки форматирования таблицы `split'

% положение названия таблицы
% \centering   --- выравнивать по центру
% \raggedleft  --- выравнивать по правому краю
% \raggedright --- выравнивать по левому краю
\newcommand{\splitformatlabel}{\raggedleft}

% положение текста подписи
% \centering   --- выравнивать по центру
% \raggedleft  --- выравнивать по правому краю
% \raggedright --- выравнивать по левому краю
\newcommand{\splitformattext}{\raggedright}

%% Подпись рисунков %%
%Разделитель записи «Рисунок #» и названия рисунка
\newcommand{\figlabelsep}{~\cyrdash\ }  % (ГОСТ 2.105, 4.3.1)
                                        % "--- здесь не работает

%Демонстрация подписи диссертанта на автореферате
\setcounter{showperssign}{1}  % 0 --- не показывать;
                              % 1 --- показывать
%Демонстрация подписи учёного секретаря на автореферате
\setcounter{showsecrsign}{1}  % 0 --- не показывать;
                              % 1 --- показывать
%Демонстрация информации об оппонентах и ведущей организации на автореферате
\setcounter{showopplead}{1}   % 0 --- не показывать;
                              % 1 --- показывать

%%% Цвета гиперссылок %%%
% Latex color definitions: http://latexcolor.com/
\definecolor{linkcolor}{rgb}{0.9,0,0}
\definecolor{citecolor}{rgb}{0,0.6,0}
\definecolor{urlcolor}{rgb}{0,0,1}
%\definecolor{linkcolor}{rgb}{0,0,0} %black
%\definecolor{citecolor}{rgb}{0,0,0} %black
%\definecolor{urlcolor}{rgb}{0,0,0} %black
        % Упрощённые настройки шаблона

%%% Основные сведения %%%
\newcommand{\thesisAuthorLastName}{Моисеенко}
\newcommand{\thesisAuthorOtherNames}{Евгений Александрович}
\newcommand{\thesisAuthorInitials}{E.\,A.}
\newcommand{\thesisAuthor}             % Диссертация, ФИО автора
{%
    \texorpdfstring{% \texorpdfstring takes two arguments and uses the first for (La)TeX and the second for pdf
        \thesisAuthorLastName~\thesisAuthorOtherNames% так будет отображаться на титульном листе или в тексте, где будет использоваться переменная
    }{%
        \thesisAuthorLastName, \thesisAuthorOtherNames% эта запись для свойств pdf-файла. В таком виде, если pdf будет обработан программами для сбора библиографических сведений, будет правильно представлена фамилия.
    }
}
\newcommand{\thesisAuthorShort}        % Диссертация, ФИО автора инициалами
{\thesisAuthorInitials~\thesisAuthorLastName}
%\newcommand{\thesisUdk}                % Диссертация, УДК
%{\fixme{xxx.xxx}}
\newcommand{\thesisTitle}              % Диссертация, название
{Семантика многопоточных систем с слабыми моделями памяти на основе структур событий}
\newcommand{\thesisSpecialtyNumber}    % Диссертация, специальность, номер
{\fixme{XX.XX.XX}}
\newcommand{\thesisSpecialtyTitle}     % Диссертация, специальность, название (название взято с сайта ВАК для примера)
{\fixme{XXX}}
%% \newcommand{\thesisSpecialtyTwoNumber} % Диссертация, вторая специальность, номер
%% {\fixme{XX.XX.XX}}
%% \newcommand{\thesisSpecialtyTwoTitle}  % Диссертация, вторая специальность, название
%% {\fixme{Теория и~методика физического воспитания, спортивной тренировки,
%% оздоровительной и~адаптивной физической культуры}}
\newcommand{\thesisDegree}             % Диссертация, ученая степень
{кандидата физико-математических наук}
\newcommand{\thesisDegreeShort}        % Диссертация, ученая степень, краткая запись
{канд. физ.-мат. наук}
\newcommand{\thesisCity}               % Диссертация, город написания диссертации
{Санкт-Петербург}
\newcommand{\thesisYear}               % Диссертация, год написания диссертации
{\the\year}
\newcommand{\thesisOrganization}       % Диссертация, организация
{\fixme{Федеральное государственное автономное образовательное учреждение высшего
образования <<Длинное название образовательного учреждения <<АББРЕВИАТУРА>>}}
\newcommand{\thesisOrganizationShort}  % Диссертация, краткое название организации для доклада
{\fixme{НазУчДисРаб}}

\newcommand{\thesisInOrganization}     % Диссертация, организация в предложном падеже: Работа выполнена в ...
{кафедре системного программирования Санкт-Петербургского государственного университета}

%% \newcommand{\supervisorDead}{}           % Рисовать рамку вокруг фамилии
\newcommand{\supervisorFio}              % Научный руководитель, ФИО
{Кознов Дмитрий Владимирович}
\newcommand{\supervisorRegalia}          % Научный руководитель, регалии
{доктор технических наук, доцент, профессор кафедры системного программирования}
\newcommand{\supervisorFioShort}         % Научный руководитель, ФИО
{Д.\,В.~Кознов}
\newcommand{\supervisorRegaliaShort}     % Научный руководитель, регалии
{д-р техн. наук, проф.}

%% \newcommand{\supervisorTwoDead}{}        % Рисовать рамку вокруг фамилии
%% \newcommand{\supervisorTwoFio}           % Второй научный руководитель, ФИО
%% {\fixme{Фамилия Имя Отчество}}
%% \newcommand{\supervisorTwoRegalia}       % Второй научный руководитель, регалии
%% {\fixme{уч. степень, уч. звание}}
%% \newcommand{\supervisorTwoFioShort}      % Второй научный руководитель, ФИО
%% {\fixme{И.\,О.~Фамилия}}
%% \newcommand{\supervisorTwoRegaliaShort}  % Второй научный руководитель, регалии
%% {\fixme{уч.~ст.,~уч.~зв.}}

\newcommand{\opponentOneFio}           % Оппонент 1, ФИО
{\fixme{Фамилия Имя Отчество}}
\newcommand{\opponentOneRegalia}       % Оппонент 1, регалии
{\fixme{доктор физико-математических наук, профессор}}
\newcommand{\opponentOneJobPlace}      % Оппонент 1, место работы
{\fixme{Не очень длинное название для места работы}}
\newcommand{\opponentOneJobPost}       % Оппонент 1, должность
{\fixme{старший научный сотрудник}}

\newcommand{\opponentTwoFio}           % Оппонент 2, ФИО
{\fixme{Фамилия Имя Отчество}}
\newcommand{\opponentTwoRegalia}       % Оппонент 2, регалии
{\fixme{кандидат физико-математических наук}}
\newcommand{\opponentTwoJobPlace}      % Оппонент 2, место работы
{\fixme{Основное место работы c длинным длинным длинным длинным названием}}
\newcommand{\opponentTwoJobPost}       % Оппонент 2, должность
{\fixme{старший научный сотрудник}}

%% \newcommand{\opponentThreeFio}         % Оппонент 3, ФИО
%% {\fixme{Фамилия Имя Отчество}}
%% \newcommand{\opponentThreeRegalia}     % Оппонент 3, регалии
%% {\fixme{кандидат физико-математических наук}}
%% \newcommand{\opponentThreeJobPlace}    % Оппонент 3, место работы
%% {\fixme{Основное место работы c длинным длинным длинным длинным названием}}
%% \newcommand{\opponentThreeJobPost}     % Оппонент 3, должность
%% {\fixme{старший научный сотрудник}}

\newcommand{\leadingOrganizationTitle} % Ведущая организация, дополнительные строки. Удалить, чтобы не отображать в автореферате
{\fixme{XXX}}

\newcommand{\defenseDate}              % Защита, дата
{\fixme{DD mmmmmmmm YYYY~г.~в~XX часов}}
\newcommand{\defenseCouncilNumber}     % Защита, номер диссертационного совета
{\fixme{Д\,123.456.78}}
\newcommand{\defenseCouncilTitle}      % Защита, учреждение диссертационного совета
{\fixme{Название учреждения}}
\newcommand{\defenseCouncilAddress}    % Защита, адрес учреждение диссертационного совета
{\fixme{Адрес}}
\newcommand{\defenseCouncilPhone}      % Телефон для справок
{\fixme{+7~(0000)~00-00-00}}

\newcommand{\defenseSecretaryFio}      % Секретарь диссертационного совета, ФИО
{\fixme{Фамилия Имя Отчество}}
\newcommand{\defenseSecretaryRegalia}  % Секретарь диссертационного совета, регалии
{\fixme{д-р~физ.-мат. наук}}            % Для сокращений есть ГОСТы, например: ГОСТ Р 7.0.12-2011 + http://base.garant.ru/179724/#block_30000

\newcommand{\synopsisLibrary}          % Автореферат, название библиотеки
{\fixme{Название библиотеки}}
\newcommand{\synopsisDate}             % Автореферат, дата рассылки
{\fixme{DD mmmmmmmm}\the\year~года}

% To avoid conflict with beamer class use \providecommand
\providecommand{\keywords}%            % Ключевые слова для метаданных PDF диссертации и автореферата
{}
           % Основные сведения
\input{common/fonts}          % Определение шрифтов (частичное)
\input{common/styles}         % Стили общие для диссертации и автореферата
\input{Synopsis/synstyles}    % Стили для автореферата
% для вертикального центрирования ячеек в tabulary
\def\zz{\ifx\[$\else\aftergroup\zzz\fi}
%$ \] % <-- чиним подсветку синтаксиса в некоторых редакторах
\def\zzz{\setbox0\lastbox
\dimen0\dimexpr\extrarowheight + \ht0-\dp0\relax
\setbox0\hbox{\raise-.5\dimen0\box0}%
\ht0=\dimexpr\ht0+\extrarowheight\relax
\dp0=\dimexpr\dp0+\extrarowheight\relax
\box0
}

\lstdefinelanguage{Renhanced}%
{keywords={abbreviate,abline,abs,acos,acosh,action,add1,add,%
        aggregate,alias,Alias,alist,all,anova,any,aov,aperm,append,apply,%
        approx,approxfun,apropos,Arg,args,array,arrows,as,asin,asinh,%
        atan,atan2,atanh,attach,attr,attributes,autoload,autoloader,ave,%
        axis,backsolve,barplot,basename,besselI,besselJ,besselK,besselY,%
        beta,binomial,body,box,boxplot,break,browser,bug,builtins,bxp,by,%
        c,C,call,Call,case,cat,category,cbind,ceiling,character,char,%
        charmatch,check,chol,chol2inv,choose,chull,class,close,cm,codes,%
        coef,coefficients,co,col,colnames,colors,colours,commandArgs,%
        comment,complete,complex,conflicts,Conj,contents,contour,%
        contrasts,contr,control,helmert,contrib,convolve,cooks,coords,%
        distance,coplot,cor,cos,cosh,count,fields,cov,covratio,wt,CRAN,%
        create,crossprod,cummax,cummin,cumprod,cumsum,curve,cut,cycle,D,%
        data,dataentry,date,dbeta,dbinom,dcauchy,dchisq,de,debug,%
        debugger,Defunct,default,delay,delete,deltat,demo,de,density,%
        deparse,dependencies,Deprecated,deriv,description,detach,%
        dev2bitmap,dev,cur,deviance,off,prev,,dexp,df,dfbetas,dffits,%
        dgamma,dgeom,dget,dhyper,diag,diff,digamma,dim,dimnames,dir,%
        dirname,dlnorm,dlogis,dnbinom,dnchisq,dnorm,do,dotplot,double,%
        download,dpois,dput,drop,drop1,dsignrank,dt,dummy,dump,dunif,%
        duplicated,dweibull,dwilcox,dyn,edit,eff,effects,eigen,else,%
        emacs,end,environment,env,erase,eval,equal,evalq,example,exists,%
        exit,exp,expand,expression,External,extract,extractAIC,factor,%
        fail,family,fft,file,filled,find,fitted,fivenum,fix,floor,for,%
        For,formals,format,formatC,formula,Fortran,forwardsolve,frame,%
        frequency,ftable,ftable2table,function,gamma,Gamma,gammaCody,%
        gaussian,gc,gcinfo,gctorture,get,getenv,geterrmessage,getOption,%
        getwd,gl,glm,globalenv,gnome,GNOME,graphics,gray,grep,grey,grid,%
        gsub,hasTsp,hat,heat,help,hist,home,hsv,httpclient,I,identify,if,%
        ifelse,Im,image,\%in\%,index,influence,measures,inherits,install,%
        installed,integer,interaction,interactive,Internal,intersect,%
        inverse,invisible,IQR,is,jitter,kappa,kronecker,labels,lapply,%
        layout,lbeta,lchoose,lcm,legend,length,levels,lgamma,library,%
        licence,license,lines,list,lm,load,local,locator,log,log10,log1p,%
        log2,logical,loglin,lower,lowess,ls,lsfit,lsf,ls,machine,Machine,%
        mad,mahalanobis,make,link,margin,match,Math,matlines,mat,matplot,%
        matpoints,matrix,max,mean,median,memory,menu,merge,methods,min,%
        missing,Mod,mode,model,response,mosaicplot,mtext,mvfft,na,nan,%
        names,omit,nargs,nchar,ncol,NCOL,new,next,NextMethod,nextn,%
        nlevels,nlm,noquote,NotYetImplemented,NotYetUsed,nrow,NROW,null,%
        numeric,\%o\%,objects,offset,old,on,Ops,optim,optimise,optimize,%
        options,or,order,ordered,outer,package,packages,page,pairlist,%
        pairs,palette,panel,par,parent,parse,paste,path,pbeta,pbinom,%
        pcauchy,pchisq,pentagamma,persp,pexp,pf,pgamma,pgeom,phyper,pico,%
        pictex,piechart,Platform,plnorm,plogis,plot,pmatch,pmax,pmin,%
        pnbinom,pnchisq,pnorm,points,poisson,poly,polygon,polyroot,pos,%
        postscript,power,ppoints,ppois,predict,preplot,pretty,Primitive,%
        print,prmatrix,proc,prod,profile,proj,prompt,prop,provide,%
        psignrank,ps,pt,ptukey,punif,pweibull,pwilcox,q,qbeta,qbinom,%
        qcauchy,qchisq,qexp,qf,qgamma,qgeom,qhyper,qlnorm,qlogis,qnbinom,%
        qnchisq,qnorm,qpois,qqline,qqnorm,qqplot,qr,Q,qty,qy,qsignrank,%
        qt,qtukey,quantile,quasi,quit,qunif,quote,qweibull,qwilcox,%
        rainbow,range,rank,rbeta,rbind,rbinom,rcauchy,rchisq,Re,read,csv,%
        csv2,fwf,readline,socket,real,Recall,rect,reformulate,regexpr,%
        relevel,remove,rep,repeat,replace,replications,report,require,%
        resid,residuals,restart,return,rev,rexp,rf,rgamma,rgb,rgeom,R,%
        rhyper,rle,rlnorm,rlogis,rm,rnbinom,RNGkind,rnorm,round,row,%
        rownames,rowsum,rpois,rsignrank,rstandard,rstudent,rt,rug,runif,%
        rweibull,rwilcox,sample,sapply,save,scale,scan,scan,screen,sd,se,%
        search,searchpaths,segments,seq,sequence,setdiff,setequal,set,%
        setwd,show,sign,signif,sin,single,sinh,sink,solve,sort,source,%
        spline,splinefun,split,sqrt,stars,start,stat,stem,step,stop,%
        storage,strstrheight,stripplot,strsplit,structure,strwidth,sub,%
        subset,substitute,substr,substring,sum,summary,sunflowerplot,svd,%
        sweep,switch,symbol,symbols,symnum,sys,status,system,t,table,%
        tabulate,tan,tanh,tapply,tempfile,terms,terrain,tetragamma,text,%
        time,title,topo,trace,traceback,transform,tri,trigamma,trunc,try,%
        ts,tsp,typeof,unclass,undebug,undoc,union,unique,uniroot,unix,%
        unlink,unlist,unname,untrace,update,upper,url,UseMethod,var,%
        variable,vector,Version,vi,warning,warnings,weighted,weights,%
        which,while,window,write,\%x\%,x11,X11,xedit,xemacs,xinch,xor,%
        xpdrows,xy,xyinch,yinch,zapsmall,zip},%
    otherkeywords={!,!=,~,$,*,\%,\&,\%/\%,\%*\%,\%\%,<-,<<-},%$
    alsoother={._$},%$
    sensitive,%
    morecomment=[l]\#,%
    morestring=[d]",%
    morestring=[d]'% 2001 Robert Denham
}%

%решаем проблему с кириллицей в комментариях (в pdflatex) https://tex.stackexchange.com/a/103712
\lstset{extendedchars=true,keepspaces=true,literate={Ö}{{\"O}}1
    {Ä}{{\"A}}1
    {Ü}{{\"U}}1
    {ß}{{\ss}}1
    {ü}{{\"u}}1
    {ä}{{\"a}}1
    {ö}{{\"o}}1
    {~}{{\textasciitilde}}1
    {а}{{\selectfont\char224}}1
    {б}{{\selectfont\char225}}1
    {в}{{\selectfont\char226}}1
    {г}{{\selectfont\char227}}1
    {д}{{\selectfont\char228}}1
    {е}{{\selectfont\char229}}1
    {ё}{{\"e}}1
    {ж}{{\selectfont\char230}}1
    {з}{{\selectfont\char231}}1
    {и}{{\selectfont\char232}}1
    {й}{{\selectfont\char233}}1
    {к}{{\selectfont\char234}}1
    {л}{{\selectfont\char235}}1
    {м}{{\selectfont\char236}}1
    {н}{{\selectfont\char237}}1
    {о}{{\selectfont\char238}}1
    {п}{{\selectfont\char239}}1
    {р}{{\selectfont\char240}}1
    {с}{{\selectfont\char241}}1
    {т}{{\selectfont\char242}}1
    {у}{{\selectfont\char243}}1
    {ф}{{\selectfont\char244}}1
    {х}{{\selectfont\char245}}1
    {ц}{{\selectfont\char246}}1
    {ч}{{\selectfont\char247}}1
    {ш}{{\selectfont\char248}}1
    {щ}{{\selectfont\char249}}1
    {ъ}{{\selectfont\char250}}1
    {ы}{{\selectfont\char251}}1
    {ь}{{\selectfont\char252}}1
    {э}{{\selectfont\char253}}1
    {ю}{{\selectfont\char254}}1
    {я}{{\selectfont\char255}}1
    {А}{{\selectfont\char192}}1
    {Б}{{\selectfont\char193}}1
    {В}{{\selectfont\char194}}1
    {Г}{{\selectfont\char195}}1
    {Д}{{\selectfont\char196}}1
    {Е}{{\selectfont\char197}}1
    {Ё}{{\"E}}1
    {Ж}{{\selectfont\char198}}1
    {З}{{\selectfont\char199}}1
    {И}{{\selectfont\char200}}1
    {Й}{{\selectfont\char201}}1
    {К}{{\selectfont\char202}}1
    {Л}{{\selectfont\char203}}1
    {М}{{\selectfont\char204}}1
    {Н}{{\selectfont\char205}}1
    {О}{{\selectfont\char206}}1
    {П}{{\selectfont\char207}}1
    {Р}{{\selectfont\char208}}1
    {С}{{\selectfont\char209}}1
    {Т}{{\selectfont\char210}}1
    {У}{{\selectfont\char211}}1
    {Ф}{{\selectfont\char212}}1
    {Х}{{\selectfont\char213}}1
    {Ц}{{\selectfont\char214}}1
    {Ч}{{\selectfont\char215}}1
    {Ш}{{\selectfont\char216}}1
    {Щ}{{\selectfont\char217}}1
    {Ъ}{{\selectfont\char218}}1
    {Ы}{{\selectfont\char219}}1
    {Ь}{{\selectfont\char220}}1
    {Э}{{\selectfont\char221}}1
    {Ю}{{\selectfont\char222}}1
    {Я}{{\selectfont\char223}}1
    {і}{{\selectfont\char105}}1
    {ї}{{\selectfont\char168}}1
    {є}{{\selectfont\char185}}1
    {ґ}{{\selectfont\char160}}1
    {І}{{\selectfont\char73}}1
    {Ї}{{\selectfont\char136}}1
    {Є}{{\selectfont\char153}}1
    {Ґ}{{\selectfont\char128}}1
}

% Ширина текста минус ширина надписи 999
\newlength{\twless}
\newlength{\lmarg}
\setlength{\lmarg}{\widthof{999}}   % ширина надписи 999
\setlength{\twless}{\textwidth-\lmarg}

\lstset{ %
%    language=R,                     %  Язык указать здесь, если во всех листингах преимущественно один язык, в результате часть настроек может пойти только для этого языка
    numbers=left,                   % where to put the line-numbers
    numberstyle=\fontsize{12pt}{14pt}\selectfont\color{Gray},  % the style that is used for the line-numbers
    firstnumber=1,                  % в этой и следующей строках задаётся поведение нумерации 5, 10, 15...
    stepnumber=5,                   % the step between two line-numbers. If it's 1, each line will be numbered
    numbersep=5pt,                  % how far the line-numbers are from the code
    backgroundcolor=\color{white},  % choose the background color. You must add \usepackage{color}
    showspaces=false,               % show spaces adding particular underscores
    showstringspaces=false,         % underline spaces within strings
    showtabs=false,                 % show tabs within strings adding particular underscores
    frame=leftline,                 % adds a frame of different types around the code
    rulecolor=\color{black},        % if not set, the frame-color may be changed on line-breaks within not-black text (e.g. commens (green here))
    tabsize=2,                      % sets default tabsize to 2 spaces
    captionpos=t,                   % sets the caption-position to top
    breaklines=true,                % sets automatic line breaking
    breakatwhitespace=false,        % sets if automatic breaks should only happen at whitespace
%    title=\lstname,                 % show the filename of files included with \lstinputlisting;
    % also try caption instead of title
    basicstyle=\fontsize{12pt}{14pt}\selectfont\ttfamily,% the size of the fonts that are used for the code
%    keywordstyle=\color{blue},      % keyword style
    commentstyle=\color{ForestGreen}\emph,% comment style
    stringstyle=\color{Mahogany},   % string literal style
    escapeinside={\%*}{*)},         % if you want to add a comment within your code
    morekeywords={*,...},           % if you want to add more keywords to the set
    inputencoding=utf8,             % кодировка кода
    xleftmargin={\lmarg},           % Чтобы весь код и полоска с номерами строк была смещена влево, так чтобы цифры не вылезали за пределы текста слева
}

%http://tex.stackexchange.com/questions/26872/smaller-frame-with-listings
% Окружение, чтобы листинг был компактнее обведен рамкой, если она задается, а не на всю ширину текста
\makeatletter
\newenvironment{SmallListing}[1][]
{\lstset{#1}\VerbatimEnvironment\begin{VerbatimOut}{VerbEnv.tmp}}
{\end{VerbatimOut}\settowidth\@tempdima{%
        \lstinputlisting{VerbEnv.tmp}}
    \minipage{\@tempdima}\lstinputlisting{VerbEnv.tmp}\endminipage}
\makeatother

\DefineVerbatimEnvironment% с шрифтом 12 пт
{Verb}{Verbatim}
{fontsize=\fontsize{12pt}{14pt}\selectfont}

%% \newfloat[chapter]{ListingEnv}{lol}{Листинг}

\renewcommand{\lstlistingname}{Листинг}

%Общие счётчики окружений листингов
%http://tex.stackexchange.com/questions/145546/how-to-make-figure-and-listing-share-their-counter
% Если смешивать плавающие и не плавающие окружения, то могут быть проблемы с нумерацией
\makeatletter
\AfterEndPreamble{% https://tex.stackexchange.com/a/252682
    \let\c@ListingEnv\relax % drop existing counter "ListingEnv"
    \newaliascnt{ListingEnv}{lstlisting} % команда требует пакет aliascnt
    \let\ftype@lstlisting\ftype@ListingEnv % give the floats the same precedence
}
\makeatother

% значок С++ — используйте команду \cpp
\newcommand{\cpp}{%
    C\nolinebreak\hspace{-.05em}%
    \raisebox{.2ex}{+}\nolinebreak\hspace{-.10em}%
    \raisebox{.2ex}{+}%
}

%%%  Чересстрочное форматирование таблиц
%% http://tex.stackexchange.com/questions/278362/apply-italic-formatting-to-every-other-row
\newcounter{rowcnt}
\newcommand\altshape{\ifnumodd{\value{rowcnt}}{\color{red}}{\vspace*{-1ex}\itshape}}
% \AtBeginEnvironment{tabular}{\setcounter{rowcnt}{1}}
% \AtEndEnvironment{tabular}{\setcounter{rowcnt}{0}}

%%% Ради примера во второй главе
\let\originalepsilon\epsilon
\let\originalphi\phi
\let\originalkappa\kappa
\let\originalle\le
\let\originalleq\leq
\let\originalge\ge
\let\originalgeq\geq
\let\originalemptyset\emptyset
\let\originaltan\tan
\let\originalcot\cot
\let\originalcsc\csc

%%% Русская традиция начертания математических знаков
\renewcommand{\le}{\ensuremath{\leqslant}}
\renewcommand{\leq}{\ensuremath{\leqslant}}
\renewcommand{\ge}{\ensuremath{\geqslant}}
\renewcommand{\geq}{\ensuremath{\geqslant}}
\renewcommand{\emptyset}{\varnothing}

%%% Русская традиция начертания математических функций (на случай копирования из зарубежных источников)
\renewcommand{\tan}{\operatorname{tg}}
\renewcommand{\cot}{\operatorname{ctg}}
\renewcommand{\csc}{\operatorname{cosec}}

%%% Русская традиция начертания греческих букв (греческие буквы вертикальные, через пакет upgreek)
\renewcommand{\epsilon}{\ensuremath{\upvarepsilon}}   %  русская традиция записи
\renewcommand{\phi}{\ensuremath{\upvarphi}}
%\renewcommand{\kappa}{\ensuremath{\varkappa}}
\renewcommand{\alpha}{\upalpha}
\renewcommand{\beta}{\upbeta}
\renewcommand{\gamma}{\upgamma}
\renewcommand{\delta}{\updelta}
\renewcommand{\varepsilon}{\upvarepsilon}
\renewcommand{\zeta}{\upzeta}
\renewcommand{\eta}{\upeta}
\renewcommand{\theta}{\uptheta}
\renewcommand{\vartheta}{\upvartheta}
\renewcommand{\iota}{\upiota}
\renewcommand{\kappa}{\upkappa}
\renewcommand{\lambda}{\uplambda}
\renewcommand{\mu}{\upmu}
\renewcommand{\nu}{\upnu}
\renewcommand{\xi}{\upxi}
\renewcommand{\pi}{\uppi}
\renewcommand{\varpi}{\upvarpi}
\renewcommand{\rho}{\uprho}
%\renewcommand{\varrho}{\upvarrho}
\renewcommand{\sigma}{\upsigma}
%\renewcommand{\varsigma}{\upvarsigma}
\renewcommand{\tau}{\uptau}
\renewcommand{\upsilon}{\upupsilon}
\renewcommand{\varphi}{\upvarphi}
\renewcommand{\chi}{\upchi}
\renewcommand{\psi}{\uppsi}
\renewcommand{\omega}{\upomega}
   % Стили для специфических пользовательских задач

%%% Библиография. Выбор движка для реализации %%%
\ifnumequal{\value{bibliosel}}{0}{%
    \input{biblio/predefined} % Встроенная реализация с загрузкой файла через движок bibtex8
}{
    %%% Реализация библиографии пакетами biblatex и biblatex-gost с использованием движка biber %%%

\usepackage{csquotes} % biblatex рекомендует его подключать. Пакет для оформления сложных блоков цитирования.
%%% Загрузка пакета с основными настройками %%%
\makeatletter
\ifnumequal{\value{draft}}{0}{% Чистовик
\usepackage[%
backend=biber,% движок
bibencoding=utf8,% кодировка bib файла
sorting=none,% настройка сортировки списка литературы
style=gost-numeric,% стиль цитирования и библиографии (по ГОСТ)
language=autobib,% получение языка из babel/polyglossia, default: autobib % если ставить autocite или auto, то цитаты в тексте с указанием страницы, получат указание страницы на языке оригинала
autolang=other,% многоязычная библиография
clearlang=true,% внутренний сброс поля language, если он совпадает с языком из babel/polyglossia
defernumbers=true,% нумерация проставляется после двух компиляций, зато позволяет выцеплять библиографию по ключевым словам и нумеровать не из большего списка
sortcites=true,% сортировать номера затекстовых ссылок при цитировании (если в квадратных скобках несколько ссылок, то отображаться будут отсортированно, а не абы как)
doi=false,% Показывать или нет ссылки на DOI
isbn=false,% Показывать или нет ISBN, ISSN, ISRN
]{biblatex}[2016/09/17]
\ltx@iffilelater{biblatex-gost.def}{2017/05/03}%
{\toggletrue{bbx:gostbibliography}%
\renewcommand*{\revsdnamepunct}{\addcomma}}{}
}{%Черновик
\usepackage[%
backend=biber,% движок
bibencoding=utf8,% кодировка bib файла
sorting=none,% настройка сортировки списка литературы
% defernumbers=true, % откомментируйте, если требуется правильная нумерация ссылок на литературу в режиме черновика. Замедляет сборку
]{biblatex}[2016/09/17]%
}
\makeatother

\providebool{blxmc} % biblatex version needs and has MakeCapital workaround
\boolfalse{blxmc} % setting our new boolean flag to default false
\ifxetexorluatex
\else
% Исправление случая неподдержки знака номера в pdflatex
    \DefineBibliographyStrings{russian}{number={\textnumero}}

% Исправление случая отсутствия прописных букв в некоторых случаях
% https://github.com/plk/biblatex/issues/960#issuecomment-596658282
    \ifdefmacro{\ExplSyntaxOn}{}{\usepackage{expl3}}
    \makeatletter
    \ltx@ifpackagelater{biblatex}{2020/02/23}{
    % Assuming this version of biblatex defines MakeCapital correctly
    }{
        \ltx@ifpackagelater{biblatex}{2019/12/01}{
            % Assuming this version of biblatex defines MakeCapital incorrectly
            \usepackage{expl3}[2020/02/25]
            \@ifpackagelater{expl3}{2020/02/25}{
                \booltrue{blxmc} % setting our new boolean flag to true
            }{}
        }{}
    }
    \makeatother
    \ifblxmc
        \typeout{Assuming this version of biblatex defines MakeCapital
        incorrectly}
        \usepackage{xparse}
        \makeatletter
        \ExplSyntaxOn
        \NewDocumentCommand \blx@maketext@lowercase {m}
          {
            \text_lowercase:n {#1}
          }

        \NewDocumentCommand \blx@maketext@uppercase {m}
          {
            \text_uppercase:n {#1}
          }

        \RenewDocumentCommand \MakeCapital {m}
          {
            \text_titlecase_first:n {#1}
          }
        \ExplSyntaxOff

        \protected\def\blx@biblcstring#1#2#3{%
          \blx@begunit
          \blx@hyphenreset
          \blx@bibstringsimple
          \lowercase{\edef\blx@tempa{#3}}%
          \ifcsundef{#2@\blx@tempa}
            {\blx@warn@nostring\blx@tempa
             \blx@endnounit}
            {#1{\blx@maketext@lowercase{\csuse{#2@\blx@tempa}}}%
             \blx@endunit}}

        \protected\def\blx@bibucstring#1#2#3{%
          \blx@begunit
          \blx@hyphenreset
          \blx@bibstringsimple
          \lowercase{\edef\blx@tempa{#3}}%
          \ifcsundef{#2@\blx@tempa}
            {\blx@warn@nostring\blx@tempa
             \blx@endnounit}
            {#1{\blx@maketext@uppercase{\csuse{#2@\blx@tempa}}}%
             \blx@endunit}}
        \makeatother
    \fi
\fi

\ifsynopsis
\ifnumgreater{\value{usefootcite}}{0}{
    \ExecuteBibliographyOptions{autocite=footnote}
    \newbibmacro*{cite:full}{%
        \printtext[bibhypertarget]{%
            \usedriver{%
                \DeclareNameAlias{sortname}{default}%
            }{%
                \thefield{entrytype}%
            }%
        }%
        \usebibmacro{shorthandintro}%
    }
    \DeclareCiteCommand{\smartcite}[\mkbibfootnote]{%
        \usebibmacro{prenote}%
    }{%
        \usebibmacro{citeindex}%
        \usebibmacro{cite:full}%
    }{%
        \multicitedelim%
    }{%
        \usebibmacro{postnote}%
    }
}{}
\fi

%%% Подключение файлов bib %%%
\addbibresource[label=bl-external]{biblio/external.bib}
\addbibresource[label=bl-author]{biblio/author.bib}
\addbibresource[label=bl-registered]{biblio/registered.bib}

%http://tex.stackexchange.com/a/141831/79756
%There is a way to automatically map the language field to the langid field. The following lines in the preamble should be enough to do that.
%This command will copy the language field into the langid field and will then delete the contents of the language field. The language field will only be deleted if it was successfully copied into the langid field.
\DeclareSourcemap{ %модификация bib файла перед тем, как им займётся biblatex
    \maps{
        \map{% перекидываем значения полей language в поля langid, которыми пользуется biblatex
            \step[fieldsource=language, fieldset=langid, origfieldval, final]
            \step[fieldset=language, null]
        }
        \map{% перекидываем значения полей numpages в поля pagetotal, которыми пользуется biblatex
            \step[fieldsource=numpages, fieldset=pagetotal, origfieldval, final]
            \step[fieldset=numpages, null]
        }
        \map{% перекидываем значения полей pagestotal в поля pagetotal, которыми пользуется biblatex
            \step[fieldsource=pagestotal, fieldset=pagetotal, origfieldval, final]
            \step[fieldset=pagestotal, null]
        }
        \map[overwrite]{% перекидываем значения полей shortjournal, если они есть, в поля journal, которыми пользуется biblatex
            \step[fieldsource=shortjournal, final]
            \step[fieldset=journal, origfieldval]
            \step[fieldset=shortjournal, null]
        }
        \map[overwrite]{% перекидываем значения полей shortbooktitle, если они есть, в поля booktitle, которыми пользуется biblatex
            \step[fieldsource=shortbooktitle, final]
            \step[fieldset=booktitle, origfieldval]
            \step[fieldset=shortbooktitle, null]
        }
        \map{% если в поле medium написано "Электронный ресурс", то устанавливаем поле media, которым пользуется biblatex, в значение eresource.
            \step[fieldsource=medium,
            match=\regexp{Электронный\s+ресурс},
            final]
            \step[fieldset=media, fieldvalue=eresource]
            \step[fieldset=medium, null]
        }
        \map[overwrite]{% стираем значения всех полей issn
            \step[fieldset=issn, null]
        }
        \map[overwrite]{% стираем значения всех полей abstract, поскольку ими не пользуемся, а там бывают "неприятные" латеху символы
            \step[fieldsource=abstract]
            \step[fieldset=abstract,null]
        }
        \map[overwrite]{ % переделка формата записи даты
            \step[fieldsource=urldate,
            match=\regexp{([0-9]{2})\.([0-9]{2})\.([0-9]{4})},
            replace={$3-$2-$1$4}, % $4 вставлен исключительно ради нормальной работы программ подсветки синтаксиса, которые некорректно обрабатывают $ в таких конструкциях
            final]
        }
        \map[overwrite]{ % стираем ключевые слова
            \step[fieldsource=keywords]
            \step[fieldset=keywords,null]
        }
        % реализация foreach различается для biblatex v3.12 и v3.13.
        % Для версии v3.13 эта конструкция заменяет последующие 7 структур map
        % \map[overwrite,foreach={authorvak,authorscopus,authorwos,authorconf,authorother,authorparent,authorprogram}]{ % записываем информацию о типе публикации в ключевые слова
        %     \step[fieldsource=$MAPLOOP,final=true]
        %     \step[fieldset=keywords,fieldvalue={,biblio$MAPLOOP},append=true]
        % }
        \map[overwrite]{ % записываем информацию о типе публикации в ключевые слова
            \step[fieldsource=authorvak,final=true]
            \step[fieldset=keywords,fieldvalue={,biblioauthorvak},append=true]
        }
        \map[overwrite]{ % записываем информацию о типе публикации в ключевые слова
            \step[fieldsource=authorscopus,final=true]
            \step[fieldset=keywords,fieldvalue={,biblioauthorscopus},append=true]
        }
        \map[overwrite]{ % записываем информацию о типе публикации в ключевые слова
            \step[fieldsource=authorwos,final=true]
            \step[fieldset=keywords,fieldvalue={,biblioauthorwos},append=true]
        }
        \map[overwrite]{ % записываем информацию о типе публикации в ключевые слова
            \step[fieldsource=authorconf,final=true]
            \step[fieldset=keywords,fieldvalue={,biblioauthorconf},append=true]
        }
        \map[overwrite]{ % записываем информацию о типе публикации в ключевые слова
            \step[fieldsource=authorother,final=true]
            \step[fieldset=keywords,fieldvalue={,biblioauthorother},append=true]
        }
        \map[overwrite]{ % записываем информацию о типе публикации в ключевые слова
            \step[fieldsource=authorpatent,final=true]
            \step[fieldset=keywords,fieldvalue={,biblioauthorpatent},append=true]
        }
        \map[overwrite]{ % записываем информацию о типе публикации в ключевые слова
            \step[fieldsource=authorprogram,final=true]
            \step[fieldset=keywords,fieldvalue={,biblioauthorprogram},append=true]
        }
        \map[overwrite]{ % добавляем ключевые слова, чтобы различать источники
            \perdatasource{biblio/external.bib}
            \step[fieldset=keywords, fieldvalue={,biblioexternal},append=true]
        }
        \map[overwrite]{ % добавляем ключевые слова, чтобы различать источники
            \perdatasource{biblio/author.bib}
            \step[fieldset=keywords, fieldvalue={,biblioauthor},append=true]
        }
        \map[overwrite]{ % добавляем ключевые слова, чтобы различать источники
            \perdatasource{biblio/registered.bib}
            \step[fieldset=keywords, fieldvalue={,biblioregistered},append=true]
        }
        \map[overwrite]{ % добавляем ключевые слова, чтобы различать источники
            \step[fieldset=keywords, fieldvalue={,bibliofull},append=true]
        }
%        \map[overwrite]{% стираем значения всех полей series
%            \step[fieldset=series, null]
%        }
        \map[overwrite]{% перекидываем значения полей howpublished в поля organization для типа online
            \step[typesource=online, typetarget=online, final]
            \step[fieldsource=howpublished, fieldset=organization, origfieldval]
            \step[fieldset=howpublished, null]
        }
    }
}

\ifnumequal{\value{mediadisplay}}{1}{
    \DeclareSourcemap{
        \maps{%
            % \map{% использование media=text по умолчанию
            %     \step[fieldset=media, fieldvalue=text]
            % }
        }
    }
}{}
\ifnumequal{\value{mediadisplay}}{2}{
    \DeclareSourcemap{
        \maps{%
            \map[overwrite]{% удаление всех записей media
                \step[fieldset=media, null]
            }
        }
    }
}{}
\ifnumequal{\value{mediadisplay}}{3}{
    \DeclareSourcemap{
        \maps{
            \map[overwrite]{% стираем значения всех полей media=text
                \step[fieldsource=media,match={text},final]
                \step[fieldset=media, null]
            }
        }
    }
}{}
\ifnumequal{\value{mediadisplay}}{4}{
    \DeclareSourcemap{
        \maps{
            \map[overwrite]{% стираем значения всех полей media=eresource
                \step[fieldsource=media,match={eresource},final]
                \step[fieldset=media, null]
            }
        }
    }
}{}

\ifsynopsis
\else
\DeclareSourcemap{ %модификация bib файла перед тем, как им займётся biblatex
    \maps{
        \map[overwrite]{% стираем значения всех полей addendum
            \perdatasource{biblio/author.bib}
            \step[fieldset=addendum, null] %чтобы избавиться от информации об объёме авторских статей, в отличие от автореферата
        }
    }
}
\fi

\ifpresentation
% удаляем лишние поля в списке литературы презентации
% их названия можно узнать в файле presentation.bbl
\DeclareSourcemap{
    \maps{
    \map[overwrite,foreach={%
        % {{{ Список лишних полей в презентации
        address,%
        chapter,%
        edition,%
        editor,%
        eid,%
        howpublished,%
        institution,%
        key,%
        month,%
        note,%
        number,%
        organization,%
        pages,%
        publisher,%
        school,%
        series,%
        type,%
        media,%
        url,%
        doi,%
        location,%
        volume,%
        % Список лишних полей в презентации }}}
    }]{
        \perdatasource{biblio/author.bib}
        \step[fieldset=$MAPLOOP,null]
    }
    }
}
\fi

\defbibfilter{vakscopuswos}{%
    keyword=biblioauthorvak or keyword=biblioauthorscopus or keyword=biblioauthorwos
}

\defbibfilter{scopuswos}{%
    keyword=biblioauthorscopus or keyword=biblioauthorwos
}

\defbibfilter{papersregistered}{%
    keyword=biblioauthor or keyword=biblioregistered
}

%%% Убираем неразрывные пробелы перед двоеточием и точкой с запятой %%%
%\makeatletter
%\ifnumequal{\value{draft}}{0}{% Чистовик
%    \renewcommand*{\addcolondelim}{%
%      \begingroup%
%      \def\abx@colon{%
%        \ifdim\lastkern>\z@\unkern\fi%
%        \abx@puncthook{:}\space}%
%      \addcolon%
%      \endgroup}
%
%    \renewcommand*{\addsemicolondelim}{%
%      \begingroup%
%      \def\abx@semicolon{%
%        \ifdim\lastkern>\z@\unkern\fi%
%        \abx@puncthook{;}\space}%
%      \addsemicolon%
%      \endgroup}
%}{}
%\makeatother

%%% Правка записей типа thesis, чтобы дважды не писался автор
%\ifnumequal{\value{draft}}{0}{% Чистовик
%\DeclareBibliographyDriver{thesis}{%
%  \usebibmacro{bibindex}%
%  \usebibmacro{begentry}%
%  \usebibmacro{heading}%
%  \newunit
%  \usebibmacro{author}%
%  \setunit*{\labelnamepunct}%
%  \usebibmacro{thesistitle}%
%  \setunit{\respdelim}%
%  %\printnames[last-first:full]{author}%Вот эту строчку нужно убрать, чтобы автор диссертации не дублировался
%  \newunit\newblock
%  \printlist[semicolondelim]{specdata}%
%  \newunit
%  \usebibmacro{institution+location+date}%
%  \newunit\newblock
%  \usebibmacro{chapter+pages}%
%  \newunit
%  \printfield{pagetotal}%
%  \newunit\newblock
%  \usebibmacro{doi+eprint+url+note}%
%  \newunit\newblock
%  \usebibmacro{addendum+pubstate}%
%  \setunit{\bibpagerefpunct}\newblock
%  \usebibmacro{pageref}%
%  \newunit\newblock
%  \usebibmacro{related:init}%
%  \usebibmacro{related}%
%  \usebibmacro{finentry}}
%}{}

%\newbibmacro{string+doi}[1]{% новая макрокоманда на простановку ссылки на doi
%    \iffieldundef{doi}{#1}{\href{http://dx.doi.org/\thefield{doi}}{#1}}}

%\ifnumequal{\value{draft}}{0}{% Чистовик
%\renewcommand*{\mkgostheading}[1]{\usebibmacro{string+doi}{#1}} % ссылка на doi с авторов. стоящих впереди записи
%\renewcommand*{\mkgostheading}[1]{#1} % только лишь убираем курсив с авторов
%}{}
%\DeclareFieldFormat{title}{\usebibmacro{string+doi}{#1}} % ссылка на doi с названия работы
%\DeclareFieldFormat{journaltitle}{\usebibmacro{string+doi}{#1}} % ссылка на doi с названия журнала
%%% Тире как разделитель в библиографии традиционной руской длины:
\renewcommand*{\newblockpunct}{\addperiod\addnbspace\cyrdash\space\bibsentence}
%%% Убрать тире из разделителей элементов в библиографии:
%\renewcommand*{\newblockpunct}{%
%    \addperiod\space\bibsentence}%block punct.,\bibsentence is for vol,etc.
%%% Изменение точки с запятой на запятую в перечислении библиографических
%%% ссылок:
%\renewcommand*{\multicitedelim}{\addcomma\space}

%%% Возвращаем запись «Режим доступа» %%%
%\DefineBibliographyStrings{english}{%
%    urlfrom = {Mode of access}
%}
%\DeclareFieldFormat{url}{\bibstring{urlfrom}\addcolon\space\url{#1}}

%%% В списке литературы обозначение одной буквой диапазона страниц англоязычного источника %%%
\DefineBibliographyStrings{english}{%
    pages = {p\adddot} %заглавность буквы затем по месту определяется работой самого biblatex
}

%%% В ссылке на источник в основном тексте с указанием конкретной страницы обозначение одной большой буквой %%%
%\DefineBibliographyStrings{russian}{%
%    page = {C\adddot}
%}

%%% Исправление длины тире в диапазонах %%%
% \cyrdash --- тире «русской» длины, \textendash --- en-dash
\DefineBibliographyExtras{russian}{%
  \protected\def\bibrangedash{%
    \cyrdash\penalty\value{abbrvpenalty}}% almost unbreakable dash
  \protected\def\bibdaterangesep{\bibrangedash}%тире для дат
}
\DefineBibliographyExtras{english}{%
  \protected\def\bibrangedash{%
    \cyrdash\penalty\value{abbrvpenalty}}% almost unbreakable dash
  \protected\def\bibdaterangesep{\bibrangedash}%тире для дат
}

%Set higher penalty for breaking in number, dates and pages ranges
\setcounter{abbrvpenalty}{10000} % default is \hyphenpenalty which is 12

%Set higher penalty for breaking in names
\setcounter{highnamepenalty}{10000} % If you prefer the traditional BibTeX behavior (no linebreaks at highnamepenalty breakpoints), set it to ‘infinite’ (10 000 or higher).
\setcounter{lownamepenalty}{10000}

%%% Set low penalties for breaks at uppercase letters and lowercase letters
%\setcounter{biburllcpenalty}{500} %управляет разрывами ссылок после маленьких букв RTFM biburllcpenalty
%\setcounter{biburlucpenalty}{3000} %управляет разрывами ссылок после больших букв, RTFM biburlucpenalty

%%% Список литературы с красной строки (без висячего отступа) %%%
%\defbibenvironment{bibliography} % переопределяем окружение библиографии из gost-numeric.bbx пакета biblatex-gost
%  {\list
%     {\printtext[labelnumberwidth]{%
%       \printfield{prefixnumber}%
%       \printfield{labelnumber}}}
%     {%
%      \setlength{\labelwidth}{\labelnumberwidth}%
%      \setlength{\leftmargin}{0pt}% default is \labelwidth
%      \setlength{\labelsep}{\widthof{\ }}% Управляет длиной отступа после точки % default is \biblabelsep
%      \setlength{\itemsep}{\bibitemsep}% Управление дополнительным вертикальным разрывом между записями. \bibitemsep по умолчанию соответствует \itemsep списков в документе.
%      \setlength{\itemindent}{\bibhang}% Пользуемся тем, что \bibhang по умолчанию принимает значение \parindent (абзацного отступа), который переназначен в styles.tex
%      \addtolength{\itemindent}{\labelwidth}% Сдвигаем правее на величину номера с точкой
%      \addtolength{\itemindent}{\labelsep}% Сдвигаем ещё правее на отступ после точки
%      \setlength{\parsep}{\bibparsep}%
%     }%
%      \renewcommand*{\makelabel}[1]{\hss##1}%
%  }
%  {\endlist}
%  {\item}

%%% Макросы автоматического подсчёта количества авторских публикаций.
% Печатают невидимую (пустую) библиографию, считая количество источников.
% http://tex.stackexchange.com/a/66851/79756
%
\makeatletter
    \newtotcounter{citenum}
    \defbibenvironment{counter}
        {\setcounter{citenum}{0}\renewcommand{\blx@driver}[1]{}} % begin code: убирает весь выводимый текст
        {} % end code
        {\stepcounter{citenum}} % item code: cчитает "печатаемые в библиографию" источники

    \newtotcounter{citeauthorvak}
    \defbibenvironment{countauthorvak}
        {\setcounter{citeauthorvak}{0}\renewcommand{\blx@driver}[1]{}}
        {}
        {\stepcounter{citeauthorvak}}

    \newtotcounter{citeauthorscopus}
    \defbibenvironment{countauthorscopus}
        {\setcounter{citeauthorscopus}{0}\renewcommand{\blx@driver}[1]{}}
        {}
        {\stepcounter{citeauthorscopus}}

    \newtotcounter{citeauthorwos}
    \defbibenvironment{countauthorwos}
        {\setcounter{citeauthorwos}{0}\renewcommand{\blx@driver}[1]{}}
        {}
        {\stepcounter{citeauthorwos}}

    \newtotcounter{citeauthorother}
    \defbibenvironment{countauthorother}
        {\setcounter{citeauthorother}{0}\renewcommand{\blx@driver}[1]{}}
        {}
        {\stepcounter{citeauthorother}}

    \newtotcounter{citeauthorconf}
    \defbibenvironment{countauthorconf}
        {\setcounter{citeauthorconf}{0}\renewcommand{\blx@driver}[1]{}}
        {}
        {\stepcounter{citeauthorconf}}

    \newtotcounter{citeauthor}
    \defbibenvironment{countauthor}
        {\setcounter{citeauthor}{0}\renewcommand{\blx@driver}[1]{}}
        {}
        {\stepcounter{citeauthor}}

    \newtotcounter{citeauthorvakscopuswos}
    \defbibenvironment{countauthorvakscopuswos}
        {\setcounter{citeauthorvakscopuswos}{0}\renewcommand{\blx@driver}[1]{}}
        {}
        {\stepcounter{citeauthorvakscopuswos}}

    \newtotcounter{citeauthorscopuswos}
    \defbibenvironment{countauthorscopuswos}
        {\setcounter{citeauthorscopuswos}{0}\renewcommand{\blx@driver}[1]{}}
        {}
        {\stepcounter{citeauthorscopuswos}}

    \newtotcounter{citeregistered}
    \defbibenvironment{countregistered}
        {\setcounter{citeregistered}{0}\renewcommand{\blx@driver}[1]{}}
        {}
        {\stepcounter{citeregistered}}

    \newtotcounter{citeauthorpatent}
    \defbibenvironment{countauthorpatent}
        {\setcounter{citeauthorpatent}{0}\renewcommand{\blx@driver}[1]{}}
        {}
        {\stepcounter{citeauthorpatent}}

    \newtotcounter{citeauthorprogram}
    \defbibenvironment{countauthorprogram}
        {\setcounter{citeauthorprogram}{0}\renewcommand{\blx@driver}[1]{}}
        {}
        {\stepcounter{citeauthorprogram}}

    \newtotcounter{citeexternal}
    \defbibenvironment{countexternal}
        {\setcounter{citeexternal}{0}\renewcommand{\blx@driver}[1]{}}
        {}
        {\stepcounter{citeexternal}}
\makeatother

\defbibheading{nobibheading}{} % пустой заголовок, для подсчёта публикаций с помощью невидимой библиографии
\defbibheading{pubgroup}{\section*{#1}} % обычный стиль, заголовок-секция
\defbibheading{pubsubgroup}{\noindent\textbf{#1}} % для подразделов "по типу источника"

%%%Сортировка списка литературы Русский-Английский (предварительно удалить dissertation.bbl) (начало)
%%%Источник: https://github.com/odomanov/biblatex-gost/wiki/%D0%9A%D0%B0%D0%BA-%D1%81%D0%B4%D0%B5%D0%BB%D0%B0%D1%82%D1%8C,-%D1%87%D1%82%D0%BE%D0%B1%D1%8B-%D1%80%D1%83%D1%81%D1%81%D0%BA%D0%BE%D1%8F%D0%B7%D1%8B%D1%87%D0%BD%D1%8B%D0%B5-%D0%B8%D1%81%D1%82%D0%BE%D1%87%D0%BD%D0%B8%D0%BA%D0%B8-%D0%BF%D1%80%D0%B5%D0%B4%D1%88%D0%B5%D1%81%D1%82%D0%B2%D0%BE%D0%B2%D0%B0%D0%BB%D0%B8-%D0%BE%D1%81%D1%82%D0%B0%D0%BB%D1%8C%D0%BD%D1%8B%D0%BC
%\DeclareSourcemap{
%    \maps[datatype=bibtex]{
%        \map{
%            \step[fieldset=langid, fieldvalue={tempruorder}]
%        }
%        \map[overwrite]{
%            \step[fieldsource=langid, match=russian, final]
%            \step[fieldsource=presort,
%            match=\regexp{(.+)},
%            replace=\regexp{aa$1}]
%        }
%        \map{
%            \step[fieldsource=langid, match=russian, final]
%            \step[fieldset=presort, fieldvalue={az}]
%        }
%        \map[overwrite]{
%            \step[fieldsource=langid, notmatch=russian, final]
%            \step[fieldsource=presort,
%            match=\regexp{(.+)},
%            replace=\regexp{za$1}]
%        }
%        \map{
%            \step[fieldsource=langid, notmatch=russian, final]
%            \step[fieldset=presort, fieldvalue={zz}]
%        }
%        \map{
%            \step[fieldsource=langid, match={tempruorder}, final]
%            \step[fieldset=langid, null]
%        }
%    }
%}
%Сортировка списка литературы (конец)

%%% Создание команд для вывода списка литературы %%%
\newcommand*{\insertbibliofull}{
    \printbibliography[keyword=bibliofull,section=0,title=\bibtitlefull]
    \ifnumequal{\value{draft}}{0}{
      \printbibliography[heading=nobibheading,env=counter,keyword=bibliofull,section=0]
    }{}
}
\newcommand*{\insertbiblioauthor}{
    \printbibliography[heading=pubgroup, section=0, filter=papersregistered, title=\bibtitleauthor]
}
\newcommand*{\insertbiblioauthorimportant}{
    \printbibliography[heading=pubgroup, section=2, filter=papersregistered, title=\bibtitleauthorimportant]
}

% Вариант вывода печатных работ автора, с группировкой по типу источника.
% Порядок команд `\printbibliography` должен соответствовать порядку в файле common/characteristic.tex
\newcommand*{\insertbiblioauthorgrouped}{
    \section*{\bibtitleauthor}
    \ifsynopsis
    \printbibliography[heading=pubsubgroup, section=0, keyword=biblioauthorvak,    title=\bibtitleauthorvak,resetnumbers=true] % Работы автора из списка ВАК (сброс нумерации)
    \else
    \printbibliography[heading=pubsubgroup, section=0, keyword=biblioauthorvak,    title=\bibtitleauthorvak,resetnumbers=false] % Работы автора из списка ВАК (сквозная нумерация)
    \fi
    \printbibliography[heading=pubsubgroup, section=0, keyword=biblioauthorwos,    title=\bibtitleauthorwos,resetnumbers=false]% Работы автора, индексируемые Web of Science
    \printbibliography[heading=pubsubgroup, section=0, keyword=biblioauthorscopus, title=\bibtitleauthorscopus,resetnumbers=false]% Работы автора, индексируемые Scopus
    \printbibliography[heading=pubsubgroup, section=0, keyword=biblioauthorpatent, title=\bibtitleauthorpatent,resetnumbers=false]% Патенты
    \printbibliography[heading=pubsubgroup, section=0, keyword=biblioauthorprogram,title=\bibtitleauthorprogram,resetnumbers=false]% Программы для ЭВМ
    \printbibliography[heading=pubsubgroup, section=0, keyword=biblioauthorconf,   title=\bibtitleauthorconf,resetnumbers=false]% Тезисы конференций
    \printbibliography[heading=pubsubgroup, section=0, keyword=biblioauthorother,  title=\bibtitleauthorother,resetnumbers=false]% Прочие работы автора
}

\newcommand*{\insertbiblioexternal}{
    \printbibliography[heading=pubgroup,    section=0, keyword=biblioexternal,     title=\bibtitlefull]
}
   % Реализация пакетом biblatex через движок biber
}

% Вывести информацию о выбранных опциях в лог сборки
\typeout{Selected options:}
\typeout{Draft mode: \arabic{draft}}
\typeout{Font: \arabic{fontfamily}}
\typeout{AltFont: \arabic{usealtfont}}
\typeout{Bibliography backend: \arabic{bibliosel}}
\typeout{Precompile images: \arabic{imgprecompile}}
% Вывести информацию о версиях используемых библиотек в лог сборки
\listfiles

\begin{document}

\thispagestyle{empty}

\noindent%
\begin{tabularx}{\textwidth}{@{}lXr@{}}%
    & & \large{На правах рукописи}\\
    % \IfFileExists{images/logo.pdf}{\includegraphics[height=2.5cm]{logo}}{\rule[0pt]{0pt}{2.5cm}}  
    & &
    \ifnumequal{\value{showperssign}}{0}{%
        \rule[0pt]{0pt}{1.5cm}
    }{
        % \includegraphics[height=1.5cm]{personal-signature.png}
    }\\
\end{tabularx}

\vspace{0pt plus1fill} %число перед fill = кратность относительно некоторого расстояния fill, кусками которого заполнены пустые места
\begin{center}
\textbf {\large \thesisAuthor}
\end{center}

\vspace{0pt plus3fill} %число перед fill = кратность относительно некоторого расстояния fill, кусками которого заполнены пустые места
\begin{center}
\textbf {\Large %\MakeUppercase
\thesisTitle}

\vspace{0pt plus3fill} %число перед fill = кратность относительно некоторого расстояния fill, кусками которого заполнены пустые места
{\large Специальность \thesisSpecialtyNumber\ "---\par <<\thesisSpecialtyTitle>>}

\ifdefined\thesisSpecialtyTwoNumber
{\large Специальность \thesisSpecialtyTwoNumber\ "---\par <<\thesisSpecialtyTwoTitle>>}
\fi

\vspace{0pt plus1.5fill} %число перед fill = кратность относительно некоторого расстояния fill, кусками которого заполнены пустые места
\Large{Автореферат}\par
\large{диссертации на соискание учёной степени\par \thesisDegree}
\end{center}

\vspace{0pt plus4fill} %число перед fill = кратность относительно некоторого расстояния fill, кусками которого заполнены пустые места
{\centering\thesisCity~--- \thesisYear\par}

\newpage
% оборотная сторона обложки
\thispagestyle{empty}
\noindent Работа выполнена на {\thesisInOrganization}.

\vspace{0.008\paperheight plus1fill}
\noindent%
\begin{tabularx}{\textwidth}{@{}lX@{}}
    \ifdefined\supervisorTwoFio
    Научные руководители:   & \supervisorRegalia\par
                              \ifdefined\supervisorDead
                              \framebox{\textbf{\supervisorFio}}
                              \else
                              \textbf{\supervisorFio}
                              \fi
                              \par
                              \vspace{0.013\paperheight}
                              \supervisorRegalia\par
                              \ifdefined\supervisorTwoDead
                              \framebox{\textbf{\supervisorTwoFio}}
                              \else
                              \textbf{\supervisorTwoFio}
                              \fi
                              \vspace{0.013\paperheight}\\
    \else
    Научный руководитель:   & \supervisorRegalia\par
                              \ifdefined\supervisorDead
                              \framebox{\textbf{\supervisorFio}}
                              \else
                              \textbf{\supervisorFio}
                              \fi
                              \vspace{0.013\paperheight}\\
    \fi
    Официальные оппоненты:  &
    \ifnumequal{\value{showopplead}}{0}{\vspace{13\onelineskip plus1fill}}{%
        \textbf{\opponentOneFio,}\par
        \opponentOneRegalia,\par
        \opponentOneJobPlace,\par
        \opponentOneJobPost\par
        \vspace{0.01\paperheight}
        \textbf{\opponentTwoFio,}\par
        \opponentTwoRegalia,\par
        \opponentTwoJobPlace,\par
        \opponentTwoJobPost
    \ifdefined\opponentThreeFio
        \par
        \vspace{0.01\paperheight}
        \textbf{\opponentThreeFio,}\par
        \opponentThreeRegalia,\par
        \opponentThreeJobPlace,\par
        \opponentThreeJobPost
    \fi
    }%
    \vspace{0.013\paperheight} \\
    \ifdefined\leadingOrganizationTitle
    Ведущая организация:    &
    \ifnumequal{\value{showopplead}}{0}{\vspace{6\onelineskip plus1fill}}{%
        \leadingOrganizationTitle
    }%
    \fi
\end{tabularx}
\vspace{0.008\paperheight plus1fill}

\noindent Защита состоится \defenseDate~на~заседании диссертационного совета \defenseCouncilNumber~при \defenseCouncilTitle~по адресу: \defenseCouncilAddress.

\vspace{0.008\paperheight plus1fill}
\noindent С диссертацией можно ознакомиться в библиотеке \synopsisLibrary.

\vspace{0.008\paperheight plus1fill}
\noindent Отзывы на автореферат в двух экземплярах, заверенные печатью учреждения, просьба направлять по адресу: \defenseCouncilAddress, ученому секретарю диссертационного совета~\defenseCouncilNumber.

\vspace{0.008\paperheight plus1fill}
\noindent{Автореферат разослан \synopsisDate.}

\noindent Телефон для справок: \defenseCouncilPhone.

\vspace{0.008\paperheight plus1fill}
\noindent%
\begin{tabularx}{\textwidth}{@{}%
>{\raggedright\arraybackslash}b{18em}@{}
>{\centering\arraybackslash}X
r
@{}}
    Ученый секретарь\par
    диссертационного совета\par
    \defenseCouncilNumber,\par
    \defenseSecretaryRegalia
    &
    \ifnumequal{\value{showsecrsign}}{0}{}{%
        % \includegraphics[width=2cm]{secretary-signature.png}%
    }%
    &
    \defenseSecretaryFio
\end{tabularx}
        % Титульный лист
%\mainmatter                   % В том числе начинает нумерацию страниц арабскими цифрами с единицы
\mainmatter*                  % Нумерация страниц не изменится, но начнётся с новой страницы
\pdfbookmark{Общая характеристика работы}{characteristic}             % Закладка pdf
\section*{Общая характеристика работы}

\newcommand{\actuality}{\pdfbookmark[1]{Актуальность}{actuality}\underline{\textbf{\actualityTXT}}}
\newcommand{\progress}{\pdfbookmark[1]{Разработанность темы}{progress}\underline{\textbf{\progressTXT}}}
\newcommand{\aim}{\pdfbookmark[1]{Цели}{aim}\underline{{\textbf\aimTXT}}}
\newcommand{\tasks}{\pdfbookmark[1]{Задачи}{tasks}\underline{\textbf{\tasksTXT}}}
\newcommand{\aimtasks}{\pdfbookmark[1]{Цели и задачи}{aimtasks}\aimtasksTXT}
\newcommand{\novelty}{\pdfbookmark[1]{Научная новизна}{novelty}\underline{\textbf{\noveltyTXT}}}
\newcommand{\influence}{\pdfbookmark[1]{Практическая значимость}{influence}\underline{\textbf{\influenceTXT}}}
\newcommand{\methods}{\pdfbookmark[1]{Методология и методы исследования}{methods}\underline{\textbf{\methodsTXT}}}
\newcommand{\defpositions}{\pdfbookmark[1]{Положения, выносимые на защиту}{defpositions}\underline{\textbf{\defpositionsTXT}}}
\newcommand{\reliability}{\pdfbookmark[1]{Достоверность}{reliability}\underline{\textbf{\reliabilityTXT}}}
\newcommand{\probation}{\pdfbookmark[1]{Апробация}{probation}\underline{\textbf{\probationTXT}}}
\newcommand{\contribution}{\pdfbookmark[1]{Личный вклад}{contribution}\underline{\textbf{\contributionTXT}}}
\newcommand{\publications}{\pdfbookmark[1]{Публикации}{publications}\underline{\textbf{\publicationsTXT}}}

% %% programming languages abbreviations

\newcommand{\Java}{Java\xspace}
\newcommand{\JVM}{JVM\xspace}
\newcommand{\CLANG}{C\xspace}
\newcommand{\CPP}{C/C++\xspace}
\newcommand{\JS}{JavaScript\xspace}
\newcommand{\LLVM}{LLVM\xspace}

%% memory models abbreviations

\newcommand{\MM}[1]{\ensuremath{\mathsf{#1}}\xspace}

\newcommand{\SC}{\MM{SC}}
\newcommand{\DRFx}{\MM{DRFx}}

\newcommand{\Intel}{\MM{x86}}
\newcommand{\TSO}{\MM{TSO}}
\newcommand{\SPARC}{\MM{SPARC}}
\newcommand{\ARM}{\MM{ARM}}
\newcommand{\ARMv}[1]{\MM{ARMv{#1}}}
\newcommand{\POWER}{\MM{POWER}}
\newcommand{\RISC}{\MM{RISC\text{-}V}}

\newcommand{\CMM}{\MM{C11}}
\newcommand{\RCMM}{\MM{RC11}}

\newcommand{\Prm}{\MM{Promising}}
\newcommand{\Wkm}{\MM{Weakestmo}}
\newcommand{\WkmS}{\MM{Weakestmo2}}
\newcommand{\MRD}{\MM{MRD}}
\newcommand{\PwP}{\MM{PwP}}

%% proof assistants 

\newcommand{\coq}{\textsc{Coq}\xspace}
\newcommand{\gallina}{\textsc{Gallina}\xspace}
\newcommand{\mathcomp}{\textsc{MathComp}\xspace}
\newcommand{\analysis}{\textsc{MathComp-Analysis}\xspace}
\newcommand{\finmap}{\textsc{finmap}\xspace}
\newcommand{\relationalgebra}{\textsc{relation-algebra}\xspace}
\newcommand{\ssreflect}{\textsc{SSReflect}\xspace}
\newcommand{\equations}{\textsc{Equations}\xspace}

\newcommand{\agda}{\textsc{Agda}\xspace}
\newcommand{\arend}{\textsc{Arend}\xspace}
\newcommand{\idris}{\textsc{Idris}\xspace}
\newcommand{\isabelle}{\textsc{Isabelle/HOL}\xspace}

%% tools 

\newcommand{\hmc}{\textsc{HMC}\xspace}
\newcommand{\hmclbf}{$\hmc_{\lbf}$\xspace}
\newcommand{\RCMC}{\textsc{RCMC}\xspace}
\newcommand{\rcmc}{\textsc{rcmc}\xspace}
\newcommand{\genmc}{\textsc{GenMC}\xspace}
\newcommand{\lockmc}{\textsc{LAPOR}\xspace}
\newcommand{\genmcmath}{\textnormal{\genmc}\xspace}
\newcommand{\Tracer}{\textsc{Tracer}\xspace}
\newcommand{\Herd}{\textsc{Herd}\xspace}
\newcommand{\PPCMEM}{\textsc{PPCMEM}\xspace}
\newcommand{\ARMMEM}{\textsc{ARMMEM}\xspace}
\newcommand{\CPPMEM}{\textsc{CPPMEM}\xspace}
\newcommand{\TriCheck}{\textsc{TriCheck}\xspace}
\newcommand{\rmem}{\textsc{rmem}\xspace}
\newcommand{\Nidhugg}{\textsc{Nidhugg}\xspace}
\newcommand{\CDSChecker}{\textsc{CDS\-Checker}\xspace}
\newcommand{\CBMC}{\textsc{CBMC}\xspace}
\newcommand{\Dartagnan}{\textsc{Dartagnan}\xspace}
\newcommand{\Verisoft}{\textsc{Verisoft}\xspace}
\newcommand{\CHESS}{\textsc{CHESS}\xspace}
\newcommand{\wmc}{\textsc{WMC}\xspace}


% Характеристика работы по структуре во введении и в автореферате не отличается (ГОСТ Р 7.0.11, пункты 5.3.1 и 9.2.1), 
% потому её загружаем из одного и того же внешнего файла, предварительно задав форму выделения некоторым параметрам

% {\actuality} Обзор, введение в тему, обозначение места данной работы в
% мировых исследованиях и~т.\:п., можно использовать ссылки на~другие
% работы~\autocite{Gosele1999161,Lermontov}
% (если их~нет, то~в~автореферате
% автоматически пропадёт раздел <<Список литературы>>). Внимание! Ссылки
% на~другие работы в~разделе общей характеристики работы можно
% использовать только при использовании \verb!biblatex! (из-за технических
% ограничений \verb!bibtex8!. Это связано с тем, что одна
% и~та~же~характеристика используются и~в~тексте диссертации, и в
% автореферате. В~последнем, согласно ГОСТ, должен присутствовать список
% работ автора по~теме диссертации, а~\verb!bibtex8! не~умеет выводить в~одном
% файле два списка литературы).
% При использовании \verb!biblatex! возможно использование исключительно
% в~автореферате подстрочных ссылок
% для других работ командой \verb!\autocite!, а~также цитирование
% собственных работ командой \verb!\cite!. Для этого в~файле
% \verb!common/setup.tex! необходимо присвоить положительное значение
% счётчику \verb!\setcounter{usefootcite}{1}!.

% Для генерации содержимого титульного листа автореферата, диссертации
% и~презентации используются данные из файла \verb!common/data.tex!. Если,
% например, вы меняете название диссертации, то оно автоматически
% появится в~итоговых файлах после очередного запуска \LaTeX. Согласно
% ГОСТ 7.0.11-2011 <<5.1.1 Титульный лист является первой страницей
% диссертации, служит источником информации, необходимой для обработки и
% поиска документа>>. Наличие логотипа организации на~титульном листе
% упрощает обработку и~поиск, для этого разметите логотип вашей
% организации в папке images в~формате PDF (лучше найти его в векторном
% варианте, чтобы он хорошо смотрелся при печати) под именем
% \verb!logo.pdf!. Настроить размер изображения с логотипом можно
% в~соответствующих местах файлов \verb!title.tex!  отдельно для
% диссертации и автореферата. Если вам логотип не~нужен, то просто
% удалите файл с~логотипом.

% \ifsynopsis
% Этот абзац появляется только в~автореферате.
% Для формирования блоков, которые будут обрабатываться только в~автореферате,
% заведена проверка условия \verb!\!\verb!ifsynopsis!.
% Значение условия задаётся в~основном файле документа (\verb!synopsis.tex! для
% автореферата).
% \else
% Этот абзац появляется только в~диссертации.
% Через проверку условия \verb!\!\verb!ifsynopsis!, задаваемого в~основном файле
% документа (\verb!dissertation.tex! для диссертации), можно сделать новую
% команду, обеспечивающую появление цитаты в~диссертации, но~не~в~автореферате.
% \fi

% {\progress}
% Этот раздел должен быть отдельным структурным элементом по
% ГОСТ, но он, как правило, включается в описание актуальности
% темы. Нужен он отдельным структурынм элемементом или нет ---
% смотрите другие диссертации вашего совета, скорее всего не нужен.

{\actuality} 
В современном мире многопоточные программные системы распространены повсеместно. 
Разработка и тестирование программного обеспечения для таких систем на порядок 
сложнее и существенно более трудозатратно, чем для последовательных систем. 
По этой причине крайне актуальной является задача 
верификации многопоточных программ.
Решение этой задачи в свою очередь требует наличия 
строгой математической спецификации семантики многопоточных программ.

Формальная семантика многопоточных программ, потоки которых работают 
с разделяемой памятью, называется \emph{моделью памяти}. 
Главной целью модели памяти является задание множества 
допустимых \emph{сценариев исполнения} программы.
Современные многопоточные системы и языки программирования 
вследствие применения компиляторами и процессорами
различных оптимизаций при сборке и выполнении программ
допускают так называемые \emph{слабые сценарии исполнения},
то есть такие сценарии, которые не могут быть получены в результате 
простого поочередного исполнения инструкций различными потоками. 
\emph{Слабые модели памяти} призваны описать множество 
допустимых слабых сценариев исполнения программы. 
Оказывается, что вопрос какие именно слабые сценарии поведения 
следует допускать, а какие нет, не является однозначным 
и зависит от требований к самой многопоточной системе 
или языку программирования. 
По этой причине в последние годы появилось (и продолжает появляться) 
множество различных моделей памяти для современных мультипроцессоров, например, \Intel~\autocite{Sewell-al:CACM10}, 
\ARM~\autocite{Pulte-al:POPL18}, 
\POWER~\autocite{Sarkar-al:PLDI11}
языков программирования, например, 
\CPP~\autocite{Batty-al:POPL11},
\Java~\autocite{Manson-al:POPL05}, 
\JS~\autocite{Watt-al:PLDI2020}, 
\OCaml~\autocite{Dolan-al:PLDI18},
а также распределенных систем%
~\autocite{Jagadeesan-al:ESOP2018,Lahav-Boker:PLDI2020}.
В связи с этим встает задача формализации 
существующих моделей памяти и создание теории
для разработки будущих моделей памяти.


Одним из способов задания моделей памяти
является использование семантических доменов
\emph{истинной конкурентности} (\emph{true concurrency semantics}).
Этот класс моделей позволяет выразить независимость (параллельность) атомарных событий, а также 
причинно-следственные связи между ними,
что ведет к более компактному представлению пространства состояний программы.
Всё это упрощает рассуждения 
о поведении многопоточных программ как для человека, 
так и для программных средств при автоматической и интерактивной верификации. 

\emph{Структуры событий} (\emph{event structures}) являются одним из семантических доменов, 
относящихся к классу моделей истинной конкурентности.
В наиболее простом варианте структура событий состоит из множества атомарных событий,
функции, присваивающей каждому событию семантическую метку,
отношения причинно-следственной связи и отношения конфликта между событиями.
Классическая теория структур событий была разработана M.Nielsen, G.Plotkin и G.Winskel
для описания семантики исчисления взаимодействующих систем (Calculus of Communicating Systems, CCS).
Следует отметить, что данное исчисление является достаточно простой моделью параллельных вычислений
и не позволяет описывать слабые сценарии поведения многопоточных программ.

Для описания слабых моделей памяти исследователями было предложено несколько формализмов,
основанных на структурах событий, в частности,
модель A.Jeffrey и J.Riely~\autocite{Jeffrey-Riely:LICS16},
модель J.Pichon-Pharabod и P. Sewell~\autocite{PichonPharabod-Sewell:POPL16},
модель \Wkm~\autocite{Chakraborty-Vafeiadis:POPL19},
модель \MRD~\autocite{Paviotti-al:ESOP20}.
Отметим, что данные модели вводят новые классы
структур событий, несовместимые с классической теорией,
что не позволяет применять известные результаты
о структурах событий к данным моделям.
Кроме того, в рамках данных моделей даже для небольших программ
вычислительно затратно перечисление возможных слабых сценариев поведения,
что препятствует разработке эффективных средств верификации многопоточных программ.

Таким образом, возникает потребность в создании формальной семантики 
многопоточных программ на основе структур событий, 
которая, с одной стороны, позволяла бы описывать слабые сценарии поведения, 
с другой стороны, допускала бы разработку  
инструментов для автоматической и интерактивной верификации. 

{\progress}

Теория структур событий была разработана M.Nielsen, G.Plotkin и G.Winskel в 1980-1990 годы%
~\autocite{Nielsen:REX93,Sassone:MFCS1993,Vaandrager:TCS1991,Winskel-TCS:09} 
для задания денотационной семантики 
исчисления параллельных взаимодействующих систем (Calculus of Communicating Systems, CCS)%
~\autocite{Winskel:ICALP1982}.
Относительно недавно эта теория также была использована 
для задания семантики пи-исчисления процессов ($\pi$-calculus)%
~\autocite{Varacca-Nobuko:TCS10,Crafa-al:FSCCS12,Hildebrandt-al:LATA2017}.
Но CCS и пи-исчисление не позволяют описывать 
слабые сценарии поведения многопоточных программ.

Теория слабых моделей памяти также активно развивалась, начиная с 1990-ых годов. 
На сегодняшний день существует множество моделей памяти, 
описывающих поведение мультипроцессоров, 
многопоточных языков программирования и распределенных систем. 
Эти модели, в свою очередь, можно разделить на несколько классов.

Модели, \emph{сохраняющие программный порядок}, образуют широкий класс,
включающий, в том числе, модель \TSO процессоров семейства \Intel~\autocite{Sewell-al:CACM10},
модели последовательной согласованности (sequential consistency)~\autocite{Lamport:TC79}
причинной согласованности (causal consistency)~\autocite{Lahav-Boker:PLDI2020},
и согласованности в конечном счёте (eventual consistency)~\autocite{Jagadeesan-al:ESOP2018},
а также модели памяти некоторых языков программирования, например,
модель памяти языка \OCaml~\autocite{Dolan-al:PLDI18}.
Общим недостатком моделей данного класса является то,
что в рамках этих моделей не поддерживаются \emph{оптимальные схемы компиляции} 
в целевой код для современных мультипроцессоров \ARM и \POWER.
Это означает, что реализация данных моделей на этих мультипроцессорах
влечет дополнительные накладные расходы и может приводить
к увелечению времени исполнения программ~\autocite{Ou-Demsky:OOPSLA18}. 

Модели памяти мультипроцессоров,
например \ARM~\autocite{Pulte-al:POPL18} и \POWER~\autocite{Sarkar-al:PLDI11}, 
как правило, принадлежат к классу моделей, \emph{сохраняющих синтаксические зависимости}. 
Основное ограничение моделей, принадлежащих к данному классу, заключается в том, 
что они не поддерживают некоторые трансформации программ, 
применяемые оптимизирующими компиляторами, например, свертку констант.
По этой причине модели данного класса, как правило,
не применяются в качестве моделей памяти для языков программирования.  

Таким образом, модели памяти двух вышеупомянутых классов 
не отвечают требованиям, предъявляемым к моделям памяти для таких языков как \CPP и \Java. 
С целью преодоления этих ограничений исследователями были предложены модели  
\Prm~\autocite{Kang-al:POPL17}, \Wkm~\autocite{Chakraborty-Vafeiadis:POPL19}, 
\MRD~\autocite{Paviotti-al:ESOP20}, \PwP~\autocite{Jagadeesan-al:OOPSLA2020}
и другие~\autocite{Jeffrey-Riely:LICS16,PichonPharabod-Sewell:POPL16},
которые обычно относят к классу моделей, \emph{сохраняющих семантические зависимости}.
Данные модели, как правило, поддерживают оптимальные схемы компиляции
для современных мультипроцессоров и поддерживают широкий спектр оптимизаций программ. 
Тем не менее, классы моделей, сохраняющих программный порядок и синтаксические зависимости, 
хорошо изучены, в то время как свойства класса моделей,
сохраняющих семантические зависимости, по-прежнему активно исследуются.
В частности, для моделей данного класса практический не исследованы
вопросы построения эффективных инструментов автоматической и интерактивной верификации. 

%% Некоторые из вышеупомянутых моделей, сохраняющих семантические зависимости,
%% основаны на теории структур событий%
%% ~\autocite{Jeffrey-Riely:LICS16,PichonPharabod-Sewell:POPL16,
%% Chakraborty-Vafeiadis:POPL19,Paviotti-al:ESOP20}.
%% Общий недостаток данных моделей заключается в том,
%% что они вводят новые классы структур событий, 
%% несовместимые с классическими определениями.
%% Это затрудняет применение уже существующей классической теории структур событий
%% для решения проблем, возникающих в теории слабых моделей памяти. 

Рассмотрим для примера язык \CPP.
Для описания модели памяти данного языка исследовательским
сообществом было выработано несколько подходов.
Модель \RCMM~\autocite{Lahav-al:PLDI17}
относится к классу моделей, сохраняющих программный порядок.
Данная модель является относительно простой и
предоставляет ряд важных и полезных свойств.
Для данной модели также были разработаны эффективные
средства верификации многопоточных программ,
например, инструмент проверки моделей \genmc~\autocite{Kokologiannakis:PLDI2019}.
Однако данная модель не поддерживает
оптимальную схему компиляции в модели мультипроцессоров \ARM и \POWER.
С другой стороны, модели \Prm и \Wkm,
относящиеся к классу моделей, сохраняющих семантические зависимости,
поддерживают оптимальную схему компиляции и широкий набор оптимизаций программ.
Но данные модели существенно более сложные, их свойства слабо изучены, и для них
не разработаны эффективные методы верификации программ. 

В контексте данной работы наибольший интерес
представляет именно модель \Wkm~\autocite{Chakraborty-Vafeiadis:POPL19},
так как она основана на теории структур событий
и для нее было формально доказано наличие ряда важных для практики свойств.
Отметим, что у данной модели тем не менее есть ряд недостаков:
данная модель не укладывается в классическую теорию структур событий,
для нее не была доказана корректность оптимальной схемы
компиляции в модели современных мультипроцессоров,
а также для данной модели не были ранее разработаны
средства верификации программ.

%% Для большей гибкости, данный язык предоставляет несколько
%% режимов доступа к разделяемым переменным (\emph{access modes}):
%% \emph{последовательно согласованный режим} (\emph{sequentiall consistent}),
%% режимы \emph{захвата и освобождения} (\emph{acquire/release}),
%% гарантирующие причинную согласованность~\autocite{Lahav-al:POPL16},
%% \emph{ослабленный режим} (\emph{relaxed}),
%% гарантирующий когерентность~\autocite{Alglave-al:TOPLAS14}
%% и \emph{неатомарный режим} для неконкурентных обращений к памяти. 

%% Для описания подмножества модели памяти \CPP
%% исследователями была разработана модель \RCMM~\autocite{Lahav-al:PLDI17}.
%% Данная модель полностью описывает все возможные сценарии
%% поведения многопоточных программ, которые
%% не используют режим ослабленных обращений.  

%% Среди слабых моделей памяти, сохраняющих семантические зависимости
%% и основанных на структурах событий, в контексте данной работы наибольший интерес
%% представляет модель \Wkm~\autocite{Chakraborty-Vafeiadis:POPL19},
%% поскольку для данной модели было формально доказано наличие ряда важных для практики свойств.
%% В частности, для этой модели была доказана корректность
%% локальных трансформаций программ и теорема о свободе от гонок.
%% Тем не менее отметим, что корректность оптимальной схемы
%% компиляции из модели \Wkm в модели современных мультипроцессоров
%% \emph{не была ранее доказана}, что является существенным недостатком,
%% так как наличие данного свойства является одним из базовых требований,
%% предъявляемых к классу моделей памяти, сохраняющих семантические зависимости.

{\aim} данной работы является адаптация теории структур событий
для описания слабых моделей памяти и разработка на основе этих исследований 
инструментов для автоматической и интерактивной верификации многопоточных программ. 

Для достижения данной цели были сформулированы следующие {\tasks}.
\begin{enumerate}[beginpenalty=10000] % https://tex.stackexchange.com/a/476052/104425
  \item
    Формализовать в системе для интерактивного доказательства теорем \coq
    классическую теорию структур событий. Показать, что
    в данную теорию укладывается класс моделей памяти,
    сохраняющих программный порядок и, в частности, модель \RCMM.
  \item
    Формализовать в системе для интерактивного доказательства теорем \coq
    теорию структур событий модели \Wkm.
    Доказать корректность оптимальной схемы компиляции
    из модели \Wkm в модели памяти современных мультипроцессоров.
  \item
    Разработать строгую версию модели \Wkm, 
    допускающую реализацию эффективных инструментов автоматической верификации
    и доказать, что для неё сохраняются основные свойства \Wkm  
    (в частности, корректность компиляции, корректность локальных трансформаций программ, 
     теорема о свободе от гонок).
  \item
    Разработать алгоритм проверки моделей (model~checking) для предложенной модели.
\end{enumerate}

~\newline

{\methods}

Диссертационное исследование базируется на теории формальных семантик. 
Используются классические и хорошо изученные формализмы, в частности, 
системы помеченных переходов, языки помеченных частично упорядоченных мультимножеств и структуры событий. 

Для формализации некоторых теорем и доказательств, представленных в данной работе, 
использовалась система интерактивного доказательства теорем \coq 
и библиотека формализованных математических теорий \mathcomp.

%% При разработке алгоритма проверки моделей использовались техника \emph{редукции частичных порядков}.
%% Предложенный алгоритм был внедрен в систему \genmc --- 
%% инструмент для автоматической верификации многопоточных программ написанных на языке \CLANG.

{\defpositions}
\begin{enumerate}[beginpenalty=10000] % https://tex.stackexchange.com/a/476052/104425
  \item Предложена формальная семантика на основе классической теории структур событий, 
    покрывающая класс слабых моделей памяти, сохраняющих программный порядок;
    данная семантика формализована в системе \coq.
  \item Доказана корректность оптимальной схемы компиляции из модели \Wkm
    в модели современных мультипроцессоров \TSO, \ARM и \POWER;
    модель \Wkm и доказательство теоремы о корректности компиляции
    формализованы в системе \coq.
  \item Предложена модель \WkmS, расширяющая модель \Wkm 
  новыми свойствами \emph{свободы от буферизации
  операций чтения} и \emph{локальности сертификациии}, 
  которые позволяют проводить эффективную верификацию программ в данной модели;
  также доказано сохранение основных свойств модели \Wkm: корректности компиляции,
  свойства корректности локальных трансформаций программ,
  теоремы о свободе от гонок.
  \item Для модели \WkmS разработан алгоритм автоматической 
    верификации программ методом проверки моделей; 
    в ряде экспериментов показана лучшая эффективность 
    данного алгоритма по сравнению с аналогами.
\end{enumerate}
% В папке Documents можно ознакомиться с решением совета из Томского~ГУ
% (в~файле \verb+Def_positions.pdf+), где обоснованно даются рекомендации
% по~формулировкам защищаемых положений.

{\novelty}
\begin{enumerate}[beginpenalty=10000] % https://tex.stackexchange.com/a/476052/104425

  \item Впервые предложена семантика, основанная на классической теории структур событий,
    которая покрывает класс слабых моделей памяти, сохраняющих программный порядок.
    %% что позволяет применить известные теоретические результаты 
    %% о структурах событий к данному классу моделей.

  \item Впервые доказана корректность оптимальной схемы компиляции
    из модели памяти, основанной на структурах событий (\Wkm), 
    в модели памяти современных мультипроцессоров.

  \item Впервые предложена модель памяти (\WkmS),
    принадлежащая к классу моделей, сохраняющих семантические зависимости, 
    и при этом допускающая реализацию эффективных методов автоматической верификации программ. 

  \item Разработан новый алгоритм проверки моделей для \WkmS,
    который является существенно более эффективным по сравнению с другими алгоритмами
    (\CDSChecker~\autocite{Norris-Demsky:OOPSLA2013}, \rmem~\autocite{Pulte-al:PLDI2019}),
    которые поддерживают класс моделей памяти, сохраняющих семантические зависимости.

\end{enumerate}

{\influence} 

Новая семантика на основе теории структур событий 
для класса слабых моделей памяти, сохраняющих программный порядок,
соединяет классическую теорию структур событий.
%% с теорией слабых моделей памяти и позволяет применить известные результаты 
%% о структурах событий в новой предметной области.  
Формализация этой семантики в системе \coq открывает 
путь к дальнейшей разработке инструментов для  
интерактивной верификации многоточных программ  
с учетом слабых сценариев исполнения. 
 
Новые свойства предложенной модели \WkmS ---
свобода от буферизации операций чтения (load buffering race freedom)
и локальности сертификации (certification locality), --- 
также могут быть добавлены в другие модели памяти 
с целью разработки методов автоматической верификации программ в этих моделях. 
Наличие данных свойств позволяет оптимизировать алгоритм 
проверки моделей и таким образом существенно увеличить его эффективность.

Предложенный  алгоритм проверки моделей может быть использован на практике
для отладки и верификации многопоточных алгоритмов и структур данных 
с учетом слабых сценариев исполнения, допустимых стандартом языка \CLANG. 

{\reliability} полученных результатов обеспечивается 
формальными доказательствами, разработанными в том числе с использованием
систем интерактивного доказательства теорем, 
а также инженерными экспериментами. 
Результаты находятся в соответствии с результатами, полученными другими авторами.

{\probation}
Основные результаты работы докладывались~на
следующих научных конференциях и семинарах:
Surrey Concurrency Workshop (23-24 июля 2019, Университет Суррея, Великобритания),
The European Conference on Object-Oriented Programming
(ECOOP, 15-17 ноября 2020, Берлин, Германия, онлайн конференция),
Spring/Summer Young Researchers' Colloquium on Software Engineering
(27-28 мая 2021, Москва, Россия),
внутренние семинары JetBrains Research
(18 ноября 2018, 13 апреля 2020, Санкт-Петербург, Россия). \\
\fixme{добавить будущие мероприятия по мере проведения}.

% {\contribution} Автор принимал активное участие \ldots

\ifnumequal{\value{bibliosel}}{0}
{%%% Встроенная реализация с загрузкой файла через движок bibtex8. (При желании, внутри можно использовать обычные ссылки, наподобие `\cite{vakbib1,vakbib2}`).
    {\publications} Основные результаты по теме диссертации изложены
    в~XX~печатных изданиях,
    X из которых изданы в журналах, рекомендованных ВАК,
    X "--- в тезисах докладов.
}%
{%%% Реализация пакетом biblatex через движок biber
    \begin{refsection}[bl-author, bl-registered]
        % Это refsection=1.
        % Процитированные здесь работы:
        %  * подсчитываются, для автоматического составления фразы "Основные результаты ..."
        %  * попадают в авторскую библиографию, при usefootcite==0 и стиле `\insertbiblioauthor` или `\insertbiblioauthorgrouped`
        %  * нумеруются там в зависимости от порядка команд `\printbibliography` в этом разделе.
        %  * при использовании `\insertbiblioauthorgrouped`, порядок команд `\printbibliography` в нём должен быть тем же (см. biblio/biblatex.tex)
        %
        % Невидимый библиографический список для подсчёта количества публикаций:
        \printbibliography[heading=nobibheading, section=1, env=countauthorvak,          keyword=biblioauthorvak]%
        \printbibliography[heading=nobibheading, section=1, env=countauthorwos,          keyword=biblioauthorwos]%
        \printbibliography[heading=nobibheading, section=1, env=countauthorscopus,       keyword=biblioauthorscopus]%
        \printbibliography[heading=nobibheading, section=1, env=countauthorconf,         keyword=biblioauthorconf]%
        \printbibliography[heading=nobibheading, section=1, env=countauthorother,        keyword=biblioauthorother]%
        \printbibliography[heading=nobibheading, section=1, env=countregistered,         keyword=biblioregistered]%
        \printbibliography[heading=nobibheading, section=1, env=countauthorpatent,       keyword=biblioauthorpatent]%
        \printbibliography[heading=nobibheading, section=1, env=countauthorprogram,      keyword=biblioauthorprogram]%
        \printbibliography[heading=nobibheading, section=1, env=countauthor,             keyword=biblioauthor]%
        \printbibliography[heading=nobibheading, section=1, env=countauthorvakscopuswos, filter=vakscopuswos]%
        \printbibliography[heading=nobibheading, section=1, env=countauthorscopuswos,    filter=scopuswos]%
        %
        \nocite{*}%
        %
        {\publications} Основные результаты по теме диссертации изложены в~\arabic{citeauthor}~печатных изданиях,
        \arabic{citeauthorvak} из которых изданы в журналах, рекомендованных ВАК\sloppy%
        \ifnum \value{citeauthorscopuswos}>0%
            , \arabic{citeauthorscopuswos} "--- в~периодических научных журналах, индексируемых Web of~Science и Scopus\sloppy%
        \fi%
        \ifnum \value{citeauthorconf}>0%
            , \arabic{citeauthorconf} "--- в~тезисах докладов.
        \else%
            .
        \fi%
        \ifnum \value{citeregistered}=1%
            \ifnum \value{citeauthorpatent}=1%
                Зарегистрирован \arabic{citeauthorpatent} патент.
            \fi%
            \ifnum \value{citeauthorprogram}=1%
                Зарегистрирована \arabic{citeauthorprogram} программа для ЭВМ.
            \fi%
        \fi%
        \ifnum \value{citeregistered}>1%
            Зарегистрированы\ %
            \ifnum \value{citeauthorpatent}>0%
            \formbytotal{citeauthorpatent}{патент}{}{а}{}\sloppy%
            \ifnum \value{citeauthorprogram}=0 . \else \ и~\fi%
            \fi%
            \ifnum \value{citeauthorprogram}>0%
            \formbytotal{citeauthorprogram}{программ}{а}{ы}{} для ЭВМ.
            \fi%
        \fi%
        % К публикациям, в которых излагаются основные научные результаты диссертации на соискание учёной
        % степени, в рецензируемых изданиях приравниваются патенты на изобретения, патенты (свидетельства) на
        % полезную модель, патенты на промышленный образец, патенты на селекционные достижения, свидетельства
        % на программу для электронных вычислительных машин, базу данных, топологию интегральных микросхем,
        % зарегистрированные в установленном порядке.(в ред. Постановления Правительства РФ от 21.04.2016 N 335)
    \end{refsection}%
    \begin{refsection}[bl-author, bl-registered]
        % Это refsection=2.
        % Процитированные здесь работы:
        %  * попадают в авторскую библиографию, при usefootcite==0 и стиле `\insertbiblioauthorimportant`.
        %  * ни на что не влияют в противном случае
        \nocite{Moiseenko-al:OOPSLA22}
        \nocite{Moiseenko-al:ECOOP20}
        \nocite{Moiseenko-al:STJITMO22}
        \nocite{Moiseenko-al:PCS21}
        \nocite{Gladstein-al:ISPRAS21}
        % \nocite{vakbib2}%vak
        % \nocite{patbib1}%patent
        % \nocite{progbib1}%program
        % \nocite{bib1}%other
        % \nocite{confbib1}%conf
    \end{refsection}%
        %
        % Всё, что вне этих двух refsection, это refsection=0,
        %  * для диссертации - это нормальные ссылки, попадающие в обычную библиографию
        %  * для автореферата:
        %     * при usefootcite==0, ссылка корректно сработает только для источника из `external.bib`. Для своих работ --- напечатает "[0]" (и даже Warning не вылезет).
        %     * при usefootcite==1, ссылка сработает нормально. В авторской библиографии будут только процитированные в refsection=0 работы.
}

% При использовании пакета \verb!biblatex! будут подсчитаны все работы, добавленные
% в файл \verb!biblio/author.bib!. Для правильного подсчёта работ в~различных
% системах цитирования требуется использовать поля:
% \begin{itemize}
%         \item \texttt{authorvak} если публикация индексирована ВАК,
%         \item \texttt{authorscopus} если публикация индексирована Scopus,
%         \item \texttt{authorwos} если публикация индексирована Web of Science,
%         \item \texttt{authorconf} для докладов конференций,
%         \item \texttt{authorpatent} для патентов,
%         \item \texttt{authorprogram} для зарегистрированных программ для ЭВМ,
%         \item \texttt{authorother} для других публикаций.
% \end{itemize}
% Для подсчёта используются счётчики:
% \begin{itemize}
%         \item \texttt{citeauthorvak} для работ, индексируемых ВАК,
%         \item \texttt{citeauthorscopus} для работ, индексируемых Scopus,
%         \item \texttt{citeauthorwos} для работ, индексируемых Web of Science,
%         \item \texttt{citeauthorvakscopuswos} для работ, индексируемых одной из трёх баз,
%         \item \texttt{citeauthorscopuswos} для работ, индексируемых Scopus или Web of~Science,
%         \item \texttt{citeauthorconf} для докладов на конференциях,
%         \item \texttt{citeauthorother} для остальных работ,
%         \item \texttt{citeauthorpatent} для патентов,
%         \item \texttt{citeauthorprogram} для зарегистрированных программ для ЭВМ,
%         \item \texttt{citeauthor} для суммарного количества работ.
% \end{itemize}

% Счётчик \texttt{citeexternal} используется для подсчёта процитированных публикаций;
% \texttt{citeregistered} "--- для подсчёта суммарного количества патентов и программ для ЭВМ.

% Для добавления в список публикаций автора работ, которые не были процитированы в
% автореферате, требуется их~перечислить с использованием команды \verb!\nocite! в
% \verb!Synopsis/content.tex!.

Личный вклад автора в публикациях, выполненных с соавторами, распределён следующим образом.
В работе \cite{Gladstein-al:ISPRAS21} автор предложил
метод кодирования семантики параллельной регистровой машины с
моделью памяти, сохраняющей программный порядок, в терминах простых структур событий;
соавторы участвовали в формализации данного метода в системе \coq.
В работе \cite{Moiseenko-al:STJITMO22} автор предложил
метод кодирования семантического домена языков частично упорядоченных мультимножеств
с использованием фактор-типов, 
соавторы участвовали в обсуждении данного метода и его формализации в системе \coq.
В работе \cite{Moiseenko-al:PCS21} автор выполнил сбор и анализ данных
о существующих моделях памяти языков программирования;
соавторы участвовали в формулировке выводов данного анализа.
В работе \cite{Moiseenko-al:ECOOP20} автор выполнил
формализацию доказательства корректности компиляции из
модели \Wkm в модели современных мультипроцессоров;
соавторы участвовали в обсуждении данного доказательства
и его формализации в системе \coq.
В работе \cite{Moiseenko-al:OOPSLA22} автор
формализовал новые свойства модели \WkmS,
а именно, свойства свободы от буферизации операций чтения и локальности сертификации,
доказал сохранение полезных свойств модели \Wkm в \WkmS,
а также разработал прототип алгоритма проверки моделей для \WkmS;
соавторы участвовали в обсуждении формализации модели \WkmS
и доказательстве ее свойств,
оказывали помощь при реализации нового алгоритма,
а также провели эксперименты по сравнению нового алгоритма с аналогами.
 

%Диссертационная работа была выполнена при поддержке грантов \dots

%\underline{\textbf{Объем и структура работы.}} Диссертация состоит из~введения,
%четырех глав, заключения и~приложения. Полный объем диссертации
%\textbf{ХХХ}~страниц текста с~\textbf{ХХ}~рисунками и~5~таблицами. Список
%литературы содержит \textbf{ХХX}~наименование.

\pdfbookmark{Содержание работы}{description}                          % Закладка pdf

\section*{Содержание работы}

Во \underline{\textbf{введении}} обосновывается актуальность
исследований, проводимых в~рамках данной диссертационной работы,
приводится обзор научной литературы по~изучаемой проблеме,
формулируется цель, ставятся задачи работы, излагается научная новизна
и практическая значимость представляемой работы. В~последующих главах
сначала описывается общий принцип, позволяющий \dots, а~потом идёт
апробация на частных примерах: \dots  и~\dots.


\underline{\textbf{Первая глава}} посвящена \dots

% картинку можно добавить так:
% \begin{figure}[ht]
%     \centerfloat{
%         \hfill
%         \subcaptionbox{\LaTeX}{%
%             \includegraphics[scale=0.27]{latex}}
%         \hfill
%         \subcaptionbox{Knuth}{%
%             \includegraphics[width=0.25\linewidth]{knuth1}}
%         \hfill
%     }
%     \caption{Подпись к картинке.}\label{fig:latex}
% \end{figure}

% Формулы в строку без номера добавляются так:
% \[
%     \lambda_{T_s} = K_x\frac{d{x}}{d{T_s}}, \qquad
%     \lambda_{q_s} = K_x\frac{d{x}}{d{q_s}},
% \]

\underline{\textbf{Вторая глава}} посвящена исследованию

\underline{\textbf{Третья глава}} посвящена исследованию

% Можно сослаться на свои работы в автореферате. Для этого в файле
% \verb!Synopsis/setup.tex! необходимо присвоить положительное значение
% счётчику \verb!\setcounter{usefootcite}{1}!. В таком случае ссылки на
% работы других авторов будут подстрочными.
% Изложенные в третьей главе результаты опубликованы в~\cite{vakbib1, vakbib2}.
% Использование подстрочных ссылок внутри таблиц может вызывать проблемы.

В \underline{\textbf{четвертой главе}} приведено описание

\FloatBarrier
\pdfbookmark{Заключение}{conclusion}                                  % Закладка pdf
В \underline{\textbf{заключении}} приведены основные результаты работы, которые заключаются в следующем:
\input{common/concl}

\pdfbookmark{Литература}{bibliography}                                % Закладка pdf
% При использовании пакета \verb!biblatex! список публикаций автора по теме
% диссертации формируется в разделе <<\publications>>\ файла
% \verb!common/characteristic.tex!  при помощи команды \verb!\nocite!

\ifdefmacro{\microtypesetup}{\microtypesetup{protrusion=false}}{} % не рекомендуется применять пакет микротипографики к автоматически генерируемому списку литературы
\urlstyle{rm}                               % ссылки URL обычным шрифтом
\ifnumequal{\value{bibliosel}}{0}{% Встроенная реализация с загрузкой файла через движок bibtex8
    \renewcommand{\bibname}{\large \bibtitleauthor}
    \nocite{*}
    \insertbiblioauthor           % Подключаем Bib-базы
    %\insertbiblioexternal   % !!! bibtex не умеет работать с несколькими библиографиями !!!
}{% Реализация пакетом biblatex через движок biber
    % Цитирования.
    %  * Порядок перечисления определяет порядок в библиографии (только внутри подраздела, если `\insertbiblioauthorgrouped`).
    %  * Если не соблюдать порядок "как для \printbibliography", нумерация в `\insertbiblioauthor` будет кривой.
    %  * Если цитировать каждый источник отдельной командой --- найти некоторые ошибки будет проще.
    %
    %% authorvak
    \nocite{Gladstein-al:ISPRAS21}
    % \nocite{vakbib1}%
    %
    %% authorwos
    % \nocite{wosbib1}%
    %
    %% authorscopus
    \nocite{Moiseenko-al:STJITMO22}%
    \nocite{Moiseenko-al:PCS21}%
    \nocite{Moiseenko-al:ECOOP20}%
    \nocite{Moiseenko-al:OOPSLA22}%
    %
    %% authorpathent
    % \nocite{patbib1}%
    %
    %% authorprogram
    % \nocite{progbib1}%
    %
    %% authorconf
    % \nocite{confbib1}%
    % \nocite{confbib2}%
    %
    %% authorother
    % \nocite{bib1}%
    % \nocite{bib2}%

    \ifnumgreater{\value{usefootcite}}{0}{
        \begin{refcontext}[labelprefix={}]
            \ifnum \value{bibgrouped}>0
                \insertbiblioauthorgrouped    % Вывод всех работ автора, сгруппированных по источникам
            \else
                \insertbiblioauthor      % Вывод всех работ автора
            \fi
        \end{refcontext}
    }{
        \ifnum \totvalue{citeexternal}>0
            \begin{refcontext}[labelprefix=A]
                \ifnum \value{bibgrouped}>0
                    \insertbiblioauthorgrouped    % Вывод всех работ автора, сгруппированных по источникам
                \else
                    \insertbiblioauthor      % Вывод всех работ автора
                \fi
            \end{refcontext}
        \else
            \ifnum \value{bibgrouped}>0
                \insertbiblioauthorgrouped    % Вывод всех работ автора, сгруппированных по источникам
            \else
                \insertbiblioauthor      % Вывод всех работ автора
            \fi
        \fi
        %  \insertbiblioauthorimportant  % Вывод наиболее значимых работ автора (определяется в файле characteristic во второй section)
        \begin{refcontext}[labelprefix={}]
            \insertbiblioexternal            % Вывод списка литературы, на которую ссылались в тексте автореферата
        \end{refcontext}
        % Невидимый библиографический список для подсчёта количества внешних публикаций
        % Используется, чтобы убрать приставку "А" у работ автора, если в автореферате нет
        % цитирований внешних источников.
        \printbibliography[heading=nobibheading, section=0, env=countexternal, keyword=biblioexternal, resetnumbers=true]%
    }
}
\ifdefmacro{\microtypesetup}{\microtypesetup{protrusion=true}}{}
\urlstyle{tt}                               % возвращаем установки шрифта ссылок URL
      % Содержание автореферата

%%% Выходные сведения типографии
\newpage\thispagestyle{empty}

\vspace*{0pt plus1fill}

\small
\begin{center}
    \textit{\thesisAuthor}
    \par\medskip

    \thesisTitle
    \par\medskip

    Автореф. дис. на соискание ученой степени \thesisDegreeShort
    \par\bigskip

    Подписано в печать \blank[\widthof{999}].\blank[\widthof{999}].\blank[\widthof{99999}].
    Заказ № \blank[\widthof{999999999999}]

    Формат 60\(\times\)90/16. Усл. печ. л. 1. Тираж 100 экз.
    %Это не совсем формат А5, но наиболее близкий, подробнее: http://ru.wikipedia.org/w/index.php?oldid=78976454

    Типография \blank[0.5\linewidth]
\end{center}
\cleardoublepage

\end{document}
