\begin{center}
\begin{table}[t]
\begin{tabular}{l@{\hskip 40pt}|l} 

    \hline

    Синтаксис инструкции & Семантика инструкции \\

    \midrule

      $\writeInst{o}{x}{v}$ 
    & \makecell[l]{
        Инструкция записи значения $v$ \\
        в разделяемую переменную $x$   \\
        с режимом доступа $o$.
      } 
    \\ 
    \hline

      $\readInst{o}{r}{x}$ 
    & \makecell[l]{
        Инструкция чтения значения    \\
        из разделяемой переменной $x$ \\
        в локальную переменную $r$    \\
        с режимом доступа $o$.
      } 
    \\ 
    \hline

      $\casInst{o_s}{o_f}{r}{x}{v_e}{v_d}$ 
    & \makecell[l]{
        Инструкция атомарного сравнения                 \\ 
        разделяемой переменной $x$                      \\
        c ожидаемым значением $v_e$ и                   \\ 
        заменой на желаемое значение $v_d$;             \\ 
        прочитанное значение присваивается              \\
        в локальную переменную $r$;                     \\
        в случае успеха операции сравнения применяется  \\ 
        режим доступа $o_s$, иначе $o_f$.
      } 
    \\ 
    \hline

      $\exchgInst{o}{r}{x}{v}$ 
    & \makecell[l]{
        Инструкция атомарного обмена значения           \\
        разделяемой переменной $x$ на значение $v_e$,   \\
        аннотированная режимом доступа $o$;             \\ 
        прочитанное значение присваивается              \\
        в локальную переменную $r$.                     \\
      } 
    \\ 
    \hline

      $\faiInst{o}{r}{x}{v}$ 
    & \makecell[l]{
        Инструкция атомарного инкремента значения       \\
        разделяемой переменной $x$ на значение $v$,     \\
        аннотированная режимом доступа $o$;             \\ 
        прочитанное значение присваивается              \\
        в локальную переменную $r$.                     \\
      } 
    \\ 
    \hline

      $\fenceInst{o}$ 
    & \makecell[l]{
        Инструкция барьера памяти                       \\
        аннотированная режимом доступа $o$.             \\ 
      } 
    \\ 
    \hline


\end{tabular}
\captionsetup{justification=centering}
\caption{Список используемых программных примитивов}
\label{table:primitives}
\end{table}
\end{center}
