\begin{figure}[t]

\begin{subfigure}[t]{.48\textwidth}
\vskip 0pt
\centering
\begin{equation*}
  \inarrII{
    \writeInst{\sco}{x}{1}   \\
    \readInst{\sco}{r_1}{y}  \\
  }{
    \writeInst{\sco}{y}{1}   \\
    \readInst{\sco}{r_2}{x}  \\
  }
\tag{SB-SC}\label{ex:sb-sc}
\end{equation*}
\end{subfigure}
%
\hfill
%
\begin{subfigure}[t]{.48\textwidth}\centering
\vskip 0pt
  \begin{tikzpicture}[xscale=3,yscale=1.5]
    \node (0)  at (0.5, 2) {$\Init$};

    \node (wx) at (0, 1) {$\wlab{\sco}{x}{1}$};
    \node (ry) at (0, 0) {$\rlab{\sco}{y}{0}$};

    \node (wy) at (1, 1) {$\wlab{\sco}{y}{1}$};
    \node (rx) at (1, 0) {$\rlab{\sco}{x}{0}$};

    \draw[po] (0) edge (wx) edge (wy);
    \draw[po] (wx) edge (ry);
    \draw[po] (wy) edge (rx);

    \draw[rb] (rx) edge (wx);
    \draw[rb] (ry) edge node[below] {\small$\lRB$} (wy);

    \draw[rf,bend left =50] (0) edge (rx);
    \draw[rf,bend right=50] (0) edge (ry);

  \end{tikzpicture}
\end{subfigure}

\caption{Пример графа сценария исполнения, 
который не является \textbf{последовательно согласованным}.}
\label{fig:sc-ex}
\end{figure}
