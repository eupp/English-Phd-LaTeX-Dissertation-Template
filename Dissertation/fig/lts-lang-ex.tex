\begin{figure}[t]
  \centering

    \begin{subfigure}[t]{0.48\textwidth}
    \centering
    \begin{tikzpicture}[xscale=1.5,yscale=1.5]

      \tikzset{
        smallstate/.style={state,
          inner sep=3pt,
          outer sep=3pt,
          minimum size=0pt,
        },
      }

      \node[smallstate] (s0) at (1,3) {$s_0$};
      \node[smallstate] (s1) at (1,2) {$s_1$};
      \node[smallstate] (s2) at (0,1) {$s_2$};
      \node[smallstate] (s3) at (2,1) {$s_3$};
      \node[smallstate] (s4) at (1,0) {$s_4$};

      \draw[->] (s0) edge[right,pos=0.5] node {$a$} (s1);

      \draw[->] (s1) edge[bend right,above,pos=0.5] node {$c$} (s2);
      \draw[->] (s1) edge[bend left ,below,pos=0.5] node {$d$} (s2);

      \draw[->] (s1) edge[right,pos=0.5] node {$b$} (s3);

      \draw[->] (s2) edge[left,pos=0.5]  node {$b$} (s4);

      \draw[->] (s3) edge[bend right,above,pos=0.5] node {$c$} (s4);
      \draw[->] (s3) edge[bend left ,below,pos=0.5] node {$d$} (s4);

    \end{tikzpicture}
    \label{fig:lts-ex}
    \caption{Система помеченных переходов}
  \end{subfigure}
  \hfill
  \begin{subfigure}[t]{0.48\textwidth}
    \centering
    \begin{tikzpicture}[xscale=1.5,yscale=1.5]
      \node (a1) at (0, 2) {$a$};
      \node (b1) at (0, 1) {$b$};
      \node (c1) at (0, 0) {$c$};

      \draw[po] (a1) edge (b1);
      \draw[po] (b1) edge (c1);

      \node (a2) at (1, 2) {$a$};
      \node (c2) at (1, 1) {$c$};
      \node (b2) at (1, 0) {$b$};

      \draw[po] (a2) edge (c2);
      \draw[po] (c2) edge (b2);

      \node (a3) at (2, 2) {$a$};
      \node (b3) at (2, 1) {$b$};
      \node (d3) at (2, 0) {$d$};

      \draw[po] (a3) edge (b3);
      \draw[po] (b3) edge (d3);

      \node (a4) at (3, 2) {$a$};
      \node (d4) at (3, 1) {$d$};
      \node (b4) at (3, 0) {$b$};

      \draw[po] (a4) edge (d4);
      \draw[po] (d4) edge (b4);
    \end{tikzpicture}
    \label{fig:lang-ex}
    \caption{Язык принимаемый системой переходов в начальном состоянии $s_0$.}
  \end{subfigure}

  
  %% \begin{subfigure}[t]{0.48\textwidth}
  %%   \centering
  %%   \begin{tikzpicture}[xscale=0.6]
  %%     \node (a1) at (1, 1) {$a$};
  %%     \node (b1) at (0, 0) {$b$};
  %%     \node (c1) at (2, 0) {$c$};

  %%     \draw[po] (a1) edge (b1);
  %%     \draw[po] (a1) edge (c1);

  %%     \node (a2) at (4, 1) {$a$};
  %%     \node (b2) at (3, 0) {$b$};
  %%     \node (d2) at (5, 0) {$d$};

  %%     \draw[po] (a2) edge (b2);
  %%     \draw[po] (a2) edge (d2);
  %%   \end{tikzpicture}
  %%   \caption{Язык помеченных частично упорядоченных мультимножеств}
  %% \end{subfigure}
  %% \hfill
  %% \begin{subfigure}[t]{0.48\textwidth}
  %%   \centering
  %%   \begin{tikzpicture}[xscale=0.8]
  %%     \node (a) at (1, 1) {$a$};
  %%     \node (b) at (0, 0) {$b$};
  %%     \node (c) at (1, 0) {$c$};
  %%     \node (d) at (2, 0) {$d$};

  %%     \draw[po] (a) edge (b);
  %%     \draw[po] (a) edge (c);
  %%     \draw[po] (a) edge (d);
  %%     \draw[cf] (c) -> (d);
  %%   \end{tikzpicture}
  %%   \caption{Простая структура событий}
  %% \end{subfigure}
  \label{fig:lts-lang-ex}
  \caption{
    Пример системы помеченных переходов и принимаемого ей языка
  }
%% как обычного языка,
%%     языка помеченных частично упорядоченных мультимножеств
    %% и структуры событий. 
\end{figure}
