\begin{figure}[b]

\begin{subfigure}[t]{.48\textwidth}
\vskip 0pt
\centering
\begin{equation*}
  \inarrII{
    \writeInst{}{x}{2}        \\
    \readInst{}{a}{x}         \\
  }{
    \faiInst{}{b}{x}{1}       \\
  }
\tag{INC}\label{ex:inc}
\end{equation*}
\end{subfigure}
%
\hfill
%
\begin{subfigure}[t]{.48\textwidth}\centering
\vskip 0pt
  \begin{tikzpicture}[xscale=3,yscale=1.5]
    \node (0)  at (0.5, 2) {$\Init$};

    \node (wx) at (0, 1) {$\wlab{}{x}{2}$};
    \node (rx) at (0, 0) {$\rlab{}{x}{1}$};

    \node (exrx) at (1, 1) {$\rlab{}{x}{0}$};
    \node (exwx) at (1, 0) {$\wlab{}{x}{1}$};

    \draw[po] (0) edge (wx) edge (exrx);
    \draw[po] (wx) edge (rx);

    \draw[rmw] (exrx) edge node[right] {\small$\lRMW$} (exwx);

    \draw[rb] (exrx) edge node[above] {\small$\lRB$} (wx);
    \draw[co] (wx)   edge node[below] {\small$\lCO$} (exwx);

    \draw[rf] (exwx) edge (rx);
    \draw[rf,bend left=20] (0) edge (exrx);

  \end{tikzpicture}
\end{subfigure}

\caption{Пример \textbf{неатомарного} графа сценария исполнения.}
\label{fig:atom-ex}
\end{figure}
