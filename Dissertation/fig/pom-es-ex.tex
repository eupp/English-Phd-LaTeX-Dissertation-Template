\begin{figure}[t]
  \centering
  \begin{subfigure}[t]{0.48\textwidth}
    \centering
    \begin{tikzpicture}[xscale=1.5,yscale=1.5]
      \node (a1) at (1, 1) {$a$};
      \node (b1) at (0, 0) {$b$};
      \node (c1) at (2, 0) {$c$};

      \draw[po] (a1) edge (b1);
      \draw[po] (a1) edge (c1);

      \node (a2) at (4, 1) {$a$};
      \node (b2) at (3, 0) {$b$};
      \node (d2) at (5, 0) {$d$};

      \draw[po] (a2) edge (b2);
      \draw[po] (a2) edge (d2);
    \end{tikzpicture}
    \label{fig:pom-ex}
    \caption{Язык помеченных частично упорядоченных мультимножеств}
  \end{subfigure}
  \hfill
  \begin{subfigure}[t]{0.48\textwidth}
    \centering
    \begin{tikzpicture}[xscale=1.5,yscale=1.5]
      \node (a) at (1, 1) {$a$};
      \node (b) at (0, 0) {$b$};
      \node (c) at (1, 0) {$c$};
      \node (d) at (2, 0) {$d$};

      \draw[po] (a) edge (b);
      \draw[po] (a) edge (c);
      \draw[po] (a) edge (d);
      \draw[cf] (c) -> (d);
    \end{tikzpicture}
    \label{fig:es-ex}
    \caption{Простая структура событий}
  \end{subfigure}

  \label{fig:pom-es-ex}
  \caption{
    Пример кодирования языка системы переходов 
    как языка помеченных частично упорядоченных мультимножеств
    и как простой структуры событий. 
  }
\end{figure}
