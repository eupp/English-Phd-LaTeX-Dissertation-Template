\chapter{Модель \WkmS и свойства свободы от буферизации операций чтения и локальности сертификациии}
\label{ch:weakestmo2}

Модели памяти, сохраняющие семантические зависимости,
и, в частности, модель \Wkm, используют различные формы
спекулятивного исполнения, для того чтобы обеспечить
корректность оптимальных схем компиляции и оптимизаций.
Требование на предоставление этими моделями некоторых базовых гарантий,
таких как свойство свободы от гонок, накладывают
ограничения на возможности спекулятивного исполнения.
Несмотря на эти ограничения, как будет показано в этой главе,
механизм спекулятивного исполнения, используемый моделью \Wkm,
все еще порождает большое количество слабых сценариев исполнения,
которые не могут быть обоснованы как результат применения
разумного набора оптимизаций.
Кроме того, наличие данных слабых сценариев
затрудняет разработку средств автоматической верификации
программ для модели \Wkm и других моделей данного класса. 

В данной главе будет представлена новая версия модели \Wkm ---
модель \WkmS, дополненная новыми свойствами
свободы от буферизации операций чтения и локальности сертификации.
Данные свойства накладывают ограничения на то
\begin{itemize}
  \item когда может использоваться спекулятивное исполнение, и
  \item как сильно ветки спекулятивного исполнения программы 
    могут отличаться от итоговой ветки фактического исполнения. 
\end{itemize}
Введение данных свойств служит двум целям. 
Во-первых, наличие данных свойств препятствует появлению 
некоторых подозрительных слабых сценариев поведения, 
которые не могут быть обоснованы как результат применения
разумного набора оптимизаций. 
Во-вторых, наличие данных свойств позволяет реализовать 
несколько ключевых оптимизаций в алгоритме проверки моделей 
и таким образом впервые разработать эффективный инструмент
автоматической верификации для модели из класса 
моделей памяти, сохраняющих семантические зависимости. 

Данная глава организована следующим образом. 
В разделах \ref{sec:lbrf} и \ref{sec:cert-loc}
вводятся свойства свободы от буферизации операций чтения 
и локальности сертификации и на примерах показывается 
какие слабые сценарии исполнения запрещаются данными свойствами. 
В разделе \ref{sec:mc-opt} кратко объясняется как 
наличие данных свойств позволяет оптимизировать 
алгоритм проверки моделей. 
Наконец, в разделе \ref{sec:wkmo2} вводится 
формальное определение модели \WkmS 
и доказывается, что она сохраняет все полезные свойства модели \Wkm.

\section{Свобода от буферизации операций чтения}
\label{sec:lbrf}

Свойство свободы от буферизации операций чтения 
(\emph{load buffering race freedom}, \LBRF) 
допускает использование спекулятивного исполнения только в случае, 
если программа содержит гонку особого вида, 
называемую \emph{гонкой с буферизацией операции чтения}.
Свойство свободы от буферизации операций 
является частным случаем свойства свободы от гонок. 
Далее введем формальное определение этого понятия.

\begin{definition}
  Рассмотрим граф сценария исполнения $G$. 
  Будем называть $\lPO$ ребро от операции чтения $r$ к операции записи $w$
  \emph{переупорядочиваемым} (\emph{reordarable}) и обозначать 
  соответствующее отношение как $\lRPO$ 
  если $r$ и $w$ это ослабленные (relaxed) обращения и между ними нет барьеров.  

  $$ \lRPO \defeq
     [\lR^{\rlx}] \seqc (\lPO \setminus (\lPO \seqc [\lF] \seqc \lPO)) \seqc [\lW^{\rlx}]
  $$
\end{definition}

\begin{definition}
\label{def:lb-race}
  Пара событий $r$ и $w$ образуют \emph{гонку с буферизацией операции чтения}, 
  что обозначается как $\tup{r, w} \in \lLBRACE$,
  в графе $G$ если $r$ это операция чтения, а $w$ это конкурентная ей операция записи,
  такая что от $r$ к $w$ существует $\lPORF$ путь, начинающийся с 
  переупорядочиваемого ребра $\lRPO$.

  \begin{equation*}
    \lLBRACE \defeq 
      ([\lR] \seqc {=_{\lLOC}} \seqc [\lW]) \cap 
      (\lRPO \seqc (\lRF \setminus \lPO) \seqc \lPORF)) \setminus 
      (\lHB \cup \lHB^{-1})
  \end{equation*}

\end{definition}

На рисунке \ref{fig:lbr-ex} показан пример графа, 
содержащего гонку с буферизацией операции чтения
(такую гонку образуют события $r_x$ и $w_x$).

\begin{figure}[t]
\centering
\begin{tikzpicture}[xscale=1.4,yscale=0.8]

  %% \node at (0,1.8) {$\circledb{C}$};

  \node (init) at (1,  1.5) {$\Init$};

  \node[color=colorRACE] (i11) at (0,  0) {$r_x: \rlab{}{x}{0}{}$};
  \node                  (i12) at (0, -2) {$w_y: \wlab{}{y}{1}{}$};

  \node                  (i21) at (2,  0) {$r_y: \rlab{}{y}{1}{}$};
  \node[color=colorRACE] (i22) at (2, -2) {$w_x: \wlab{}{x}{1}{}$};

  \draw[po] (i11) edge node[right] {\small$\lPO$} (i12);
  \draw[po] (i21) edge node[left ] {\small$\lPO$} (i22);
  %% \draw[ppo,bend left=10] (i21) edge node[right] {\small$\lPPO$} (i22);

  \draw[rf] (init) edge node[below] {} (i11);
  \draw[rf] (i12)  edge node[below] {\small$\lRF$} (i21);

  \draw[po] (init) edge node[left]  {\small$\lPO$} (i11);
  \draw[po] (init) edge node[right] {\small$\lPO$} (i21);

\end{tikzpicture}
\caption{Пример гонки с буферизацией операции чтения.}
\label{fig:lbr-ex}
\end{figure}


\begin{definition}
Программа $P$ называется свободной от буферизации операций чтения 
в модели памяти $M$ если все консистентные с точки зрения 
модели $M$ графы этой программы не содержат 
гонки с буферизацией операции чтения.
\end{definition}

\begin{definition}[\LBRF]
Будем говорить, что модель памяти $M$ предоставляет 
свойство свободы от буферизации операций чтения 
относительно более строгой модели $M'$,
что обозначается как $M \in \LBRF_{M'}$, если 
для любой программы $P$ свободной от буферизации операций чтения 
согласно модели $M'$, множество консистентных сценариев исполнения 
программы $P$ с точки зрения модели $M$ совпадает 
с множеством консистентных сценариев исполнения 
с точки зрения модели~$M'$.
\end{definition}

В качестве более строгой модели $M'$ рассмотрим 
модель памяти \RCMM, сохраняющую программный порядок
(то есть запрещающую $\lPORF$ циклы).
Тогда свойство $\LBRF_{\RCMM}$ для модели \Wkm 
утверждает, что если ни один \RCMM консистентный граф 
программы $P$ не содержит гонки с буферизацией операции чтения,
тогда модель \Wkm также должна допускать для программы $P$ 
только \RCMM консистентные графы, то есть 
все такие графы должны быть $\lPORF$ ацикличны.

К сожалению, как показывает следующий пример, 
модель \Wkm не обладает свойством 
свободы от буферизации операций чтения, 
то есть ${\Wkm \not\in \LBRF_{M'}}$.
Рассмотрим программу \ref{prog:lbf-cex} и соответствующие ей 
\Wkm консистентные графы сценариев исполнения, 
показанные на рисунке~\ref{fig:lbf-cex}. 
Среди этих графов \circledb{A}, \circledb{B}, \circledb{C}, \circledb{D} 
также являются \RCMM консистентными и при этом не содержат 
гонок с буферизацией операций чтения.
Несмотря на это, модель \Wkm также допускает граф \circledr{E},
который содержит $\lPORF$ цикл и, следовательно, не является \RCMM консистентным. 

%% This execution can arise in the following manner. First, thread 2
%% promises the $\wlab{}{y}{1}$ store; the promise is allowed because thread 2
%% can read $z = 0$ and fulfill it.
%% Then, thread 3 executes: it reads $y = 1$ and writes $1$ to $x$.
%% Finally, threads 1 and 2 execute: thread 2 reads $z = 1$ and $x = 1$,
%% and subsequently writes $1$ to $y$ thereby fulfilling its promise.

%!TEX root = ../main.tex

\begin{figure}[t]\centering\small
\begin{minipage}{.35\textwidth}
\vspace{-12pt}
\begin{equation}
\inarrIII{
  \writeInst{}{z}{1}
}{
  \ifStmt{\readExpr{}{z}}  \\
  \quad\readInst{}{r}{x}   \\
  \quad\writeInst{}{y}{r}  \\
  \elseStmt                \\
  \quad\writeInst{}{y}{1}  \\
}{
  \ifStmt{\readExpr{}{y}}  \\
  \quad\writeInst{}{x}{1}  \\
}
\tag{\LBRF-cex}\label{prog:lbf-cex}
\end{equation}
\end{minipage}\hfill
\begin{minipage}{.5\textwidth}
  \begin{tikzpicture}[yscale=1,xscale=1.4]
    \node at (-1,1) {$\circledb{A}$};
    \node (0)  at (0,1) {$[init]$};
    \node (wz) at (-1, 0) {$\wlab{}{z}{\!\,1}$};
    \node (rz) at (0, 0) {$\rlab{}{z}{0}$};
    \node (wy) at (0, -1) {$\wlab{}{y}{1}$};
    \node (ry) at (1, 0) {$\rlab{}{y}{0}$};
%    \node (wx) at (1, -1) {$\wlab{}{x}{0}$};
    \node      at (0, -2) {\phantom{$\rlab{}{y}{0}$}};
    \draw[po] (0) edge (wz) edge (rz) edge (ry);
    \draw[po] (rz) edge (wy);
%    \draw[po] (ry) edge (wx);
    \draw[rf] (0) edge[bend left=20] (rz);
    \draw[rf] (0) edge[bend left=20] (ry);
  \end{tikzpicture}
  \hfill
  \begin{tikzpicture}[yscale=1,xscale=1.4]
    \node at (-1,1) {$\circledb{B}$};
    \node (0)  at (0,1) {$[init]$};
    \node (wz) at (-1, 0) {$\wlab{}{z}{\!\,1}$};
    \node (rz) at (0, 0) {$\rlab{}{z}{0}$};
    \node (wy) at (0, -1) {$\wlab{}{y}{1}$};
    \node (ry) at (1, 0) {$\rlab{}{y}{1}$};
    \node (wx) at (1, -1) {$\wlab{}{x}{1}$};
    \node      at (0, -2) {\phantom{$\rlab{}{y}{0}$}};
    \draw[po] (0) edge (wz) edge (rz) edge (ry);
    \draw[po] (rz) edge (wy);
    \draw[po] (ry) edge (wx);
    \draw[rf, bend left=20] (0) edge (rz);
    \draw[rf] (wy) edge (ry);
  \end{tikzpicture}
\end{minipage}

  \begin{tikzpicture}[yscale=1,xscale=1.4]
    \node at (-1,1) {$\circledb{C}$};
    \node (0)  at (0,1) {$[init]$};
    \node (wz) at (-1, 0) {$\wlab{}{z}{1}$};
    \node (rz) at (0, 0) {$\rlab{}{z}{1}$};
    \node (rx) at (0, -1) {$\rlab{}{x}{0}$};
    \node (wy) at (0, -2) {$\wlab{}{y}{0}$};
    \node (ry) at (1, 0) {$\rlab{}{y}{0}$};
%    \node (wx) at (1, -1) {$\wlab{}{x}{0}$};
    \draw[po] (0) edge (wz) edge (rz) edge (ry);
    \draw[po] (rz) edge (rx);
    \draw[po] (rx) edge (wy);
%    \draw[po] (ry) edge (wx);
    \draw[rf] (wz) edge (rz);
    \draw[rf, bend left=30] (0) edge (rx);
    \draw[rf, bend left=20] (0) edge (ry);
  \end{tikzpicture}
  \hfill
  \begin{tikzpicture}[yscale=1,xscale=1.4]
    \node at (-1,1) {$\circledb{D}$};
    \node (0)  at (0,1) {$[init]$};
    \node (wz) at (-1, 0) {$\wlab{}{z}{1}$};
    \node (rz) at (0, 0) {$\rlab{}{z}{1}$};
    \node (rx) at (0, -1) {$\rlab{}{x}{0}$};
    \node (wy) at (0, -2) {$\wlab{}{y}{0}$};
    \node (ry) at (1, 0) {$\rlab{}{y}{0}$};
%    \node (wx) at (1, -1) {$\wlab{}{x}{0}$};
    \draw[po] (0) edge (wz) edge (rz) edge (ry);
    \draw[po] (rz) edge (rx);
    \draw[po] (rx) edge (wy);
%    \draw[po] (ry) edge (wx);
    \draw[rf] (wz) edge (rz);
    \draw[rf, bend left=30] (0) edge (rx);
    \draw[rf] (wy) edge (ry);
  \end{tikzpicture}
  \hfill 
  \begin{tikzpicture}[yscale=1,xscale=1.6]
%    \draw[color=brown,fill=yellow!20,dashed,thick] (-2.4,0.5) rectangle (-1.6,-2.5);
    \node at (-1,1) {$\circledr{E}$};
    \node (0)  at (0,1) {$[init]$};
    \node (wz) at (-1, 0) {$\wlab{}{z}{1}$};
    \node (rz) at (0, 0) {$\rlab{}{z}{1}$};
    \node (rx) at (0, -1) {$\rlab{}{x}{1}$};
    \node (wy) at (0, -2) {$\wlab{}{y}{1}$};
    \node (ry) at (1, 0) {$\rlab{}{y}{1}$};
    \node (wx) at (1, -1) {$\wlab{}{x}{1}$};
%    \node (crz) at (-2, 0) {$\rlab{}{z}{1}$};
%    \node (cwy) at (-2, -2) {$\wlab{}{y}{1}$};
    \draw[po] (0) edge (wz) edge (rz) edge (ry);
    \draw[po] (rz) edge (rx);
    \draw[po] (rx) edge (wy);
    \draw[po] (ry) edge (wx);
%    \draw[po] (crz) edge (cwy);
    \draw[rf,bend left=20] (wz) edge (rz);
    \draw[rf] (wx) edge (rx);
    \draw[rf] (wy) edge (ry);
%    \draw[rf] (wz) edge (crz);
  \end{tikzpicture}
  \caption{Пример свободной от буферизации операций чтения программы и 
    соответствующих ей \Wkm консистентных графов сценариев исполнения.}
  \label{fig:lbf-cex}
\end{figure}


%% Normally, we take $M'$ to be some standard model that forbids $\lPORF$ cycles,
%% such as \RCMM\@.
%% Similar to $\drf$, $\lbf_\RCMM$ allows programmers to program defensively
%% against $M$ avoiding LB races, and relieves them from the need to learn and
%% understand the definition of $M$.
%% Since absence of LB races is to be checked with respect to \RCMM, one can use
%% any of the existing tools and methodologies that reason about program
%% correctness under \RCMM (\eg \igps, \Herd, \rcmc, \genmc, etc).

\section{Локальность сертификации}
\label{sec:cert-loc}

Введение свойства локальности сертификации.
Мотивация: отсеять часть необоснованных сценариев исполнения.
Формальное определение. Примеры. 

\section{Оптимизация алгоритма проверки моделей}
\label{sec:mc-opt}


\section{Формализация модели \WkmS}
\label{sec:wkmo2}

Формальное определение новой уточненной версии модели \WkmS.

\subsection{Свобода от гонок и буферизации операций чтения}

Доказательтсво, что новая версия модели
удовлетворяет свойствам свободы от гонок и от буферизации операция чтения.

\subsection{Корректность компиляции}

Доказательтсво, что новая версия модели
удовлетворяет свойству корректности компиляции в модели
современных мультипроцессоров.
Описание модификации доказательства из раздела \ref{ch:weakestmo-imm}.

\subsection{Корректность локальных трансформаций}

Доказательтсво, что в новой версии модели
сохраняется корректность локальных трансформаций:
перестановки независимых инструкций и удаления дублирующей инструкции. 

