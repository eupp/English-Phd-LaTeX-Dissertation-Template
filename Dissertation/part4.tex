\chapter{Модель \WkmS и свойства свободы от буферизации операций чтения и локальности сертификациии}
\label{ch:weakestmo2}

Модели памяти, сохраняющие семантические зависимости,
и, в частности, модель \Wkm, используют различные формы
спекулятивного исполнения, для того чтобы обеспечить
корректность оптимальных схем компиляции и оптимизаций.
Требование на предоставление этими моделями некоторых базовых гарантий,
таких как свойство свободы от гонок, накладывают
ограничения на возможности спекулятивного исполнения.
Несмотря на эти ограничения, как будет показано в этой главе,
механизм спекулятивного исполнения, используемый моделью \Wkm,
все еще порождает большое количество слабых сценариев исполнения,
которые не могут быть обоснованы как результат применения
разумного набора оптимизаций.
Кроме того, наличие данных слабых сценариев
затрудняет разработку средств автоматической верификации
программ для модели \Wkm и других моделей данного класса. 

В данной главе будет представлена новая версия модели \Wkm ---
модель \WkmS, дополненная новыми свойствами
свободы от буферизации операций чтения и локальности сертификации.
Данные свойства накладывют ограничения на то
\begin{itemize}
  \item когда может использоваться спекулятивное исполнение, и
  \item как сильно ветки спекулятивного исполнения программы 
    могут отличаться от итоговой ветки фактического исполнения. 
\end{itemize}
Введение данных свойств служит двум целям. 
Во-первых, наличие данных свойств препятствует появлению 
некоторых подозрительных слабых сценариев поведения, 
которые не могут быть обоснованы как результат применения
разумного набора оптимизаций. 
Во-вторых, наличие данных свойств позволяет реализовать 
несколько ключевых оптимизаций в алгоритме проверки моделей 
и таким образом впервые разработать эффективный инструмент
автоматической верификации для модели из класса 
моделей памяти, сохраняющих семантические зависимости. 

Данная глава организована следующим образом. 
В разделах \ref{sec:lbrf} и \ref{sec:cert-loc}
вводятся свойства свободы от буферизации операций чтения 
и локальности сертификации и на примерах показывается 
какие слабые сценарии исполнения запрещаются данными свойствами. 
В разделе \ref{sec:mc-opt} кратко объясняется как 
наличие данных свойств позволяет оптимизировать 
алгоритм проверки моделей. 
Наконец, в разделе \ref{sec:wkmo2} вводится 
формальное определение модели \WkmS 
и доказывается, что она сохраняет все полезные свойства модели \Wkm.

\section{Cвобода от буферизации операций чтения}
\label{sec:lbrf}

Введение свойства свободы от буферизации операция чтения.
Мотивация: отсеять часть необоснованных сценариев исполнения.
Формальное определение. Примеры. 

\section{Локальность сертификации}
\label{sec:cert-loc}

Введение свойства локальности сертификации.
Мотивация: отсеять часть необоснованных сценариев исполнения.
Формальное определение. Примеры. 

\section{Оптимизация алгоритма проверки моделей}
\label{sec:mc-opt}


\section{Формализация модели \WkmS}
\label{sec:wkmo2}

Формальное определение новой уточненной версии модели \WkmS.

\subsection{Свобода от гонок и буферизации операций чтения}

Доказательтсво, что новая версия модели
удовлетворяет свойствам свободы от гонок и от буферизации операция чтения.

\subsection{Корректность компиляции}

Доказательтсво, что новая версия модели
удовлетворяет свойству корректности компиляции в модели
современных мультипроцессоров.
Описание модификации доказательства из раздела \ref{ch:weakestmo-imm}.

\subsection{Корректность локальных трансформаций}

Доказательтсво, что в новой версии модели
сохраняется корректность локальных трансформаций:
перестановки независимых инструкций и удаления дублирующей инструкции. 

