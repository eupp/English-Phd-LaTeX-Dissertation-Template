\chapter{Модель \WkmS и свойства свободы от буфферезации операций чтения и локальности сертификациии}
\label{ch:weakestmo2}

Постановка проблемы: модель \Wkm допускает слишком много
сценариев исполнения (большое пространство перебора),
при этом данные сценарии не могут наблюдаться в реальности
(т.е. являются артефактами модели).
Эта проблема затрудняет разработку средств верификации
программ для модели \Wkm. 
Обоснование необходимости уточнения модели.

\section{Недостатки модели \Wkm}

Примеры программ, на которых можно наблюдать
артефакты модели \Wkm (необоснованные сценарии исполнения).
Неформальное описание причин возникновения этих артефактов. 

\section{Cвобода от буфферезации операций чтения}

Введение свойства свободы от буфферизации операция чтения.
Мотивация: отсеять часть необоснованных сценариев исполнения.
Формальное определение. Примеры. 

\section{Локальность сертификации}

Введение свойства локальности сертификации.
Мотивация: отсеять часть необоснованных сценариев исполнения.
Формальное определение. Примеры. 

\section{Формализация модели \WkmS}

Формальное определение новой уточненной версии модели \WkmS.

\subsection{Свобода от гонок и буфферезации операций чтения}

Доказательтсво, что новая версия модели
удовлетворяет свойствам свободы от гонок и от буфферизации операция чтения.

\subsection{Корректность компиляции}

Доказательтсво, что новая версия модели
удовлетворяет свойству корректности компиляции в модели
современных мультипроцессоров.
Описание модификации доказательства из раздела \ref{ch:weakestmo-imm}.

\subsection{Корректность локальных трансформаций}

Доказательтсво, что в новой версии модели
сохраняется корректность локальных трансформаций:
перестановки независимых инструкций и удаления дублирующей инструкции. 

