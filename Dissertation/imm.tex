\subsection*{Модель \IMM}

Модель \IMM относится к классу моделей, сохраняющих синтаксические зависимости. 
В рамках данных моделей, как правило, 
определяется отношение \emph{сохраняемого программного порядка}
(\emph{preserved program order}) $\lPPO$, 
являющееся подмножеством обычного программного порядка. 
События, связанные сохраняемым программным порядком, 
должны выполнятся согласно этому порядку, 
а несвязанные события могут выполняться в произвольном порядке. 
 
Рассмотрим программы \ref{ex:LB-nodep}, 
\ref{ex:LB-fakedep} и \ref{ex:LB-dep} показанные ниже.

\begin{center}
\begin{minipage}{.32\linewidth}
{\small
\begin{equation}
\inarrII{
  \readInst{}{a}{x} \rfcomment{1} \\
  \writeInst{}{y}{1} \\
}{\readInst{}{b}{y} \rfcomment{1} \\
  \writeInst{}{x}{b}  \\
}%
\tag{LB-nodep}\label{ex:LB-nodep}
\end{equation}
}
\end{minipage}
%
\hfill\vline\hfill
\begin{minipage}{.32\linewidth}
{\small
\begin{equation}
\inarrII{
  \readInst{}{a}{x} \rfcomment{1} \\
  \writeInst{}{y}{1 + a * 0} \\
}{\readInst{}{b}{y} \rfcomment{1} \\
  \writeInst{}{x}{b}  \\
}
\tag{LB-fakedep}\label{ex:LB-fakedep}
\end{equation}
}
\end{minipage}
%
\hfill\vline\hfill
%
\begin{minipage}{.32\linewidth}
{\small
\begin{equation}
\inarrII{
  \readInst{}{a}{x} \nocomment{1} \\
  \writeInst{}{y}{a} \\
}{\readInst{}{b}{y} \nocomment{1} \\
  \writeInst{}{x}{b}  \\
}
\tag{LB-dep}\label{ex:LB-dep}
\end{equation}
}
\end{minipage}
\end{center}


\eupp{В этом разделе далее необходимо переформулировать предложения,
  чтобы избежать плагиата с ВКР.}

Модель \IMM допускает сценарий исполнений 
с результатом $a=b=1$ для программы \ref{ex:LB-nodep}, 
но не для \ref{ex:LB-fakedep} и \ref{ex:LB-dep}.
Соответствующий этому сценарию граф для 
программы \ref{ex:LB-nodep} показан на~\cref{fig:LB-nodep-ppo-exec},
а для для \ref{ex:LB-fakedep} и \ref{ex:LB-dep} на~\cref{fig:LB-dep-ppo-exec}.
Заметим, что в графе, показанном на~\cref{fig:LB-nodep-ppo-exec}, 
события в левом потоке не связаны отношением $\lPPO$.
В графе, показанном на~~\cref{fig:LB-nodep-ppo-exec}, напротив, 
события в обоих потоках связанны отношением $\lPPO$,
так как между соответствющими инструкциями 
есть \emph{зависимость по данным}.
Кроме того, объединение отношений $\lPPO$ и $\lRFE$ образует цикл. 
Именно из-за наличия этого цикла данный 
граф считается неконсистентным с точки зрения модели~\IMM.

{
\newcommand{\XScale}{1}
\newcommand{\YScale}{0.7}

\begin{figure}[b]
  \begin{subfigure}[b]{.44\textwidth}\centering
  \begin{tikzpicture}[xscale=\XScale,yscale=\YScale]

  %% \node at (0,1.5) {$\circledb{A}$};

  \node (init) at (1,  1.5) {$\Init$};
  \node (i11) at ( 0,  0) {$\rlab{}{x}{1}{}$};
  \node (i12) at ( 0, -2) {$\wlab{}{y}{1}{}$};
  \node (i21) at ( 2,  0) {$\rlab{}{y}{1}{}$};
  \node (i22) at ( 2, -2) {$\wlab{}{x}{1}{}$};

  \draw[po] (i11) edge node[right] {\small$\lPO$} (i12);
  \draw[po] (i21) edge node[left ] {\small$\lPO$} (i22);
  \draw[ppo,bend left=10] (i21) edge node[right] {\small$\lPPO$} (i22);

  \draw[rf] (i22) edge node[below,pos=0.5] {}             (i11);
  \draw[rf] (i12) edge node[below,pos=0.5] {\small$\lRF$} (i21);

  \draw[po] (init) edge node[left]  {\small$\lPO$} (i11);
  \draw[po] (init) edge node[right] {\small$\lPO$} (i21);
  \end{tikzpicture}
  \caption{Граф соответствующий программе \ref{ex:lb-nodep}.}
  \label{fig:LB-nodep-ppo-exec}
  \end{subfigure}\hfill
  %
  \begin{subfigure}[b]{.55\textwidth}\centering
  \begin{tikzpicture}[xscale=\XScale,yscale=\YScale]

  %% \node at (0,1.5) {$\circledb{B}$};

  \node (init) at (1,  1.5) {$\Init$};
  \node (i11) at ( 0,  0) {$\rlab{}{x}{1}{}$};
  \node (i12) at ( 0, -2) {$\wlab{}{y}{1}{}$};
  \node (i21) at ( 2,  0) {$\rlab{}{y}{1}{}$};
  \node (i22) at ( 2, -2) {$\wlab{}{x}{1}{}$};

  \draw[po] (i11) edge node[right] {\small$\lPO$} (i12);
  \draw[po] (i21) edge node[left ] {\small$\lPO$} (i22);
  \draw[ppo,bend right=10] (i11) edge node[left ] {\small$\lPPO$} (i12);
  \draw[ppo,bend left =10] (i21) edge node[right] {\small$\lPPO$} (i22);

  \draw[rf] (i22)  edge node[below]         {}             (i11);
  \draw[rf] (i12)  edge node[below,pos=0.5] {\small$\lRF$} (i21);

  \draw[po] (init) edge node[left]  {\small$\lPO$} (i11);
  \draw[po] (init) edge node[right] {\small$\lPO$} (i21);
  \end{tikzpicture}
  \caption{Граф соответствующий программам 
    \ref{ex:lb-fakedep}~и~\ref{ex:lb-dep}.
  }
  \label{fig:LB-dep-ppo-exec}
  \end{subfigure}

\caption{Графы сценариев исполнения обосновывающие результат ${a=b=1}$.}
\label{fig:LB-ppo-execs}
\end{figure}
}


Далее приводится формальное определение понятия 
зависимостей и модели \IMM.

\begin{definition}
  \label{def:imm-exec-graph}
  Графом сценария исполнения в модели \IMM называется 
  обычный граф сценария исполнения, дополненный отношениями 
  \emph{зависимости по данным} (\emph{data dependency}) $\lDATA$, 
  \emph{зависимости по потоку управления} (\emph{control dependency}) $\lCTRL$, 
  и \emph{зависимости по целевому адресу} (\emph{address dependency}) $\lADDR$, 
  и \emph{зависимость по операции \CAS} (\emph{\CAS dependency}) $\lRMWDEP$.
  Объединенное отношение \emph{зависимости} (\emph{dependency}) 
  определяется следующим образом: 
  $$ \lDEPS \defeq \lDATA \cup \lCTRL \cup \lADDR \seqc \lPO^? \cup \lRMWDEP. $$
  В контексте модели \IMM под графом сценария исполнения будем 
  подразумевать граф, дополненный отношениями зависимости. 
\end{definition}

\begin{definition}
  \label{def:imm-aux-rel}
  В модели \IMM для графа $G$ вводятся следующием производные отношения%
  \footnote{Подробное описание приведенных здесь отношений может 
   быть найдено в~\cite{Podkopaev-al:POPL19,Moiseenko-al:ECOOP20}}.

  \begin{itemize}

    \item Отношение \emph{порядка барьеров} (\emph{barrier-order-before}):
      $$ \lBOB \defeq \lPO \seqc [\lW^{\rel\squq}] \cup 
                      [\lR^{\acq\squq}] \seqc \lPO \cup 
                      \lPO \seqc [\lF] \cup [\lF] \seqc \lPO \cup 
                      [\lW^{\rel\squq}] \seqc \lPO_{\lLOC} \seqc [\lW]. $$

    \item Отношение \emph{сохраняемого программного порядка} 
      (\emph{preserved program order}):
      $$ \lPPO \defeq [\lR] \seqc (\lDEPS \cup \lRFI)^+ \seqc [W] $$

    \item Отношение \emph{обхода} (\emph{detour}):
      $$ \lDETOUR \defeq (\lCOE \seqc \lRFE) \cap \lPO. $$

    \item Отношение \emph{синхронизируется-с} (\emph{synchronizes-with}):
     $$ \lSW  \defeq [\lE^{\rel\squq}]             \seqc 
                     ([\lF] \seqc \lPO)^?           \seqc 
                     ([\lW] \seqc \lPO_{\lLOC})^?   \seqc
                     (\lRF \seqc \lRMW)^*           \seqc 
                     \lRF \seqc (\lPO \seqc [\lF])^? \seqc 
                     [\lE^{\acq\squq}]. 
     $$

    \item Отношение \emph{произошло-до} (\emph{happens-before}):
      $$ \lHB \defeq (\lPO \cup \lSW)^+. $$

    \item Отношение \emph{читает-до} (\emph{reads-before} или \emph{from-reads}):
      $$ \lFR \defeq \lRF^{-1} \seqc \lCO. $$

    \item \emph{Расширенный порядок когерентности} 
      (\emph{extended coherence order}):
      $$ \lECO \defeq (\lCO \cup \lRF \cup \lFR)^+. $$

    \item Отношение \emph{последовательно-упорядочен-до}
      (\emph{sequentially consistent before}):
      $$ \lSCB \defeq \lPO \cup
                      \lPO\rst{\neq \lLOC} \seqc \lHB \seqc 
                      \lPO\rst{\neq \lLOC} \cup
                      \lHB\rst{\lLOC} \cup
                      \lCO \cup \lFR. $$

    \item Частиный порядок \emph{базовой последовательной упорядоченности}:
      $$ \lPSCB \defeq ([\lE^\sco] \cup [\lF^\sco] \seqc \lHB^?) \seqc 
                         \lSCB \seqc 
                       ([\lE^\sco] \cup \lHB^?\seqc[\lF^\sco]). 
      $$ 

    \item Частиный порядок \emph{последовательной упорядоченности барьеров}:
      $$ \lPSCF \defeq [\lF^\sco] \seqc 
                       (\lHB \cup \lHB \seqc \lECO \seqc \lHB) \seqc 
                       [\lF^\sco]. 
      $$ 

    \item Частиный порядок \emph{последовательной упорядоченности}:
      $$ \lPSC \defeq \lPSCB \cup \lPSCF. $$ 

    \item Вспомогательное \emph{ацикличное} отношение (\emph{acyclic relation}):
      $$ \lAR \defeq \lRFE \cup \lBOB \cup \lPPO \cup \lDETOUR \cup \lPSCF. $$
    
  \end{itemize}

\end{definition}

\begin{definition}
  \label{def:imm-cons}
  Граф $G$ является консистентным с точки зрения \IMM 
  если выполняются следующие условия:
  
  \begin{itemize}

    \item $\lR \suq \cod{\lRF}$;
      \labelAxiom{$\lRF$-completeness}{ax:rf-complete}

    \item $\lAR$ ациклично;
      \labelAxiom{imm-no-thin-air}{ax:imm-noota}

    \item $\lHB_{\IMM} \seqc \lECO^?$ иррефлексивно;
      \labelAxiom{imm-coherent}{ax:imm-coh}

    \item $\lRMW \cap (\lFR \seqc \lCO) = \emptyset$;
      \labelAxiom{rmw-atomic}{ax:imm-atom}

    \item $\lPSC$ ациклично.
      \labelAxiom{imm-sequential-consistency}{ax:imm-sc}

  \end{itemize}
\end{definition}
