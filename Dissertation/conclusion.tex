\chapter*{Заключение}                       % Заголовок
\addcontentsline{toc}{chapter}{Заключение}  % Добавляем его в оглавление

%% Согласно ГОСТ Р 7.0.11-2011:
%% 5.3.3 В заключении диссертации излагают итоги выполненного исследования, рекомендации, перспективы дальнейшей разработки темы.
%% 9.2.3 В заключении автореферата диссертации излагают итоги данного исследования, рекомендации и перспективы дальнейшей разработки темы.
%% Поэтому имеет смысл сделать эту часть общей и загрузить из одного файла в автореферат и в диссертацию:

В ходе работы были достигнуты результаты, перечисленные ниже.

\begin{itemize}

  \item Формализована в системе \coq модель \Wkm, 
    а также доказательство корректности компиляции
    из данной модели в модели памяти современных мультипроцессоров.

  \item Предложена новая версия модели \Wkm --- \WkmS.
    Доказано сохранение основных свойств \Wkm для модели \WkmS:
    корректности компиляции, корректности локальных трансформаций программ, 
    теорема о свободе от гонок.

  \item Разработан и апробирован новый инструмент проверки моделей \wmc для модели \WkmS.

  \item Формализована в системе \coq классическая теория
    простых структур событий. Выделен фрагмент модели памяти \Wkm, 
    укладывающийся в классическую теорию структур событий.  

\end{itemize}

Можно выделить следующие потенциальные
\textbf{направления дальнейшей разработки тематики}
данного исследования.

%% Во-первых, интересной задачей представляется формализация
%% в системе \coq помимо простых структур событий
%% также других классов структур событий, например,
%% класса стабильных структур событий~\cite{}.
%% Данный класс структур событий представляет интерес, в частности,
%% в контексте задачи задания формальной \emph{денотационной} cемантики
%% многопоточных программ в слабых моделей памяти.

Во-первых, интересной задачей представляется
дальнейшее изучение свойств модели \WkmS.
В частности, было бы интересно узнать,
оказываются ли полезными новые свойства \WkmS
(свобода от буферезации операций чтения и локальности сертификациии)
при разработке других методов верификации многопоточных программ,
например, программной логики для дедуктивной верификации программ,
по аналогии с тем, как эти свойства оказываются полезны
при реализации алгоритма проверки моделей. 

Во-вторых, необходимо обобщить теорию, представленную в главе \ref{ch:porf-evenstruct}
для случая моделей памяти сохраняющих синтаксические зависимости.
Также было бы крайне полезно разработать вариант модели \Wkm,
полностью укладывающийся в классическую теорию простых структур событий. 
Достичь этого можно ослабив определение причинно-следственной связи
событий в модели \Wkm таким образом, чтобы удовлетворить
ограничению наследственности отношения конфликта.
Выполнение данных задач позволит создать единообразную
теорию простых структур событий, покрывающую
все три класса слабых моделей памяти.

%% Основные результаты работы заключаются в следующем.
%% \input{common/concl}
%% И какая-нибудь заключающая фраза.

%% Последний параграф может включать благодарности.  В заключение автор
%% выражает благодарность и большую признательность научному руководителю
%% Иванову~И.\,И. за поддержку, помощь, обсуждение результатов и~научное
%% руководство. Также автор благодарит Сидорова~А.\,А. и~Петрова~Б.\,Б.
%% за помощь в~работе с~образцами, Рабиновича~В.\,В. за предоставленные
%% образцы и~обсуждение результатов, Занудятину~Г.\,Г. и авторов шаблона
%% *Russian-Phd-LaTeX-Dissertation-Template* за~помощь в оформлении
%% диссертации. Автор также благодарит много разных людей
%% и~всех, кто сделал настоящую работу автора возможной.
