\chapter{Модель \Wkm и корректность компиляции}
\label{ch:weakestmo-imm}

Напоминание о модели \Wkm и проблемах, которые она призвана решить.
Описание формализации модели \Wkm в системе \coq.

\section{Теорема о корректности компиляции}

Постановка задачи о корректности компиляции
из модели \Wkm в модели современных мультипроцессоров. Мотивация.
Формулировка теоремы о корректности компиляции. 

\subsection{Схема доказательства}

Верхнеуровневое описание схемы доказательства этой теоремы,
которое будет раскрыто более подробно в последующих разделах.
Сведение задачи к доказательству корректности компиляции в модель \IMM. 

\section{Модель \IMM}

Краткое описание модели \IMM. 

\subsection{Обход графа сценария исполнения}

Описание процесса обхода графов в модели \IMM
и его формальная операционная семантика. 

\section{Симуляция обхода графа \IMM}

Описание симуляции обхода графа \IMM путем построения структуры событий \Wkm.
Формальное определение отношения симуляции. 
