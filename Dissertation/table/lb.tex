\begin{table}
\centering

\newcommand{\execs}{Уникальные}
\newcommand{\dupls}{Повторяющиеся}
\newcommand{\blckd}{Блокированные}
\newcommand{\pcent}{\%\!\!\!\!}
%% \resizebox{.4\textwidth}{!}{
\begin{tabular}{@{}l@{\sep}S[table-format=4]@{\hsep}S[table-format=2.1]@{\hsep}S[table-format=2.1]@{}}

\toprule

~ & \multicolumn{1}{c}{\execs} & \multicolumn{1}{c}{\dupls} & \multicolumn{1}{c}{\blckd} \\

\midrule
%
LB+ctrl(10)   &    11 & 0 & 0  \\
LB+ctrl(12)   &    13 & 0 & 0  \\
LB+ctrl(14)   &    15 & 0 & 0  \\
%
LB+data(10)   &  1024 & 0 & 0  \\
LB+data(12)   &  4096 & 0 & 0  \\
LB+data(14)   & 16384 & 0 & 0  \\
%
LB-nodep(10)   &   1024 &   0.9 \pcent &  0.9 \pcent \\
LB-nodep(12)   &   4096 &   0.3 \pcent &  0.3 \pcent \\
LB-nodep(14)   &  16384 &   0.1 \pcent &  0.1 \pcent \\
%
LB-pairs(10)   &   1024 & 205.2 \pcent & 33.3 \pcent \\
LB-pairs(12)   &   4096 & 281.5 \pcent & 33.3 \pcent \\
LB-pairs(14)   &  16384 & 376.8 \pcent & 33.3 \pcent \\

\bottomrule
\end{tabular}%
%% }

\captionsetup{justification=centering}
\caption{Количество исследуемых \wmc сценариев исполнения для синтетических программ}
\label{tab:lb}
\end{table}
