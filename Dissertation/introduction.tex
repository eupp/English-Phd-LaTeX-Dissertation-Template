\chapter*{Введение}                         % Заголовок
\addcontentsline{toc}{chapter}{Введение}    % Добавляем его в оглавление

\newcommand{\actuality}{\textbf{\actualityTXT}}
\newcommand{\progress}{\textbf{\progressTXT}}
\newcommand{\aim}{{\textbf\aimTXT}}
\newcommand{\tasks}{\textbf{\tasksTXT}}
\newcommand{\novelty}{\textbf{\noveltyTXT}}
\newcommand{\influence}{\textbf{\influenceTXT}}
\newcommand{\methods}{\textbf{\methodsTXT}}
\newcommand{\defpositions}{\textbf{\defpositionsTXT}}
\newcommand{\reliability}{\textbf{\reliabilityTXT}}
\newcommand{\probation}{\textbf{\probationTXT}}
\newcommand{\contribution}{\textbf{\contributionTXT}}
\newcommand{\publications}{\textbf{\publicationsTXT}}


% {\actuality} Обзор, введение в тему, обозначение места данной работы в
% мировых исследованиях и~т.\:п., можно использовать ссылки на~другие
% работы~\autocite{Gosele1999161,Lermontov}
% (если их~нет, то~в~автореферате
% автоматически пропадёт раздел <<Список литературы>>). Внимание! Ссылки
% на~другие работы в~разделе общей характеристики работы можно
% использовать только при использовании \verb!biblatex! (из-за технических
% ограничений \verb!bibtex8!. Это связано с тем, что одна
% и~та~же~характеристика используются и~в~тексте диссертации, и в
% автореферате. В~последнем, согласно ГОСТ, должен присутствовать список
% работ автора по~теме диссертации, а~\verb!bibtex8! не~умеет выводить в~одном
% файле два списка литературы).
% При использовании \verb!biblatex! возможно использование исключительно
% в~автореферате подстрочных ссылок
% для других работ командой \verb!\autocite!, а~также цитирование
% собственных работ командой \verb!\cite!. Для этого в~файле
% \verb!common/setup.tex! необходимо присвоить положительное значение
% счётчику \verb!\setcounter{usefootcite}{1}!.

% Для генерации содержимого титульного листа автореферата, диссертации
% и~презентации используются данные из файла \verb!common/data.tex!. Если,
% например, вы меняете название диссертации, то оно автоматически
% появится в~итоговых файлах после очередного запуска \LaTeX. Согласно
% ГОСТ 7.0.11-2011 <<5.1.1 Титульный лист является первой страницей
% диссертации, служит источником информации, необходимой для обработки и
% поиска документа>>. Наличие логотипа организации на~титульном листе
% упрощает обработку и~поиск, для этого разметите логотип вашей
% организации в папке images в~формате PDF (лучше найти его в векторном
% варианте, чтобы он хорошо смотрелся при печати) под именем
% \verb!logo.pdf!. Настроить размер изображения с логотипом можно
% в~соответствующих местах файлов \verb!title.tex!  отдельно для
% диссертации и автореферата. Если вам логотип не~нужен, то просто
% удалите файл с~логотипом.

% \ifsynopsis
% Этот абзац появляется только в~автореферате.
% Для формирования блоков, которые будут обрабатываться только в~автореферате,
% заведена проверка условия \verb!\!\verb!ifsynopsis!.
% Значение условия задаётся в~основном файле документа (\verb!synopsis.tex! для
% автореферата).
% \else
% Этот абзац появляется только в~диссертации.
% Через проверку условия \verb!\!\verb!ifsynopsis!, задаваемого в~основном файле
% документа (\verb!dissertation.tex! для диссертации), можно сделать новую
% команду, обеспечивающую появление цитаты в~диссертации, но~не~в~автореферате.
% \fi

% {\progress}
% Этот раздел должен быть отдельным структурным элементом по
% ГОСТ, но он, как правило, включается в описание актуальности
% темы. Нужен он отдельным структурынм элемементом или нет ---
% смотрите другие диссертации вашего совета, скорее всего не нужен.

{\actuality} 
В современном мире многопоточные программные системы распространены повсеместно. 
Разработка и тестирование программного обеспечения для таких систем на порядок 
сложнее и существенно более трудозатратно, чем для последовательных систем. 
По этой причине крайне актуальной является задача 
верификации многопоточных программ.
Решение этой задачи в свою очередь требует наличия 
строгой математической спецификации семантики многопоточных программ.

Формальная семантика многопоточных программ, потоки которых работают 
с разделяемой памятью, называется \emph{моделью памяти}. 
Главной целью модели памяти является задание множества 
допустимых \emph{сценариев исполнения} программы.
Современные многопоточные системы и языки программирования 
вследствие применения компиляторами и процессорами
различных оптимизаций при сборке и выполнении программ
допускают так называемые \emph{слабые сценарии исполнения},
то есть такие сценарии, которые не могут быть получены в результате 
простого поочередного исполнения инструкций различными потоками. 
\emph{Слабые модели памяти} призваны описать множество 
допустимых слабых сценариев исполнения программы. 
Оказывается, что вопрос какие именно слабые сценарии поведения 
следует допускать, а какие нет, не является однозначным 
и зависит от требований к самой многопоточной системе 
или языку программирования. 
По этой причине в последние годы появилось (и продолжает появляться) 
множество различных моделей памяти для современных мультипроцессоров, например, \Intel~\autocite{Sewell-al:CACM10}, 
\ARM~\autocite{Pulte-al:POPL18}, 
\POWER~\autocite{Sarkar-al:PLDI11}
языков программирования, например, 
\CPP~\autocite{Batty-al:POPL11},
\Java~\autocite{Manson-al:POPL05}, 
\JS~\autocite{Watt-al:PLDI2020}, 
\OCaml~\autocite{Dolan-al:PLDI18},
а также распределенных систем%
~\autocite{Jagadeesan-al:ESOP2018,Lahav-Boker:PLDI2020}.
В связи с этим встает задача формализации 
существующих моделей памяти и создание теории
для разработки будущих моделей памяти.


Одним из способов задания моделей памяти
является использование семантических доменов
\emph{истинной конкурентности} (\emph{true concurrency semantics}).
Этот класс моделей позволяет выразить независимость (параллельность) атомарных событий, а также 
причинно-следственные связи между ними,
что ведет к более компактному представлению пространства состояний программы.
Всё это упрощает рассуждения 
о поведении многопоточных программ как для человека, 
так и для программных средств при автоматической и интерактивной верификации. 

\emph{Структуры событий} (\emph{event structures}) являются одним из семантических доменов, 
относящихся к классу моделей истинной конкурентности.
В наиболее простом варианте структура событий состоит из множества атомарных событий,
функции, присваивающей каждому событию семантическую метку,
отношения причинно-следственной связи и отношения конфликта между событиями.
Классическая теория структур событий была разработана M.Nielsen, G.Plotkin и G.Winskel
для описания семантики исчисления взаимодействующих систем (Calculus of Communicating Systems, CCS).
Следует отметить, что данное исчисление является достаточно простой моделью параллельных вычислений
и не позволяет описывать слабые сценарии поведения многопоточных программ.

Для описания слабых моделей памяти исследователями было предложено несколько формализмов,
основанных на структурах событий, в частности,
модель A.Jeffrey и J.Riely~\autocite{Jeffrey-Riely:LICS16},
модель J.Pichon-Pharabod и P. Sewell~\autocite{PichonPharabod-Sewell:POPL16},
модель \Wkm~\autocite{Chakraborty-Vafeiadis:POPL19},
модель \MRD~\autocite{Paviotti-al:ESOP20}.
Отметим, что данные модели вводят новые классы
структур событий, несовместимые с классической теорией,
что не позволяет применять известные результаты
о структурах событий к данным моделям.
Кроме того, в рамках данных моделей даже для небольших программ
вычислительно затратно перечисление возможных слабых сценариев поведения,
что препятствует разработке эффективных средств верификации многопоточных программ.

Таким образом, возникает потребность в создании формальной семантики 
многопоточных программ на основе структур событий, 
которая, с одной стороны, позволяла бы описывать слабые сценарии поведения, 
с другой стороны, допускала бы разработку  
инструментов для автоматической и интерактивной верификации. 

{\progress}

Теория структур событий была разработана M.Nielsen, G.Plotkin и G.Winskel в 1980-1990 годы%
~\autocite{Nielsen:REX93,Sassone:MFCS1993,Vaandrager:TCS1991,Winskel-TCS:09} 
для задания денотационной семантики 
исчисления параллельных взаимодействующих систем (Calculus of Communicating Systems, CCS)%
~\autocite{Winskel:ICALP1982}.
Относительно недавно эта теория также была использована 
для задания семантики пи-исчисления процессов ($\pi$-calculus)%
~\autocite{Varacca-Nobuko:TCS10,Crafa-al:FSCCS12,Hildebrandt-al:LATA2017}.
Но CCS и пи-исчисление не позволяют описывать 
слабые сценарии поведения многопоточных программ.

Теория слабых моделей памяти также активно развивалась, начиная с 1990-ых годов. 
На сегодняшний день существует множество моделей памяти, 
описывающих поведение мультипроцессоров, 
многопоточных языков программирования и распределенных систем. 
Эти модели, в свою очередь, можно разделить на несколько классов.

Модели, \emph{сохраняющие программный порядок}, образуют широкий класс,
включающий, в том числе, модель \TSO процессоров семейства \Intel~\autocite{Sewell-al:CACM10},
модели последовательной согласованности (sequential consistency)~\autocite{Lamport:TC79}
причинной согласованности (causal consistency)~\autocite{Lahav-Boker:PLDI2020},
и согласованности в конечном счёте (eventual consistency)~\autocite{Jagadeesan-al:ESOP2018},
а также модели памяти некоторых языков программирования, например,
модель памяти языка \OCaml~\autocite{Dolan-al:PLDI18}.
Общим недостатком моделей данного класса является то,
что в рамках этих моделей не поддерживаются \emph{оптимальные схемы компиляции} 
в целевой код для современных мультипроцессоров \ARM и \POWER.
Это означает, что реализация данных моделей на этих мультипроцессорах
влечет дополнительные накладные расходы и может приводить
к увелечению времени исполнения программ~\autocite{Ou-Demsky:OOPSLA18}. 

Модели памяти мультипроцессоров,
например \ARM~\autocite{Pulte-al:POPL18} и \POWER~\autocite{Sarkar-al:PLDI11}, 
как правило, принадлежат к классу моделей, \emph{сохраняющих синтаксические зависимости}. 
Основное ограничение моделей, принадлежащих к данному классу, заключается в том, 
что они не поддерживают некоторые трансформации программ, 
применяемые оптимизирующими компиляторами, например, свертку констант.
По этой причине модели данного класса, как правило,
не применяются в качестве моделей памяти для языков программирования.  

Таким образом, модели памяти двух вышеупомянутых классов 
не отвечают требованиям, предъявляемым к моделям памяти для таких языков как \CPP и \Java. 
С целью преодоления этих ограничений исследователями были предложены модели  
\Prm~\autocite{Kang-al:POPL17}, \Wkm~\autocite{Chakraborty-Vafeiadis:POPL19}, 
\MRD~\autocite{Paviotti-al:ESOP20}, \PwP~\autocite{Jagadeesan-al:OOPSLA2020}
и другие~\autocite{Jeffrey-Riely:LICS16,PichonPharabod-Sewell:POPL16},
которые обычно относят к классу моделей, \emph{сохраняющих семантические зависимости}.
Данные модели, как правило, поддерживают оптимальные схемы компиляции
для современных мультипроцессоров и поддерживают широкий спектр оптимизаций программ. 
Тем не менее, классы моделей, сохраняющих программный порядок и синтаксические зависимости, 
хорошо изучены, в то время как свойства класса моделей,
сохраняющих семантические зависимости, по-прежнему активно исследуются.
В частности, для моделей данного класса практический не исследованы
вопросы построения эффективных инструментов автоматической и интерактивной верификации. 

%% Некоторые из вышеупомянутых моделей, сохраняющих семантические зависимости,
%% основаны на теории структур событий%
%% ~\autocite{Jeffrey-Riely:LICS16,PichonPharabod-Sewell:POPL16,
%% Chakraborty-Vafeiadis:POPL19,Paviotti-al:ESOP20}.
%% Общий недостаток данных моделей заключается в том,
%% что они вводят новые классы структур событий, 
%% несовместимые с классическими определениями.
%% Это затрудняет применение уже существующей классической теории структур событий
%% для решения проблем, возникающих в теории слабых моделей памяти. 

Рассмотрим для примера язык \CPP.
Для описания модели памяти данного языка исследовательским
сообществом было выработано несколько подходов.
Модель \RCMM~\autocite{Lahav-al:PLDI17}
относится к классу моделей, сохраняющих программный порядок.
Данная модель является относительно простой и
предоставляет ряд важных и полезных свойств.
Для данной модели также были разработаны эффективные
средства верификации многопоточных программ,
например, инструмент проверки моделей \genmc~\autocite{Kokologiannakis:PLDI2019}.
Однако данная модель не поддерживает
оптимальную схему компиляции в модели мультипроцессоров \ARM и \POWER.
С другой стороны, модели \Prm и \Wkm,
относящиеся к классу моделей, сохраняющих семантические зависимости,
поддерживают оптимальную схему компиляции и широкий набор оптимизаций программ.
Но данные модели существенно более сложные, их свойства слабо изучены, и для них
не разработаны эффективные методы верификации программ. 

В контексте данной работы наибольший интерес
представляет именно модель \Wkm~\autocite{Chakraborty-Vafeiadis:POPL19},
так как она основана на теории структур событий
и для нее было формально доказано наличие ряда важных для практики свойств.
Отметим, что у данной модели тем не менее есть ряд недостаков:
данная модель не укладывается в классическую теорию структур событий,
для нее не была доказана корректность оптимальной схемы
компиляции в модели современных мультипроцессоров,
а также для данной модели не были ранее разработаны
средства верификации программ.

%% Для большей гибкости, данный язык предоставляет несколько
%% режимов доступа к разделяемым переменным (\emph{access modes}):
%% \emph{последовательно согласованный режим} (\emph{sequentiall consistent}),
%% режимы \emph{захвата и освобождения} (\emph{acquire/release}),
%% гарантирующие причинную согласованность~\autocite{Lahav-al:POPL16},
%% \emph{ослабленный режим} (\emph{relaxed}),
%% гарантирующий когерентность~\autocite{Alglave-al:TOPLAS14}
%% и \emph{неатомарный режим} для неконкурентных обращений к памяти. 

%% Для описания подмножества модели памяти \CPP
%% исследователями была разработана модель \RCMM~\autocite{Lahav-al:PLDI17}.
%% Данная модель полностью описывает все возможные сценарии
%% поведения многопоточных программ, которые
%% не используют режим ослабленных обращений.  

%% Среди слабых моделей памяти, сохраняющих семантические зависимости
%% и основанных на структурах событий, в контексте данной работы наибольший интерес
%% представляет модель \Wkm~\autocite{Chakraborty-Vafeiadis:POPL19},
%% поскольку для данной модели было формально доказано наличие ряда важных для практики свойств.
%% В частности, для этой модели была доказана корректность
%% локальных трансформаций программ и теорема о свободе от гонок.
%% Тем не менее отметим, что корректность оптимальной схемы
%% компиляции из модели \Wkm в модели современных мультипроцессоров
%% \emph{не была ранее доказана}, что является существенным недостатком,
%% так как наличие данного свойства является одним из базовых требований,
%% предъявляемых к классу моделей памяти, сохраняющих семантические зависимости.

{\aim} данной работы является адаптация теории структур событий
для описания слабых моделей памяти и разработка на основе этих исследований 
инструментов для автоматической и интерактивной верификации многопоточных программ. 

Для достижения данной цели были сформулированы следующие {\tasks}.
\begin{enumerate}[beginpenalty=10000] % https://tex.stackexchange.com/a/476052/104425
  \item
    Формализовать в системе для интерактивного доказательства теорем \coq
    классическую теорию структур событий. Показать, что
    в данную теорию укладывается класс моделей памяти,
    сохраняющих программный порядок и, в частности, модель \RCMM.
  \item
    Формализовать в системе для интерактивного доказательства теорем \coq
    теорию структур событий модели \Wkm.
    Доказать корректность оптимальной схемы компиляции
    из модели \Wkm в модели памяти современных мультипроцессоров.
  \item
    Разработать строгую версию модели \Wkm, 
    допускающую реализацию эффективных инструментов автоматической верификации
    и доказать, что для неё сохраняются основные свойства \Wkm  
    (в частности, корректность компиляции, корректность локальных трансформаций программ, 
     теорема о свободе от гонок).
  \item
    Разработать алгоритм проверки моделей (model~checking) для предложенной модели.
\end{enumerate}

~\newline

{\methods}

Диссертационное исследование базируется на теории формальных семантик. 
Используются классические и хорошо изученные формализмы, в частности, 
системы помеченных переходов, языки помеченных частично упорядоченных мультимножеств и структуры событий. 

Для формализации некоторых теорем и доказательств, представленных в данной работе, 
использовалась система интерактивного доказательства теорем \coq 
и библиотека формализованных математических теорий \mathcomp.

%% При разработке алгоритма проверки моделей использовались техника \emph{редукции частичных порядков}.
%% Предложенный алгоритм был внедрен в систему \genmc --- 
%% инструмент для автоматической верификации многопоточных программ написанных на языке \CLANG.

{\defpositions}
\begin{enumerate}[beginpenalty=10000] % https://tex.stackexchange.com/a/476052/104425
  \item Предложена формальная семантика на основе классической теории структур событий, 
    покрывающая класс слабых моделей памяти, сохраняющих программный порядок;
    данная семантика формализована в системе \coq.
  \item Доказана корректность оптимальной схемы компиляции из модели \Wkm
    в модели современных мультипроцессоров \TSO, \ARM и \POWER;
    модель \Wkm и доказательство теоремы о корректности компиляции
    формализованы в системе \coq.
  \item Предложена модель \WkmS, расширяющая модель \Wkm 
  новыми свойствами \emph{свободы от буферизации
  операций чтения} и \emph{локальности сертификациии}, 
  которые позволяют проводить эффективную верификацию программ в данной модели;
  также доказано сохранение основных свойств модели \Wkm: корректности компиляции,
  свойства корректности локальных трансформаций программ,
  теоремы о свободе от гонок.
  \item Для модели \WkmS разработан алгоритм автоматической 
    верификации программ методом проверки моделей; 
    в ряде экспериментов показана лучшая эффективность 
    данного алгоритма по сравнению с аналогами.
\end{enumerate}
% В папке Documents можно ознакомиться с решением совета из Томского~ГУ
% (в~файле \verb+Def_positions.pdf+), где обоснованно даются рекомендации
% по~формулировкам защищаемых положений.

{\novelty}
\begin{enumerate}[beginpenalty=10000] % https://tex.stackexchange.com/a/476052/104425

  \item Впервые предложена семантика, основанная на классической теории структур событий,
    которая покрывает класс слабых моделей памяти, сохраняющих программный порядок.
    %% что позволяет применить известные теоретические результаты 
    %% о структурах событий к данному классу моделей.

  \item Впервые доказана корректность оптимальной схемы компиляции
    из модели памяти, основанной на структурах событий (\Wkm), 
    в модели памяти современных мультипроцессоров.

  \item Впервые предложена модель памяти (\WkmS),
    принадлежащая к классу моделей, сохраняющих семантические зависимости, 
    и при этом допускающая реализацию эффективных методов автоматической верификации программ. 

  \item Разработан новый алгоритм проверки моделей для \WkmS,
    который является существенно более эффективным по сравнению с другими алгоритмами
    (\CDSChecker~\autocite{Norris-Demsky:OOPSLA2013}, \rmem~\autocite{Pulte-al:PLDI2019}),
    которые поддерживают класс моделей памяти, сохраняющих семантические зависимости.

\end{enumerate}

{\influence} 

Новая семантика на основе теории структур событий 
для класса слабых моделей памяти, сохраняющих программный порядок,
соединяет классическую теорию структур событий.
%% с теорией слабых моделей памяти и позволяет применить известные результаты 
%% о структурах событий в новой предметной области.  
Формализация этой семантики в системе \coq открывает 
путь к дальнейшей разработке инструментов для  
интерактивной верификации многоточных программ  
с учетом слабых сценариев исполнения. 
 
Новые свойства предложенной модели \WkmS ---
свобода от буферизации операций чтения (load buffering race freedom)
и локальности сертификации (certification locality), --- 
также могут быть добавлены в другие модели памяти 
с целью разработки методов автоматической верификации программ в этих моделях. 
Наличие данных свойств позволяет оптимизировать алгоритм 
проверки моделей и таким образом существенно увеличить его эффективность.

Предложенный  алгоритм проверки моделей может быть использован на практике
для отладки и верификации многопоточных алгоритмов и структур данных 
с учетом слабых сценариев исполнения, допустимых стандартом языка \CLANG. 

{\reliability} полученных результатов обеспечивается 
формальными доказательствами, разработанными в том числе с использованием
систем интерактивного доказательства теорем, 
а также инженерными экспериментами. 
Результаты находятся в соответствии с результатами, полученными другими авторами.

{\probation}
Основные результаты работы докладывались~на
следующих научных конференциях и семинарах:
Surrey Concurrency Workshop (23-24 июля 2019, Университет Суррея, Великобритания),
The European Conference on Object-Oriented Programming
(ECOOP, 15-17 ноября 2020, Берлин, Германия, онлайн конференция),
Spring/Summer Young Researchers' Colloquium on Software Engineering
(27-28 мая 2021, Москва, Россия),
внутренние семинары JetBrains Research
(18 ноября 2018, 13 апреля 2020, Санкт-Петербург, Россия). \\
\fixme{добавить будущие мероприятия по мере проведения}.

% {\contribution} Автор принимал активное участие \ldots

\ifnumequal{\value{bibliosel}}{0}
{%%% Встроенная реализация с загрузкой файла через движок bibtex8. (При желании, внутри можно использовать обычные ссылки, наподобие `\cite{vakbib1,vakbib2}`).
    {\publications} Основные результаты по теме диссертации изложены
    в~XX~печатных изданиях,
    X из которых изданы в журналах, рекомендованных ВАК,
    X "--- в тезисах докладов.
}%
{%%% Реализация пакетом biblatex через движок biber
    \begin{refsection}[bl-author, bl-registered]
        % Это refsection=1.
        % Процитированные здесь работы:
        %  * подсчитываются, для автоматического составления фразы "Основные результаты ..."
        %  * попадают в авторскую библиографию, при usefootcite==0 и стиле `\insertbiblioauthor` или `\insertbiblioauthorgrouped`
        %  * нумеруются там в зависимости от порядка команд `\printbibliography` в этом разделе.
        %  * при использовании `\insertbiblioauthorgrouped`, порядок команд `\printbibliography` в нём должен быть тем же (см. biblio/biblatex.tex)
        %
        % Невидимый библиографический список для подсчёта количества публикаций:
        \printbibliography[heading=nobibheading, section=1, env=countauthorvak,          keyword=biblioauthorvak]%
        \printbibliography[heading=nobibheading, section=1, env=countauthorwos,          keyword=biblioauthorwos]%
        \printbibliography[heading=nobibheading, section=1, env=countauthorscopus,       keyword=biblioauthorscopus]%
        \printbibliography[heading=nobibheading, section=1, env=countauthorconf,         keyword=biblioauthorconf]%
        \printbibliography[heading=nobibheading, section=1, env=countauthorother,        keyword=biblioauthorother]%
        \printbibliography[heading=nobibheading, section=1, env=countregistered,         keyword=biblioregistered]%
        \printbibliography[heading=nobibheading, section=1, env=countauthorpatent,       keyword=biblioauthorpatent]%
        \printbibliography[heading=nobibheading, section=1, env=countauthorprogram,      keyword=biblioauthorprogram]%
        \printbibliography[heading=nobibheading, section=1, env=countauthor,             keyword=biblioauthor]%
        \printbibliography[heading=nobibheading, section=1, env=countauthorvakscopuswos, filter=vakscopuswos]%
        \printbibliography[heading=nobibheading, section=1, env=countauthorscopuswos,    filter=scopuswos]%
        %
        \nocite{*}%
        %
        {\publications} Основные результаты по теме диссертации изложены в~\arabic{citeauthor}~печатных изданиях,
        \arabic{citeauthorvak} из которых изданы в журналах, рекомендованных ВАК\sloppy%
        \ifnum \value{citeauthorscopuswos}>0%
            , \arabic{citeauthorscopuswos} "--- в~периодических научных журналах, индексируемых Web of~Science и Scopus\sloppy%
        \fi%
        \ifnum \value{citeauthorconf}>0%
            , \arabic{citeauthorconf} "--- в~тезисах докладов.
        \else%
            .
        \fi%
        \ifnum \value{citeregistered}=1%
            \ifnum \value{citeauthorpatent}=1%
                Зарегистрирован \arabic{citeauthorpatent} патент.
            \fi%
            \ifnum \value{citeauthorprogram}=1%
                Зарегистрирована \arabic{citeauthorprogram} программа для ЭВМ.
            \fi%
        \fi%
        \ifnum \value{citeregistered}>1%
            Зарегистрированы\ %
            \ifnum \value{citeauthorpatent}>0%
            \formbytotal{citeauthorpatent}{патент}{}{а}{}\sloppy%
            \ifnum \value{citeauthorprogram}=0 . \else \ и~\fi%
            \fi%
            \ifnum \value{citeauthorprogram}>0%
            \formbytotal{citeauthorprogram}{программ}{а}{ы}{} для ЭВМ.
            \fi%
        \fi%
        % К публикациям, в которых излагаются основные научные результаты диссертации на соискание учёной
        % степени, в рецензируемых изданиях приравниваются патенты на изобретения, патенты (свидетельства) на
        % полезную модель, патенты на промышленный образец, патенты на селекционные достижения, свидетельства
        % на программу для электронных вычислительных машин, базу данных, топологию интегральных микросхем,
        % зарегистрированные в установленном порядке.(в ред. Постановления Правительства РФ от 21.04.2016 N 335)
    \end{refsection}%
    \begin{refsection}[bl-author, bl-registered]
        % Это refsection=2.
        % Процитированные здесь работы:
        %  * попадают в авторскую библиографию, при usefootcite==0 и стиле `\insertbiblioauthorimportant`.
        %  * ни на что не влияют в противном случае
        \nocite{Moiseenko-al:OOPSLA22}
        \nocite{Moiseenko-al:ECOOP20}
        \nocite{Moiseenko-al:STJITMO22}
        \nocite{Moiseenko-al:PCS21}
        \nocite{Gladstein-al:ISPRAS21}
        % \nocite{vakbib2}%vak
        % \nocite{patbib1}%patent
        % \nocite{progbib1}%program
        % \nocite{bib1}%other
        % \nocite{confbib1}%conf
    \end{refsection}%
        %
        % Всё, что вне этих двух refsection, это refsection=0,
        %  * для диссертации - это нормальные ссылки, попадающие в обычную библиографию
        %  * для автореферата:
        %     * при usefootcite==0, ссылка корректно сработает только для источника из `external.bib`. Для своих работ --- напечатает "[0]" (и даже Warning не вылезет).
        %     * при usefootcite==1, ссылка сработает нормально. В авторской библиографии будут только процитированные в refsection=0 работы.
}

% При использовании пакета \verb!biblatex! будут подсчитаны все работы, добавленные
% в файл \verb!biblio/author.bib!. Для правильного подсчёта работ в~различных
% системах цитирования требуется использовать поля:
% \begin{itemize}
%         \item \texttt{authorvak} если публикация индексирована ВАК,
%         \item \texttt{authorscopus} если публикация индексирована Scopus,
%         \item \texttt{authorwos} если публикация индексирована Web of Science,
%         \item \texttt{authorconf} для докладов конференций,
%         \item \texttt{authorpatent} для патентов,
%         \item \texttt{authorprogram} для зарегистрированных программ для ЭВМ,
%         \item \texttt{authorother} для других публикаций.
% \end{itemize}
% Для подсчёта используются счётчики:
% \begin{itemize}
%         \item \texttt{citeauthorvak} для работ, индексируемых ВАК,
%         \item \texttt{citeauthorscopus} для работ, индексируемых Scopus,
%         \item \texttt{citeauthorwos} для работ, индексируемых Web of Science,
%         \item \texttt{citeauthorvakscopuswos} для работ, индексируемых одной из трёх баз,
%         \item \texttt{citeauthorscopuswos} для работ, индексируемых Scopus или Web of~Science,
%         \item \texttt{citeauthorconf} для докладов на конференциях,
%         \item \texttt{citeauthorother} для остальных работ,
%         \item \texttt{citeauthorpatent} для патентов,
%         \item \texttt{citeauthorprogram} для зарегистрированных программ для ЭВМ,
%         \item \texttt{citeauthor} для суммарного количества работ.
% \end{itemize}

% Счётчик \texttt{citeexternal} используется для подсчёта процитированных публикаций;
% \texttt{citeregistered} "--- для подсчёта суммарного количества патентов и программ для ЭВМ.

% Для добавления в список публикаций автора работ, которые не были процитированы в
% автореферате, требуется их~перечислить с использованием команды \verb!\nocite! в
% \verb!Synopsis/content.tex!.

Личный вклад автора в публикациях, выполненных с соавторами, распределён следующим образом.
В работе \cite{Gladstein-al:ISPRAS21} автор предложил
метод кодирования семантики параллельной регистровой машины с
моделью памяти, сохраняющей программный порядок, в терминах простых структур событий;
соавторы участвовали в формализации данного метода в системе \coq.
В работе \cite{Moiseenko-al:STJITMO22} автор предложил
метод кодирования семантического домена языков частично упорядоченных мультимножеств
с использованием фактор-типов, 
соавторы участвовали в обсуждении данного метода и его формализации в системе \coq.
В работе \cite{Moiseenko-al:PCS21} автор выполнил сбор и анализ данных
о существующих моделях памяти языков программирования;
соавторы участвовали в формулировке выводов данного анализа.
В работе \cite{Moiseenko-al:ECOOP20} автор выполнил
формализацию доказательства корректности компиляции из
модели \Wkm в модели современных мультипроцессоров;
соавторы участвовали в обсуждении данного доказательства
и его формализации в системе \coq.
В работе \cite{Moiseenko-al:OOPSLA22} автор
формализовал новые свойства модели \WkmS,
а именно, свойства свободы от буферизации операций чтения и локальности сертификации,
доказал сохранение полезных свойств модели \Wkm в \WkmS,
а также разработал прототип алгоритма проверки моделей для \WkmS;
соавторы участвовали в обсуждении формализации модели \WkmS
и доказательстве ее свойств,
оказывали помощь при реализации нового алгоритма,
а также провели эксперименты по сравнению нового алгоритма с аналогами.
 % Характеристика работы по структуре во введении и в автореферате не отличается (ГОСТ Р 7.0.11, пункты 5.3.1 и 9.2.1), потому её загружаем из одного и того же внешнего файла, предварительно задав форму выделения некоторым параметрам

\textbf{Объем и структура работы.} Диссертация состоит из~введения,
\formbytotal{totalchapter}{глав}{ы}{}{},
заключения и
\formbytotal{totalappendix}{приложен}{ия}{ий}{}.
%% на случай ошибок оставляю исходный кусок на месте, закомментированным
%Полный объём диссертации составляет  \ref*{TotPages}~страницу
%с~\totalfigures{}~рисунками и~\totaltables{}~таблицами. Список литературы
%содержит \total{citenum}~наименований.
%
Полный объём диссертации составляет
\formbytotal{TotPages}{страниц}{у}{ы}{}, включая
\formbytotal{totalcount@figure}{рисун}{ок}{ка}{ков} и
\formbytotal{totalcount@table}{таблиц}{у}{ы}{}.
Список литературы содержит
\formbytotal{citenum}{наименован}{ие}{ия}{ий}.
