\chapter{Обзор}
\label{ch:review}

\section{Слабые модели памяти}

\eupp{Данный подраздел будет в значительной степени основан
  на материле обзорной статьи~\cite{Moiseenko-al:PCS21}}

Краткое введение в область слабых моделей памяти, пример. 
Обоснование важности исследования этой области.  

\subsection{Требования к моделям памяти}

Основные требования предъявлемые к моделям памяти,
их краткое описание и мотивация их введения:

\begin{itemize}
  \item Корректность компиляции.
  \item Корректность трансформаций программ.
  \item Предоставляемые гарантии для рассуждения о поведении программ.
    Здесь также будет сделан акцент на инструментах
    для верификации многопоточных программ. 
\end{itemize}

\subsection{Классы моделей памяти}

Основные классы моделей памяти (по материалам обзорной статьи~\cite{Moiseenko-al:PCS21}).
Описание свойств этих классов и компромиссов в их дизайне
в соответствии с требованиями, предъявленными в предыдущем разделе. 

\subsection{Итог}

Подведение итога обзора моделей памяти.
Фрагмент сравнительной таблицы из обзорной статьи~\cite{Moiseenko-al:PCS21}.
Обозначение research gaps, которые закрывает данная диссертация, а именно:

\begin{itemize}
  \item классическая теория стуктур событий для моделей памяти
    сохраняющих программный порядок;
  \item корректность компиляции для модели \Wkm;
  \item автоматическая верификация многопоточных программ
    (model checking) в модели \WkmS.
\end{itemize}

\section{Формальная семантика параллельных программ и моделей памяти}

В этом разделе будет введен необходимый математический аппарат. 

\subsection{Языки помсетов и простые структуры событий}
\label{sec:pomsets-eventstruct}

Определение языков помсетов и класса простых структур событий.
Связанные определения и простейшие свойства. 

\subsection{Графы сценариев исполнения}

Определение аксиоматических моделей памяти и графов сценариев исполнения.

\subsection{Структуры событий в модели \Wkm}

Определение класса структур событий, использующихся в модели \Wkm.
