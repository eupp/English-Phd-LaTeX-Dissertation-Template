\subsection*{Шаг симуляции}
\label{sec:simstep}

В данном разделе приводится схема доказательства леммы \ref{lm:simstep}.
А именно, будет показано каким образом
операционная семантика построения структуры событий
симулирует шаг обхода графа \IMM.

Предположим, что для некоторых $P$, $G$, $\TC$, $S$ и $X$
выполняется отношение симуляции $\simrel(P, G, \TC, S, X)$.
Также положим, что в рамках обхода выполняется шаг
$G \vdash \TC \travstep{} \TC'$, который покрывает
или выпускает событие из потока с идентификатором $t$.
По условиям леммы \ref{lm:simstep} требуется предъявить
структуру $S'$ и конфигурацию $X'$,
такие что выполняется $\simrel(P, G, \TC', S', X')$.
Если поток $t$ содержит ещё непокрытые, но уже
выпущенные события записи, то необходимо выполнить
несколько шагов для построения из структуры $S$ структуры $S'$
чтобы добавить все события, $\lPO$-предшествующие непокрытым
событиям записи в потоке $t$.
Будем называть множество этих событий
\emph{сертификационной веткой},
а процесс добавление этих событий --- \emph{сертификацией}.

\begin{figure}[h]
$\hfill\inarr{\begin{tikzpicture}[xscale=1,yscale=1.5]
  \node (init) at (2,  1)   {$\Init$};
  \node (i11)  at (0,  0)   {$\mese{1}{1}{} \rlab{}{x}{1}$};
  \node (i12)  at (0, -1)   {$\mese{1}{2}{} \wlab{}{y}{1}$};
  \node (i13)  at (0, -2)   {$\mese{1}{3}{} \wlab{}{z}{1}$};
  \node (i21)  at (4,  0)   {$\mese{2}{1}{} \rlab{}{y}{1}$};
  \node (i22)  at (4, -1)   {$\mese{2}{2}{} \rlab{}{z}{1}$};
  \node (i23)  at (4, -2)   {$\mese{2}{3}{} \wlab{}{x}{1}$};
  %% \node (hh) at (2, -3.5) {$\inarrC{\text{The traversal configuration } \TCb}$};
  \begin{scope}[on background layer]
     \issuedCoveredBox{init};
     \issuedBox{i13};
     \issuedBox{i23};
  \end{scope}
  \draw[rf] (i13) edge node[above] {} (i22);
  \draw[rf] (i23) edge node[above] {} (i11);
  \draw[rf] (i12) edge node[above] {} (i21);
  \draw[vf] (init) edge[bend right=20]  node[above left, pos=0.9] {$\lVF$} (i11);
  \draw[vf] (init) edge[bend left=20]  node[above right, pos=0.9] {$\lVF$} (i21);
  \draw[vf] (i13)  edge[bend right=20] node[above] {$\lVF$} (i22);
  \draw[ppo,out=230,in=130] (i11) edge node[left ,pos=0.8] {\small$\lPPO$} (i12);
  \draw[ppo,out=310,in=50 ] (i22) edge node[right,pos=0.3] {\small$\lPPO$} (i23);
  \draw[po] (init) edge (i11);
  \draw[po] (init) edge (i21);
  \draw[po] (i11)  edge (i12);
  \draw[po] (i12)  edge (i13);
  \draw[po] (i21)  edge (i22);
  \draw[po] (i22)  edge (i23);
\end{tikzpicture}}
\hfill\vrule\hfill
\inarr{\begin{tikzpicture}[xscale=1,yscale=1.5]

  \node (init)  at (0, 1)      {$\Init$};

  \node (i111)  at (-1.5,  0)  {$\mese{1}{1}{1} \rlab{}{x}{0}$};
  \node (i121)  at (-1.5, -1)  {$\mese{1}{2}{1} \wlab{}{y}{0}$};
  \node (i131)  at (-1.5, -2)  {$\mese{1}{3}{1} \wlab{}{z}{1}$};

  \node (i211)  at (1.5,  0)   {$\mese{2}{1}{1} \rlab{}{y}{0}$};
  \node (i221)  at (1.5, -1)   {$\mese{2}{2}{1} \rlab{}{z}{1}$};
  \node (i231)  at (1.5, -2)   {$\mese{2}{3}{1} \wlab{}{x}{1}$};

  \draw[jf] (init) edge[bend right] node[above]        {\small{$\lJF$}} (i111);
  \draw[jf] (init) edge[bend left ] node[above]        {\small{$\lJF$}} (i211);
  \draw[jf] (i131) edge             node[pos=.5,below] {\small{$\lJF$}} (i221);

  \draw[po] (init)  edge (i111);
  \draw[po] (i111)  edge (i121);
  \draw[po] (i121)  edge (i131);

  \draw[po] (init)  edge (i211);
  \draw[po] (i211)  edge (i221);
  \draw[po] (i221)  edge (i231);

  \begin{scope}[on background layer]
    \draw[extractStyle] (-3, 1.5) rectangle (3,-2.5);
  \end{scope}

  %% \node (hh) at (0, -3.5) {$\inarrC{\text{The event structure } \ESb \text{ and} \\\text{the selected execution } \SXb}$};
\end{tikzpicture}}\hfill$
\caption{%
Граф сценария исполнения $G$, 
конфигурация обхода~$\TC_b$
и соответствующая этой конфигурации
структура событий~$S_b$ вместе с конфигурацией~$X_b$.
%% Покрытые события выделены как 
%% {\protect\tikz \protect\draw[coveredStyle] (0,0) rectangle ++(0.35,0.35);}
%% , а выпущенные как
%% {\protect\tikz \protect\draw[issuedStyle] (0,0) rectangle ++(0.35,0.35);}.
%% События, принадлежащие конфигурации $X_b$, выдены как 
%% {\protect\tikz \protect\draw[extractStyle] (0,0) rectangle ++(0.35,0.35);}.
}
\label{fig:lb-sim-ex-travB}
\end{figure}


Рассмотрим процесс построения сертификационной ветки
на примере шага обхода из конфигурации $\TC_a$ (\cref{fig:lb-sim-ex-travA})
в конфигурацию $\TC_b$ (\cref{fig:lb-sim-ex-travB})
путем выпуска события $\ese{2}{3}{}$.
Чтобы симулировать этот шаг, необходимо выполнить инструкции правого потока
и добавить в структуру событий ветку
$\Br_b = \set{\ese{2}{1}{1},\ese{2}{2}{1},\ese{2}{3}{1}}$
(смотри \cref{fig:lb-sim-ex-travB}).
Для того чтобы добавить эти события, в свою очередь,
выполняется построение трассы операционной семантики потока
${\state \thrdstep{\ese{2}{1}{1}}
         \thrdstep{\ese{2}{2}{1}}
         \thrdstep{\ese{2}{3}{1}}
         \state'}$, 
такой что
(i) она содержит все события правого потока
вплоть до последнего выпущенного события записи $\ese{2}{3}{}$ в графе $G$,
(ii) все эти события должны иметь тот же идентификатор потока,
тип обращения и локацию как и соответствующие события в графе
(то есть $\ese{2}{1}{}, \ese{2}{2}{}, \ese{2}{3}{}$),
(iii) все события, соответствующие покрытым и выпущенным событиям
(в данном случае~$\ese{2}{3}{1}$) должны иметь то же значение,
что и в графе $G$.
Для построения этой трассы используется свойство
\emph{восприимчивости} (\emph{receptiveness})
операционной семантики потока.
Это свойство позволяет выбрать произвольные значения
для всех промежуточных событий чтения в конструируемой трассе,
от которых не зависят (согласно отношению $\lDEPS$) выпущенные события записи%
\footnote{Формальное определение восприимчивости опущено
в данной работе для краткости и может быть найдено в \coq репозитории,
сопровождающем работу~\cite{Podkopaev-al:POPL19}.}.

Помимо этого, при добавлении новой ветки $\Br_b$ в структуру событий
необходимо выполнить следующие требования.
\begin{itemize}
  \item Для каждого события чтения (в данном случае $\ese{2}{1}{1}$ и $\ese{2}{2}{1}$)
    необходимо выбрать событие запись, обосновывающее это чтение.  
  \item Для каждого события записи необходимо определить позицию
    этого события в отношении частичного порядка $\lCO$.
\end{itemize}
Наконец, после завершения процесса сертификации,
новая ветка заменяет собой ветку потока $t$ в конфигурации $X$:
$$ X_b \defeq X_a \setminus S.\lE\rst{t} \cup \Br_b $$
где $S.\lE\rst{t} \defeq \set{e \in S.\lE \sth S.\lTID(e) = t}$.

\paragraph{Обоснование событий чтения.}

Далее рассмотрим процесс выбора обосновывающего события записи
для добавляемого события чтения.
Для этой цели определим отношение \emph{стабильной обоснованности}
в несколько этапов. 

Сначала по графу $G$ и текущей конфигурации обхода $\tup{C, I}$
зададим множество \emph{зафиксированных} (\emph{determined}) событий.
Метки зафиксированных событий а также события записи,
обосновывающие зафиксированные чтения, должны
совпадать в графе $G$, текущей структуре событий $S$,
а также в конструируемой сертификационной ветке $\Br$.

\begin{definition}
\label{def:det}
Множество \emph{зафиксированных событий}
определяется следующим соотношением.
\begin{align*}
  G.D_{\tup{C, I}} &\defeq {}
           C \cup I {}\cup{} \\
     %% &\cup G.\lW \setminus \codom{G.\lPPO} {}\cup{} \\
     &\cup \dom{G.\lRFI^? \seqc G.\lPPO \seqc [I]} {}\cup{} \\
     &\cup \cod{[I] \seqc G.\lRFI} {}\cup{} \\
     &\cup \cod{G.\lRFE \seqc [G.\lE^{\squq\acq}]}
\end{align*}
\end{definition}

Помимо покрытых и выпущенных событий
в множество зафиксированных событий также входят
все $\lPPO$-предшественники выпущенных событий,
все события чтения, читающие локально из некоторого выпущенного события,
а также события захватывающего ($\acq$) чтения,
читающие из другого потока. 

Для графа $G$ и конфигурации обхода $\TC_b$,
показанных на \cref{fig:lb-sim-ex-travB},
множество зафиксированных событий 
состоит из событий $\ese{1}{3}{}$, $\ese{2}{2}{}$ и $\ese{2}{3}{}$.
В то же время события $\ese{1}{1}{}$, $\ese{1}{2}{}$ и $\ese{2}{1}{}$
не являются зафиксированными, и следовательно
метки соответствующих им событий в структуре $S_b$
могут отличаться от меток в графе $G$.

Далее, введем понятие \emph{фронта} с помощью отношения $\lVF$.
Множество $\dom{\lVF \seqc [e]}$ будем называть \emph{фронтом}
события $e$. Это множество содержит все события записи
``наблюдаемые'' событием $e$.
Будем говорить что $e$ \emph{наблюдает} событие записи $w$,
то есть $\tup{w, e} \in G.\lVF_{\TC}$, если
$w$ ``происходит-до'' $e$, либо оно было
прочитано некоторым покрытым событием, ``происходящим-до''~$e$,
либо оно было ранее прочитано некоторым зафиксированным событием
принадлежащем тому же потококу, что и~$e$. 

\begin{definition}
\label{def:vf}
Отношение $\lVF$ определено как:
\begin{align*}
  G.\lVF_{\tup{C,I}} \defeq {}
    [G.\lW] \seqc (G.\lRF \seqc [C])^? \seqc G.\lHB^? \cup
    G.\lRF \seqc [G.D_{\tup{C, I}}] \seqc G.\lPO^?.
\end{align*}
\end{definition}

На \cref{fig:lb-sim-ex-travB} изображено три ребра отношения $G.\lVF_{\TC_b}$.
Все остальные ребра этого отношения могут быть выведены
при помощи следующего наблюдения:

$$ {G.\lVF_{\TC} \seqc G.\lPO \subseteq G.\lVF_{\TC}}. $$

Наконец, можно привести определение отношения стабильной обоснованости.
Оно соединяет событие чтения с $\lCO$ максимальным
наблюдаемым событием записи в ту же локацию.

\begin{definition}
\label{def:sjf}
Отношение \emph{стабильной обоснованности} определяется следующим образом.
\begin{equation*}
  G.\lSRF_{TC} \defeq
    ([G.\lW] \seqc (G.\lVF_{TC} \cap \lEQLOC) \seqc [G.\lR])
    \setminus (G.\lCO \seqc G.\lVF_{TC})
\end{equation*}
\end{definition}

Для графа $G$ и конфигурации $\TC_b$
отношение $\lSRF$ совпадает c показанными
на \cref{fig:lb-sim-ex-travB} ребрами отношения $\lVF$:
$$\tup{\Init, \ese{1}{1}{}}, \tup{\Init, \ese{2}{1}{}},
  \tup{\ese{1}{3}{}, \ese{2}{2}{}} \in G.\lSRF_{\TC_b}.$$

\begin{lemma}
\label{lm:sjf-det}
Если граф $G$ консистентен согласно модели \IMM,
тогда отношение $G.\lSRF$ совпадает с отношением $G.\lRF$
на множестве зафиксированных событий чтения.
$$  G.\lSRF_{\TC} ; [G.D_{\TC}] \subseteq G.\lRF $$
\end{lemma}

Лемма \ref{lm:sjf-det}, в частности, гарантирует, что
выбранные метки для событий чтения в сертификационной ветке,
от которых зависят (согласно отношению $\lDEPS$) выпущенные события записи,
будут согласованы с метками соответствующих событий в графе $G$.
Для всех остальных событий чтения, согласно свойству восприимчивости,
можно безопасно заменить прочитанные значения.

\begin{lemma}
\label{lm:sjf-iss-po}
Если граф $G$ консистентен согласно модели \IMM,
тогда для отношения $G.\lSRF$ выполняется следующие соотношение:
$$  G.\lSRF_{\TC} \suq [I] \seqc G.\lSRF_{\TC} \cup G.\lPO. $$
\end{lemma}

Лемма \ref{lm:sjf-iss-po} позволяет выбрать обосновывающее
событие записи в структуре событий.
Пусть $\tup{w, r} \in G.\lSRF_{\TC}$.
Если при этом $\tup{w, r} \in [I] \seqc G.\lSRF_{\TC}$
тогда, согласно свойству \ref{simrel:ew-iss} отношения симуляции
можно выбрать событие записи $w' \in S.\lE$, которое принадлежит
конфигурации $X$ и при этом соответствует выпущенной записи, то есть $\ea(w') = w$.
Например, в случае конфигурации обхода $\TC_b$, показанной на \cref{fig:lb-sim-ex-travB},
для события чтения $\ese{2}{2}{1}$
обосновывающим событием записи будет $\ese{2}{3}{1}$
Иначе $\tup{w, r} \in G.\lPO$.
В таком случае достаточно просто выбрать $S.\lPO$
предшествующее событие записи, принадлежащее сертификационной ветке $\Br$.

\paragraph{Упорядочивание событий записи.}

Позиция добавляемых событий записи в отношении порядка
$S.\lCO$ структуры событий выбирается на основе порядка
$G.\lCO$ \IMM графа. Тем не менее, из-за наличия конфликтующих событий,
можно гарантировать только лишь что отношение $S.\lCO$ вложено
в рефлексивное замыкание отношения $G.\lCO$,
то есть $\fmap{S.\lCO} \subseteq G.\lCO^?$.

\begin{figure}[h]
\hfill$\inarr{\begin{tikzpicture}[xscale=1,yscale=1.5]
  \node (init) at (2,  1)   {$\Init$};
  \node (i11)  at (0,  0)   {$\mese{1}{1}{} \rlab{}{x}{1}$};
  \node (i12)  at (0, -1)   {$\mese{1}{2}{} \wlab{}{y}{1}$};
  \node (i13)  at (0, -2)   {$\mese{1}{3}{} \wlab{}{z}{1}$};
  \node (i21)  at (4,  0)   {$\mese{2}{1}{} \rlab{}{y}{1}$};
  \node (i22)  at (4, -1)   {$\mese{2}{2}{} \rlab{}{z}{1}$};
  \node (i23)  at (4, -2)   {$\mese{2}{3}{} \wlab{}{x}{1}$};
  %% \node (hh) at (2, -3) {$\inarrC{\text{The traversal configuration } \TCc}$};
  \begin{scope}[on background layer]
     \issuedCoveredBox{init};
     \issuedBox{i13};
     \issuedBox{i23};
     \coveredBox{i11};
  \end{scope}
  \draw[rf] (i13) edge node[above] {} (i22);
  \draw[rf] (i23) edge node[above] {} (i11);
  \draw[rf] (i12) edge node[above] {} (i21);
  %\draw[vf] (init) edge[bend left=20]  node[above right, pos=0.9] {$\lVF$} (i21);
  %\draw[vf] (i13)  edge[bend right=20] node[above] {$\lVF$} (i22);
  \draw[ppo,out=230,in=130] (i11) edge node[left ,pos=0.8] {\small$\lPPO$} (i12);
  \draw[ppo,out=310,in=50 ] (i22) edge node[right,pos=0.3] {\small$\lPPO$} (i23);
  \draw[po] (init) edge (i11);
  \draw[po] (init) edge (i21);
  \draw[po] (i11)  edge (i12);
  \draw[po] (i12)  edge (i13);
  \draw[po] (i21)  edge (i22);
  \draw[po] (i22)  edge (i23);
\end{tikzpicture}}
\hfill\vrule\hfill
\inarr{\begin{tikzpicture}[xscale=1,yscale=1.5]
  \node (init) at (3, 1)     {$\Init$};

  \node (i111)  at (0,  0)   {$\mese{1}{1}{1} \rlab{}{x}{0}$};
  \node (i121)  at (0, -1)   {$\mese{1}{2}{1} \wlab{}{y}{0}$};
  \node (i131)  at (0, -2)   {$\mese{1}{3}{1} \wlab{}{z}{1}$};

  \node (i112)  at (3,  0)   {$\mese{1}{1}{2} \rlab{}{x}{1}$};
  \node (i122)  at (3, -1)   {$\mese{1}{2}{2} \wlab{}{y}{1}$};
  \node (i132)  at (3, -2)   {$\mese{1}{3}{2} \wlab{}{z}{1}$};

  \node (i211)  at (6,  0)   {$\mese{2}{1}{1} \rlab{}{y}{0}$};
  \node (i221)  at (6, -1)   {$\mese{2}{2}{1} \rlab{}{z}{1}$};
  \node (i231)  at (6, -2)   {$\mese{2}{3}{1} \wlab{}{x}{1}$};

  \draw[jf] (init) edge[bend right] node[above]        {} (i111);
  \draw[jf] (init) edge[bend left ] node[above]        {} (i211);
  \draw[jf] (i131) edge             node[pos=.5,below] {} (i221);
  \draw[jf] (i231) edge             node[pos=.5,below] {} (i112);

  \draw[cf] (i111) -- (i112);
  \node at ($.5*(i111) + .5*(i112) - (0, 0.2)$) {\small$\lCF$};

  \draw[co] (i122) edge node[pos=.5,below] {\small$\lCO$} (i121);
  \draw[ew] (i131) edge node[pos=.5,below] {\small$\lEW$} (i132);

  \draw[po] (init)  edge (i111);
  \draw[po] (i111)  edge (i121);
  \draw[po] (i121)  edge (i131);

  \draw[po] (init)  edge (i112);
  \draw[po] (i112)  edge (i122);
  \draw[po] (i122)  edge (i132);

  \draw[po] (init)  edge (i211);
  \draw[po] (i211)  edge (i221);
  \draw[po] (i221)  edge (i231);

  \begin{scope}[on background layer]
    \draw[extractStyle] (1.8,1.5) rectangle (7.2,-2.5);
  \end{scope}

  %% \node (hh) at (3, -3.5) {$\inarrC{\text{The event structure } \ESc \text{ and} \\\text{the selected execution } \SXc}$};
\end{tikzpicture}}$\hfill
\caption{
Граф сценария исполнения~$G$, конфигурация обхода~$\TC_c$
и соответствующая этой конфигурации
структура событий~$S_c$ вместе с конфигурацией~$X_c$.
}
\label{fig:lb-sim-ex-travC}
\end{figure}


Данную особенность можно продемонстрировать 
на примере шага обхода из конфигурации $\TC_b$ (\cref{fig:lb-sim-ex-travB})
в конфигурацию $\TC_c$ (\cref{fig:lb-sim-ex-travC})
путем покрытия события $\ese{1}{1}{}$.
Для симуляции этого шага выполняется построение
структуры событий $S_c$, содержашей новую ветку 
$\Br_c \defeq \set{\ese{1}{1}{2}, \ese{1}{2}{2}, \ese{1}{3}{2}}$.

Рассмотрим события записи $\ese{1}{2}{1}$ и $\ese{1}{2}{2}$.
Так как два этих события имеют различные метки,
они не могут быть объявлены $\lEW$-эквивалентными.
С другой стороны, требуется, чтобы отношение
$S_c.\lCO$ полностью упорядочивало все события записи в одну и ту же локацию
(по модулю $\lEW$-эквивалентных событий).
То есть требуется каким-либо образом упорядочить
события $\ese{1}{2}{1}$ и $\ese{1}{2}{2}$ между собой.
Так как оба эти события отображаются в одно и то же событие $\ese{1}{2}{}$
в графе $G$, отношение $G.\lCO$ не может быть использовано
при выборе направления $S_c.\lCO$ ребра. 

На самом деле может быть выбран любой из двух способов
упорядочить эти события. Тем не менее,
в целях упрощения доказательства оказалось
удобнее выбрать порядок, при котором новые события
оказываются упорядочены ранее в отношении когеретности.
То есть, возвращаясь к примеру на \cref{fig:lb-sim-ex-travC},
стоит добавить ребро $\tup{\ese{1}{2}{2}, \ese{1}{2}{1}} \in S_c.\lCO$.
Используя данное соглашение можно показать, что
$S.\lCO$ ребро, оканчивающиеся событием из новой ветки $\Br_c$,
должно отображаться строго в ребро $G.\lCO$ в графе: 
$\fmap{S_c.\lCO \seqc [\Br_c]} \suq G.\lCO$.

Далее рассмотрим события $\ese{1}{3}{1}$ и $\ese{1}{3}{2}$.
Эти события имеют одинаковую метку и отображаются в одно
и тоже событие $\ese{1}{3}{}$ в графе $G$.
Следовательно, они могут быть объявлены $\lEW$-эквивалентными.
На самом деле, для корректности построения их необходимо объявить таковыми.
Иначе события ветки $\Br_c$ окажутся невидимыми из-за того,
что существует путь $S_c.\lCF \cap (S_c.\lJFE \seqc (S_c.\lPO \cup S_c.\lJF)^*)$
из события $\ese{1}{3}{1}$ в событие $\ese{1}{1}{2}$.
Напомним, что только видимые события могут быть использованы
для извлечения графа сценария исполнения из структуры событий
(смотри \cref{def:cfg}).

В общем случае, новое событие записи $e$
прикрепляется к классу эквивалентности по отношению $S.\lEW$,
представленному событием $w$, таким что
(i) $w$ имет тот же образ графе, что и $e$, то есть $\ea(w) = \ea(e)$;
(ii) $w$ принадлежит конфигурации $X$, а его образ в графе
принадлежит множеству выпущенных событий: $w \in X \cap \fcomap{I}$.
Если такого события $w$ не существует, тогда $e$
упорядочивается в отношении $S.\lCO$ до
множества событий, чьи образы в графе $G.\lCO$-предшествуют $\ea(e)$
и после событий, чьи образы в графе равны или $G.\lCO$-следуют за $\ea(e)$.
Благодаря свойству \ref{simrel:jfe-iss} отношения симуляции,
а именно $\dom{S.\lJFE} \suq \dom{S.\lEW \seqc [X \cap \fcomap{I}]}$,
подобный выбор отношения $S'.\lEW$ гарантирует,
что все события из новой сертификационной ветки будут видимы. 
