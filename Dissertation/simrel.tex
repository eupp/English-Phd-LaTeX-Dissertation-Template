\subsection*{Отношение симуляции}
\label{sec:simrel}

Далее опишем отношение симуляции $\simrel$.
В целях ясности и простоты изложения, 
в данном разделе будет приведена упрощенная версия формального
определения этого отношения, которая опускает некоторые
технические детали. Полная версия отношения симуляции
может быть найдена в \coq репозитории. 

Отношение симуляции $\simrel(P, T, G, TC, S, X)$
устанавливает взаимосвязь между структурой событий $S$
и графом сценария исполнения $G$ с помощью
функции $\ea : S.\lE \fun G.\lE$, которая отображает
события структуры $S$ в события графа $G$.
Эта функция может быть натуральным образом
расширена на множества событий следующим образом%
\footnote{аналогично образом функций $\ea$ может быть расширена
на бинарные отношения на событиях.}:

\begin{align*}
\text{for } A_S \subseteq S.\lE        & :
  \fmap{A_S} \defeq \set{\ea(e) \in G.\lE \mid e \in A_S} \\
\text{for } A_G \subseteq G.\lE        & :
  \fcomap{A_G} \defeq \set{e \in S.\lE \mid \ea(e) \in A_G}.
\end{align*}

Отношение симуляции $\simrel(P, T, G, \TC, S, X)$ состоит из следующих свойств.

\begin{enumerate}

  \item \label{simrel:events}
    События $S$, принадлежащие потокам из $T$, а также события,
    принадлежащие конфигурации $X$, соответствуют покрытым событиям,
    а также выпущенным событиям и их $\lPO$-предшественникам: 
    \begin{itemize}
      \item $\fmap{S.\lE\rst{T}} = \fmap{X} = C \cup \dom{G.\lPO^? \seqc [I]}$
    \end{itemize}

  \item \label{simrel:lab}
    Метки событий из $S$ совпадают с метками событий из $G$
    по модулю прочитанных или записанных значений. 
    \begin{enumerate}
      \setcounter{enumii}{0}
      \item \label{simrel:lab-eqmval}
        $\forall e \in S.\lE \ldotp\;
          S.\set{\lTID, \lTYP, \lLOC, \lMOD}(e) =
          G.\set{\lTID, \lTYP, \lLOC, \lMOD}(\fmap{e}) $
    \end{enumerate}
    Метки покрытых и выпущенных событий, принадлежащих конфигурации $X$,
    сопадают полностью.
    \begin{enumerate}
      \setcounter{enumii}{1}
      \item \label{simrel:lab-det}
        $\forall e \in X \cap \fcomap{C \cup I} \ldotp~
          S.\lVAL(e) = G.\lVAL(\ea(e))$
    \end{enumerate}

  \item \label{simrel:po}
    Программный порядок в структуре событий $S$
    совпадает с программным порядком в графе $G$:
    \begin{itemize}
      \item $\fmap{S.\lPO} \suq G.\lPO$
    \end{itemize}

  \item \label{simrel:cf}
    Если два события имеют одинаковый образ под действием функции $\ea$,
    то эти события равны или находятся в конфликте.
    \begin{itemize}
      \item $\fcomap{\mathtt{id}} \suq S.\lCF^?$
    \end{itemize}

  \item \label{simrel:jf}
    События чтения в $S$ должны быть обоснованы событиями записи,
    которые наблюдаются соответствующим событием чтением в $G$.
    \begin{enumerate}
      \item \label{simrel:jf-obs}
      \setcounter{enumii}{0}
        $\fmap{S.\lJF} \suq G.\lRF^?\seqc G.\lHB^?$
    \end{enumerate}
    Более того, отношение $\lJF$, ограниченное на события чтения,
    принадлежащие конфигурации $X$, соответствуют
    отношению \emph{стабильной обоснованности} (\emph{stable justification})
    (смотри \cref{def:sjf}) в графе $G$.
    \begin{enumerate}
      \setcounter{enumii}{1}
      \item \label{simrel:jf-sjf}
        $\fmap{S.\lJF \seqc [X]} \suq G.\lSRF_{TC}$
    \end{enumerate}
    %% As a consequence it is possible to derive that
    %% justification for covered events in $X$
    %% corresponds to their justification in $G$:
    %% $\fmap{S.\lJF \seqc [X \cap \fcomap{C}]} \subseteq G.\lRF$. \\
    Только выпущенные события могут быть использованы для
    внешнего обоснования событий чтения. 
    \begin{enumerate}
      \setcounter{enumii}{2}
      \item \label{simrel:jfe-iss}
         $\dom{S.\lJFE} \suq \dom{S.\lEW \seqc [X \cap \fcomap{I}]}$
    \end{enumerate}

  \item \label{simrel:ew}
    Все эквивалентные события записи в $S$ отображаются
    в одно и то же событие записи $G$.
    \begin{enumerate}
      \setcounter{enumii}{0}
      \item \label{simrel:ew-id}
        $\fmap{S.\lEW} \suq \mathtt{id}$
    \end{enumerate}
    Также каждый класс эквивалентности по отношению $S.\lEW^*$
    должен иметь представителя среди выпущенных событий,
    принадлежащих конфигурации $X$.
    \begin{enumerate}
      \setcounter{enumii}{1}
      \item \label{simrel:ew-iss}
        $S.\lEW \suq (S.\lEW \seqc [X \cap \fcomap{I}] \seqc S.\lEW)^?$
    \end{enumerate}

  \item \label{simrel:co}
    Если два события структуры $S$ находящихся в отношении когерентности,
    то их образы под действием функции либо также находятся
    в отношении когерентности, либо равны. 
    \begin{enumerate}
      \setcounter{enumii}{0}
      \item \label{simrel:co-co}
         $\fmap{S.\lCO} \suq G.\lCO^?$
    \end{enumerate}
    Если же ребро отношения когерентности оканчивается
    в событии, принадлежащем конфигурации $X$ и одному из потоков из $T$,
    тогда образ этого ребра принадлежит отношению когерентности в графе $G$.
    \begin{enumerate}
      \setcounter{enumii}{1}
      \item \label{simrel:co-cfg}
         $\fmap{S.\lCO \seqc [X\rst{T}]} \suq G.\lCO$
    \end{enumerate}

  \item \label{simrel:sw-hb}
    Отношения ``синхронизируется-с'' и ``происходит-до''
    в структуре событий $S$ согласованы с соответствующими
    отношениями в графе $G$.
    \begin{enumerate}
      \item \label{simrel:sw}
        $\fmap{S.\lSW} \suq G.\lSW$
      \item \label{simrel:hb}
        $\fmap{S.\lHB} \suq G.\lHB$
    \end{enumerate}
\end{enumerate}

\begin{figure}[h]
$\hfill\inarr{\begin{tikzpicture}[xscale=1,yscale=1.5]
  \node (init) at (2,  1)   {$\Init$};
  \node (i11)  at (0,  0)   {$\mese{1}{1}{} \rlab{}{x}{1}$};
  \node (i12)  at (0, -1)   {$\mese{1}{2}{} \wlab{}{y}{1}$};
  \node (i13)  at (0, -2)   {$\mese{1}{3}{} \wlab{}{z}{1}$};
  \node (i21)  at (4,  0)   {$\mese{2}{1}{} \rlab{}{y}{1}$};
  \node (i22)  at (4, -1)   {$\mese{2}{2}{} \rlab{}{z}{1}$};
  \node (i23)  at (4, -2)   {$\mese{2}{3}{} \wlab{}{x}{1}$};
  %% \node (hh)   at (2, -3.5) {$\inarrC{\text{The execution graph } G \text{ and} \\\text{its traversal configuration } \TCa}$};
  \begin{scope}[on background layer]
     \issuedCoveredBox{init};
     \issuedBox{i13};
%     \issuedBox{i23};
  \end{scope}
  \draw[rf] (i13) edge node[above] {} (i22);
  \draw[rf] (i23) edge node[above] {} (i11);
  \draw[rf] (i12) edge node[above] {} (i21);
% \draw[vf] (init) edge[bend right=20]  node[above left, pos=0.9] {$\lVF$} (i11);
% \draw[vf] (init) edge[bend left=20]  node[above right, pos=0.9] {$\lVF$} (i21);
% \draw[vf] (i13)  edge[bend right=20] node[above] {$\lVF$} (i22);
  \draw[ppo,out=230,in=130] (i11) edge node[left ,pos=0.8] {\small$\lPPO$} (i12);
  \draw[ppo,out=310,in=50 ] (i22) edge node[right,pos=0.3] {\small$\lPPO$} (i23);
  \draw[po] (init) edge (i11);
  \draw[po] (init) edge (i21);
  \draw[po] (i11)  edge (i12);
  \draw[po] (i12)  edge (i13);
  \draw[po] (i21)  edge (i22);
  \draw[po] (i22)  edge (i23);
\end{tikzpicture}}
\hfill\vrule\hfill
\inarr{\begin{tikzpicture}[xscale=1,yscale=1.5]

  \node (init) at (0, 1)   {$\Init$};

  \node (i111) at (-1.5,  0)   {$\mese{1}{1}{1} \rlab{}{x}{0}$};
  \node (i121) at (-1.5, -1)   {$\mese{1}{2}{1} \wlab{}{y}{0}$};
  \node (i131) at (-1.5, -2)   {$\mese{1}{3}{1} \wlab{}{z}{1}$};

  \node (i211) at (0.5,  0)   {\phantom{$\mese{2}{1}{1} \rlab{}{y}{0}$}};
  \node (i221) at (0.5, -1)   {\phantom{$\mese{2}{2}{1} \rlab{}{z}{1}$}};
  \node (i231) at (0.5, -2)   {\phantom{$\mese{2}{3}{1} \wlab{}{x}{1}$}};

  \draw[jf] (init) edge[bend right] node[above]        {\small{$\lJF$}} (i111);

  \draw[po] (init)  edge (i111);
  \draw[po] (i111)  edge (i121);
  \draw[po] (i121)  edge (i131);

  \begin{scope}[on background layer]
    \draw[extractStyle] (-3, 1.5) rectangle (1,-2.5);
  \end{scope}

  %% \node (hh) at (0, -3.5) {$\inarrC{\text{The event structure } \ESa \text{ and} \\\text{the selected execution } \SXa}$};
\end{tikzpicture}}\hfill$
\caption{%
Граф сценария исполнения $G$, 
конфигурация обхода $\TC_a$
и соответствующая этой конфигурации
структура событий $S_a$ вместе с конфигурацией $X_a$.
Покрытые события выделены как 
{\protect\tikz \protect\draw[coveredStyle] (0,0) rectangle ++(0.35,0.35);}
, а выпущенные как
{\protect\tikz \protect\draw[issuedStyle] (0,0) rectangle ++(0.35,0.35);}.
События, принадлежащие конфигурации $X_a$, выдены как 
{\protect\tikz \protect\draw[extractStyle] (0,0) rectangle ++(0.35,0.35);}.
}
\label{fig:lb-sim-ex-travA}
\end{figure}


\TODO{пример}
