\chapter{Обзор}
\label{ch:review}

В данной главе приводится введение в предметную область
данного диссертационного исследования --- слабые модели памяти.
Рассматриваются требования, предъявляемые к моделям памяти,
обсуждаются существующие модели памяти и их классификация.
Рассматриваются недостатки существующих моделей памяти
и открытые исследовательские вопросы.
Также вводятся необходимые математические формализмы,
используемые для описания семантики многопоточных программ
и слабых моделей памяти. 

\section{Слабые модели памяти}
\label{sec:models-intro}

Напомним, что в контексте данного исследования
моделью памяти называется формальная семантика
многопоточных программ, оперирующих с разделяемой памятью.
В данной работе будем преимущественно говорить о моделях памяти 
высокоуровневых языков программирования~\cite{Moiseenko-al:PCS21}, 
таких как, например, \CPP, \Java и другие. 

Одной из наиболее простых для понимания моделей памяти
является модель \emph{последовательной согласованности}
(\emph{sequential consistency})~\cite{Lamport:TC79}.
В рамках данной модели каждый допустимый
сценарий поведения многопоточной программы
является результатом поочередного исполнения
атомарных обращений к разделяемой памяти из параллельных потоков.

Рассмотрим, например, параллельную программу, показанную ниже.

\begin{equation*}
\inarrII{
\writeInst{}{x}{1} \\
\readInst{}{r_1}{y} \\
\kw{if} {r_1 = 0} ~\{ \\
~~\rfcomment{критическая секция} \\
\}
}{
\writeInst{}{y}{1} \\
\readInst{}{r_2}{x} \\
\kw{if} {r_2 = 0} ~\{ \\
~~\rfcomment{критическая секция} \\
\}
}
\tag{Dekker's Lock}\label{ex:Dekker}
\end{equation*}


Данная программа является упрощенной версией
алгоритма блокировки Деккера~\cite{Dijkstra:68}.
В этой программе два потока соревнуются за доступ к критической секции.
Каждый поток, для того чтобы обозначить свое намерение войти в критическую секцию,
устанавливает значение переменной $x$ или $y$ соответственно
\footnote{В данной работе разделяемые переменные
будем обозначать как $x$, $y$, $z$..., 
а локальные переменные как $r_1$, $r_2$, $r_3$...}.
Право войти в критическую секцию получает тот поток, 
который успеет прочитать значение переменной до его установки другим потоком.

В рамках модели последовательной согласованности
в результате выполнения данной программы 
либо один из потоков прочитает значение~\tcode{1}, а другой~\tcode{0}, 
либо оба прочитают значение~\tcode{1} (в этом случае ни один из потоков
не войдет в критическую секцию).
То есть в результате получим один из следующих исходов:
${[r_1=0, r_2=1]}$, ${[r_1=1,r_2=0]}$ или ${[r_1=1,r_2=1]}$. 
Данные сценарии поведения будем называть 
\emph{последовательно согласованными}.

Алгоритм Деккера полагается на тот факт, что оба 
потока не могут одновременно прочитать значение~\tcode{0}.
В противном случае не гарантируется свойство \emph{взаимного исключения}, 
то есть два потока могут одновременно войти в критическую секцию. 
Тем не менее на практике можно также наблюдать
сценарий поведения данной программы, нарушающий это предположение, 
то есть в результате которого имеем ${[r_1=0,r_2=0]}$.
Например, данный сценарий поведения можно наблюдать,
если перевести приведенный выше псевдокод алгоритма Деккера
на язык \CLANG, скомпилировать его с помощью \GCC
и запустить получившийся код на процессорах семейства \IntelX.

Подобные сценарии поведения, 
не укладывающиеся в модель последовательной согласованности, 
принято называть \emph{слабыми сценариями}.
Слабые сценарии поведения могут появляться 
в результате выполнения различных оптимизаций 
компилятором при сборке программы или процессором при ее исполнении. 
Например, в случае программы \ref{ex:Dekker}, оптимизатор может выполнить 
\emph{переупорядочивание независимых инструкций} 
записи в переменную $x$ и чтения из переменной $y$ в левом потоке.
Для оптимизированной версии программы сценарий поведения, 
ведущий к результату ${[r_1=0, r_2=0]}$, уже является последовательно согласованным.

Современные многопоточные языки программирования 
как правило предоставляют \emph{слабые модели памяти},
то есть модели, допускающие слабые сценарии поведения,
поскольку более строгая модель последовательной согласованности
не допускает применение широкого спектра оптимизаций
и, следовательно, реализация данной модели на практике
приводит к значительным накладным расходам%
~\cite{Marino-al:PLDI11,Singh-al:ISCA12,Liu-al:OOPSLA17,Liu-al:PLDI19}. 

Как было продемонстрировано выше на примере алгоритма Деккера, 
чтобы гарантировать корректность многопоточных программ
разработчикам необходимо учитывать слабые сценарии поведения.
Таким образом, важно понимать, насколько слабую модель памяти
предоставляет язык программирования, какие оптимизации 
допускаются этой моделью памяти и какие гарантии 
эта модель предлагает программисту. 

Далее в разделе \ref{sec:models-primitives} 
вводятся программные примитивы для работы с разделяемой памятью. 
В разделе \ref{sec:models-requirements} более подробно обсуждаются 
различные требования, предъявляемые к моделям памяти, 
а в разделе \ref{sec:models-classes} обсуждаются существующие модели и их классификация. 
Наконец в разделе \ref{sec:models-summary} приводится сравнение различных 
классов моделей памяти и обозначаются открытые исследовательские проблемы, 
некоторые из которых были решены в рамках данного 
диссертационного исследования. 

\subsection{Программные примитивы}
\label{sec:models-primitives}

В рамках данного исследования будем считать, что
разделяемая память представляет собой отображение
из адресов переменных\footnote{В данной работе адрес переменной 
также иногда будем называть \emph{локацией}.} в их значения. 
Таким образом, будем подразумевать, 
что разделяемая память состоит из взаимно непересекающихся, 
типизированных адресуемых ячеек памяти%
\footnote{В теории моделей памяти также иногда определяют 
разделяемую память как нетипизированную последовательность байт, 
допускающую обращения \emph{смешанного размера} (\emph{mixed-size accesses}). 
В контексте данной работы смешанные обращения не рассматриваются.}

Основные операции, которые предоставляет абстракция разделяемой памяти --- 
это операция записи в разделяемую переменную и операция чтения из разделяемой переменной. 
Также, будем считать, что разделяемая память предоставляет атомарные операции 
\emph{чтения-модификации-записи} (\emph{read-modify-write}), 
в частности, операцию сравнения и замены (\emph{compare-and-swap}, \CAS), 
операцию атомарного обмена (\emph{exchange}, \EXCHG) 
и операцию атомарного инкремента (\emph{fetch-and-add}, \FADD).
Операция сравнения и замены атомарно выполняет сравнение 
текущего и ожидаемого значений переменной и в случае 
их совпадения заменяет значение переменной на желаемое.
Операция обмена атомарно заменяет значение переменной 
и возвращает ее прежнее значение. 
Наконец, операция атомарного инкремента прибавляет 
к значению переменной заданную величину и
возвращает ее значение до модификации.
Все инструкции обращения к разделяемой памяти 
приведены в таблице \ref{table:primitives}. 

\begin{center}
\begin{figure}[hb]
\begin{tabular}{l@{\hskip 40pt}|l} 

    \hline

      $\writeInst{o}{x}{v}$ 
    & \makecell[l]{
        Инструкция записи значения $v$ \\
        в разделяемую переменную $x$   \\
        с режимом доступа $o$.
      } 
    \\ 
    \hline

      $\readInst{o}{r}{x}$ 
    & \makecell[l]{
        Инструкция чтения значения    \\
        из разделяемой переменной $x$ \\
        в локальную переменную $r$    \\
        с режимом доступа $o$.
      } 
    \\ 
    \hline

      $\casInst{o_s}{o_f}{r}{x}{v_e}{v_d}$ 
    & \makecell[l]{
        Инструкция атомарного сравнения                 \\ 
        разделяемой переменной $x$                      \\
        c ожидаемым значением $v_e$ и                   \\ 
        заменой на желаемое значение $v_d$;             \\ 
        прочитанное значение присваивается              \\
        в локальную переменную $r$;                     \\
        в случае успеха операции сравнения применяется  \\ 
        режим доступа $o_s$, иначе $o_f$.
      } 
    \\ 
    \hline

      $\exchgInst{o}{r}{x}{v}$ 
    & \makecell[l]{
        Инструкция атомарного обмена значения           \\
        разделяемой переменной $x$ на значение $v_e$,   \\
        аннотированная режимом доступа $o$;             \\ 
        прочитанное значение присваивается              \\
        в локальную переменную $r$.                     \\
      } 
    \\ 
    \hline

      $\faiInst{o}{r}{x}{v}$ 
    & \makecell[l]{
        Инструкция атомарного инкремента значения       \\
        разделяемой переменной $x$ на значение $v$,     \\
        аннотированная режимом доступа $o$;             \\ 
        прочитанное значение присваивается              \\
        в локальную переменную $r$.                     \\
      } 
    \\ 
    \hline

\end{tabular}
\caption{Список используемых программных примитивов}
\label{fig:primitives}
\end{figure}
\end{center}


Помимо этого, модели памяти, как правило, различают 
несколько видов обращений к разделяемой памяти и позволяют 
программисту аннотировать эти обращения 
\emph{режимом доступа} (\emph{access mode}).
Режимы доступа отличаются гарантиями, 
которые они предоставляют пользователю. 
Выделяют следующие режимы доступа: 
\emph{неатомарный режим} (\emph{non-atomic}), 
\emph{ослабленный режим} (\emph{relaxed} или \emph{opaque} в терминологии \Java),
режим \emph{захвата} (\emph{acquire}), 
режим \emph{освобождения} (\emph{release}), 
их комбинированный режим \emph{захвата-освобождения} (\emph{acquire-release}), 
а также \emph{последовательно согласованный режим} 
(\emph{sequentially consistent} или \emph{volatile} в \Java).
Эти режимы обозначаются как $\na$, $\rlx$, $\acq$, $\rel$, $\acqrel$ и $\sco$ соответственно.
При этом режим $\acq$ применим только к операциям чтения,
а режим $\rel$ --- только к операциям записи.
Режимы обращения упорядочены согласно строгости предоставляемых гарантий, 
как показано на следующей диаграмме. 

 \[\inarr{
 \begin{tikzpicture}[yscale=0.6,xscale=1.3]

   \node (na)     at (-2.6,  0)  {$\na$};
   \node (rlx)    at (-1.3,  0)  {$\rlx$};
   \node (rel)    at (0   ,  1)  {$\rel$};
   \node (acq)    at (0   , -1)  {$\acq$};
   \node (acqrel) at (1.5 ,  0)  {$\acqrel$};
   \node (sc)     at (3   ,  0)  {$\sco$};

   \path[->] (na)   edge[line width=0.742mm] 
                    node[fill=white, anchor=center, pos=0.5] {$\squ$} 
             (rlx);

   \path[->] (rlx) edge[line width=0.742mm] 
                   node[fill=white,anchor=center,pos=0.5] 
                   {\rotatebox[origin=c]{45} {$\squ$}} 
             (rel); 

   \path[->] (rlx) edge[line width=0.742mm] 
                   node[fill=white,anchor=center,pos=0.5] 
                   {\rotatebox[origin=c]{-45} {$\squ$}}
             (acq); 
   
   \path[->] (rel) edge[line width=0.742mm] 
                   node[fill=white,anchor=center,pos=0.5] 
                   {\rotatebox[origin=c]{-45} {$\squ$}}
             (acqrel); 

   \path[->] (acq) edge[line width=0.742mm] 
                   node[fill=white,anchor=center,pos=0.5] 
                   {\rotatebox[origin=c]{45} {$\squ$}}
             (acqrel); 

   \path[->] (acqrel) edge[line width=0.742mm] 
                      node[fill=white, anchor=center, pos=0.5] {$\squ$} 
             (sc);

 \end{tikzpicture}
 }\]


Неатомарные обращения, аннотированные режимом $\na$, 
не предполагается использовать для конкурентного доступа 
к разделяемой переменной из параллельных потоков программы. 
В зависимости от конкретного языка программирования
конкурентные неатомарные обращения либо полностью запрещены
(например, в \Haskell~\cite{Vollmer-al:PPoPP17} и \Rust~\cite{RustBook:19}), 
либо могут приводить к неопределенному поведению
(например, в \CPP~\cite{Batty-al:POPL11}),
либо не предоставляют практический никаких гарантий о порядке,
в котором потоки могут наблюдать эти обращения
(например, в \Java~\cite{Manson-al:POPL05}). 

Для ослабленных обращений, аннотированных режимом $\rlx$, 
как правило, гарантируется только выполнение свойства 
\emph{когерентности}~\cite{Alglave-al:TOPLAS14}.
Это свойство обеспечивает \emph{последовательную согласованность 
по каждой отдельной локации в памяти}.
В частности, из этого следует, что программа, 
состоящая из ослабленных обращений только к одной переменной, 
допускает только последовательно согласованные сценарии исполнения.

Обращения, аннотированные режимами захвата $\acq$ и освобождения $\rel$,
используются для поддержки идиомы передачи сообщений~\cite{Lahav-al:POPL16}.
Поток, выполняющий отправку сообщения, должен аннотировать соответствующую 
операцию записи в разделяемую память режимом доступа $\rel$, 
а поток, ожидающий это сообщение, должен аннотировать 
операцию чтения режимом доступа $\acq$.

Наконец, последовательно согласованные обращения, 
то есть аннотированные режимом $\sco$, 
при правильном использовании гарантируют 
семантику последовательной согласованности%
~\cite{Boehm-Adve:PLDI08, Lahav-al:PLDI17}.

\subsection{Требования к моделям памяти}
\label{sec:models-requirements}

Как уже упоминалось ранее на примере алгоритма Деккера, 
при разработке многопоточных программ необходимо 
учитывать модель памяти, предоставляемую языком программирования.
К моделям памяти предъявляются противоречивые требования. 
С одной стороны, более строгая модель допускает меньше сценариев поведения 
и предоставляет больше гарантий разработчику.
С другой стороны, более слабая модель позволяет 
выполнять большее количество различных оптимизаций, 
что приводит к повышению производительности программы. 
Таким образом, в дизайне модели памяти необходимо 
найти разумный компромисс между этими конфликтующими запросами.

В этом разделе будут более подробно описан 
набор типичных требований и критериев, предъявляемых 
к моделям памяти языков программирования. 

\subsubsection*{Оптимальность и корректность схемы компиляции}

\emph{Схемой компиляции} называется отображение
примитивов языка программирования в инструкции 
языка ассемблера конкретного семейства процессоров.
Будем подразумевать, что и высокоуровневый язык программирования и 
язык ассемблера в данном случае предоставляют одинаковый 
набор программных примитивов, описанных в разделе \ref{sec:models-primitives}.

\emph{Оптимальная} схема компиляции позволяет 
компилировать инструкций обращения к разделяемой памяти 
из языка программирования в инструкции целевого процессора
без необходимости вставки барьеров памяти и 
без усиления режима доступа обращений. 
Другими словами, наличие у языка программирование оптимальной схемы компиляции
позволяет компилировать программы на этом языке 
в эффективный код для целевого процессора.  
Напротив, использование неоптимальных схем компиляции
может приводить к снижению производительности кода
из-за наличия многочисленных барьеров памяти.
Но в то же время вставка дополнительных барьеров памяти
может предотвратить появление слабых сценариев поведения, 
допустимых спецификацией архитектуры процессора. 

\emph{Корректность} схемы компиляции гарантирует,
что множество сценариев поведения, допустимых моделью памяти процессора 
для скомпилированной версии программы,  
является подмножеством сценариев поведения исходной программы, 
допустимой моделью памяти языка программирования. 

Чтобы пояснить введенные выше понятия оптимальности 
и корректности схемы компиляции, рассмотрим пример. 
Программа \ref{ex:sb}, показанная ниже, является 
еще более упрощенным фрагментом алгоритма Деккера. 

\begin{equation*}
\inarrII{
   \writeInst{}{x}{1}   \\
   \readInst{}{r_1}{y}  \\
}{
  \writeInst{}{y}{1}   \\
  \readInst{}{r_2}{x}  \\
}
\tag{SB}\label{ex:sb}
\end{equation*}

Допустим, что язык программирования должен предоставлять
последовательно согласованную модель памяти и должен
поддерживать компиляцию в ассемблерный код процессоров семейства \IntelX.

Рассмотрим схему компиляции, которая 
компилирует инструкции чтения и записи 
разделяемых переменных в инструкцию \texttt{MOV}%
\footnote{В архитектуре \IntelX инструкция \texttt{MOV} 
используется для обычного чтения и записи в память.}. 
Такая схема компиляции является оптимальной, 
так как она не вставляет никакие дополнительные барьры памяти
и не усиливает режимы доступа обращений. 
Однако данная схема не является корректной, так как,
спецификации модели памяти~\IntelX, в частности, 
допускает для программы \ref{ex:sb} сценарий исполнения 
с результатом ${[r_1=0, r_2=0]}$, который 
не является последовательно согласованным.
Данный результат может появиться вследствие 
\emph{буферизации операций записи} --- 
операция записи ${\writeInst{}{x}{1}}$ может быть буферизована
и исполнена процессором после выполнения всех остальных инструкций программы.

С другой стороны, рассмотрим схему компиляции, 
которая вставляет инструкцию \texttt{mfence}
%% \footnote{В архитектуре \IntelX инструкция \texttt{mfence} 
%% является барьером памяти.}. 
после каждой операции записи%
~\cite{Sewell-al:CACM10, Batty-al:POPL11}.
Инструкция \texttt{mfence} является специальным барьером памяти 
в системе команд процессоров \IntelX. 
Выполнение данной инструкции приводит к сбросу буфера записей в основную память. 
Для программы \ref{ex:sb}, скомпилированной описанным выше способом,
результат ${[r_1=0, r_2=0]}$ уже является запрещенным 
моделью памяти процессоров \IntelX. 
Таким образом, альтернативная схема компиляции 
является корректной, но не оптимальной%
\footnote{На практике использование данной схемы 
компиляции может приводить к замедлению 
на 10-30\%~\cite{Marino-al:PLDI11, Liu-al:OOPSLA17}.}. 

К сожалению, модель последовательной согласованности 
не обладает оптимальной и корректной схемой компиляции 
для современных мультипроцессоров семейств 
\IntelX, \ARM и \POWER.
Это является одной из причин ослабления моделей памяти 
высокопроизводительных языков программирования. 

\subsubsection*{Корректность трансформаций кода}

Другим немаловажным требованием, предъявляемым к моделям памяти, 
является корректность трансформаций исходного кода, 
то есть правил переписывания исходного кода, 
применяемых при оптимизации программы компилятором.

\emph{Корректность} трансформации гарантирует,
что множество сценариев поведения программы, 
полученной после применения трансформации, 
является подмножеством допустимых сценариев 
поведения оригинальной программы.

Возвращаясь к программе \ref{ex:sb},
вновь рассмотрим модель последовательной согласованности 
и трансформацию \emph{перестановки независимых инструкций}.
Допустим что данная трансформация применяется к левому потоку 
и переставляет местами операции записи и чтения,
как показано ниже. 

\bigskip

\begin{minipage}{0.42\linewidth}
\begin{equation*}
\inarrII{
   \writeInst{}{x}{1}   \\
   \readInst{}{r_1}{y}  \\
}{
  \writeInst{}{y}{1}   \\
  \readInst{}{r_2}{x}  \\
}
% \tag{SB}\label{ex:sb-src}
\end{equation*}
\end{minipage}\hfill%
\begin{minipage}{0.05\linewidth}
\Large~\\ $\leadsto$
\end{minipage}\hfill%
\begin{minipage}{0.42\linewidth}
\begin{equation*}
\inarrII{
   \readInst{}{r_1}{y}  \\
   \writeInst{}{x}{1}   \\
}{
  \writeInst{}{y}{1}   \\
  \readInst{}{r_2}{x}  \\
}
% \tag{SBtr}\label{ex:sb-tgt}
\end{equation*}
\end{minipage}

\bigskip

Для исходной версии программы (слева), 
результат $[r_1=0, r_2=0]$ \textbf{не является} 
последовательно согласованным, но для трансформированной 
версии программы этот результат уже \textbf{является} 
последовательно согласованным. 
Из чего можно сделать вывод, что перестановка независимых инструкций 
не является корректной трансформацией с точки зрения 
модели последовательной согласованности. 

В теории моделей памяти рассматривается вопрос корректности 
широкого набора базовых трансформаций.
Подробный список этих трансформаций с пояснениями 
может быть найден в работе~\cite{Moiseenko-al:PCS21}.
В данной работе будут обсуждаться только некоторые 
конкретные трансформации, которые будут вводиться по мере необходимости. 

%% Далее от требований о поддержке эффективной компиляции
%% и возможности проводить различные оптимизации, 
%% которые влекут к ослаблению модели памяти,
%% перейдем к требованиям о предоставляемых гарантиях, 
%% которые наоборот влекут к необходимости усиления модели памяти.

\subsubsection*{Гарантии для программ свободных от гонок}

Наиболее базовая гарантия, ожидаемая от модели памяти, 
требует, чтобы для программ, не содержащих \emph{гонок по данным}%
\footnote{Напомним, что гонкой по данным называется пара конкурентных 
обращений к одной и той же разделяемой переменной,
такая что как минимум одно из этих обращений является операцией записи.} 
допускались только последовательно согласованные сценарии исполнения. 
Это свойство также называется \emph{свободой от гонок}
(\emph{data-race freedom}, \DRF)~\cite{Manson-al:POPL05}.

В чуть более формальной формулировке, утверждается, 
что слабая модель памяти $M$ удовлетворяет свойству \DRF
если для любой программы $P$, которая не содержит 
гонок ни в одном последовательно согласованном сценарии исполнения,
модель $M$ допускает только последовательно согласованном сценарии%
\footnote{Свойство свободы от гонок в приведенной выше формулировке
также называется \DRFM{SC} по названию модели памяти 
sequential consistency. 
Можно также рассматривать свойство \DRF от другой 
произвольной модели $\mathsf{M}$, в этом случае это 
свойство называется \DRFM{M}.}.

Таким образом, свойство \DRF позволяет свести рассуждения о поведении 
многопоточной программы в слабой модели к рассуждениям о поведении 
этой же программы в более простой модели последовательной согласованности.
Для этого достаточно показать, что программа не имеет гонок в модели \SC. 

\subsubsection*{Спекулятивное исполнение}

Модели памяти также можно разделить по тому, 
требуют ли они спекулятивного исполнения инструкций или нет.
Рассмотрим еще один пример. 

\bigskip

\begin{equation*}
\inarrII{
  \readInst{}{r_1}{x}     \\
  \writeInst{}{y}{1}      \\
}{
  \readInst{}{r_2}{y}     \\
  \writeInst{}{x}{r_2}    \\
}
\tag{LB}\label{ex:lb-spec}
\end{equation*}

\bigskip

Некоторые модели памяти, в частности, 
модели семейств мультипроцессоров \ARM и \POWER,
допускают для данной программы сценарий поведения, 
ведущий к результату ${[r_1=1, r_2=1]}$. 
Однако данный результат не может быть получен 
путем исполнения инструкции согласно их 
порядку внутри потоков (\emph{in-order execution}).
Для того чтобы получить этот результат, необходимо 
использовать \emph{спекулятивное исполнение}
(\emph{speculative execution})~\cite{Boudol-Petri:ESOP10,Boehm-Demsky:MSPC14}.
Например, данный результат можно получить, если 
буферизировать операцию чтения $\readInst{}{r_1}{x}$ в левом потоке
и исполнить инструкцию записи $\writeInst{}{y}{1}$ вне очереди%
\footnote{Из этого происходит название приведенной программы --- 
буферизация операции чтения (\emph{load buffering}, \ref{ex:lb-spec})}.

Важно отметить, что неограниченное использование 
спекулятивного исполнения может привести к нежелательным последствиям. 
Чтобы продемонстрировать проблему, рассмотрим следующий вариант 
программы с буферизацией операции чтения.

\bigskip

\begin{equation*}
\inarrII{
  \readInst{}{r_1}{x}   \\
  \writeInst{}{y}{r_1}  \\
}{
  \readInst{}{r_2}{y}   \\
  \writeInst{}{x}{r_2}  \\
}
\tag{LB+dep}\label{ex:lb+dep-spec}
\end{equation*}

\bigskip

Сценарий исполнения, в котором сначала происходит 
спекулятивное исполнение операции записи 
в переменную \tcode{y} значения \tcode{1} в левом потоке, 
затем чтение этого значения и его запись в переменную \tcode{x}
в правом потоке, а затем вновь чтение его обратно в левом потоке, 
ведет к циклу причинно-следственных связей 
и неожиданному результату ${[r_1=1, r_2=1]}$.
Прочитанные значения \tcode{1} в примере выше 
также называются \emph{значениями из воздуха} 
(\emph{out of thin-air})~\cite{Batty-al:ESOP15}.
Для того чтобы запретить появление подобных значений 
из воздуха модель памяти должна ограничивать использование
спекулятивного исполнения. 
Более подробно возможные решения этой проблемы 
обсуждаются в разделах \ref{sec:models-classes}, 
\ref{sec:exec-graphs} и \ref{sec:wkmo-eventstruct}.

\subsubsection*{Поддержка методов автоматической верификации программ}

Многопоточные программы являются источниками нетривиальных
трудновоспроизводимых ошибок. 
Тестирование многопоточных программ, как правило, 
оказывается недостаточно эффективным методом поиска 
и предотвращения ошибок из-за недетерминированного 
поведения данного типа программ.
В контексте слабых моделей памяти эта проблема встает еще более остро 
из-за того, что слабые модели памяти допускают еще больше 
возможных сценариев поведения программы. 

По этой причине крайне актуальной становится проблема
разработки средств автоматической верификации программ, 
например, методом проверки моделей~\cite{Baier:2008},
которые учитывали бы слабые сценарии поведения.
Однако для некоторых моделей памяти проблема верификации 
программ оказываются слишком вычислительно сложной, 
из-за огромного пространства возможный состояний программы. 
В частности, процесс верификации может быть существенно затруднен, 
если модель памяти допускает спекулятивное исполнение.
Более подробно эта проблема обсуждается в разделе \ref{sec:models-classes}.  

\subsection{Классы моделей памяти}
\label{sec:models-classes}

Существующие слабые модели памяти языков программирования 
можно разбить на несколько классов в зависимости от того, 
поддерживают ли они оптимальную схему компиляции, 
насколько широкий класс трансформаций они поддерживают 
и насколько сильные гарантии для рассуждения о поведении многопоточных программ
они предоставляют. 

В данном исследовании будем рассматривать четыре класса моделей памяти: 
\emph{модели, сохраняющие программный порядок}
(\emph{program order preserving models}); 
\emph{модели, сохраняющие синтаксические зависимости} 
(\emph{syntactic dependency preserving models});
\emph{модели, сохраняющие семантические зависимости} 
(\emph{semantic dependency preserving models}) и 
\emph{модели, допускающие значения из воздуха} 
(\emph{out of thin-air models}).
Чтобы продемонстрировать разницу между этими классами 
будем использовать несколько вариаций программы 
буферизации операции чтений 
\ref{ex:lb-nodep}, \ref{ex:lb-fakedep} и \ref{ex:lb-dep}, 
которые приведены ниже. 

\begin{center}
\begin{minipage}{.32\linewidth}
{\small
\begin{equation}
\inarrII{
  \readInst{}{a}{x} \rfcomment{1} \\
  \writeInst{}{y}{1} \\
}{\readInst{}{b}{y} \rfcomment{1} \\
  \writeInst{}{x}{b}  \\
}%
\tag{LB-nodep}\label{ex:lb-nodep}
\end{equation}
}
\end{minipage}
%
\hfill\vline\hfill
\begin{minipage}{.32\linewidth}
{\small
\begin{equation}
\inarrII{
  \readInst{}{a}{x} \rfcomment{1} \\
  \writeInst{}{y}{1 + a * 0} \\
}{\readInst{}{b}{y} \rfcomment{1} \\
  \writeInst{}{x}{b}  \\
}
\tag{LB-fakedep}\label{ex:lb-fakedep}
\end{equation}
}
\end{minipage}
%
\hfill\vline\hfill
%
\begin{minipage}{.32\linewidth}
{\small
\begin{equation}
\inarrII{
  \readInst{}{a}{x} \nocomment{1} \\
  \writeInst{}{y}{a} \\
}{\readInst{}{b}{y} \nocomment{1} \\
  \writeInst{}{x}{b}  \\
}
\tag{LB-dep}\label{ex:lb-dep}
\end{equation}
}
\end{minipage}
\end{center}


\subsubsection*{Модели памяти, сохраняющие программный порядок}

В рамках моделей, сохраняющих программный порядок, 
эффекты от выполнения обращений к разделяемой памяти
наблюдаются потоками согласно их \emph{программному порядку}
то есть в линейном порядке в котором соответствующие 
инструкции обращения к памяти расположены в потоке%
\footnote{Заметим, что при этом данный класс моделей 
все же позволяет операциям чтения наблюдать ``устаревшие'' значение, 
в отличие от модели последовательной согласованности.}. 
Другими словами, в рамках данных моделей запрещено 
спекулятивное исполнение инструкций. 
Модели памяти, принадлежащие к этому классу 
запрещают сценарий поведения с результатом ${[r_1=1,r_2=1]}$
для всех трех программ \ref{ex:lb-nodep}, \ref{ex:lb-fakedep} и \ref{ex:lb-dep}.

Модели памяти, сохраняющие программный порядок, 
предоставляют довольно много гарантий о поведении программ. 
В частности, они обладают свойством свободы от гонок \DRF, 
запрещают спекулятивное исполнение и, следовательно, 
появление значений из воздуха~\cite{Lahav-al:PLDI17}. 
Также для данного класса моделей разработаны эффективные 
алгоритмы автоматической верификации 
методом проверки моделей~\cite{Kokologiannakis-al:POPL2017, Kokologiannakis:PLDI2019}.
С другой стороны, данный класс моделей не поддерживает
оптимальную схему компиляции в модели мультипроцессоров
\ARM и \POWER, а также не поддерживает трансформацию 
перестановки операции чтения после независимой
от нее операции записи (\emph{load/store reordering}). 

К данному классу относится модель~\RCMM~\cite{Lahav-al:PLDI17}, 
покрывающая подмножество сценариев поведения, допустимых моделью памяти языка \CLANG,
модель \TSO процессоров семейства \Intel~\cite{Sewell-al:CACM10},
модели последовательной согласованности (sequential consistency)~\cite{Lamport:TC79},
причинной согласованности (causal consistency)~\autocite{Lahav-Boker:PLDI2020}
и согласованности в конечном счёте (eventual consistency)~\cite{Jagadeesan-al:ESOP2018},
а также, например, модель памяти языка \OCaml~\cite{Dolan-al:PLDI18}.

\subsubsection*{Модели памяти, сохраняющие синтаксические зависимости}

Модели памяти, сохраняющих синтаксические зависимости, 
ослабляют требование линейности программного порядка 
и вводят понятие частичного \emph{сохраняемого программного порядка}. 
Операции обращения к разделяемой памяти из одного и того же потока
находятся в отношении сохраняемого программного порядка 
если между соответствующими им инструкциями есть 
\emph{синтаксические зависимости}, например, 
\emph{зависимость по данным} или \emph{по управлению}. 

Например, в программе \ref{ex:lb-nodep} между инструкциями в 
левом потоке нет синтаксической зависимости, 
поэтому эти инструкции могут быть выполнены в произвольном порядке. 
Следовательно, модель памяти, сохраняющая синтаксические зависимости, 
допускает сценарий поведения с результатом ${[r_1=1,r_2=1]}$ для этой программы. 
В то же время этот результат запрещен для программ 
\ref{ex:lb-dep} и \ref{ex:lb-fakedep}, 
поскольку в этих программах существует зависимость 
между инструкциями в обоих потоках. 

Модели данного класса поддерживают оптимальные схемы компиляции
в модели современных мультипроцессоров. 
Также, в отличие от моделей, сохраняющих программный порядок, 
данный класс моделей допускает переупорядочивание 
инструкции чтения после независящей от нее инструкции записи. 
Но при этом, данный класс запрещает множество других трансформаций
которые могут удалять синтаксические зависимости между инструкциями. 
Примером такой трансформации является \emph{свертка констант}~\cite{Muchnick:ACDI97}.
Можно видеть, например, что в случае программы 
свертка констант может преобразовать инструкцию 
$\writeInst{}{y}{1 + a * 0}$ в инструкцию $\writeInst{}{y}{1}$, 
удалив зависимость от предшествующей инструкции чтения $\readInst{}{a}{x}$.
После применения этой трансформации сценарий поведения с результатом ${[r_1=1,r_2=1]}$
становится допустим моделью памяти.  

Модели памяти, сохраняющие синтаксические зависимости, 
предоставляют более слабые гарантии, по сравнению с 
моделями, сохраняющими программный порядок. 
Эти модели все еще обладают свойством свободы от гонок (\DRF), 
но в отличие от моделей предыдущего класса 
допускают спекулятивное исполнение 
синтаксический независимых инструкций. 
Несмотря на это, для данного класса моделей все же 
существуют достаточно эффективные алгоритмы автоматической верификации 
методом проверки моделей%
~\cite{Abdulla-al:CAV2016,Pulte-al:PLDI2019,Kokologiannakis-Vafeiadis:ASPLOS2020}.

Модели, сохраняющие синтаксические зависимости,
практический не используются как модели памяти для
языков программирования, в частности, потому что 
модели этого класса запрещают применение широкого 
класса крайне важных трансформаций (например, свертку констант).  
Вместе с тем, большинство моделей современных мултьтипроцессоров, 
например, \ARM~\cite{Pulte-al:POPL18} и \POWER~\cite{Sarkar-al:PLDI11}, 
попадают именно в этот класс. 
Также к данному классу относится 
модель памяти ядра \Linux~\cite{Alglave-al:ASPLOS18}

\subsubsection*{Модели памяти, сохраняющие семантические зависимости}

Модели памяти, сохраняющие семантические зависимости,
отслеживают вместо синтаксических зависимостей между операциями
\emph{семантические зависимости}.
Например, в случае программы \ref{ex:lb-fakedep} можно сказать,
что хотя между инструкциями $\readInst{}{a}{x}$ и $\writeInst{}{y}{1 + a * 0}$
есть синтаксическая зависимость, семантический они независимы,
так как на самом деле записываемое значение $1 + a * 0$
не зависит от значения $a$, прочитанного первой инструкцией.
Таким образом, модели данного класса допускают
сценарий исполнения с результатом ${[r_1=1,r_2=1]}$ для программ
\ref{ex:lb-nodep} и \ref{ex:lb-fakedep}, но не для \ref{ex:lb-dep}.

Модели памяти, сохраняющие семантические зависимости,
поддерживают оптимальные схемы компиляции и
применение широкого класса различных трансформаций программ.
В то же время данные модели предоставляют гарантию \DRF,
но также допускают спекулятивное исполнение.
Модели данного класса использую различные
концептуально сложные формализмы, чтобы дать строгое определение
понятию семантических зависимостей между операциями.
Эта сложность, в частности, приводит к тому,
что автоматическая верификация программ в этих моделях
существенно затруднена, а вопрос построения
инструментов для верификации программ не изучен.

К данному классу относятся различные модели,
предложенные в качестве моделей памяти для
языков \CPP и \Java, в частности, модели
\Prm~\cite{Kang-al:POPL17},
\Wkm~\cite{Chakraborty-Vafeiadis:POPL19}, 
\PwP~\cite{Jagadeesan-al:OOPSLA2020}
и другие~\cite{Jeffrey-Riely:LICS16,PichonPharabod-Sewell:POPL16,Paviotti-al:ESOP20}.

\subsubsection*{Модели памяти допускающие значения из воздуха}

Также упомянем класс моделей памяти допускающих значения из воздуха.
Такие модели допускают сценарий исполнения с результатом ${[r_1=1,r_2=1]}$
для всех трех программ \ref{ex:lb-nodep}, \ref{ex:lb-fakedep} и \ref{ex:lb-dep}.

Модели данного класса предоставляют оптимальные схемы компиляции и
допускают применение широкого класса различных трансформаций программ.
Однако ценой этого является возможность появления значений из воздуха.
Наличие значений из воздуха препятствует как неформальному
так строго формальному рассуждению о поведении программ,
приводит к нарушению гарантии \DRF и невозможности
построения каких-либо инструментов верификации программ%
~\cite{Boehm-Demsky:MSPC14, Batty-al:ESOP15}. 

Все эти недостатки данного класса моделей
привели к консенсусу в исследовательском сообществе,
что модели, допускающие значения из воздуха,
не подходят на роль моделей памяти 
для языков программирования.
Тем не менее, с точки зрения истории развития
теории моделей памяти, стоит отметить, например,
что первоначальная версия модели памяти для языков \CPP
допускала значения из воздуха~\cite{Batty-al:POPL11}.

\subsection{Сравнение моделей памяти и открытые проблемы}
\label{sec:models-summary}

Подведем итоги сравнения различных классов моделей памяти. 
Таблица~\ref{table:models-classes} резюмирует 
анализ классов моделей памяти, приведенный в 
предыдущем разделе.

\begin{table}[t]
\small

\newcommand{\rotateAngle}{270}

\newcolumntype{Y}{>{\centering\arraybackslash}X}

\def\arraystretch{2}
\setlength\tabcolsep{3pt} %\setlength\tabcolsep{2pt}
\setlength\extrarowheight{6pt}

\begin{center}
\begin{tabularx}{\linewidth}{|*{7}{Y|}} 

%% \hline

%% \rotatebox[origin=c]{\rotateAngle}{
%%   \makecell{Класс моделей памяти}  
%% } &

%% \rotatebox[origin=c]{\rotateAngle}{
%%   \makecell{Оптимальность\\схемы компиляции}
%% } &

%% \rotatebox[origin=c]{\rotateAngle}{
%%   \makecell{Корректность\\трансформаций}
%% } &

%% \rotatebox[origin=c]{\rotateAngle}{
%%   \makecell{Гарантии для программ\\свободных от гонок\\(\DRF)} 
%% } &

%% \rotatebox[origin=c]{\rotateAngle}{
%%   \makecell{Спекулятивное\\исполнение}
%% } &

%% \rotatebox[origin=c]{\rotateAngle}{
%%   \makecell{Поддержка методов\\автоматической верификации}
%% }

%% \\

\hline 

Сlass  & 
Compil. &
Transf.~ &
\DRF &
In-Order &
no-OOTA &
Auto. Ver.

\\
\hline

\makecell{Program\\Order\\Preserving} & 
 \badcell & \badcell & \okcell & \okcell & \okcell & \okcell

\\
\hline

\makecell{Syntax.\\Dep.\\Preserving} &
 \okcell & \badcell & \okcell & \badcell & \okcell & \okcell

\\
\hline

\makecell{Semantic\\Dep.\\Preserving} & 
 \okcell & \okcell & \okcell & \badcell & \okcell & \textbf{?}

\\
\hline

\makecell{Out of\\Thin-Air} & 
 \okcell & \okcell & \badcell & \badcell & \badcell & \badcell

\\
\hline


\end{tabularx}
\end{center}

\captionsetup{justification=centering}
\caption{Классы моделей памяти и их свойства}
\label{table:models-classes}
\end{table}



Можно видеть, что модели памяти,
сохраняющих программный порядок, или синтаксические зависимости 
предоставляют набор базовых гарантий о поведении программ, 
который включает свойство \DRF. Также для моделей данных классов 
доступны средства автоматической верификации программ.
%% ~\cite{Abdulla-al:CAV2016,Pulte-al:PLDI2019,Kokologiannakis-al:POPL2017,
%% Kokologiannakis:PLDI2019,Kokologiannakis-Vafeiadis:ASPLOS2020}.
Однако использование моделей данного класса влечет 
дополнительные накладные расходы на время исполнения программ, 
из-за того что они не поддерживают оптимальные схемы компиляции
и/или некоторые важные трансформации программ.
В дополнение к этому отметим, что модели,
сохраняющие синтаксические зависимости используют спекулятивное исполнение, 
что приводит к увеличению концептуальной сложности данных моделей 
и к росту числа допустимых сценариев исполнения программ. 

С другой стороны, модели допускающие значения из воздуха 
поддерживают оптимальные схемы компиляции и широкий спектр 
трансформации программ, но из-за наличия значений из воздуха 
рассуждение о поведении программ в этих моделях становится невозможным. 

Модели, сохраняющие семантические зависимости, стремятся 
решить это противоречие, и, с одной стороны, 
поддержать оптимальные схемы компиляции и широкий спектр 
трансформаций программ, а с другой стороны 
гарантировать отсутствие значений из воздуха 
и предоставить разумный набор гарантий о поведении программ.
Достигается эта цель с помощью использования различных 
сложных формализмов, с помощью которых строится строгое определение
семантических зависимостей между операциями.
Концептуальная сложность моделей данного класса, 
а также огромное число допустимых сценариев поведения программ,
привели к тому, что проблема построения эффективных 
средств автоматической верификации программ в 
моделях памяти данного класса до сих пор не была решена. 

Данное диссертационное исследование посвящено 
применению теории структур событий в контексте слабых моделей памяти. 
В частности, рассматривается модель \Wkm~\cite{Chakraborty-Vafeiadis:POPL19}, 
которая принадлежит классу моделей, сохраняющих семантические зависимости. 
Список свойств данной модели также приводится в таблице \ref{table:models-classes}.

Отметим, что хотя модель \Wkm и основана на 
теории структур событий, в рамках данной модели 
вводится особый класс структур событий, 
который не удовлетворяет аксиомам классической теории.
В главе \ref{ch:porf-evenstruct} данного диссертационного исследования 
приводится альтернативная семантика на основе 
классической теории структур событий,  
которая позволяет закодировать произвольную 
модель памяти, сохраняющую программный порядок.

Также отметим, что для модели \Wkm ранее не была 
доказана корректность оптимальных схем компиляции 
в модели памяти современных мультипроцессоров. 
В главе \ref{ch:weakestmo-imm} данного диссертационного исследования 
исправляется данный недостаток. 

Наконец, в главе \ref{ch:weakestmo2} предлагается новая версия модели \Wkm --- 
\WkmS~ --- которая допускает реализацию эффективных методов 
автоматической верификации программ.
На основе этого результата в главе \ref{ch:mc-weakestmo2} 
впервые предлагается инструмент 
автоматической верификации программ для модели памяти, 
сохраняющей семантические зависимости.  

\section{Формальная семантика параллельных программ и моделей памяти}

В этом разделе приводятся определения различных формализмов,
используемых для задания семантики многопоточных программ и моделей памяти.
В разделе~\ref{sec:lts} дано определение \emph{систем помеченных переходов}
и \emph{операционных семантик с чередованием} 
(\emph{interleaving operational semanitcs}).
В разделе~\ref{sec:pomsets-eventstruct} приводится альтернативный способ
задания семантики многопоточных программ без чередования 
(\emph{non-interleaving}) 
с помощью семантических доменов \emph{истинной конкурентности} 
(\emph{true concurrency}), а именно 
\emph{языков частично упорядоченных мультимножеств} и \emph{структур событий}.
В разделе~\ref{sec:exec-graphs} вводится понятие графов сценариев исполнения
и аксиоматических моделей памяти, а также приводится краткое сравнение
графов сценариев исполнения и частично упорядоченных мультимножеств.
Наконец, в разделе~\ref{sec:wkmo-eventstruct} вводится
специальный тип структур событий, используемых в модели \Wkm.

\subsection{Системы помеченных переходов}
\label{sec:lts}

Системы помеченных переходов являются традиционным 
способом задания операционной семантики. 
Системы помеченных переходов это (потенциально бесконечный) граф, 
вершины в котором представляют внутреннее состояния системы, а
ребра соответствуют выполнению шага вычислений. 
Метка ребра задает видимый эффект выполнения данного шага вычислений.
%% Множество трасс автомата задает его ``последовательную'' спецификацию, 
%% а список меток, индуцируемый трассой, определяет
%% наблюдаемое поведение автомата. 

\begin{definition}
  \label{def:lts}
  \emph{Система помеченных переходов} --- это тройка
    $\LTS \defeq \tup{\State, \Label, \TrRel}$, где 
  \begin{itemize}
    \item $\State$ --- множество состояний;
    \item $\Label$ --- множество меток, также называемое \emph{алфавитом};
    \item $R \subseteq L \times S \times S$ --- помеченное отношение перехода.
  \end{itemize}
\end{definition}

Для обозначения наличия перехода между состояниями используется следующая нотация:
\[
\begin{array}{lcr@{\hspace{3em}}lcr}
  \ltr[R]{\ell}{s}{s'} & \defeq & \step{\ell}{s}{s'} \in \TrRel                     &
  \tr[R]{s}{s'}        & \defeq & \exists \ell \ldotp \step{\ell}{s}{s'} \in \TrRel \\
\end{array}
\]

\begin{definition}
  \label{def:lts-trace}
  \emph{Трассой} помеченной системой переходов называется чередующаяся последовательность  
  состояний $s_0, s_1, \ldots, s_n \in \Label$ 
  и меток $\ell_1, \ldots, \ell_n \in L$, 
  такая что выполняется условие
  $$s_0 \xrightarrow{\ell_1} s_1 \xrightarrow{\ell_2} s_2 \xrightarrow{\ell_3} \ldots \xrightarrow{\ell_n} s_n$$
\end{definition}

\begin{definition}
  \label{def:lts-lang}
  \emph{Язык принимаемый системой переходов в начальном состоянии $s_0$} ---
  это множество последовательностей слов над алфавитом $\Label$, 
  таких что для каждого слова существует соответствующая 
  трасса, начинающаяся в $s_0$:
  $$ \langof{\LTS, s_0} \defeq \set{ 
      \ell_1 \ldots \ell_n ~|~ \exists s_1, \ldots, s_n \ldotp 
       s_0 \xrightarrow{\ell_1} s_1 \xrightarrow{\ell_2} \ldots \xrightarrow{\ell_n} s_n
     } 
  $$
\end{definition}
  
\begin{figure}[t]
  \centering

    \begin{subfigure}[t]{0.48\textwidth}
    \centering
    \begin{tikzpicture}[xscale=1.5,yscale=1.5]

      \tikzset{
        smallstate/.style={state,
          inner sep=3pt,
          outer sep=3pt,
          minimum size=0pt,
        },
      }

      \node[smallstate] (s0) at (1,3) {$s_0$};
      \node[smallstate] (s1) at (1,2) {$s_1$};
      \node[smallstate] (s2) at (0,1) {$s_2$};
      \node[smallstate] (s3) at (2,1) {$s_3$};
      \node[smallstate] (s4) at (1,0) {$s_4$};

      \draw[->] (s0) edge[right,pos=0.5] node {$a$} (s1);

      \draw[->] (s1) edge[bend right,above,pos=0.5] node {$c$} (s2);
      \draw[->] (s1) edge[bend left ,below,pos=0.5] node {$d$} (s2);

      \draw[->] (s1) edge[right,pos=0.5] node {$b$} (s3);

      \draw[->] (s2) edge[left,pos=0.5]  node {$b$} (s4);

      \draw[->] (s3) edge[bend right,above,pos=0.5] node {$c$} (s4);
      \draw[->] (s3) edge[bend left ,below,pos=0.5] node {$d$} (s4);

    \end{tikzpicture}

    \caption{Система помеченных переходов}
    \label{fig:lts-ex}
    \end{subfigure}
    \hfill
    %
    \begin{subfigure}[t]{0.48\textwidth}
    \centering
    \begin{tikzpicture}[xscale=1.5,yscale=1.5]
      \node (a1) at (0, 2) {$a$};
      \node (b1) at (0, 1) {$b$};
      \node (c1) at (0, 0) {$c$};

      \draw[po] (a1) edge (b1);
      \draw[po] (b1) edge (c1);

      \node (a2) at (1, 2) {$a$};
      \node (c2) at (1, 1) {$c$};
      \node (b2) at (1, 0) {$b$};

      \draw[po] (a2) edge (c2);
      \draw[po] (c2) edge (b2);

      \node (a3) at (2, 2) {$a$};
      \node (b3) at (2, 1) {$b$};
      \node (d3) at (2, 0) {$d$};

      \draw[po] (a3) edge (b3);
      \draw[po] (b3) edge (d3);

      \node (a4) at (3, 2) {$a$};
      \node (d4) at (3, 1) {$d$};
      \node (b4) at (3, 0) {$b$};

      \draw[po] (a4) edge (d4);
      \draw[po] (d4) edge (b4);
    \end{tikzpicture}

    \caption{Язык принимаемый системой переходов в начальном состоянии $s_0$.}
    \label{fig:lang-ex}
    \end{subfigure}

  \label{fig:lts-lang-ex}
  \caption{
    Пример системы помеченных переходов и принимаемого ей языка
  }
%% как обычного языка,
%%     языка помеченных частично упорядоченных мультимножеств
    %% и структуры событий. 
\end{figure}


На рисунке~\ref{fig:lts-ex} показан пример системы помеченных переходов, 
а на рисунке~\ref{fig:lang-ex} пример языка, 
принимаемого этой системой в состоянии $s_0$.

В рамках так называемой операционной семантики с чередованием
(interleaving semantics) поведение многопоточной программы
определяется как поочередное исполнение атомарных действий параллельных потоков.

\begin{definition}
  \label{def:lts-par}
  \emph{Параллельной композицией двух систем переходов} $\LTS_1$ и $\LTS_2$
  будем называть систему переходов
  $\parlts{\LTS_1}{\LTS_2} \defeq \tup{\State_1 \times \State_2, \Label, \TrRel_{\parSymb}}$
  где $\TrRel_{\parSymb}$ определяется следующим образом:
  \begin{itemize}
    \item $\ltr[\TrRel_1]{\ell}{s_1}{s'_1}$ влечет
          $\ltr[\TrRel_{\parSymb}]{\ell}{\tup{s_1, s_2}}{\tup{s'_1, s_2}}$, и
    \item $\ltr[\TrRel_2]{\ell}{s_2}{s'_2}$ влечет
          $\ltr[\TrRel_{\parSymb}]{\ell}{\tup{s_1, s_2}}{\tup{s_1, s'_2}}$.
  \end{itemize}
\end{definition}

Если при этом потоки имеют доступ к общему ресурсу 
(например, разделяемой памяти), то в таком случае можно
семантику ресурса также задать с помощью системы переходов
а затем рассмотреть произведение параллельной композиции потоков и ресурса.  

\begin{definition}
  \label{def:lts-par}
  \emph{Произведением двух систем переходов} $\LTS_1$ и $\LTS_2$
  будем называть систему переходов
  $\prodlts{\LTS_1}{\LTS_2} \defeq \tup{\State_1 \times \State_2, \Label, \TrRel_{\prodSymb}}$
  где $\TrRel_{\prodSymb}$ определяется следующим образом:
  \begin{itemize}
    \item $\ltr[\TrRel_{\prodSymb}]{\ell}{\tup{s_1, s_2}}{\tup{s'_1, s'_2}}$ 
      тогда и только тогда, когда 
      $\ltr[\TrRel_1]{\ell}{s_1}{s'_1}$ и $\ltr[\TrRel_2]{\ell}{s_2}{s'_2}$.
  \end{itemize}
\end{definition}

Таким образом, если система помеченных переходов $\LTS_{\thrdSymb}$ 
задает семантику потоков, а система $\LTS_{\resSymb}$ --- 
семантику разделяемого ресурса, тогда 
$(\LTS_{\thrdSymb} \parSymb \dots \parSymb \LTS_{\thrdSymb}) \prodSymb \LTS_{\resSymb}$
задает семантику всей системы, состоящей из $n$ потоков и разделяемого ресурса.

\subsection{Языки помсетов и простые структуры событий}
\label{sec:pomsets-eventstruct}

Операционные семантики с чередованием представляют 
простой и интуитивно понятный подход для моделирования
многопоточных программ. Однако его недостаток заключается в том, 
что с ростом программы экспоненциально растет количество трасс, 
допустимых операционной семантикой. 

В попытке преодолеть эту проблему, исследователями 
были предложены различные альтернативные 
подходы к заданию семантики многопоточных программ, 
которые позволяют более компактно представить пространство 
возможных сценариев поведения таких программ. 
Данный класс семантик принято называть 
\emph{семантиками без чередования} (\emph{non-interleaving semantics})
или также \emph{истинно конкурентными семантиками}
(\emph{true concurrent semantics})~\cite{Nielsen:REX93}.
Из данного класса семантик в рамках 
данной диссертации особый интерес представляют 
\emph{языки частично упорядоченных мультимножеств}
(кратко --- языки помсетов)~\cite{Pratt:CONCUR84,Gischer:TCS88}, 
и \emph{структуры событий}~\cite{Winskel:86}.

Языки помеченных частично упорядоченных множеств является
обобщением понятия обычных ``последовательных'' языков, 
то есть множества слов данного алфавита. 
Обобщение заключается в переходе от линейного порядка 
на символах алфавита в рамках слова к частичному порядку.
Частично упорядоченное множество соответствует одному
сценарию исполнения многопоточной программы.
Элементы этого множества представляют
атомарные шаги вычисления и называются \emph{событиями}.
Каждому событию ставится в соответствие семантическая \emph{метка} ---
символ заданного алфавита.
Если событие $e_1$ упорядочено перед ~$e_2$, $e_1 \ca e_2$, 
тогда считается что событие~$e_2$ в
сценарии исполнения программы зависит от события~$e_1$.
Если не выполняется ни $e_1 \ca e_2$ ни $e_2 \ca e_1$ 
тогда события $e_1$ и $e_2$ считаются параллельными, 
$e_1 \co e_2$. 

\begin{definition}
  \label{def:lposet}
  \emph{Помеченное частично упорядоченное множество} над множеством меток $\Label$, 
  это тройка $\tup{\Event, \lab, \ca}$, где 
  \begin{itemize}
    \item $\Event$ это множество \emph{событий};
    \item $\lab : \Event \fun \Label$ \emph{функция разметки событий};
    \item $\ca \subseteq \Event \times \Event$ это частичный порядок 
      \emph{причинно-следственной связи} между событиями. 
  \end{itemize}
\end{definition}

Множество всех помеченных частично упорядоченных множеств над алфавитом $\Label$
будем обозначать как $\lPoset[\Label]$. 

Отметим, что при работе c помеченными частично упорядоченными множествами
конкретные идентификаторы событий как правило неважны,
важна лишь их разметка и отношение причинно-следственной связи между ними.
Таким образом, с помеченными частично упорядоченными множествами 
работают по модулю переименования событий, то есть с точностью до изоморфизма.

Рассмотрим $p, q \in \lPoset[\Label]$. Функция $f : E_p \fun E_q$ называется:
\begin{itemize}
  \item \emph{сохраняющей метки} если ${\lab_q(f(e)) = \lab_p(e)}$;
  \item \emph{сохраняющей порядок} если ${e_1 \ca_p e_2}$ влечет ${f(e_1) \ca_q f(e_2)}$;
  \item \emph{вкладывающей порядок} если ${e_1 \ca_p e_2}$ тогда и только тогда, когда ${f(e_1) \ca_q f(e_2)}$.
\end{itemize}

\begin{definition}
  \label{def:lposet-hom}
  \emph{Гомоморфизмом} частично упорядоченных помеченных множеств называется
  функция, сохраняющая метки и порядок. 
\end{definition}

\begin{definition}
  \label{def:lposet-subs}
  Будем говорить что частично упорядоченных помеченное множество 
  $p$ \emph{поглощается} $q$, или что $p$ \emph{более упорядочено чем} $q$, 
  что обозначается как $p \subs q$, если существует гомоморфизм из $q$ в $p$.
\end{definition}

\begin{definition}
  \label{def:lposet-iso}
  \emph{Изоморфизмом} частично упорядоченных помеченных множеств называется
  биективная функция, сохраняющая метки и вкладывающая порядок. 
\end{definition}

\begin{definition}
  \label{def:lposet-subs}
  Будем говорить что частично упорядоченные помеченные множества
  $p$ и $q$ \emph{изоморфны}, что обозначается как $p \iso q$,
  если существует изоморфная функция между ними.
\end{definition}

\begin{definition}
  \label{def:pomset}
  \emph{Помеченные частично упорядоченные мультимножества}, 
  или, кратко, \emph{помсеты} (от англ. \emph{partially ordered multiset, pomset}), 
  это классы помеченных частично упорядоченных множеств по модулю изоморфизма: 
  $${\Pom[\Label] \defeq \lPoset[\Label] / {\iso}}.$$ 
\end{definition}

\begin{definition}
  \label{def:pomset}
  \emph{Язык помсетов} --- это множество помсетов: 
  $${\Pomlang[\Label] \defeq \pwset{\Pom[\Label]}}.$$ 
\end{definition}

\begin{figure}[t]
  \centering
  \begin{subfigure}[t]{0.48\textwidth}
    \centering
    \begin{tikzpicture}[xscale=1.5,yscale=1.5]
      \node (a1) at (1, 1) {$a$};
      \node (b1) at (0, 0) {$b$};
      \node (c1) at (2, 0) {$c$};

      \draw[po] (a1) edge (b1);
      \draw[po] (a1) edge (c1);

      \node (a2) at (4, 1) {$a$};
      \node (b2) at (3, 0) {$b$};
      \node (d2) at (5, 0) {$d$};

      \draw[po] (a2) edge (b2);
      \draw[po] (a2) edge (d2);
    \end{tikzpicture}
    \caption{Язык помсетов}
    \label{fig:pom-ex}
  \end{subfigure}
  \hfill
  \begin{subfigure}[t]{0.48\textwidth}
    \centering
    \begin{tikzpicture}[xscale=1.5,yscale=1.5]
      \node (a) at (1, 1) {$a$};
      \node (b) at (0, 0) {$b$};
      \node (c) at (1, 0) {$c$};
      \node (d) at (2, 0) {$d$};

      \draw[po] (a) edge (b);
      \draw[po] (a) edge (c);
      \draw[po] (a) edge (d);
      \draw[cf] (c) -> (d);
    \end{tikzpicture}
    \caption{Простая структура событий}
    \label{fig:es-ex}
  \end{subfigure}

  \label{fig:pom-es-ex}
  \caption{
    Пример кодирования языка системы переходов 
    как языка помеченных частично упорядоченных мультимножеств
    и как простой структуры событий. 
  }
\end{figure}


На рисунке~\ref{fig:pom-ex} можно видеть пример языка помсетов. 
Этот язык помсетов кодирует обычный язык, 
показанный на рисунке~\ref{fig:lang-ex}, 
так как каждое слово из обычного языка является дополнением некоторого 
частичного упорядоченного множества из языка помcетов
до линейно упорядоченного множества. 
Формально, связь языка помсетов и обычного языка можно установить 
с помощью понятия \emph{линеаризации помсета}.

\begin{definition}
  \label{def:pomset-subs}
  Помсет $p$ \emph{поглощается} $q$, $p \subs q$, 
  если существует биективный гомоморфизм из $q$ в $p$.
  В таком случае также говорят, что $p$ более упорядочено чем $q$.
\end{definition}

\begin{definition}
  \label{def:pomset-lin}
  Помсет $p$ является \emph{линеаризацией} $q$, 
  что обозначается как $p \in \Lin{q}$,
  если $p$ является линейно упорядоченным и 
  поглощается $q$, $p \subs q$.
\end{definition}

Линеаризация языка помсетов $P$ определяется 
как объединение линеаризации всех входящих в язык помсетов:
$$ \Lin{P} \defeq \bigcup_{p \in P}\Lin{p} $$
Наконец, можно сказать что язык помсетов $P \in \Pomlang[\Label]$
соответствует обычному языку $L \in \Lang{\Label}$, если $\Lin{P} = L$.

Множество помсетов можно объединить в одну \emph{структуру событий},
и таким образом представить язык помсетов как одно частично упорядоченное множество.
Существует множество видов структур событий~\cite{}, 
в контексте данной работы будем рассматривать класс \emph{простых структур событий}.
Везде далее под термином \emph{структура событий} будем подразумевать 
именно простую структуру событий, если иное не сказано явно. 

По сравнению с помеченным частично упорядоченным множеством, 
простые структуры событий позволяют дополнительно выразить тот факт, 
что два события $e_1$ и $e_2$ находятся в конфликте.
Это означает, что эти два события не могут одновременно 
принадлежать одному сценарию исполнения программы. 

\begin{definition}
  \label{def:lposet-dwfin}
  Будем говорить, что помеченное частично упорядоченное множество 
  $p = \tup{\Event, \lab, \ca}$ является \emph{префикс конечным} 
  если каждое событие имеет конечное число предшественников, 
  то есть для любого $e \in \Event$ множество 
  $\dwset{e} \defeq \set{e' ~|~ e' \ca e}$ конечно.
\end{definition}

\begin{definition}
  \label{def:prime-es}
  \emph{Простая структура событий с бинарным конфликтом} над множеством меток $\Label$ 
  это кортеж $\tup{\Event, \lab, \ca, \cf}$, где 
  $\tup{\Event, \lab, \ca}$ это префикс-конечное помеченное 
  частично упорядоченное множество, 
  а $\cf \suq \Event \times \Event$ --- это \emph{бинарное отношение конфликта}, 
  которое является иррефлексивным, симметричным и 
  удовлетворяет свойству \emph{наследственности}:
  $$ e_1 \cf e_2 ~\text{и}~ e_2 \ca e_3 ~\text{влечет}~ e_1 \cf e_3.$$
\end{definition}

Отметим, что зачастую язык помсетов может иметь более сложную структуру 
конфликтности между событиями, которая не может быть сведена 
к бинарному конфликту между парой событий. 
В таком случае рассматривают простые структуры событий более общего вида. 

\begin{definition}
  \label{def:prime-cons-es}
  \emph{Простая структура событий с предикатом консистентности} над множеством меток $\Label$ 
  это кортеж $\tup{\Event, \lab, \ca, \Cons}$, где 
  $\tup{\Event, \lab, \ca}$ это префикс-конечное помеченное 
  частично упорядоченное множество, 
  а $\Cons \suq \pwfset{\Event}$ --- это \emph{предикат консистентности}, 
  который должен удовлетворять следующим условиям:
  \begin{enumerate}
    \item \label{ax:prime-cons-emp}
      $\emptyset \in \Cons$,
    \item \label{ax:prime-cons-subs}
      $X \subseteq Y$ и $Y \in \Cons$ влечет $X \in \Cons$,
    \item \label{ax:prime-cons-ca}
      $e_1 \ca e_2$ и $\set{e_2} \cup X \in \Cons$ 
      влечет $\set{e_1} \cup X \in \Cons$.
  \end{enumerate}
\end{definition}

Можно видеть, что простые структуры событий с бинарным конфликтом
являются частным случаем простых структур событий 
с предикатом консистентности. 
Действительно, для простой структуры событий с бинарным конфликтом
можно определить предикат консистентности следующим образом:
$$X \in \Cons \iff \forall e_1~e_2 \in X \ldotp \neg e_1 \cf e_2.$$

Наконец, формально определим язык помсетов, порождаемый структурой событий. 

\begin{definition}
  \label{def:es-cfg}
  Пусть $S = \tup{\Event, \lab, \ca, \Cons}$ простая структура событий 
  с предикатом консистентности. Тогда подмножество событий 
  $X \suq \Event$ называется \emph{конфигурацией} структуры $S$ 
  если оно является префикс-замкнутым, а все его конечные подмножества 
  являются консистентными, то есть 
  \begin{itemize}
    \item $\dwset{X} \defeq {e' ~|~ \exists e \in X \ldotp~ e' \ca e } \suq X$, 
    \item $Y \finsubseteq X$ влечет $Y \in \Cons$.
  \end{itemize}
\end{definition}

Будем обозначать как $\Cfg{S}$ множество всех конфигураций структуры событий $S$.

\begin{definition}
  \label{def:es-pomlang}
  Язык помсетов, порождаемый структурой событий $S = \tup{\Event, \lab, \ca, \Cons}$, 
  определяется следующим образом:
  $$ \pomlang{S} \defeq \set{p ~|~ \exists X \in \Cfg{S} \ldotp p = S\rst{X} }$$
  где $S\rst{X} \defeq {X, \lab\rst{X}, \ca\rst{X}}$ это сужение 
  структуры событий $S$ на консистентное подмножество событий $X$.
\end{definition}

\subsection{Графы сценариев исполнения}
\label{sec:exec-graphs}

Далее перейдем к описанию формализмов, используемых 
для определения моделей памяти.
Напомним, что под моделью памяти понимается
семантика многопоточной системы, оперирующей с разделяемой памятью.
Поэтому, как уже упоминалось в разделе~\ref{sec:lts},
модель памяти можно определить в терминах операционной семантики.
В таком случае система переходов
$(\LTS_{\thrdSymb} \parSymb \dots \parSymb \LTS_{\thrdSymb}) \prodSymb \LTS_{\memSymb}$
будет описывать многопоточную систему, состоящую из $n$ потоков
и разделяемой памяти, где $\LTS_{\thrdSymb}$ --- это система переходов,
описывающая поведение потоков, а $\LTS_{\memSymb}$ --- система переходов,
описывающая поведение разделяемой памяти. 

Как уже отмечалось, в контексте моделирования
многопоточных систем одним из недостатков подхода,
основанного на операционной семантике, 
является экспоненциально рост количество трасс системы.
В контексте слабых моделей памяти у данного подхода
также есть и другой недостаток.
Проблема заключается в том, что для кодирования каждой
отдельной модели памяти необходимо разработать
собственное представление этой модели памяти 
в терминах системы переходов $\LTS_{\memSymb}$.
При этом такая система может быть устроено
достаточно сложным образом и требовать моделирования
множества различных структур данных, например,
буферов операций, очередей сообщений,
многоуровневых кэшей и так далее.  

Поэтому для спецификации моделей памяти
зачастую используют альтернативный \emph{аксиоматический стиль}.
В аксиоматическом стиле задано большинство моделей
современных мультипроцессоров, например,
\Intel~\cite{Sewell-al:CACM10}, 
\POWER~\cite{Sarkar-al:PLDI11,Alglave-al:TOPLAS14}),
\ARM~\cite{Pulte-al:POPL18,Alglave-al:TOPLAS14}),
и некоторых языков программирования,
например, \OCaml~\cite{Dolan-al:PLDI18}, \JS~\cite{Watt-al:PLDI2020}.

Модель памяти в аксиоматическом стиле
определяется как множество консистентных 
\emph{графов сценариев исполнения} (\emph{execution graphs}).
В этом графе вершинами являются атомарные события,
а ребра формируют различные отношения между этими событиями.
Графы сценариев исполнения похожи на помсеты,
главное отличие между ними заключается в том, что 
помсет состоит из единственного
отношения причинно-следственной связи, 
а граф сценариев исполнения состоит из
нескольких отношений, наделенных различной семантикой.
Например, отношение \emph{программного порядка} (\emph{program order}) $\lPO$ 
задает порядок, в котором выполняются события в каждом потоке,
а отношение \emph{читает-из} (\emph{reads-from}) $\lRF$, 
для каждого события записи указывает 
какие события чтения выполняют операцию чтения из него. 
На рисунке~\ref{fig:lb-nodep-execs} показаны примеры графов сценариев исполнения 
соответствующих программе \ref{ex:lb-nodep}.

{
\newcommand{\XScale}{1}
\newcommand{\YScale}{0.7}

\begin{figure}[t]
  \begin{subfigure}[b]{.24\textwidth}\centering
  \begin{tikzpicture}[xscale=\XScale,yscale=\YScale]

  \node at (-1,1.8) {$\circledb{A}$};

  \node (init) at (1,  1.5) {$\Init$};

  \node (i11) at ( 0,  0) {$\rlab{}{x}{0}{}$};
  \node (i12) at ( 0, -2) {$\wlab{}{y}{1}{}$};

  \node (i21) at ( 2,  0) {$\rlab{}{y}{0}{}$};
  \node (i22) at ( 2, -2) {$\wlab{}{x}{0}{}$};

  \draw[po] (i11) edge node[right] {\small$\lPO$} (i12);
  \draw[po] (i21) edge node[left ] {\small$\lPO$} (i22);
  %% \draw[ppo,bend left=10] (i21) edge node[right] {\small$\lPPO$} (i22);

  \draw[rf,bend right=60] (init) edge node[above,pos=0.5] {\small$\lRF$} (i11);
  \draw[rf,bend left =60] (init) edge node[above,pos=0.5] {\small$\lRF$} (i21);

  \draw[po] (init) edge node[left]  {\small$\lPO$} (i11);
  \draw[po] (init) edge node[right] {\small$\lPO$} (i21);
  \end{tikzpicture}
  %% \caption{$\Glb$: Execution graph of \ref{ex:LB}.}
  %% \label{fig:lbWeak1}
  \end{subfigure}\hfill
  %
  \begin{subfigure}[b]{.24\textwidth}\centering
  \begin{tikzpicture}[xscale=\XScale,yscale=\YScale]

  \node at (0,1.8) {$\circledb{B}$};

  \node (init) at (1,  1.5) {$\Init$};

  \node (i11) at ( 0,  0) {$\rlab{}{x}{0}{}$};
  \node (i12) at ( 0, -2) {$\wlab{}{y}{1}{}$};

  \node (i21) at ( 2,  0) {$\rlab{}{y}{0}{}$};
  \node (i22) at ( 2, -2) {$\wlab{}{x}{0}{}$};

  \draw[po] (i11) edge node[right] {\small$\lPO$} (i12);
  \draw[po] (i21) edge node[left ] {\small$\lPO$} (i22);
  %% \draw[ppo,bend left=10] (i21) edge node[right] {\small$\lPPO$} (i22);

  \draw[rf] (i22)  edge node[below] {\small$\lRF$} (i11);
  \draw[rf,bend left=60] (init) edge node[above,pos=0.5] {\small$\lRF$} (i21);

  \draw[po] (init) edge node[left]  {\small$\lPO$} (i11);
  \draw[po] (init) edge node[right] {\small$\lPO$} (i21);
  \end{tikzpicture}
  %% \caption{Execution of \ref{ex:LB-TA} and \ref{ex:LB-fake}.}
  %% \label{fig:LB-nodep-execs}
  \end{subfigure}\hfill
  %
  \begin{subfigure}[b]{.24\textwidth}\centering
  \begin{tikzpicture}[xscale=\XScale,yscale=\YScale]

  \node at (0,1.8) {$\circledb{C}$};

  \node (init) at (1,  1.5) {$\Init$};

  \node (i11) at ( 0,  0) {$\rlab{}{x}{0}{}$};
  \node (i12) at ( 0, -2) {$\wlab{}{y}{1}{}$};

  \node (i21) at ( 2,  0) {$\rlab{}{y}{1}{}$};
  \node (i22) at ( 2, -2) {$\wlab{}{x}{1}{}$};

  \draw[po] (i11) edge node[right] {\small$\lPO$} (i12);
  \draw[po] (i21) edge node[left ] {\small$\lPO$} (i22);
  %% \draw[ppo,bend left=10] (i21) edge node[right] {\small$\lPPO$} (i22);

  \draw[rf] (init) edge node[below] {} (i11);
  \draw[rf] (i12)  edge node[below] {\small$\lRF$} (i21);

  \draw[po] (init) edge node[left]  {\small$\lPO$} (i11);
  \draw[po] (init) edge node[right] {\small$\lPO$} (i21);
  \end{tikzpicture}
  %% \caption{$\Glb$: Execution graph of \ref{ex:LB}.}
  %% \label{fig:lbWeak1}
  \end{subfigure}\hfill
  %
  \begin{subfigure}[b]{.24\textwidth}\centering
  \begin{tikzpicture}[xscale=\XScale,yscale=\YScale]

  \node at (0,1.8) {$\circledb{D}$};

  \node (init) at (1,  1.5) {$\Init$};

  \node (i11) at ( 0,  0) {$\rlab{}{x}{1}{}$};
  \node (i12) at ( 0, -2) {$\wlab{}{y}{1}{}$};

  \node (i21) at ( 2,  0) {$\rlab{}{y}{1}{}$};
  \node (i22) at ( 2, -2) {$\wlab{}{x}{1}{}$};

  \draw[po] (i11) edge node[right] {\small$\lPO$} (i12);
  \draw[po] (i21) edge node[left ] {\small$\lPO$} (i22);
  %% \draw[ppo,bend left=10] (i21) edge node[right] {\small$\lPPO$} (i22);

  \draw[rf] (i22) edge node[below] {}             (i11);
  \draw[rf] (i12) edge node[below] {\small$\lRF$} (i21);

  \draw[po] (init) edge node[left]  {\small$\lPO$} (i11);
  \draw[po] (init) edge node[right] {\small$\lPO$} (i21);
  \end{tikzpicture}
  %% \caption{Execution of \ref{ex:LB-TA} and \ref{ex:LB-fake}.}
  %% \label{fig:LB-nodep-execs}
  \end{subfigure}

\caption{Графы сценариев исполнения программы \ref{ex:LB-nodep}.}
\label{fig:LB-nodep-execs}
\end{figure}

}


Далее введем формальное определение графов сценариев исполнения.
Но сначала необходимо также ввести тип семантических меток (алфавита)
для описания абстракции разделяемой памяти.

Введем следующие множества:
\begin{itemize}
  \item $\Tid \suq \N$ обозначает множество \emph{идентификаторов потоков}, 
    а поток с идентификатором $t_0 \defeq 0$
    обозначает выделенный \emph{инициализирующий} поток;
  \item $\Loc$ обозначает множество \emph{разделяемых переменных} 
    (или \emph{локаций});
  \item $\Mod \defeq \set{\na, \rlx, \acq, \rel, \acqrel, \sco}$
    обозначает множество \emph{режимов доступа} (\emph{access modes})
    к разделяемым переменным;
  \item $\Val$ обозначает множество возможных \emph{значений}. 
\end{itemize}  

Также определим множество меток $\MemLab$, 
соответствующих абстракции разделяемой памяти. 
Метка $l \in \MemLab$ принимает одну из следующих форм:
\begin{itemize}
  \item $\rlab{o}{x}{v}$ --- метка операции чтения значения $v$ из переменной $x$, 
    аннотированная режимом доступа $o$;
  \item $\wlab{o}{x}{v}$ --- метка операции записи значения $v$ в переменную $x$, 
    аннотированная режимом доступа $o$;
  \item $\lF^o$ --- метка операции барьера, аннотированная режимом $o$.
\end{itemize}
Если у метки опущен режим доступа, то будем считать что 
она аннотирована режимом $\rlx$.

\begin{definition}
  \label{def:exec-graph}
  \emph{Граф сценария исполнения} (\emph{execution graph}) $G$ 
  это кортеж $\tup{\lE, \lLAB, \lPO, \lRMW, \lRF, \lCO}$.
  Компоненты этого кортежа определены следующим образом.
  \begin{itemize}

    \item $\lE \suq \N$ --- это множество событий.

    \item $\lTID : \lE \fun \Tid$ --- это функция, 
      которая присваивает каждому событию идентификатор потока.
      Множество событий, принадлежащих инициализирующему потоку,
      определяется как ${\lEi \defeq \set{e \in \lE \sth \lTID(e) = t_0}}$.

    \item $\lLAB : \lE \fun \MemLab$ --- это функция, 
      которая назначает каждому событию метку. 
      Данная функция также индуцирует частично определенные функции
      $\lTYP$, $\lLOC$, $\lMOD$, $\lVAL$, которые возвращают
      тип, локацию, режим доступа и значение метки соответственно. 
      Также положим, что $\lR$, $\lW$ и $\lF$ обозначают подмножества 
      событий с меткой операции чтения, записи и барьера соответственно.

    \item $\lPO \suq \lE \times \lE$ --- это отношение 
      \emph{программного порядка} (\emph{program order}).
      Это отношение строгого частичного порядка на событиях, 
      которое полностью упорядочивает все события внутри одного потока
      согласно потоку управления программы. 
      Дополнительно полагается, что инициализирующие события $\lEi$ 
      упорядочены программным порядком раньше всех других событий.
      Также введем отношение \emph{непосредственного программного порядка}
      (\emph{immediate program order}): 
      будем считать событие $e_1$ непосредственным $\lPO$-предшественником 
      события $e_2$ если $e_1$ предшествует $e_2$ 
      и между ними нет других событий.
      \begin{equation*}
        \lPOimm \defeq \lPO \setminus (\lPO \seqc \lPO)
      \end{equation*}

    \item $\lRMW \suq \lRex \seqc \lPOimm \cap \lEQLOC \seqc \lWex$ ---
      отношение соединяющие \emph{атомарные пары событий чтения-записи}. 
      Если $\tup{r, w} \in \lRMW$ тогда считается, что данная пара событий
      возникла в ходе исполнения одной инструкции атомарного чтения-записи, 
      например, инструкции \emph{атомарного сравнения с обменом} (\CAS).

    \item $\lRF \suq [\lW] \seqc \lEQLOC \cap \lEQVAL \seqc [\lR]$ --- отношение 
      \emph{читает-из} (\emph{reads-from}). 
      Это отношение связывает событие-запись с событиями-чтениями, 
      которые выполняют операцию чтения из него. 
      Для каждого события чтения должно существовать 
      событие записи, из которого выполняется чтение: 
      $$ r \in \lR \implies \exists w \in \lW \ldotp \tup{w, r} \in \lRF.$$
      Более того, каждое событие чтения может быть связано только с одним событием записи:
      $$ \tup{w_1,r} \in \lRF \wedge \tup{w_2,r} \in \lRF \implies w_1 = w_2.$$

      Дополнительно будем рассматривать внутреннею (\emph{internal}) 
      и внешнюю (\emph{external}) $\lRFE$ версию отношения ``читает-из''
      (обозначается как $\lRFI$ и $\lRFE$ соответственно), 
      в зависимости от того принадлежит ли пара событий записи и чтения
      одному потоку или разным потокам.
      \[\def\arraystretch{1}
       \begin{array}{c@{\qquad}c@{\qquad}c@{\qquad}c}
         \lRFI \defeq \lRF \cap \lPO      &
         \lRFE \defeq \lRF \setminus \lPO
       \end{array}
      \]

    \item $\lCO \suq [\lW] \seqc \lEQLOC \seqc [\lW]$ --- это отношение 
      \emph{когерентности}. Это отношение строгого частичного порядка на событиях, 
      которое полностью упорядочивает все операции записи в одну локацию. 
      Это отношение представляет порядок, в котором операции записи 
      продвигаются в основную память и становятся видимы другим потокам. 
      \begin{equation*}
        \forall w_1, w_2 \in \lW \ldotp~ 
          \lLOC(w_1) = \lLOC(w_2) \implies \tup{w_1, w_2} \in \lCO \cup \lCO^{-1}
      \end{equation*}
      По аналогии с отношением ``читает-из'' также определим
      внутреннею и внешнюю версии отношения когерентности.
      \[\def\arraystretch{1}
       \begin{array}{c@{\qquad}c@{\qquad}c@{\qquad}c}
         \lCOI \defeq \lCO \cap \lPO      &
         \lCOE \defeq \lCO \setminus \lPO
       \end{array}
      \]

  \end{itemize}
\end{definition}

Множество всех графов сценариев исполнения будем обозначать как~$\ExecG$.

\begin{definition}
  \label{def:ax-memory-model}
  \emph{Аксиоматическая модель памяти} (\emph{axiomatic memory model}) $M$ 
  задается как подмножество графов сценариев исполнения: $M \suq \ExecG$.
\end{definition}

\begin{definition}
  \label{def:memory-model-cons}
  Граф $G$ называется \emph{консистентным} с точки зрения модели $M$, 
  или просто $M$-\emph{консистентным}, если $G \in M$.
\end{definition}

Модели памяти, сохраняющие программный порядок, накладывают 
ограничение консистентности требующее, чтобы объединение 
отношений программного порядка и ``читает-из'' было ацикличным. 

\begin{definition}
Будем говорить, что граф сценария исполнения $G$ 
\emph{сохраняет программный порядок}, если выполняются следующее условие: 
\begin{itemize}
  \item $\lPO \cup \lRF$ является ацикличным отношением.
    \labelAxiom{$\lPORF$-acyclic}{ax:porf-acyc}
\end{itemize}
\end{definition}

Обозначим множество всех таких графов как $\PorfExecG$.
Также будем говорить, что модель памяти $M$, 
заданная в аксиоматическом стиле, сохраняет программный порядок, 
если любой $M$-консистентный граф сохраняет программный порядок, 
то есть ${M \suq \PorfExecG}$.

Например, среди графов, показанных на рисунке~\ref{fig:lb-nodep-execs}, 
графы \circledb{A}, \circledb{B} и \circledb{C} сохраняют программный порядок, 
а граф \circledb{D} --- нет, так как он содержит $\lPO \cup \lRF$ цикл.
Таким образом, сценарий исполнения программы \ref{ex:lb-nodep},
соответствующий графу \circledb{D},
и в результате которого в локальные переменные $a$ и $b$ записано значение $1$,
запрещен моделями памяти, сохраняющими программный порядок.

Модели памяти, сохраняющие синтаксические зависимости, 
могут допускать некоторые $\lPO \cup \lRF$ цикличные графы. 
Данные модели гарантируют сохранение порядка между событиями
одного потока только если они связаны отношением 
\emph{сохраняемого программного порядка} (\emph{preserved program order}) $\lPPO$, 
которое является подмножеством отношения программного порядка $\lPO$. 
Отношение сохраняемого программного порядка
строится с помощью отношения \emph{синтаксических зависимостей} между событиями, 
данное отношение включает отношения зависимости по данным, по управлению, и другие. 

\begin{definition}
  \label{def:ext-exec-graph}
  \emph{Расширенным графом сценария исполнения} будем называть
  обычный граф сценария исполнения (определение~\ref{def:exec-graph}), 
  дополненный отношениями 
  \emph{зависимости по данным} (\emph{data dependency}) $\lDATA$, 
  \emph{зависимости по потоку управления} (\emph{control dependency}) $\lCTRL$, 
  \emph{зависимости по целевому адресу} (\emph{address dependency}) $\lADDR$, 
  и \emph{зависимость по операции \CAS} (\emph{\CAS dependency}) $\lRMWDEP$.
\end{definition}

В контексте моделей памяти, сохраняющих синтаксические зависимости,
под графом сценария исполнения будем подразумевать расширенный граф, 
который дополнен отношениями зависимости. 

\begin{definition}
  \label{def:imm-deps-rel}
  Для расширенного графа сценария исполнения определим 
  объединенное отношение \emph{зависимости} (\emph{dependency}) 
  следующим образом:
  $$ \lDEPS \defeq \lDATA \cup \lCTRL \cup \lADDR \seqc \lPO^? \cup \lRMWDEP. $$
\end{definition}

Точное определение сохраняемого программного порядка может 
варьироваться в зависимости от конкретной модели памяти, 
но как правило оно включает как минимум объединение 
отношений зависимости, представленных выше. 
Далее модели памяти, сохраняющие синтаксические зависимости, 
накладывают ограничение консистентности требующее, чтобы объединение 
отношений сохраняемого программного порядка и 
внешнего отношения ``читает-из'' было ацикличным. 

\begin{definition}
Будем говорить, что граф сценария исполнения $G$ 
\emph{сохраняет синтаксические зависимости}, если выполняются следующие условия: 
\begin{itemize}
  \item $\lDEPS \suq \lPPO$;
    \labelAxiom{$\lPPO$-deps}{ax:ppo-deps}
  \item $\lPPO \cup \lRFE$ является ацикличным отношением.
    \labelAxiom{$\lPPORF$-acyclic}{ax:pporf-acyc}
\end{itemize}
\end{definition}

Обозначим множество всех таких графов как $\PporfExecG$.
Также будем говорить, что модель памяти $M$, 
заданная в аксиоматическом стиле, сохраняет программный порядок, 
если любой $M$-консистентный граф сохраняет программный порядок, 
то есть ${M \suq \PporfExecG}$.

{
\newcommand{\XScale}{1}
\newcommand{\YScale}{0.7}

\begin{figure}[b]
  \begin{subfigure}[b]{.44\textwidth}\centering
  \begin{tikzpicture}[xscale=\XScale,yscale=\YScale]

  %% \node at (0,1.5) {$\circledb{A}$};

  \node (init) at (1,  1.5) {$\Init$};
  \node (i11) at ( 0,  0) {$\rlab{}{x}{1}{}$};
  \node (i12) at ( 0, -2) {$\wlab{}{y}{1}{}$};
  \node (i21) at ( 2,  0) {$\rlab{}{y}{1}{}$};
  \node (i22) at ( 2, -2) {$\wlab{}{x}{1}{}$};

  \draw[po] (i11) edge node[right] {\small$\lPO$} (i12);
  \draw[po] (i21) edge node[left ] {\small$\lPO$} (i22);
  \draw[ppo,bend left=10] (i21) edge node[right] {\small$\lPPO$} (i22);

  \draw[rf] (i22) edge node[below,pos=0.5] {}             (i11);
  \draw[rf] (i12) edge node[below,pos=0.5] {\small$\lRF$} (i21);

  \draw[po] (init) edge node[left]  {\small$\lPO$} (i11);
  \draw[po] (init) edge node[right] {\small$\lPO$} (i21);
  \end{tikzpicture}
  \caption{Граф соответствующий программе \ref{ex:lb-nodep}.}
  \label{fig:LB-nodep-ppo-exec}
  \end{subfigure}\hfill
  %
  \begin{subfigure}[b]{.55\textwidth}\centering
  \begin{tikzpicture}[xscale=\XScale,yscale=\YScale]

  %% \node at (0,1.5) {$\circledb{B}$};

  \node (init) at (1,  1.5) {$\Init$};
  \node (i11) at ( 0,  0) {$\rlab{}{x}{1}{}$};
  \node (i12) at ( 0, -2) {$\wlab{}{y}{1}{}$};
  \node (i21) at ( 2,  0) {$\rlab{}{y}{1}{}$};
  \node (i22) at ( 2, -2) {$\wlab{}{x}{1}{}$};

  \draw[po] (i11) edge node[right] {\small$\lPO$} (i12);
  \draw[po] (i21) edge node[left ] {\small$\lPO$} (i22);
  \draw[ppo,bend right=10] (i11) edge node[left ] {\small$\lPPO$} (i12);
  \draw[ppo,bend left =10] (i21) edge node[right] {\small$\lPPO$} (i22);

  \draw[rf] (i22)  edge node[below]         {}             (i11);
  \draw[rf] (i12)  edge node[below,pos=0.5] {\small$\lRF$} (i21);

  \draw[po] (init) edge node[left]  {\small$\lPO$} (i11);
  \draw[po] (init) edge node[right] {\small$\lPO$} (i21);
  \end{tikzpicture}
  \caption{Граф соответствующий программам 
    \ref{ex:lb-fakedep}~и~\ref{ex:lb-dep}.
  }
  \label{fig:LB-dep-ppo-exec}
  \end{subfigure}

\caption{Графы сценариев исполнения обосновывающие результат ${a=b=1}$.}
\label{fig:LB-ppo-execs}
\end{figure}
}


Рассмотрим, например, пару $\lPO \cup \lRF$ цикличных графов, 
изображенных на рисунке~\ref{fig:LB-ppo-execs}.
Данные графы соответствуют сценарию исполнения 
с результатом $a = b = 1$. 
Отметим, что при этом граф, показанный на 
Рисунок~\ref{fig:LB-nodep-ppo-exec}, является $\lPPO \cup \lRFE$
ацикличным, а граф на рисунке~\ref{fig:LB-dep-ppo-exec} содержит такой цикл. 
Это объясняется тем, что в программу \ref{ex:lb-nodep} инстукции 
в левом потоке не связаны зависистью по данным, 
а в программах \ref{ex:lb-fakedep}~и~\ref{ex:lb-dep}
инструкции в обоих потоках связаны зависистью по данным. 
Таким образом, модели памяти, сохраняющие синтаксические зависимости, 
допускают сценарию исполнения с результатом $a = b = 1$ 
для программы \ref{ex:lb-nodep}, 
но не для программ \ref{ex:lb-fakedep}~и~\ref{ex:lb-dep}. 

\subsection{Структуры событий в модели \Wkm}
\label{sec:wkmo-eventstruct}

В предыдущем разделе было показано, что 
модели памяти, сохраняющие программный порядок 
или синтаксические зависимости, могут быть 
заданы в аксиоматическом стиле, 
то есть как множество консистентных графов 
сценариев исполнения.
В этом случае каждому сценарию исполнения, 
допустимому моделью памяти, соответствует единственный граф. 

К сожалению, модели памяти, сохраняющие семантические зависимости,
не могут быть заданы в таком стиле~\cite{Batty-al:ESOP15}.
Для того, чтобы обосновать некоторый результат исполнения программы, 
модели данного класса вынуждены рассматривать сразу 
несколько сценариев исполнения, которые 
моделируют спекулятивное исполнение программы.
В рамках модели \Wkm это множество сценариев исполнения 
объединяется в структуру событий особого вида.

\begin{definition}
  \label{def:eventstruct}
  \emph{Структура событий модели \Wkm} (\emph{\Wkm event structure}) $S$ 
  это кортеж $\tup{\lE, \lTID, \lLAB, \lPO, \lRMW, \lJF, \lEW, \lCO}$.
  Компоненты этого кортежа определены следующим образом.
  \begin{itemize}

    \item $\lE$, $\lTID$, $\lLAB$ определены по аналогии
      с графами сценариев исполнения (определение~\ref{def:exec-graph}).

    \item $\lPO \suq \lE \times \lE$ ---
      это отношение программного порядка, определенное по аналогии
      с графами сценариев исполнения (определение~\ref{def:exec-graph}).
      В отличие от случая графов сценариев исполнения
      в случае структуры событий отношение программного порядка
      не обязано полностью упорядочивать все события внутри одного потока.
      События, принадлежащие одному потоку не связанные 
      отношением программного порядка, называются \emph{конфликтующими}.
      Формально, отношение конфликта $\lCF$ задается следующим образом:
      \begin{equation*}
        \lCF \defeq ([\lE \setminus \lEi] \seqc {=_{\lTID}} \seqc [\lE \setminus \lEi])
                    \setminus (\lPO \cup \lPO^{-1})^?.
      \end{equation*}
      Также считается что два события находятся в отношении
      \emph{непосредственного конфликта} если они 
      имеют общего предка относительно программного порядка. 
      \begin{equation*}
        \lCFimm \defeq \lCF \cap (\lPOimm^{-1} \seqc \lPOimm)
      \end{equation*}

    \item Отношение $\lRMW$ определено по аналогии
      с графами сценариев исполнения (определение~\ref{def:exec-graph}).
    
    \item $\lJF \suq [\lW] \seqc (\lEQLOC \cap \lEQVAL) \seqc [\lR]$ --- 
      отношение \emph{обоснован-из} (\emph{justified-from}). 
      Данное отношение во многом аналогично отношению ``читает-из'' 
      из определения графов сценариев исполнения, 
      однако, как будет показано далее, для структур событий 
      отношение ``читает-из'' является производным. 
      Отношение $\lJF$ связывает событие записи с событиями чтения, 
      которые обоснованы данной записью.
      Каждое событие чтения должно быть обосновано: 
      $$ r \in \lR \implies \exists w \in \lW \ldotp \tup{w, r} \in \lJF.$$
      Требуется, чтобы связанные события имели одну и ту же локацию и значение, 
      кроме того, каждое событие чтения должно быть связано
      только с одним событием записи.
      \begin{equation*} 
        \begin{array}{rcl}
          \tup{w_1,r} \in \lJF \wedge \tup{w_2,r} \in \lJF 
             & \implies & w_1 = w_2 \\
        \end{array}
      \end{equation*} 
      Обоснование события чтения конфликтующим событием записи запрещено:
      ${\lJF \cap \lCF = \emptyset}$.
      По аналогии с отношением ``читает-из'' в случае
      графов сценариев исполнения (определение~\ref{def:exec-graph}) также 
      определяется внутреннея и внешняя версии отношения ``обоснован-из''.
      \[\def\arraystretch{1}
       \begin{array}{c@{\qquad}c@{\qquad}c@{\qquad}c}
         \lJFI \defeq \lJF \cap \lPO      &
         \lJFE \defeq \lJF \setminus \lPO
       \end{array}
      \]

    \item $\lEW \suq [\lW] \seqc (\lCF \cap \lEQLOC \cap \lEQVAL)^? \seqc [\lW]$ ---
      отношение \emph{эквивалентности на записях} (\emph{equal-writes}).
      Это отношение эквивалентности связывает конфликтующие
      (или идентичные) события записи с одной и той же локацией и значением.

    \item $\lCO \suq [\lW] \seqc \lEQLOC \seqc [\lW]$ ---
      отношение \emph{когерентности}, которое определено по аналогии с графами
      сценариев исполнения (определение~\ref{def:exec-graph}).
      В структуре событий \Wkm отношение когерентности
      полностью упорядочивает записи в одну и ту же локацию
      только с точностью до отношения эквивалентности на записях.
      \begin{equation*}
       \forall x \in \Loc \ldotp~ \forall w_1,w_2 \in \lW_{x} \ldotp~
          \tup{w_1, w_2} \in \lEW \cup \lCO \cup \lCO^{-1}
      \end{equation*}
      Также полагается, что отношение $\lCO$ замкнуто
      относительно отношения $\lEW$:
      $$\lEW \seqc \lCO \seqc \lEW \subseteq \lCO.$$

  \end{itemize}
\end{definition}

\begin{figure}[t]
  \centering
  \begin{tikzpicture}[xscale=3.5,yscale=1.5]

    \node (0)   at (1, 1.8) {$[\Init]$};

    \node (rxO) at (0, 1) {$\mese{1}{1}{1} \rlab{}{x}{0}$};
    \node (wyA) at (0, 0) {$\mese{1}{2}{1} \wlab{}{y}{1}$};

    \node (rxI) at (1, 1) {$\mese{1}{1}{2} \rlab{}{x}{1}$};
    \node (wyB) at (1, 0) {$\mese{1}{2}{2} \wlab{}{y}{1}$};

    \node (ryI) at (2, 1) {$\mese{2}{1}{} \rlab{}{y}{1}$};
    \node (wx)  at (2, 0) {$\mese{2}{2}{} \wlab{}{x}{1}$};

    \draw[po] (0) edge (rxO) edge (rxI) edge (ryI);
    \draw[jf] (0) edge[bend right] (rxO);

    \draw[po] (rxO) edge (wyA);
    \draw[po] (rxI) edge (wyB);
    \draw[po] (ryI) edge (wx);

    \draw[cf] (rxO) -- node[below] {\small$\lCF$} (rxI);

    \draw[jf] (wyA) edge node[below]          {}             (ryI);
    \draw[jf] (wx)  edge node[below, pos=0.8] {\small$\lJF$} (rxI);

    \draw[ew] (wyA) edge node[below] {\small$\lEW$} (wyB);

    \begin{scope}[on background layer]
      \draw[extractStyle] ($(rxI.north west)$) ++(0, 0.8) rectangle ($(wx.south east)$);
    \end{scope}
  \end{tikzpicture}
  \caption{\Wkm-консистентная структура событий программы \ref{ex:lb-nodep} 
           и извлеченный из нее граф сценария исполнения.}
  \label{fig:es-lb}
\end{figure}



В качестве примера рассмотрим структуру событий, 
показанную на рисунке~\ref{fig:es-lb}. 
Данная структура соответствует программам 
\ref{ex:lb-nodep} и \ref{ex:lb-fakedep}.
Можно видеть, что события ${\ese{1}{1}{1} \arrowPO \ese{1}{2}{1}}$
формируют ветку левого потока в которой 
инструкция чтения $\readInst{}{r_1}{x}$ читает
инициализирующее значение \tcode{0}.
Рядом находится конфликтующая ветка ${\ese{1}{1}{2} \arrowPO \ese{1}{2}{2}}$.
Заметим, что показаны только ребра отношения непосредственного конфликта.
Например, показано ребро ${\ese{1}{1}{1} \arrowCF \ese{1}{1}{2}}$,
но не ${\ese{1}{1}{1} \arrowCF \ese{1}{2}{2}}$, 
потому что последнее может быть выведено из первого.
Также обратим внимание, что каждое событие чтения 
обосновано некоторым событием записи, например,
можно видеть ребра
${\Init \arrowJF \ese{1}{1}{1}}$
${\ese{1}{2}{1} \arrowJF \ese{2}{1}{}}$
и ${\ese{2}{2}{} \arrowJF \ese{1}{1}{2}}$.
Наконец, две записи в локацию \tcode{y} помечены 
как эквивалентные: ${\ese{1}{2}{1} \arrowEW \ese{1}{2}{2}}$.

Отметим, что в отличие от классического определения простых структур событий, 
структуры событий в модели \Wkm содержат не одно отношение 
причинно-следственной связи $\ca$, а множество отношений 
c различной семантикой. 
В качестве аналога отношения причинно-следственной связи $\ca$
для \Wkm структуры событий можно было бы рассматривать 
транзитивное замыкание объединения отношений $\lPO$ и $\lJF$, 
то есть отношение $(\lPO \cup \lJF)^*$. 
Однако, можно видеть, что в этом случае \Wkm структура событий 
нарушает аксиомы иррефлексивности и наследственности отношений конфликта. 
Например, в структуре, показанной на рисунке~\ref{fig:es-lb}, 
событие $\ese{1}{1}{2}$ зависит от события $\ese{1}{1}{1}$, 
которое находится с ним в конфликте.
Если допустить, что выполняется аксиома наследственности конфликта,
тогда событие $\ese{1}{1}{2}$ должно быть в конфликте с самим собой, 
что нарушает аксиому иррефлексивности конфликта. 
Таким образом, оказывается, что классическая теория простых структур событий 
неприменима к модели \Wkm.

Структура событий \Wkm может кодировать несколько графов сценариев исполнения. 
Чтобы задать множество извлекаемых графов введем аналог понятия 
конфигурации для структур событий \Wkm.
Заметим, что здесь вновь проявляется расхождение теории 
структур событий \Wkm от классической теории простых структур событий. 
Рассмотрим, например, множество состоящее из одного события $\{\ese{1}{1}{2}\}$ 
структуры, показанной на рисунке~\ref{fig:es-lb}.
Чтобы сформировать конфигурацию, недостаточно просто 
взять замыкание этого множества относительно отношений $\lPO$ и $\lJF$,
так как, например, событие $\ese{2}{1}{}$ обосновано событием $\ese{1}{2}{1}$,
которое в свою очередь находится в конфликте с исходным событием $\ese{1}{1}{2}$.
Чтобы обойти данное ограничение, можно вместо 
события записи $\ese{1}{2}{1}$ взять эквивалентное ему событие записи $\ese{1}{2}{2}$.
Для того чтобы формализовать это построение, 
в модели \Wkm вводится производное отношение ``читает-из'' $\lRF$, 
которое расширяет отношение $\lJF$ на классы эквивалентности по отношению $\lEW^*$.

\begin{definition}
  \label{def:wkm-rf}
  Для \Wkm структуры событий $S$ отношение \emph{читает-из} $\lRF$
  определяется следующим образом:
  $$\lRF \defeq (\lEW^* \seq \lJF) \setminus \lCF.$$
\end{definition}

\begin{definition}
\label{def:wkm-cfg}
\emph{Обоснованной конфигурацией} структуры событий $S$
называется подмножество событий $C \subseteq S.\lE$, такое что:
\begin{itemize}
  \item $C$ бесконфликтно: $\lCF \cap C \times C = \emptyset$;
  \item $C$ замкнуто относительно $\lPO$: $\dom{\lPO \seqc [C]} \suq C$;
  \item $C$ полно относительно $\lRF$: $C \cap \lR \suq \cod{[C] \seqc \lRF}$.
\end{itemize}
\end{definition}

\begin{definition}
\label{def:wkm-extracted}
Будем говорить, что граф сценария исполнения $G$
может быть \emph{извлечен} из структуры событий $S$, 
что обозначается как $S \rhd G$,
если множество событий графа $G.\lE$ является 
обоснованной конфигурацией структуры событий $S$, 
такой что $G = S\rst{G.\lE}$.
\end{definition}

%% TODO: example of inconsistent event-struct, refer to PO+JF acyc and consistency def.
